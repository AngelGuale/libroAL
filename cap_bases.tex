%bases
\chapter{Base y Dimensión}

\section{Base}
\begin{dfn}[Base]
Un conjunto $B=\conjvect{v}{n}$\ es una $base$ de un espacio vectorial $V$ si y s\'olo si satisface lo siguiente:
\begin{itemize}
\item B genera a V
\item B es linealmente independiente
\end{itemize}

%Se conviene decir que si $V=\llav{0_v}$, entonces su base es el conjunto vac\'io $B_V=\llav{\ }$
\end{dfn}

\begin{ejemplo}
Determine si el conjunto $B=\llav{
1+x, 3x+x^2, x^2+1
}$\ es una base de \pdos
\end{ejemplo}
\sol
Para determinar si B genera a V debemos comprobar que existen escalares tales:~\\

\[a_0+a_1 x+a_2 x^2=
\beta_1(1+x)+
\beta_2(3x+x^2)+
\beta_3(x^2+1)
\]
Lo cual provoca el siguiente sistema de ecuaciones, el cual tendría que ser consistente:
$$\sisteq{\beta_1 + \beta_3 = a_0\\\beta_1 + 3\beta_2 = a_1\\\beta_2 + \beta_3 = a_2}$$
que desemboca en el siguiente sistema reducido:
$$\reducir{{rrr|l}}{1&0&1&a_0\\1&3&0&a_1\\0&1&1&a_2}
\underrightarrow{-f_1+f_2}
\reducir{rrr|l}{1&0&1&a_0\\0&3&-1&a_1-a_0\\0&1&1&a_2}
\underrightarrow{f_2 - 3f_3}
\reducir{rrr|l}{1&0&1&a_0\\0&3&-1&a_1-a_0\\0&0&-4&a_1-a_0-3a_2}$$

Este sistema tiene solución siempre, por lo que no hay condiciones sobre $a_0, a_1, a_2$
\[gen(B)=
\llaves{a_0+a_1 x+a_2 x^2 \in \pdos}{
a_0, a_1, a_2 \in \dobler
}=\pdos
\]

Ahora veamos la independencia lineal, debe haber solución única para

\[\beta_1(1+x)+
\beta_2(3x+x^2)+
\beta_3(x^2+1)=
0+0x+0x^2\]
Esto nos conduce a un sistema similar al anterior
$$\reducir{{rrr|l}}{1&0&1&0\\1&3&0&0\\0&1&1&0}
\underrightarrow{-f_1+f_2}
\reducir{rrr|l}{1&0&1&0\\0&3&-1&0\\0&1&1&0}
\underrightarrow{f_2 - 3f_3}
\reducir{rrr|l}{1&0&1&0\\0&3&-1&0\\0&0&-4&0}$$
Como vemos, tiene solución única para los escalares, la solución trivial $\beta_1=0, \beta_2=0, \beta_3=0$. El conjunto es linealmente independiente.


Por tanto como B genera a V y es linealmente independiente, B es una Base de V.



\begin{theorem}\label{th_masde_n}
Sea $B=\conjvect{v}{n}$ una base del espacio vectorial $V$, entonces cualquier conjunto de m\'as de n vectores es linealmente dependiente

\end{theorem}

\begin{proof}
Por contradicción ~\\
Sea $S=\conjvect{w}{m}$ un conjunto de vectores de $V$ con $m>n$, supongamos que $S$ es linealmente independiente. Como $B$ es una base entonces cualquier elemento se puede escribir como combinación lineal de los elementos de $B$

\begin{align*}
    \begin{array}{ccccccccc}
        w_1 &=& a_{11} v_1&+&a_{12} v_2 &+&\ldots &+&a_{1n} v_n\\
        w_2&=&a_{21} v_1&+&a_{22} v_2 &+&\ldots &+&a_{2n} v_n\\
        \vdots&\vdots&\vdots&\vdots&\cdots&\vdots&\vdots&\vdots&\vdots\\
        w_m&=&a_{m1} v_1&+&a_{n2} v_2 &+&\ldots &+&a_{mn} v_n\\
    \end{array}
\end{align*}

Si $S$ es linealmente independiente, entonces al tener una combinación lineal igualada al vector cero implicaría que todos esos escalares son iguales a cero

\begin{eqnarray*}
\beta_1 w_1+\beta_2 w_2 +\ldots +\beta_m w_m=0_v
\end{eqnarray*}

Al combinar estas ecuaciones tenemos
\begin{eqnarray*}
\beta_1 (a_{11} v_1+a_{12} v_2 +\ldots +a_{1n} v_n)+
\beta_2 (a_{21} v_1+a_{22} v_2 +\ldots +a_{2n} v_n) +\ldots \\+
\beta_m (a_{m1} v_1+a_{n2} v_2 +\ldots +a_{mn} v_n)
=0_v
\end{eqnarray*}

Reordenando esta ecuación tendríamos


\begin{eqnarray*}
(\beta_1 a_{11}+\beta_2 a_{21}+\ldots+\beta_m a_{m1})v1+
(\beta_1 a_{12}+\beta_2 a_{22}+\ldots+\beta_m a_{m2})v_2+
\ldots\\+
(\beta_1 a_{1n}+\beta_2 a_{2n}+\ldots+\beta_m a_{mn})v_n=0_v
\end{eqnarray*}

Pero $B$ es una base, por tanto es linealmente independiente, de esta manera, los escalares de esta combinación lineal deben ser igual a cero

\begin{align*}
\begin{array}{ccccccccc}
\beta_1 a_{11}&+&\beta_2 a_{21}&+&\ldots&+&\beta_m a_{m1}&=&0\\
\beta_1 a_{12}&+&\beta_2 a_{22}&+&\ldots&+&\beta_m a_{m2}&=&0\\
\vdots&\vdots&\vdots&\vdots&\cdots&\vdots&\vdots&\vdots&\vdots\\
\beta_1 a_{1n}&+&\beta_2 a_{2n}&+&\ldots&+&\beta_m a_{mn}&=&0\\
\end{array}
\end{align*}


Este es un sistema homogéneo (un sistema igualado todo a cero), si $m>n$ habrían más incógnitas que ecuaciones y por tanto el sistema tendría infinitas soluciones para $\beta_i$, lo cual contradice la suposición de que $S$ es un conjunto linealmente independiente. Contradicción. Luego, $S$ es linealmente dependiente.

\end{proof}

\begin{theorem}
Sea $V$ un espacio vectorial, entonces cualquier base $B$ de $V$ tiene el mismo número de elementos
\end{theorem}

\begin{proof}
Sean $B_1=\conjvect{v}{n}$ y $B_2=\conjvect{u}{m}$ dos bases de $V$, demostraremos que estos conjuntos tienen igual cantidad de elementos, es decir $m=n$.
Si $m>n$ entonces por el teorema ~\ref{th_masde_n}, ya que $B_1$ es base de $V$, $B_2$ ser\'ia linealmente dependiente, lo cual es una contradicción a la hipótesis de que $B_2$ es una base de $V$.

Si $n>m$ entonces por el mismo teorema ~\ref{th_masde_n}, debido a que $B_2$ es base de $V$, entonces $B_1$ ser\'ia linealmente dependiente, lo cual es una contradicción.

Luego, la única posibilidad es que $m=n$. $\blacksquare$
\end{proof}

Este resultado anterior nos indica que todas las bases de un mismo espacio vectorial tienen el mismo número de vectores, esto permite dar una caracter\'istica propia del espacio vectorial, que a continuación ser\'a definido como dimensión.

~\\
~\\
%\begin{quote}
%\small Un matemático, como un pintor o un poeta, es un fabricante de modelos. Si sus modelos son más duraderos que los de estos últimos, es debido a que están hechos de ideas. Los modelos del matemático, como los del pintor o los del poeta deben ser hermosos. La belleza es la primera prueba; no hay lugar permanente en el mundo para unas matemáticas feas. ~\\-G.H.HARDY
%\end{quote}
\newpage
\section{Dimensión de un espacio vectorial}
\begin{dfn}[Dimensión]
Sea $V$ un espacio vectorial sobre un campo \doblek, y $B$ una base de $V$, con un número finito de vectores. Se define como dimensión de $V$ al número de elementos de $B$ y se denota por $dimV$.
~\\
%Se conviene decir que si $V=\llav{0_v}$ entonces $dimV=0$
\end{dfn}

Si la base de un espacio vectorial $V$ posee infinitos vectores, entonces se dice que $V$ es de  \textit{dimensión infinita} 

~\\
Por ejemplo, los espacios clásicos que hemos trabajado tienen dimensiones bien conocidas. El espacio $\rdos $ posee dimensión 2 ya que $\llav{\vectrdos{1}{0}, \vectrdos{0}{1}}$ es una base de $\rdos$ y posee dos vectores. Además esto nos asegura que cualquier otra base de $ \rdos$ tendrá exactamente 2 vectores.
~\\

Las dimensiones del resto de espacios conocidos la mostramos a continuación:

\begin{itemize}
\item $\rtres $ posee dimensión 3
\item $\rn$ posee dimensión $n$
\item $\puno$ posee dimensión 2
\item $\pdos$ posee dimensión 3
\item $\pn$ posee dimensión $n+1$
\item $\mdosxdos$ posee  dimensión 4
\item $\mmxn$ posee  dimensión $m\times n$ 
\item $S_{nxn}$ posee  dimensión $\frac{n(n+1)}{2}$ 
\item El espacio trivial $\{\mathbf{n}_v\}$ posee dimensión 0

\end{itemize}
\begin{ejemplo}
Calcule la dimensión del subespacio $\svrtres{H}{a}{b}{c}{\begin{array}{c}
a+b-c=0\\
b=2c
\end{array}}$
~\\
\sol
Si obtenemos una base del subespacio H tendríamos:
\[b=2c\]
\[ a=c-b=c-2c=-c\]
\[B_H=\llav{\vectrtres{-1}{2}{1}}\]
Como solo hay un vector en la base, la dimensión de $H$ es igual a 1.

\end{ejemplo}
%%%%%%%%%%%%%%%%%%%%%%%%%%%%%%%%%%%%%%%%

\begin{obsimp}
En cualquier espacio vectorial, el vector nulo $\mathbf{0}_v$ es combinación lineal de cualquier conjunto finito de elementos de $V$ \conjvect {v}{n} por $\mathbf{0}_v = 0 v_1 + 0 v_2 + \ldots + 0 v_n$.\\
\end{obsimp}

\begin{dfn}
Un subconjunto $S$ de un espacio vectorial $(V, +, \cdot)$ se dice que es un subconjunto linealmente independiente maximal de $V$ si\\
\begin{enumerate}
\item $S$ es linealmente independiente.
\item Para cualquier $w \in V$ y $w \notin S$ el subconjunto $S'$ de $V$ formado por $S \cup \{w\}$ es linealmente dependiente.
\end{enumerate}
\end{dfn}

\begin{theorem}
Sea A un subconjunto de un espacio vectorial V. Si A es linealmente independiente maximal, entonces A es una base para V. Recíprocamente si A es una base entonces es un conjunto linealmente independiente maximal.
\end{theorem}
\begin{proof}
Veamos que $A$ genera a $V$. Sea $w \in V$ y $w \notin A$, entonces $\{w\}\cup A$ es un conjunto linealmente dependiente. Así la combinación lineal $\alpha_0 w+\alpha_1v_1+\alpha_2v_2+\hdots+\alpha_nv_n=0_V$ donde $A=\conjvect{v}{n}$ tiene solución no trivial. Si $\alpha_0=0$, entonces algún $\alpha_i;\ i=1, 2, \hdots, n$, debe ser diferente de cero. Esto último implica que A no es un conjunto linealmente independiente, por tanto $\alpha_0\neq 0$ y así obtendríamos que $w=\left(\frac{-\alpha_1}{\alpha_0}\right)v_1+\left(\frac{-\alpha_2}{\alpha_0}\right)v_2+\hdots+\left(\frac{-\alpha_n}{\alpha_0}\right)v_n$, por lo que $A$ es un conjunto generador de $V$.
\end{proof}
\begin{dfn}
Un subconjunto $M$ de un espacio vectorial se dice que es un subconjunto generador minimal de $V$ si
\begin{enumerate}
\item $M$ genera a $V$.
\item Si a $M$ se le quita un elemento $v_i$ entonces $M \setminus \{v_i\}$ no es un conjunto generador.
\end{enumerate}
\end{dfn}

\begin{theorem}
Si el espacio vectorial $(V , + , \cdot)$ tiene un subconjunto generador minimal $B$, entonces $B$ es una base para $V$.
\end{theorem}
\begin{proof}
Veamos que $B$ es linealmente independiente, para eso igualamos una combinación lineal de esos vectores al cero vector:
$$\alpha_1 v_1 + \alpha_2  v_2 + \hdots + \alpha_n v_n = \mathbf{0}_V $$
Por contradicción, asumiremos que $B$ es linealmente dependiente, eso implica que: $\exists \alpha_i \neq 0: i \leq n$\\
Por lo tanto, podemos escribir el vector $v_i$ correspondiente al escalar $\alpha_i$ como combinación lineal de los $n-1$ vectores restantes:
$$v_i = \left( \frac{-\alpha_1}{\alpha_i} \right)v_1 + \left( \frac{-\alpha_2}{\alpha_i} \right)v_2 + \hdots + \left( \frac{-\alpha_n}{\alpha_i} \right)v_n$$
Y esto implica que: 
$$gen\conjvect{v}{n} \subseteq gen\conjvect{v}{{n-1}}$$
Luego:
$$gen\{B\} \subseteq gen\{B\} \setminus \{v_i\}$$
Pero esto contradice la hipótesis de que $B$ es generador minimal, por lo tanto, este conjunto debe ser linealmente independiente.

\end{proof}


\section{Propiedades de las bases}
%
%Si $V$ es un subespacio vectorial y $B = \conjvect{v}{n}$ es una base para $V$, entonces los coeficientes de la combinación lineal $\alpha_1 v_1 + \alpha_2 v_2 + \ldots + \alpha_n v_n = v$ se denominan las coordenadas de $v$ respecto a la base $B$ y será representado por 
%$$(v)_B = \left( \begin{array}{c}
%\alpha_1\\
%\alpha_2\\
%\vdots\\
%\alpha_n  \end{array} \right)$$

%\begin{obsimp}
%Sea $V$ un espacio vectorial. Si una base está conformada por $n$ vectores, entonces todas las demás bases tendrán $n$ vectores.
%\end{obsimp}

%\begin{obsimp}
%Sea $V$ un espacio vectorial. Si existe una base con $n$ vectores, diremos que $n$ es la dimensión de $V$ y será denotado como $dim (V) = n$.
%\end{obsimp}
%
%\begin{theorem}
%Si el espacio vectorial $(V , + , \odot)$ tiene un subconjunto linealmente independiente maximal \conjvect{v}{n} entonces es una base para $V$.
%
%\end{theorem}

\begin{theorem}
Sea A=$\conjvect{v}{n}$ un subconjunto linealmente independiente de un espacio vectorial V. Sea $w\in V$ y $A'=A\cup\llav{w}$. Entonces A' es linealmente independiente si y solo si $ w \notin gen\{A\}$
\end{theorem}
 
\begin{proof}
Si $w$ es un elemento de $gen\{A\}$, entonces existen escalares $\alpha_1, \alpha_2,\\ \hdots, \alpha_n$ tales que $w=\alpha_1v_1+\alpha_2v_2+\hdots+\alpha_nv_n$. Así $(-1)w+\alpha_1v_1+\alpha_2v_2+\hdots+\alpha_nv_n=0_V$ y esto implica que $A'$ es linealmente dependiente.\\
Supongamos ahora que $A'$ es linealmente dependiente entonces existen escalares  $\alpha_1, \alpha_2, \hdots, \alpha_n$ no todos iguales a cero tales que $\alpha_0w+\alpha_1v_1+\alpha_2v_2+\hdots+\alpha_nv_n=0_V$. Si $\alpha_0=0$ entonces $\alpha_1v_1+\alpha_2v_2+\hdots+\alpha_nv_n=0_V$ esto implicaría que $\alpha_1= \alpha_2= \hdots=\alpha_n=0$ ya que $A$ es linealmente independiente, pero esto es imposible ya que $A'$ es linealmente dependiente; por lo tanto $\alpha_0\neq0$ y así $w$ es una combinación lineal de $A$, por lo tanto $w\in gen\{A\}$.
\end{proof}
\begin{theorem}
Si  A=$\conjvect{v}{n}$ es un conjunto generador de un espacio vectorial $V$ y la dimensión de $V$ es mayor o igual a 2, entonces existe un conjunto linealmente independiente contenido en $A$ que también genera a $V$.
\end{theorem}

\begin{proof}
Si $A$ es linealmente independiente no hay nada que demostrar. Supongamos que $A$ es linealmente dependiente. Sea $w_r$ el primer elemento no cero de $A$. Así $gen\{w_r\}$ es un subespacio de $V$ y $\{w_r\}$ es un conjunto linealmente independiente. Si $gen\{w_r\}=V$ se termina la prueba. Supongamos que $gen\{w_r\}\neq V$, entonces existe un elemento $w_q \in A$ que no pertenece a $gen\{w_r\}$ y por el teorema anterior, $\{w_r,w_q\}$ es un conjunto linealmente independiente. Recursivamente se obtienen vectores que no pertenezcan a el espacio generado por el conjunto linealmente independiente hasta que en un momento existirá un vector $w_p \in gen\{w_q,w_r, \hdots \}$, que al agregarlo, hará al conjunto linealmente dependiente, demostrando así que ese conjunto será linealmente independiente maximal y al mismo tiempo generador minimal de $V$.

\end{proof} 



