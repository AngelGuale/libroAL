%bases
\chapter{Base y Dimensión}

\section{Base}
\begin{dfn}[Base]
Un conjunto $B=\conjvect{v}{n}$\ es una $base$ de un espacio vectorial $V$ si y s\'olo si satisface lo siguiente:
\begin{itemize}
\item[$\spadesuit$]B genera a V
\item[$\spadesuit$]B es linealmente independiente
\end{itemize}

%Se conviene decir que si $V=\llav{0_v}$, entonces su base es el conjunto vac\'io $B_V=\llav{\ }$
\end{dfn}

\begin{ejemplo}
Determine si el conjunto $B=\llav{
1+x, 3x+x^2, x^2+1
}$\ es una base de \pdos
\end{ejemplo}
\sol
Para determinar si B genera a V debemos comprobar que existen escalares tales:~\\

\[a_0+a_1 x+a_2 x^2=
\beta_1(1+x)+
\beta_2(3x+x^2)+
\beta_3(x^2+1)
\]
Lo cual provoca el siguiente sistema de ecuaciones, el cual tendr\'ia que ser consistente:
\[
\left(
\begin{array}{rrr|r}
1&0&1&a_0\\
1&3&0&a_1\\
0&1&1&a_2
\end{array}
\right)
\sim
\left(
\begin{array}{rrr|r}
1&0&1&a_0\\
0&3&-1&a_1-a_0\\
0&1&1&a_2
\end{array}
\right)
\sim
\left(
\begin{array}{rrr|r}
1&0&1&a_0\\
0&3&-1&a_1-a_0\\
0&0&-4&a_1-a_0-3a_2
\end{array}
\right)
\]
Este sistema tiene soluci\'on siempre, por lo que no hay condiciones sobre $a_0, a_1, a_2$
\[gen(B)=
\llaves{a_0+a_1 x+a_2 x^2}{
a_0, a_1, a_2 \in \dobler
}=\pdos
\]

Ahora veamos la independencia lineal, debe haber soluci\'0n \'unica para

\[\beta_1(1+x)+
\beta_2(3x+x^2)+
\beta_3(x^2+1)=
0+0x+0x^2\]
Esto nos conduce a un sistema similar al anterior
\[
\left(
\begin{array}{rrr|r}
1&0&1&0\\
1&3&0&0\\
0&1&1&0
\end{array}
\right)
\sim
\left(
\begin{array}{rrr|r}
1&0&1&0\\
0&3&-1&0\\
0&1&1&0
\end{array}
\right)
\sim
\left(
\begin{array}{rrr|r}
1&0&1&0\\
0&3&-1&0\\
0&0&-4&0
\end{array}
\right)
\]
Como vemos, tiene soluci\'on \'unica para los escalares, la soluci\'on trivial $\beta_1=0, \beta_2=0, \beta_3=0$. El conjunto es linealmente independiente.


Por tanto como B genera a V y es linealmente independiente, B es una Base de V.



\begin{theorem}\label{th_masde_n}
Sea $B=\conjvect{v}{n}$ una base del espacio vectorial $V$, entonces cualquier conjunto de m\'as de n vectores es linealmente dependiente

\end{theorem}

\begin{proof}
Por contradicci\'on ~\\
Sea $S=\conjvect{w}{m}$ un conjunto de vectores de $V$ con $m>n$, supongamos que S es linealmente independiente. Como $B$ es una base entonces cualquier elemento se puede escribir como combinaci\'on lineal de los elementos de $B$

\begin{eqnarray*}
&w_1=a_{11} v_1+a_{12} v_2 +\ldots +a_{1n} v_n\\
&w_2=a_{21} v_1+a_{22} v_2 +\ldots +a_{2n} v_n\\
&\vdots\\
&w_m=a_{m1} v_1+a_{n2} v_2 +\ldots +a_{mn} v_n\\
\end{eqnarray*}

Si S es linealmente independiente, entonces al tener una combinaci\'on lineal igualada al vector cero implicar\'ia que todos esos escalares son iguales a cero

\begin{eqnarray*}
\beta_1 w_1+\beta_2 w_2 +\ldots +\beta_m w_m=0_v
\end{eqnarray*}

Al combinar estas ecuaciones tenemos
\begin{eqnarray*}
\beta_1 (a_{11} v_1+a_{12} v_2 +\ldots +a_{1n} v_n)+
\beta_2 (a_{21} v_1+a_{22} v_2 +\ldots +a_{2n} v_n) +\ldots \\+
\beta_m (a_{m1} v_1+a_{n2} v_2 +\ldots +a_{mn} v_n)
=0_v
\end{eqnarray*}

Reordenando esta ecuaci\'on tendr\'iamos


\begin{eqnarray*}
(\beta_1 a_{11}+\beta_2 a_{21}+\ldots+\beta_m a_{m1})v1+
(\beta_1 a_{12}+\beta_2 a_{22}+\ldots+\beta_m a_{m2})v_2+
\ldots\\+
(\beta_1 a_{1n}+\beta_2 a_{2n}+\ldots+\beta_m a_{mn})v_n=0_v
\end{eqnarray*}

Pero $B$ es una base, por tanto es linealmente independiente, de esta manera, los escalares de esta combinaci\'on lineal deben ser igual a cero

\begin{eqnarray*}
&\beta_1 a_{11}+\beta_2 a_{21}+\ldots+\beta_m a_{m1}=0\\
&\beta_1 a_{12}+\beta_2 a_{22}+\ldots+\beta_m a_{m2}=0\\
&\vdots\\
&\beta_1 a_{1n}+\beta_2 a_{2n}+\ldots+\beta_m a_{mn}=0\\
\end{eqnarray*}

Este es un sistema homog\'{e}neo (un sistema igualado a todo cero), si m>n habr\'{i}an m\'{a}s inc\'ognitas que ecuaciones y por tanto el sistema tendr\'{i}a infinitas soluciones para $\beta_i$, lo cual contradice la suposici\'on de que S es un conjunto linealmente independiente. Contradicci\'on. Luego, S es dependiente.

\end{proof}

\begin{theorem}
Sea $V$ un espacio vectorial, entonces cualquier base $B$ de $V$ tiene el mismo n\'umero de elementos
\end{theorem}

\begin{proof}
Sean $B_1=\conjvect{v}{n}$ y $B_2=\conjvect{u}{m}$ dos bases de $V$, demostraremos que estos conjuntos tienen igual cantidad de elementos, es decir $m=n$.
Si $m>n$ entonces por el teorema ~\ref{th_masde_n}, ya que $B_1$ es base de $V$, $B_2$ ser\'ia linealmente dependiente, lo cual es una contradicci\'on a la hip\'otesis de que $B_2$ es una base de $V$.

Si $n>m$ entonces por el mismo teorema ~\ref{th_masde_n}, debido a que $B_2$ es base de $V$, entonces $B_1$ ser\'ia linealmente dependiente, lo cual es una contradicci\'on.

Luego, la \'unica posibilidad es que $m=n$. $\blacksquare$
\end{proof}

Este resultado anterior nos indica que todas las bases de un mismo espacio vectorial tienen el mismo n\'umero de vectores, esto permite dar una caracter\'istica propia del espacio vectorial, que a continuaci\'on ser\'a definido como dimensi\'on.

~\\
~\\
%\begin{quote}
%\small Un matemático, como un pintor o un poeta, es un fabricante de modelos. Si sus modelos son más duraderos que los de estos últimos, es debido a que están hechos de ideas. Los modelos del matemático, como los del pintor o los del poeta deben ser hermosos. La belleza es la primera prueba; no hay lugar permanente en el mundo para unas matemáticas feas. ~\\-G.H.HARDY
%\end{quote}
\newpage
\section{Dimensión de un espacio vectorial}
\begin{dfn}[Dimensi\'on]
Sea $V$ un espacio vectorial sobre un campo \doblek, y $B$ una base de $V$, con un n\'umero finito de vectores. Se define como dimensi\'on de $V$ al n\'umero de elementos de $B$ y se denota por $dimV$.
~\\
%Se conviene decir que si $V=\llav{0_v}$ entonces $dimV=0$
\end{dfn}

Si la base de un espacio vectorial $V$ posee infinitos vectores, entonces se dice que $V$ es de  \textit{dimensi\'on infinita} 

~\\
Por ejemplo, los espacios clásicos que hemos trabajado tienen dimensiones bien conocidas. El subespacio $\rdos $ posee dimensión 2 ya que $\llav{\vectrdos{1}{0}, \vectrdos{0}{1}}$ es una base de $\rdos$ y posee dos vectores. Además esto nos asegura que cualquier otra base de $ \rdos$ tendrá exactamente 2 vectores.
~\\

Las dimensiones del resto de espacios conocidos la mostramos a continuación:

\begin{itemize}
\item $\rtres $ posee dimensión 3
\item $\rn$ posee dimensión $n$
\item $\puno$ posee dimensión 2
\item $\pdos$ posee dimensión 3
\item $\pn$ posee dimensión $n+1$
\item $\mdosxdos$ posee  dimensión 4
\item $\mmxn$ posee  dimensión $m\times n$ 
\item $S_{nxn}$ posee  dimensión $\frac{n(n+1)}{2}$ 

\end{itemize}
\begin{ejemplo}
Calcule la dimensión del subespacio $H=\llaves{\vectrtres{a}{b}{c}}{\begin{array}{r}
a+b-c=0\\
b=2c\\
\end{array}}$
~\\
\sol
Si obtenemos una base del subespacio H tendríamos:
\[b=2c\]
\[ a=c-b=c-2c=-c\]
\[B_H=\llav{\vectrtres{-1}{2}{1}}\]
Como solo hay un vector en la base, la dimensión de $H$ es igual a 1.

\end{ejemplo}
%%%%%%%%%%%%%%%%%%%%%%%%%%%%%%%%%%%%%%%%

\begin{obsimp}
En cualquier espacio vectorial, el vector nulo $\mathbf{0}_v$ es combinación lineal de cualquier familia finita de elementos de $V$ \conjvect {v}{n} por $\mathbf{0}_v = 0 v_1 + 0 v_2 + \ldots + 0 v_n$.\\
\end{obsimp}

\begin{dfn}
Un subconjunto $S$ de un espacio vectorial $(V, +, \odot)$ se dice que es un subconjunto linealmente independiente maximal de $V$ si\\
\begin{enumerate}
\item $S$ es linealmente independiente.
\item Para cualquier $w \in V$ y $w \notin S$ el subconjunto $S'$ de $V$ formado por $S$ unido con $\{w\}$ es linealmente independiente.
\end{enumerate}
\end{dfn}

\begin{theorem}
Sea A un subconjunto de un espacio vectorial V. Si A es linealmente independiente maximal, entonces A es una base para V. Recíprocamente si A es una base entonces es un conjunto linealmente independiente maximal.
\end{theorem}
\begin{proof}
Veamos que A genera a V. Sea w $\in V\setminus A$, entonces $\{w\}\cup A$ es un conjunto linealmente dependiente. Así la combinación lineal $\alpha_0 w+\alpha_1v_1+\alpha_2v_2+..\alpha_nv_n=0_V$ donde $A=\llav{v_1, v_2, .., v_n}$ tiene solución no trivial. Si $\alpha_0=0$, entonces algún $\alpha_i;\ i=1, 2.., n$, debe ser diferente de cero. Esto último implica que A no es un conjunto linealmente independiente, por tanto $\alpha_0\neq 0$ y así obtendríamos que $w=\frac{-\alpha_1}{\alpha_0}v_1+\frac{-\alpha_2}{\alpha_0}v_2+..\frac{-\alpha_n}{\alpha_0}v_n$, por lo que A es un conjunto generador de V.
\end{proof}
\begin{dfn}
Un subconjunto de $M$ de un espacio vectorial se dice que es un subconjunto generador minimal de $V$ si
\begin{enumerate}
\item $M$ genera a $V$.
\item Si a $M$ se le quita un elemento $v_0$ entonces $M \setminus \{w\}$ no es un conjunto generador.
\end{enumerate}
\end{dfn}

\begin{theorem}
Si el espacio vectorial $(V , +, \odot)$ tiene un subconjunto generador minimal $B$, entonces $B$ es una base para $V$.
\end{theorem}


\section{Propiedades de las bases}
%
%Si $V$ es un subespacio vectorial y $B = \conjvect{v}{n}$ es una base para $V$, entonces los coeficientes de la combinación lineal $\alpha_1 v_1 + \alpha_2 v_2 + \ldots + \alpha_n v_n = v$ se denominan las coordenadas de $v$ respecto a la base $B$ y será representado por 
%$$(v)_B = \left( \begin{array}{c}
%\alpha_1\\
%\alpha_2\\
%\vdots\\
%\alpha_n  \end{array} \right)$$

%\begin{obsimp}
%Sea $V$ un espacio vectorial. Si una base está conformada por $n$ vectores, entonces todas las demás bases tendrán $n$ vectores.
%\end{obsimp}

%\begin{obsimp}
%Sea $V$ un espacio vectorial. Si existe una base con $n$ vectores, diremos que $n$ es la dimensión de $V$ y será denotado como $dim (V) = n$.
%\end{obsimp}
%
%\begin{theorem}
%Si el espacio vectorial $(V , + , \odot)$ tiene un subconjunto linealmente independiente maximal \conjvect{v}{n} entonces es una base para $V$.
%
%\end{theorem}

\begin{theorem}
Sea A=$\llav{v_1, v_2, .., v_n}$ un subconjunto linealmente independiente de un espacio vectorial V. Sea $w\in V$ y $A'=A\cup\llav{w}$. Entonces A' es linealmente independiente si y solo si w es un elemento de $E\setminus gen(A)$
\end{theorem}

\begin{proof}
Si w es un elemento de gen(A), entonces existen escalares $\alpha_1, \alpha_2, ..\alpha_n$ tales que $w=\alpha_1v_1+\alpha_2v_2+..+\alpha_nv_n$. Así $(-1)w+\alpha_1v_1+\alpha_2v_2+..+\alpha_nv_n=0_V$ y esto implica que A' es linealmente dependiente.\\
Supongamos ahora que A' es linealmente dependiente entonces existen escalares  $\alpha_1, \alpha_2, ..\alpha_n$ no todos iguales a cero tales que $\alpha_0w+\alpha_1v_1+\alpha_2v_2+..+\alpha_nv_n=0_V$. Si $\alpha_0=0$ entonces $\alpha_1v_1+\alpha_2v_2+..+\alpha_nv_n=0_V$ esto implicaría que $\alpha_1= \alpha_2= ..=\alpha_n=0$ ya que A es linealmente independiente, pero esto es imposible ya que A' es linealmente dependiente; por lo tanto $\alpha_0\neq0$ y así w es una combinación lineal de A, por lo tanto $w\in gen(A)$.
\end{proof}
\begin{theorem}
Si  A=$\llav{v_1, v_2, .., v_n}$ es un conjunto generador de un espacio vectorial V y la cardinalidad de E es mayor o igual a 2, entonces existe un conjunto linealmente independiente contenido en A que también genera a V.
\end{theorem}
\begin{proof}
Si A es linealmente independiente no hay nada que demostrar. Supongamos que A no es linealmente independiente. Sea $w_r$ el primer elemento no cero de A. Así $gen\{w_r\}$ es un subespacio de V y $\{w_r\}$ es un conjunto linealmente independiente. Si $gen\{w_r\}=V$ se termina la prueba. Supongamos que $gen\{w_r\}\neq V$, entonces existe un elemento $w_q\subset A$ que no pertenece a $gen\{w_r\}$
\end{proof}