
\chapter{Coordenadas y Cambio de Base}
\section{Coordenadas de un vector}
\begin{dfn} [Coordenadas de un vector]

Sea $B=\conjvect{v}{n}$ una base ordenada del espacio vectorial $V$ sobre un campo \dobleK. Se denominan coordenadas de $v$ con respecto a $B; \forall v \in V$ a los escalares $\alpha _1,\alpha_2, \hdots, \alpha _n$ tales que
$v=\alpha_1 v_1+\alpha_2 v_2+ \hdots + \alpha_n v_n$
Y se define el vector $$[v]_B = \vectrcuatcent{\alpha_1}{\alpha_2}{\vdots}{\alpha_n} \in  \rn$$ el cual se denomina vector de coordenadas de $v$ con respecto a $B$. 
\end{dfn}

Por ejemplo, en $P_1$, si tenemos la base $B=\{1+x,1-2x\}$, entonces el vector $1+7x$ se escribe como
$$1+7x=3(1+x)-2(1-2x)$$
Entonces, las coordenadas de $1+7x$ con respecto a B son 3 y -2, el vector de coordenadas de $v$ con respecto a $B$ es $$[v]_B=\vectrdos{3}{-2}$$

Como podemos observar, el vector de coordenadas de $v$, depende directamente de la base, inclusive, si cambiamos el orden de la base, cambia el vector de coordenadas, de ahí la importancia de la “base ordenada” en la definición.

\section{Matriz de Cambio de Base}
\begin{dfn} Sea $V$ un espacio vectorial de dimensión $n$, $ B_1$  y$ B_2$  dos bases ordenadas $V$, y $C$ una matriz cuadrada de orden $n$, se dice que $C$ es la matriz de cambio de base de $B_1$ a $B_2$ si se cumple que:
$$\forall v \in V \ ; \ [v]_{B_2}=C[v]_{B_1}$$
\end{dfn}

\begin{theorem}


Sea V un espacio vectorial de dimensión n, $B_1={v_1,v_2,…,v_n }$  y $B_2={u_1,u_2,…,u_n}$  dos bases ordenadas V, y C es la matriz de cambio de base de $B_1$ a $B_2$ entonces:
\[C_{[B_1 \rightarrow B_2]}= \begin{pmatrix}
\vdots & \vdots & \vdots & \vdots\\
\left[v_1\right]_{B_2}&\left[v_2\right]_{B_2}& \hdots & \left[v_n\right]_{B_2}\\
\vdots & \vdots & \vdots & \vdots\\
\end{pmatrix}\]
\end{theorem}

\begin{theorem}[Corolario]
Sea V un espacio vectorial de dimensión $n$, $B_1$ y $B_2$  dos bases ordenadas $V$, y $C$ es la matriz de cambio de base de $B_1$ a $B_2$ entonces $C$ es invertible, y $C^{-1}$ es la matriz de cambio de base de $B_2$ a $B_1$.
\[
C_{[B_2\rightarrow B_1] }={C^{-1}}_{[B_1 \rightarrow B_2]}\]
\end{theorem}

\begin{ejemplo}
Sea $V=\pdos$ , y sean $B_1, B_2$ dos bases de $V$. Determine la matriz de cambio de base de $B_2$ a $B_1$. $ B_1=\llav{x^2+4x-2,x-3,x^2+1}$  y $B_2=\llav{2x^2+5x-4,x^2+2x+4,-x^2+x-4}$
~\\

La matriz de cambio de base de $B_2$ a $B_1$, estará formada por las coordenadas de los vectores de la base $B_2$ con respecto a la base $B_1$, es decir:
\[C_{[B_2 \rightarrow B_1]}= \left(\begin{array}{ccc}
\vdots & \vdots & \vdots \\
\left[2x^2+5x-4\right]_{B_1}&\left[x^2+2x+4\right]_{B_1}&  \left[-x^2+x-4\right]_{B_1}\\
\vdots & \vdots &  \vdots\\
\end{array}
\right)\]

Calculando estas coordenadas tenemos:
\[\left[2x^2+5x-4\right]_{B_1}=\vectrtres{1}{1}{1}\]
\[\left[x^2+2x+4\right]_{B_1}=\vectrtres{1}{-2}{0}\]
\[\left[-x^2+x-4\right]_{B_1}=\vectrtres{0}{1}{-1}\]

Por tanto, la matriz que nos pide el ejercicio es:

\[C_{B_2 \rightarrow B_1}= \left(\begin{array}{rrr}
1 & 1 & 0 \\
1&-2& 1\\
1 & 0 &  -1\\
\end{array}
\right)\]




\end{ejemplo}

\begin{ejercicio}
Sean $B_1=\llav{\vectrtres{1}{0}{2}, \vectrtres{1}{-2}{0}, \vectrtres{0}{1}{2}}$ y $B=\{u_1, u_2, u_3\}$ bases del espacio vectorial \rtres, y 

\[M_{B_1 \rightarrow B}=\left(\begin{array}{rrr}
1&-1&1\\
0&-2&1\\
1&1&1\\
\end{array}\right)\]
a)Determine los vectores de la base B
~\\b) Determine $[v]_{B_1}$ si $v=\vectrtres{1}{2}{3}$

\sol
Recordemos que las columnas de la matriz son las coordenadas de los vectores de la base de partida con respecto a la base de llegada.~\\
Por lo tanto, para resolver el literal (a) de nuestro problema necesitaríamos la matriz inversa de $M_{B_1 \rightarrow B}$~\\
Al calcularla obtenemos:~\\

\[M_{B \rightarrow B_1}=\left(\begin{array}{rrr}
3/2&-1&-1/2\\
-1/2&0&1/2\\
-1&1&1\\
\end{array}\right)\]
Las columnas de esta matriz son las coordenadas de los vectores de la base B, con respecto a la base B1:
~\\
Para el vector $u_1$~\\
$$[u_1]_{B_1}=\vectrtres{3/2}{-1/2}{-1}$$
$$u_1=(3/2)\vectrtres{1}{0}{2}+(-1/2)\vectrtres{1}{-2}{0}+(-1)\vectrtres{0}{1}{2}$$
$$u_1=\vectrtres{1}{0}{1}$$
Para el vector $u_2$~\\
$$[u_2]_{B_1}=\vectrtres{-1}{0}{1}$$
$$u_1=(-1)\vectrtres{1}{0}{2}+(0)\vectrtres{1}{-2}{0}+(1)\vectrtres{0}{1}{2}$$
$$u_1=\vectrtres{-1}{1}{0}$$
Para el vector $u_3$~\\
$$[u_1]_{B_1}=\vectrtres{-1/2}{1/2}{1}$$
$$u_1=(-1/2)\vectrtres{1}{0}{2}+(1/2)\vectrtres{1}{-2}{0}+(1)\vectrtres{0}{1}{2}$$
$$u_1=\vectrtres{0}{0}{1}$$
~\\
~\\
El literal b se reduce a simplemente calcular las coordenadas del vector $v=\vectrtres{1}{2}{3}$ con respecto a los vectores de la base $B_1$ los cuales se los coloca como columnas del sistema:
\[\left(\begin{array}{rrr|r}
1&1&0&1\\
0&-2&1&2\\
2&0&2&3\\
\end{array}\right)
\sim...\sim
\left(\begin{array}{rrr|r}
1&1&0&1\\
0&-2&1&2\\
0&0&-1&1\\
\end{array}\right)
\]

\[\alpha_3=-1, \alpha_2=-3/2, \alpha_1=5/2\]
Por lo tanto, el vector de coordenadas del vector $v$ con respecto a la base B1 es:
\[[v]_{B_1}=\vectrtres{5/2}{-3/2}{-1}\]


\end{ejercicio}

\section{Problemas}
\begin{enumerate}

\begin{prob}[]

Sean $B_1=\llav{x^2+x+1, x-1, 1-x^2}$ y $B_2=\{v_1, v_2, v_3\}$ bases del espacio vectorial \pdos, y 

\[M_{B_1 \rightarrow B_2}=\left(\begin{array}{rrr}
2&0&2\\
1&1&0\\
-1&-1&1\\
\end{array}\right)\]
a)Determine los vectores de la base $B_2$
~\\b) Determine $[v]_{B_1}=\vectrtres{1}{2}{3}$ y $[u]_{B_2}=\vectrtres{2}{1}{-1}$, determine $[2v-3u]_{B_2}$
\end{prob}

\begin{prob}[(1ra Evaluación Diciembre 2013)]
(10 puntos) Sean $B_1=\llav{v_1, v_2, v_3}$ y $B_2=\llav{u_1, u_2, u_3}$ dos bases ordenadas del espacio vectorial funcional V. Se conoce que $V=Gen\llav{\sen(x), \cos(x), x}$ y que $u_1=v_1+v_3, u_2=v_2+v_3 $ y $u_3= v_1+v_2$.
a) Calcule la matriz de transición de $B_1$ a $B2$ ~\\
b) Si $[2x]_{B_2}=\vectrtres{1}{1}{1}, [x-\sen(x)]_{B_2}=\vectrtres{1}{0}{0}, [2\sen(x)+\cos(x)]_{B_2}=\vectrtres{1}{0}{1}$, determine los vectores de $B_1$
\end{prob}



\begin{prob}[(1ra Evaluación Julio 2013)]
(10 puntos) Sean $v_1, v_2, v_3 $ vectores de un espacio vectorial $V$. Si $u_1, u_2, u_3 $ son linealmente independientes y son, respectivamente los vectores coordenados de $ v_1, v_2, v_3$ respecto a de una base $ B $ de $V$ entonces $dim V \geq 3 $
\end{prob}
\newpage
\begin{prob}[(1ra Evaluación Julio 2012)]
(10 puntos)Sean $B_1=\{v_1, v_2, v_3\}$ y $B_2=\{v_1-v_2, v_2+v_3, 2v_1\}$ dos bases de un espacio vectorial $V$. Si $[E]_{B_1}=[F]_{B_2}=\vectrtres{1}{1}{1}$ determine $[5E-2F]_{B_2}$
\end{prob}

\end{enumerate}