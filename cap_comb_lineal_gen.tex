%espacio generado combinación lineal
\chapter{Combinaciones Lineales}

\section{Combinación Lineal}
\begin{dfn}[Combinación Lineal]
Sea $v\in V$, sean \conjvect{v}{n} vectores de un espacio vectorial $V$ sobre el campo \doblek , se dice que $v$ es una combinación lineal de los vectores \conjvect{v}{n} si y solo si existen los escalares $\alpha_1, \alpha_2, \ldots, \alpha_n \in \dobleK$ tal que 
\[
v=\left(\alpha_1\odot v_1\right)\oplus
\left(\alpha_2\odot v_2\right)\oplus
\ldots \oplus
\left(\alpha_n\odot v_n\right)
\]
\end{dfn}
Nótese que la palabra clave de la definición anterior es EXISTEN, para ilustrar esto considere los
dos siguientes ejemplos:
\begin{ejemplo}
Considere el espacio vectorial \rdos; ?` es el vector \vectrdos{6}{-1} combinación lineal de los vectores \vectrdos{3}{2}, \vectrdos{1}{-1}?
\end{ejemplo}

Para resolver este problema debemos verificar si EXISTEN escalares $\alpha$, $\beta$ tales que hagan posible la combinación lineal:
\[\mbox{\vectrdos{6}{-1}} =\alpha\mbox{\vectrdos{3}{2}}+\beta\mbox{\vectrdos{1}{-1}}\]

Esto nos llevar\'a a un sistema de ecuaciones
\[
\left\{
\begin{array}{l}
3\alpha+\beta=\ 6\\
2\alpha-\beta=-1
\end{array}
\right.
\]

Si el sistema es consistente, entonces los escalares si existen y por lo tanto, dicho vector es combinación lineal de los otros dos, pero; si el sistema es inconsistente, los escalares no existen
por lo que el vector no sería combinación lineal de los otros dos.

~\\
Al resolver este sistema llegamos a la solución:
$\alpha=1$ y $\beta=3$.

~\\
Los escalares $\alpha$ y $\beta$ existen, el sistema es consistente con solución \'unica, luego el vector \vectrdos{6}{-1} es combinación lineal de los vectores \vectrdos{3}{2}, \vectrdos{1}{-1}. Ver figura ~\ref{fig:figura1}

\imagen{dosvect1}{combinación lineal de vectores en \rdos. Método del paralelogramo}{figura1}{0.5}

\begin{ejemplo}
?`El vector \vectrdos{2}{2} es combinación lineal de los vectores 
\vectrdos{-1}{2}, \vectrdos{2}{-4}?
\end{ejemplo}


Para que sea combinación lineal deben existir los escalares
$\alpha$ y $\beta$ tales que:

\[\mbox{\vectrdos{2}{2}}=\alpha\mbox{\vectrdos{-1}{2}}+
\beta\mbox{\vectrdos{2}{-4}}\]
~\\
Resolviendo el sistema por el metodo de Gauss-Jordan:

\[
\left(
\begin{array}{rr|r}
-1 & 2 & 2\\
2&-4&2
\end{array}
\right)
\underrightarrow{2f_1 + f_2}
\left(
\begin{array}{rr|r}
-1 & 2 & 2\\
0&0&6
\end{array}
\right)
\]
~\\
La última fila indica un absurdo matem\'atico, lo cual implica que el sistema es inconsistente, no
tiene solución. Como los escalares no existen, el vector \vectrdos{2}{2} NO es combinación lineal de los vectores 
\vectrdos{-1}{2}, \vectrdos{2}{-4}

\newpage
%\subsection{?`Espacio de colores?}
%Uno de los m\'etodos de explicar la combinación lineal con un ejemplo inteligible son los colores:
%
%~\\
%Suponga que usted tiene un conjunto de bolitas de plastilina con los colores Azul, Rojo, Amarillo.
%Tomamos una “bolita” amarilla y otra “bolita” roja y las combinamos (esperamos que el lector conozca qué color se formar\'a), como era de suponerse, tendremos ahora una bolita de color Naranja; es decir, el color Naranja es una Combinación Lineal de los colores Amarillo y Rojo.
%
%\imagen{amarillorojo}{combinación lineal}{figura2}{0.5}
%
%Asimismo si tomamos una bolita de color Azul y una bolita de color Amarillo, obtendremos una
%de color Verde, luego el color Verde es Combinación Lineal de los colores Azul y Amarillo.
%
%\imagen{azulamarillo}{combinación lineal}{figura3}{0.5}
%
%Ahora suponga que no toma una bolita de Azul, sino TRES bolitas de Azul y una de Amarillo; y las
%combinamos, ¡por supuesto! Tendremos otra clase de color Verde, pero un poco más oscuro, es
%decir la cantidad de bolitas son los ESCALARES de la combinación lineal y si los escalares
%cambian tendremos otro resultado.
%
%
%\imagen{verdeoscuro}{combinación lineal}{figura4}{0.4}
%
%?`Ahora est\'a más claro el concepto de combinación lineal?

\subsection{M\'as ejemplos de combinación lineal}

De ahora en adelante trabajaremos sobre espacios vectoriales sobre campo Real y con operaciones usuales, por lo que obviaremos $\oplus$ y $\odot$ por los usuales $+$ y $\cdot$

\begin{ejemplo}
Considere el espacio vectorial \pdos\ sobre el campo real, determinar si el vector $x^2+x-1$ es combinación lineal de los vectores $x+2, x^2-3, x^2+2x+1$.
\end{ejemplo}
~\\
solución

~\\
Veamos si existen los escalares $\alpha_1, \alpha_2, \alpha_3$ tales que
\[x^2+x-1=\alpha_1\left(x+2\right)+
\alpha_2\left(x^2-3\right)+
\alpha_3\left(x^2+2x+1\right)
\]

~\\
Dos polinomios son iguales si y solos si cada uno de sus coeficientes lo son, por tanto:

~\\
\[x^2+x-1=\left(\alpha_2+\alpha_3\right)x^2+
\left(\alpha_1+2\alpha_3\right)x+
\left(2\alpha_2-3\alpha_2+\alpha_3\right)
\]

Al igualar coeficientes, se origina el siguiente sistema de ecuaciones:

\begin{eqnarray*}
\alpha_2+\alpha_3=1\\
\alpha_1+2\alpha_3=1\\
2\alpha_1-3\alpha_2+\alpha_3=-1\\
\end{eqnarray*}

Resolviéndolo por Gauss:

\[
\reducir{rrr|r}{
0 &1 &1 &1\\
1&0&2&1\\
2 &-3&1&-1
}
\underrightarrow{\begin{array}{c}
    -2f_2 + f_3\\
    f_2 \longleftrightarrow f_1
\end{array}}
\reducir{rrr|r}{
1 &0 &2 &1\\
0&1&1&1\\
0 &-3&-3&-3
}
\underrightarrow{3f_2 + f_3}
\reducir{rrr|r}{
1 &0 &2 &1\\
0&1&1&1\\
0 &0&0&0
}\]

De la última fila del sistema inferimos que el sistema tiene infinitas soluciones, es decir; es consistente, que es lo que nos interesa. Como es consistente entonces EXISTEN escalares (en este caso infinitos) que hacen posible la combinación lineal.

\begin{ejemplo}
En el espacio vectorial \mdosxdos, determinar si 
\matrdxd{2&1}{-1&0}
 es combinación lineal de los vectores 
\matrdxd{1&0}{-1&0}, \matrdxd{-1&2}{0&1}, \matrdxd{0&1}{2&-1}.
\end{ejemplo}


\[\matrdxd{2&1}{-1&0}=
\alpha_1\matrdxd{1&0}{-1&0}
+{\alpha_2}{\matrdxd{-1&2}{0&1}}
+{\alpha_3}{\matrdxd{0&1}{2&-1}}
\]

Dos matrices son iguales si y solo si cada uno de sus coeficientes
lo son, por lo que resolver la ecuación anterior es equivalente a resolver el sistema de ecuaciones siguiente:
\begin{align*}
\alpha_1+\alpha_2=&2\\
2\alpha_2+\alpha_3=&1\\
-\alpha_1+2\alpha_3=&-1\\
\alpha_2-\alpha_3=&0\\
\end{align*}
Por el método de Gauss obtenemos un sistema equivalente por filas
\[
\reducir{rrr|r}{
1 &-1 &0 &2\\
0&2&1&1\\
-1&0&2&-1\\
0&1&-1&0
}
\underrightarrow{f_1 + f_3}
\reducir{rrr|r}{
1 &-1 &0 &2\\
0&2&1&1\\
0&-1&2&1\\
0&1&-1&0
}
\underrightarrow{\begin{array}{c}
    f_2 + 2f_3\\
    f_2 - 2f_4 
\end{array}}
\reducir{rrr|r}{
1 &-1 &0 &2\\
0&2&1&1\\
0&0&5&3\\
0&0&3&1
}\]

~\\
En las dos últimas filas se obtiene una inconsistencia matem\'atica, ya que en la tercera tenemos que 
$\alpha_3=\frac{3}{5}$ 
pero en la cuarta fila 
$\alpha_3=\frac{1}{3}$
lo cual es una contradicción. Por lo tanto el sistema es inconsistente, no existen escalares que satisfagan dicho sistema. Por tanto \matrdxd{2&1}{-1&0} NO
 es combinación lineal de los vectores \matrdxd{1&0}{-1&0}, \matrdxd{-1&2}{0&1}, \matrdxd{0&1}{2&-1}.

~\\
\begin{ejercicio}
Determinar de ser posible los valores de $k$ para los cuales el vector 
\vectrtres{1}{1}{k} es una combinación lineal de los vectores
 \vectrtres{1}{2}{k}, \vectrtres{k}{-k}{1}.
\end{ejercicio}
~\\
\underline{solución}:
\\
analicémoslo de la misma manera y planteemos la combinación lineal
\[
\vectrtres{1}{1}{k}=
 \alpha_1\vectrtres{1}{2}{k}+\alpha_2\vectrtres{k}{-k}{1}
\]
Para que la combinación lineal exista el sistema debe ser consistente. Realizando operaciones elementales tenemos:

\[
\reducir{rr|r}{
1 &k&1\\
2&-k&1\\
k&1&k
}
\underrightarrow{\begin{array}{c}
    -2f_1 + f_2\\
    -kf_1 + f_3 
\end{array}}
\reducir{rr|r}{
1 &k&1\\
0&-3k&-1\\
0&1-k^2&0
}
\underrightarrow{(1-k^2)f_2 - (-3k)f_3 }
\reducir{rr|l}{
1 &k&1\\
0&-3k&-1\\
0&0&k^2-1
}
\]

Atención a las dos últimas filas, para que el sistema sea consistente el valor de $k^2-1$ debe ser igual a cero (si es diferente de cero: $k^2-1=m, m\neq 0$ el sistema es inconsistente ya que quedar\'ia una ecuación $0\alpha_1+0\alpha_2=m$\, lo cual es un absurdo), es decir es consistente cuando:
\[k^2-1=0  \longrightarrow  k=-1 \vee k=1\]

Debemos tener en cuenta también que $k\neq 0$
ya que con eso el sistema es inconsistente en la
segunda fila, pero esto no afecta los valores obtenidos ya que ninguno de ellos es cero.


Por lo tanto para que \vectrtres {1}{1}{k} sea combinación lineal de \vectrtres{1}{2}{k}, \vectrtres{k}{-k}{1}; $k=-1 \vee k=1$.

\newpage
\section{Cápsula Lineal}
\begin{dfn}[Cápsula Lineal]
Sean \conjvect{v}{k} k vectores de un espacio vectorial $(V,\dobleK,\oplus,\odot)$, se denomina Cápsula Lineal de \conjvect{v}{k} al conjunto de todas las combinaciones lineales posibles de los vectores \conjvect{v}{k}
\[
\mathcal{L} \left(\kvect{v}{k}\right)=
\lbrace v \in V|v= \av{}{\alpha_1 \odot v_1} \oplus
\av{}{\alpha_2 \odot v_2} \oplus 
\hdots \oplus
\av{}{\alpha_k \odot v_k}
\rbrace
\]

\end{dfn}

\begin{theorem}
Sean \conjvect{v}{k} vectores de un espacio vectorial $(V,\dobleK,\oplus,\odot)$, entonces la cápsula lineal de \conjvect{v}{k} es un subespacio vectorial de $V$
\end{theorem}
\begin{proof}~\\
\begin{enumerate}
\item Puesto que $0_V= \av{}{0 \odot v_1} \oplus
\av{}{0 \odot v_2} \oplus 
\hdots \oplus
\av{}{0 \odot v_k}$ se tiene que $0_V$ es un elemento de la cápsula lineal de \conjvect{v}{k}
\item Sean $w_1, w_2$ dos elementos de la cápsula lineal de \conjvect{v}{k}, entonces existen escalares $\alpha_i, \beta_i \in \dobleK$ tales que \begin{align*}
w_1=\av{}{\alpha_1 \odot v_1} \oplus
\av{}{\alpha_2 \odot v_2} \oplus 
\hdots \oplus
\av{}{\alpha_k \odot v_k}\\
w_2=\av{}{\beta_1 \odot v_1} \oplus
\av{}{\beta_2 \odot v_2} \oplus 
\hdots \oplus
\av{}{\beta_k \odot v_k}\\
\end{align*}
Ahora como
\begin{align*}
w_1\oplus w_2=\av{}{(\alpha_1 + \beta_1) \odot v_1} \oplus
\av{}{(\alpha_2 + \beta_2) \odot v_2} \oplus 
\hdots \oplus
\av{}{(\alpha_k +\beta_k) \odot v_k}
\end{align*}
Concluimos que $w_1\oplus w_2$ es un elemento de la cápsula lineal de \conjvect{v}{k}.

\item Sean $c \in \dobleK$ y $w$ un elemento de la cápsula lineal de \conjvect{v}{k}. Entonces 
\begin{align*}
c\odot w&=c\odot (\av{}{\alpha_1 \odot v_1} \oplus
\av{}{\alpha_2 \odot v_2} \oplus 
\hdots \oplus
\av{}{\alpha_k \odot v_k})\\
c\odot w&=\av{}{(c\alpha_1) \odot v_1} \oplus
\av{}{(c\alpha_2) \odot v_2} \oplus 
\hdots \oplus
\av{}{(c\alpha_k) \odot v_k}\\
\end{align*}
Y así, $c\odot w$ es un elemento de la cápsula lineal de \conjvect{v}{k}.
\end{enumerate}
Por lo tanto la cápsula lineal es un subespacio de V.
\end{proof}


~\\
Por esto a la cápsula lineal se le denomina espacio generado.

\section{Conjunto Generador y Espacio Generado}(?`Qué?\ ?`no es lo mismo?)


\begin{dfn}
Sea $S=$\conjvect{v}{k} un subconjunto de un espacio vectorial $V$, se denomina Espacio Generado por S al conjunto de todas las combinaciones posibles de los vectores de S. Dado esto, el conjunto S se denomina Conjunto Generador. Se denota:
\[gen(S)=
\lbrace v \in V|v=\av{}{\alpha_1 \odot v_1} \oplus
\av{}{\alpha_2 \odot v_2} \oplus 
\hdots \oplus
\av{}{\alpha_k \odot v_k}
\rbrace
\]
al espacio generado por S.
\end{dfn}

\begin{ejemplo}
Halle el espacio generado por el subconjuto S de \pdos; $S=\lbrace
x+1, x^2-1, 2x^2+x-1\rbrace$. Luego, determine si 
\begin{enumerate}
\item[(a)]$x^2-2x+5 \in gen(S)$
\item[(b)]$3x^2+4x+1 \in gen(S)$
\end{enumerate}
\end{ejemplo}

~\\
\underline{solución}
~\\
Por definición $gen(S)=
\lbrace v \in V|v=
\beta_1 v_2+
\beta_2 v_2+
\ldots+
\beta_k v_k
\rbrace
$

~\\
Elegimos ahora un vector típico de \pdos, que es el espacio referencial
\[p(x)=a+b x+c x^2\]
$$\forall p(x)\in gen(S):\  
a+b x+c x^2=
\av{\beta_1}{x+1}+
\av{\beta_2}{x^2-1}+
\av{\beta_3}{2x^2+x-1}
$$
Igualando coeficientes llegamos al sistema de ecuaciones
\begin{eqnarray*}
\beta_1-\beta_2-\beta_3=a\\
\beta_1+\beta_3=b\\
\beta_2+2\beta_3=c
\end{eqnarray*}

\[
\reducir{rrr|c}{
1&-1&-1&a\\
1&0&1&b\\
0&1&2&c
}
\underrightarrow{-f_1 + f_2}
\reducir{rrr|c}{
1&-1&-1&a\\
0&1&2&b-a\\
0&1&2&c
}
\underrightarrow{f_2 - f_3}
\reducir{rrr|c}{
1&-1&-1&a\\
0&1&2&b-a\\
0&0&0&b-a-c
}
\]

\underline{Observación}: Atención aquí, las incógnitas de este sistema por naturaleza serían $\beta_1, \beta_2, \beta_3$
; sin embargo cuando vamos a determinar el espacio generado no nos interesa quienes sean los escalares, sólo nos interesa que existan, lo que si nos interesa son las condiciones que recaen sobre $a,b,c$
para que el sistema sea consistente (en otras palabras, para que los escalares existan).


En la última fila tenemos una fila de llena de ceros pero al final tenemos el factor $b-a-c$
analicémosla, si $b-a-c\neq 0$
,entonces el sistema no tendría solución (?`Por qué?), lo cual va
en contra de la definición de combinación lineal, que es esencial en la definición de espacio
generado, por lo que esto es lo que debe ser evitado. Para que el sistema sea consistente debe cumplirse que $b-a-c=0$
, y ésta será la condición de todos los vectores que estén en el
espacio generado por S.
\[gen(S)=
\lbrace
a+b x+c x^2 \in \pdos|
b-a-c=0
\rbrace
\]
\begin{enumerate}

\item[(a)] El vector $x^2-2x+5$ no pertenece a $gen(S)$\ ya que $(-2)-1-5=-8\neq 0$
\item[(b)] El vector $3x^2+4x+1 \in gen(S)$, ya que $4-3-1=0$
\end{enumerate}

\begin{ejemplo}
Determine el espacio generado por los vectores $\llav{ \vectrtres{1}{2}{1}, \vectrtres{3}{0}{2}, \vectrtres{3}{2}{0}}$
\end{ejemplo}

~\\
\underline{solución}:
~\\

Sea $v=\vectrtres{a}{b}{c}; a, b, c, \in \dobler$

\[
\forall v\in gen(S):\ 
\vectrtres{a}{b}{c}=
\beta_1\vectrtres{1}{2}{1}+
\beta_2\vectrtres{3}{0}{2}+
\beta_3\vectrtres{3}{2}{0}
\]

\[
\reducir{rrr|r}{
1&3&3&a\\
2&0&2&b\\
1&2&0&c
}
\underrightarrow{\begin{array}{cc}
    -2f_1 + f_2\\
    f_1 - f_3
\end{array}}
\reducir{rrr|l}{
1&3&3&a\\
0&-6&-4&-2a+b\\
0&1&3&a-c
}
\underrightarrow{f_2 + 6f_3}
\reducir{rrr|l}{
1&3&3&a\\
0&-6&-4&-2a+b\\
0&0&14&4a+b-6c
}
\]
Este sistema siempre ser\'a consistente para cualesquiera que sean los valores de $a, b, c$ por lo
tanto se dice que no hay condiciones sobre ellos, en otras palabras $a, b, c \in \dobler$; puede ser
cualquier real.
\[
gen(S)=
\left\{
\left.
\vectrtres{a}{b}{c} \in \rtres \right|
a, b, c \in \dobler
\right\}=\rtres
\]
Se dice que S genera a \rtres.

\begin{ejercicio}
Determine el valor de $k$ para que el vector $C=\matrdxd{2k&1}{-3&k}$ pertenezca al espacio generado por $S=\left\{\matrdxd{-1&2}{1&2}, \matrdxd{3&-1}{-2&0}, \matrdxd{-2&0}{3&1}\right\}$.
\end{ejercicio}
%%\sol
Hallemos el espacio generado por dichos vectores. Sea $A=\matrdxd{a&b}{c&d} \in gen(S)$.

\[\forall A\in gen(S):\ 
\matrdxd{a&b}{c&d}=
\beta_1\matrdxd{-1&2}{1&2}+ \beta_2\matrdxd{3&-1}{-2&0}+ \beta_3\matrdxd{-2&0}{3&1}
\]
\[
\left(
\begin{array}{rrr|r}
-1&3&-2&a\\
2&-1&0&b\\
1&-2&3&c\\
2&0&1&d
\end{array}
\right)
\sim
\ldots
\sim
\left(
\begin{array}{rrr|l}
-1&3&-2&a\\
0&5&-4&2a+b\\
0&0&-9&-3a+b-5c\\
0&0&0&-5a-5b-5c+5d
\end{array}
\right)
\]
Para que el sistema sea consistente debe cumplirse que $-5a-5b-5c+5d=0$. Por lo que el espacio generado por S es:
\[gen(S)=\llaves{
\matrdxd{a&b}{c&d}
}{d=a+b+c}
\]
Para que $C$\ pertenezca a $gen(S)$, debe cumplirse que
\[k=2k+1-3\ \rightarrow\ k=2\]



