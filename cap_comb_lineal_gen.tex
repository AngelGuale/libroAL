%espacio generado combinación lineal
\chapter{Combinaciones Lineales}

\section{Combinación Lineal}
\begin{dfn}[Combinación Lineal]
Sea $v\in V$, sean $\llav{v_1, v_2, v_3, \ldots, v_n}$ vectores de un espacio vectorial $V$ sobre el campo \doblek , se dice que $v$ es una combinación lineal de los vectores $\llav{v_1, v_2, v_3, \ldots, v_n}$ si y solo si existen los escalares $\alpha_1, \alpha_2, \alpha_3, \ldots, \alpha_n \in$ \doblek tal que
\[
v=\left(\alpha_1\odot v_1\right)\oplus
\left(\alpha_2\odot v_2\right)\oplus
\left(\alpha_3\odot v_3\right)\oplus
\ldots
\left(\alpha_n\odot v_n\right)
\]
\end{dfn}
Nótese que la palabra clave de la definici\'on anterior es EXISTEN, para ilustrar esto considere los
dos siguientes ejemplos:
\begin{ejemplo}
Considere el espacio vectorial \rdos; ?` es el vector \vectrdos{6}{-1} combinaci\'on lineal de los vectores \vectrdos{3}{2}, \vectrdos{1}{-1}?
\end{ejemplo}

Para resolver este problema debemos verificar si EXISTEN escalares $\alpha$, $\beta$ tales que hagan posible la combinaci\'on lineal:
\[\mbox{\vectrdos{6}{-1}} =\alpha\mbox{\vectrdos{3}{2}}+\beta\mbox{\vectrdos{1}{-1}}\]

Esto nos llevar\'a a un sistema de ecuaciones
\[
\left\{
\begin{array}{l}
3\alpha+\beta=\ 6\\
2\alpha-\beta=-1
\end{array}
\right.
\]

Si el sistema es consistente, entonces los escalares si existen y por lo tanto, dicho vector es combinación lineal de los otros dos, pero; si el sistema es inconsistente, los escalares no existen
por lo que el vector no sería combinación lineal de los otros dos.

~\\
Al resolver este sistema llegamos a la soluci\'on:
$\alpha=1$ y $\beta=3$.

~\\
Los escalares $\alpha$ y $\beta$ existen, el sistema es consistente con soluci\'on \'unica, luego el vector \vectrdos{6}{-1} es combinaci\'on lineal de los vectores \vectrdos{3}{2}, \vectrdos{1}{-1}. Ver figura ~\ref{fig:figura1}

\imagen{dosvect1}{Combinaci\'on lineal de vectores en \rdos. Método del paralelogramo}{figura1}{0.5}

\begin{ejemplo}
?`El vector \vectrdos{2}{2} es combinaci\'on lineal de los vectores 
\vectrdos{-1}{2}, \vectrdos{2}{-4}?
\end{ejemplo}


Para que sea combinación lineal deben existir los escalares
$\alpha$ y $\beta$ tales que:

\[\mbox{\vectrdos{2}{2}}=\alpha\mbox{\vectrdos{-1}{2}}+
\beta\mbox{\vectrdos{2}{-4}}\]
~\\
Resolviendo el sistema por el metodo de Gauss-Jordan:

\[
\left(
\begin{array}{rr|r}
-1 & 2 & 2\\
2&-4&2
\end{array}
\right)
\sim
\left(
\begin{array}{rr|r}
-1 & 2 & 2\\
0&0&6
\end{array}
\right)
\]
~\\
La \'ultima fila indica un absurdo matem\'atico, lo cual implica que el sistema es inconsistente, no
tiene soluci\'on. Como los escalares no existen, el vector \vectrdos{2}{2} NO es combinaci\'on lineal de los vectores 
\vectrdos{-1}{2}, \vectrdos{2}{-4}

\newpage
%\subsection{?`Espacio de colores?}
%Uno de los m\'etodos de explicar la combinación lineal con un ejemplo inteligible son los colores:
%
%~\\
%Suponga que usted tiene un conjunto de bolitas de plastilina con los colores Azul, Rojo, Amarillo.
%Tomamos una “bolita” amarilla y otra “bolita” roja y las combinamos (esperamos que el lector conozca qué color se formar\'a), como era de suponerse, tendremos ahora una bolita de color Naranja; es decir, el color Naranja es una Combinación Lineal de los colores Amarillo y Rojo.
%
%\imagen{amarillorojo}{Combinaci\'on lineal}{figura2}{0.5}
%
%Asimismo si tomamos una bolita de color Azul y una bolita de color Amarillo, obtendremos una
%de color Verde, luego el color Verde es Combinación Lineal de los colores Azul y Amarillo.
%
%\imagen{azulamarillo}{Combinaci\'on lineal}{figura3}{0.5}
%
%Ahora suponga que no toma una bolita de Azul, sino TRES bolitas de Azul y una de Amarillo; y las
%combinamos, ¡por supuesto! Tendremos otra clase de color Verde, pero un poco más oscuro, es
%decir la cantidad de bolitas son los ESCALARES de la combinación lineal y si los escalares
%cambian tendremos otro resultado.
%
%
%\imagen{verdeoscuro}{Combinaci\'on lineal}{figura4}{0.4}
%
%?`Ahora est\'a más claro el concepto de combinación lineal?

\subsection{M\'as ejemplos de combinaci\'on lineal}

De ahora en adelante trabajaremos sobre espacios vectoriales sobre campo Real y con operaciones usuales, por lo que obviaremos $\oplus$ y $\odot$ por los usuales $+$ y $\cdot$

\begin{ejemplo}
Considere el espacio vectorial \pdos\ sobre el campo real, determinar si el vector $x^2+x-1$ es combinación lineal de los vectores $x+2, x^2-3, x^2+2x+1$.
\end{ejemplo}
~\\
Soluci\'on

~\\
Veamos si existen los escalares $\alpha_1, \alpha_2, \alpha_3$ tales que
\[x^2+x-1=\alpha_1\left(x+2\right)+
\alpha_2\left(x^2-3\right)+
\alpha_3\left(x^2+2x+1\right)
\]

~\\
Dos polinomios son iguales si y solos si cada uno de sus coeficientes lo son, por tanto:

~\\
\[x^2+x-1=\left(\alpha_2+\alpha_3\right)x^2+
\left(\alpha_1+2\alpha_3\right)x+
\left(2\alpha_2-3\alpha_2+\alpha_3\right)
\]

Al igualar coeficientes, se origina el siguiente sistema de ecuaciones:

\begin{eqnarray*}
\alpha_2+\alpha_3=1\\
\alpha_1+2\alpha_3=1\\
2\alpha_1-3\alpha_2+\alpha_3=-1\\
\end{eqnarray*}

Resolvi\'endolo por Gauss:

\[
\left(
\begin{array}{rrr|r}
0 &1 &1 &1\\
1&0&2&1\\
2 &-3&1&-1
\end{array}
\right)
\sim
\left(
\begin{array}{rrr|r}
1 &0 &2 &1\\
0&1&1&1\\
0 &-3&-3&-3
\end{array}
\right)
\sim
\left(
\begin{array}{rrr|r}
1 &0 &2 &1\\
0&1&1&1\\
0 &0&0&0
\end{array}
\right)\]

De la \'ultima fila del sistema inferimos que el sistema tiene infinitas soluciones, es decir; es consistente, que es lo que nos interesa. Como es consistente entonces EXISTEN escalares (en este caso infinitos) que hacen posible la combinaci\'on lineal.

\begin{ejemplo}
En el espacio vectorial \mdosxdos, determinar si 
\matrdxd{2&1}{-1&0}
 es combinaci\'on lineal de los vectores 
\matrdxd{1&0}{-1&0}, \matrdxd{-1&2}{0&1}, \matrdxd{0&1}{2&-1}.
\end{ejemplo}


\[\matrdxd{2&1}{-1&0}=
\alpha_1\matrdxd{1&0}{-1&0}
+{\alpha_2}{\matrdxd{-1&2}{0&1}}
+{\alpha_3}{\matrdxd{0&1}{2&-1}}
\]

Dos matrices son iguales si y solo si cada uno de sus coeficientes
lo son, por lo que resolver la ecuación anterior es equivalente a resolver el sistema de ecuaciones siguiente:
\begin{align*}
\alpha_1+\alpha_2=&2\\
2\alpha_2+\alpha_3=&1\\
-\alpha_1+2\alpha_3=&-1\\
\alpha_2-\alpha_3=&0\\
\end{align*}
Por el método de Gauss obtenemos un sistema equivalente por filas
\[
\left(
\begin{array}{rrr|r}
1 &-1 &0 &2\\
0&2&1&1\\
-1&0&2&-1\\
0&1&-1&0
\end{array}
\right)
\sim
\left(
\begin{array}{rrr|r}
1 &-1 &0 &2\\
0&2&1&1\\
0&-1&2&1\\
0&1&-1&0
\end{array}
\right)
\sim
\left(
\begin{array}{rrr|r}
1 &-1 &0 &2\\
0&2&1&1\\
0&0&5&3\\
0&0&3&1
\end{array}
\right)\]

~\\
En las dos \'ultimas filas se obtiene una inconsistencia matem\'atica, ya que en la tercera tenemos que 
$\alpha_3=\frac{3}{5}$ 
pero en la cuarta fila 
$\alpha_3=\frac{1}{3}$
lo cual es una contradicci\'on. Por lo tanto el sistema es inconsistente, no existen escalares que satisfagan dicho sistema. Por tanto \matrdxd{2&1}{-1&0} NO
 es combinaci\'on lineal de los vectores \matrdxd{1&0}{-1&0}, \matrdxd{-1&2}{0&1}, \matrdxd{0&1}{2&-1}.

~\\
\begin{ejercicio}
Determinar de ser posible los valores de $k$ para los cuales el vector 
\vectrtres{1}{1}{k} es una combinaci\'on lineal de los vectores
 \vectrtres{1}{2}{k}, \vectrtres{k}{-k}{1}.
\end{ejercicio}
~\\
\underline{Soluci\'on}:
\\
Analic\'emoslo de la misma manera y planteemos la combinaci\'on lineal
\[
\vectrtres{1}{1}{k}=
 \alpha_1\vectrtres{1}{2}{k}+\alpha_2\vectrtres{k}{-k}{1}
\]
Para que la combinaci\'on lineal exista el sistema debe ser consistente. Realizando operaciones elementales tenemos:

\[
\left(
\begin{array}{rr|r}
1 &k&1\\
2&-k&1\\
k&1&k
\end{array}
\right)
\sim
\left(
\begin{array}{rr|r}
1 &k&1\\
0&-3k&-1\\
0&1-k^2&0
\end{array}
\right)
\sim
\left(
\begin{array}{rr|r}
1 &k&1\\
0&-3k&-1\\
0&0&k^2-1
\end{array}
\right)
\]

Atenci\'on a las dos \'ultimas filas, para que el sistema sea consistente el valor de $k^2-1$ debe ser igual a cero (si es diferente de cero: $k^2-1=m, m\neq 0$ el sistema es inconsistente ya que quedar\'ia una ecuaci\'on $0\alpha_1+0\alpha_2=m$\, lo cual es un absurdo), es decir es consistente cuando:
\[k^2-1=0  \longrightarrow  k=-1 \vee k=1\]

Debemos tener en cuenta también que $k\neq 0$
ya que con eso el sistema es inconsistente en la
segunda fila, pero esto no afecta los valores obtenidos ya que ninguno de ellos es cero.


Por lo tanto para que \vectrtres {1}{1}{k} sea combinaci\'on lineal de \vectrtres{1}{2}{k}, \vectrtres{k}{-k}{1}; $k=-1 \vee k=1$.

\newpage
\section{Cápsula Lineal}
\begin{dfn}[C\'apsula Lineal]
Sean \conjvect{v}{k} k vectores de un espacio vectorial $V$, se denomina C\'apsula Lineal de \conjvect{v}{k} al conjunto de todas las combinaciones lineales posibles de los vectores \conjvect{v}{k}
\[
\mathcal{L} \left(\kvect{v}{k}\right)=
\lbrace v|v=\av{\alpha_1}{v_1}+
\av{\alpha_2}{v_2}+
\ldots+
\av{\alpha_k}{v_k}
\rbrace
\]

\end{dfn}

\begin{theorem}
Sean \conjvect{v}{k} vectores de un espacio vectorial $V$, entonces la c\'apsula lineal de \conjvect{v}{k} es un subespacio vectorial de $V$
\end{theorem}
\begin{proof}~\\
\begin{enumerate}
\item Puesto que $0_V=0 v_1+0 v_2+..0 v_n$ se tiene que $0_V$ es un elemento de la cápsula lineal de $\llav{v_1, v_2, .., v_n}$
\item Sean $w_1, w_2$ dos elementos de la cápsula lineal de $\llav{v_1, v_2, .., v_n}$, entonces exiten escalares $\alpha_i, \beta_i \in \dobler$ tales que \begin{align*}
w_1=\alpha_1 v_1+\alpha_2 v_2+..\alpha_n v_n\\
w_2=\beta_1 v_1+\beta_2 v_2+..\beta_n v_n\\
\end{align*}
Ahora como
\begin{align*}
w_1+w_2=(\alpha_1+\beta_1) v_1+(\alpha_2+\beta_2) v_2+..(\alpha_n+\beta_n) v_n\\
\end{align*}
Concluimos que $w_1+w_2$ es un elemento de la cápsula lineal de $\llav{v_1, v_2, .., v_n}$.
\item Sean $c \in \dobler$ y w un elemento de la cápsula lineal de $\llav{v_1, v_2, .., v_n}$. Entonces 
\begin{align*}
cw&=c(\alpha_1 v_1+\alpha_2 v_2+..\alpha_n v_n)\\
cw&=(c\alpha_1) v_1+(c\alpha_2) v_2+..(c\alpha_n) v_n\\
\end{align*}
Y así, $cw $ es un elemento de la cápsula lineal de \llav{v_1, v_2, .., v_n}.
\end{enumerate}
Por lo tanto la cápsula lineal es un subespacio de V.
\end{proof}


~\\
Por esto a la c\'apsula lineal se le denomina espacio generado.

\section{Conjunto Generador y Espacio Generado}(?`Qu\'e?\ ?` no es lo mismo?)


\begin{dfn}
Sea $S=$\conjvect{v}{k} un subconjunto de un espacio vectorial $V$, de denomina Espacio Generado por S al conjunto de todas las combinaciones posibles de los vectores de S. Dado esto, el conjunto S se demonida Conjunto Generador. Se denota:
\[gen(S)=
\lbrace v \in V|v=\alpha_1 v_1+
\alpha_2 v_2+
\ldots+
\alpha_k v_k
\rbrace
\]
al espacio generado por S.
\end{dfn}

\begin{ejemplo}
Halle el espacio generado por el subconjuto S de \pdos; $S=\lbrace
x+1, x^2-1, 2x^2+x-1\rbrace$. Luego, determine si 
\begin{enumerate}
\item[(a)]$x^2-2x+5 \in gen(S)$
\item[(b)]$3x^2+4x+1 \in gen(S)$
\end{enumerate}
\end{ejemplo}

~\\
\underline{Soluci\'on}
~\\
Por definici\'on $gen(S)=
\lbrace v\in V|v=
\beta_1 v_2+
\beta_2 v_2+
\ldots+
\beta_k v_k
\rbrace
$

~\\
Elegimos ahora un vector t\'ipico de \pdos, que es el espacio referencial
\[p(x)=a_0+a_1 x+a_2 x^2\]
$$\forall p(x)\in gen(S):\  
a_0+a_1 x+a_2 x^2=
\av{\beta_1}{x+1}+
\av{\beta_2}{x^2-1}+
\av{\beta_3}{2x^2+x-1}
$$
Igualando coeficientes llegamos al sistema de ecuaciones
\begin{eqnarray*}
\beta_1+\beta_2-\beta_3=a_0\\
\beta_1+\beta_3=a_1\\
\beta_2+2\beta_3=a_2
\end{eqnarray*}

\[
\left(
\begin{array}{rrr|r}
1&-1&-1&a_0\\
1&0&1&a_1\\
0&1&2&a_2
\end{array}
\right)
\sim
\left(
\begin{array}{rrr|r}
1&-1&-1&a_0\\
0&1&2&a_1-a_0\\
0&1&2&a_2
\end{array}
\right)
\sim
\left(
\begin{array}{rrr|r}
1&-1&-1&a_0\\
0&1&2&a_1-a_0\\
0&0&0&a_1-a_0-a_2
\end{array}
\right)
\]

\underline{Observaci\'on}: Atenci\'on aqu\'i, las inc\'ognitas de este sistema por naturaleza ser\'ian $\beta_1, \beta_2, \beta_3$
; sin embargo cuando vamos a determinar el espacio generado no nos interesa quienes sean los escalares, s\'olo nos interesa que existan, lo que si nos interesa son las condiciones que recaen sobre $a_0, a_1, a_2 $
para que el sistema sea consistente (en otras palabras, para que los escalares existan).


En la \'ultima fila tenemos una fila de llena de ceros pero al final tenemos el factor $a_1-a_0-a_2$
analic\'emosla, si $a_1-a_0-a_2\neq 0$
,entonces el sistema no tendría soluci\'on (?`Por qu\'e?), lo cual va
en contra de la definici\'on de combinaci\'on lineal, que es esencial en la definición de espacio
generado, por lo que esto es lo que debe ser evitado. Para que el sistema sea consistente debe cumplirse que $a_1-a_0-a_2=0$
, y ésta será la condición de todos los vectores que est\'en en el
espacio generado por S.
\[gen(S)=
\lbrace
a_0+a_1 x+a_2 x^2|
a_1-a_0-a_2=0
\rbrace
\]
\begin{enumerate}

\item[(a)] El vector $x^2-2x+5$ no pertenece a $gen(S)$\ ya que $(-2)-1-5=-8\neq 0$
\item[(b)] El vector $3x^2+4x+1 \in gen(S)$, ya que $4-3-1=0$
\end{enumerate}

\begin{ejemplo}
Determine el espacio generado por los vectores $\left\{ \vectrtres{1}{2}{1}, \vectrtres{3}{0}{2}, \vectrtres{3}{2}{0}\right\}$.
\end{ejemplo}

~\\
\underline{Soluci\'on}:
~\\

Sea $v=\vectrtres{a}{b}{c}; a, b, c, \in \dobler$

\[
\forall v\in gen(S):\ 
\vectrtres{a}{b}{c}=
\beta_1\vectrtres{1}{2}{1}+
\beta_2\vectrtres{4}{0}{2}+
\beta_3\vectrtres{3}{2}{0}
\]

\[
\left(
\begin{array}{rrr|r}
1&3&3&a\\
2&0&2&b\\
1&2&0&c
\end{array}
\right)
\sim
\left(
\begin{array}{rrr|r}
1&3&3&a\\
0&-6&-4&-2a+b\\
0&1&3&a-c
\end{array}
\right)
\sim
\left(
\begin{array}{rrr|r}
1&3&3&a\\
0&-6&-4&-2a+b\\
0&0&14&4a+b-6c
\end{array}
\right)
\]
Este sistema siempre ser\'a consistente para cualesquiera que sean los valores de $a, b, c$ por lo
tanto se dice que no hay condiciones sobre ellos, en otras palabras $a, b, c \in \dobler$; puede ser
cualquier real.
\[
gen(S)=
\left\{
\left.
\vectrtres{a}{b}{c}\right|
a, b, c \in \dobler
\right\}=\rtres
\]
Se dice que S genera a \rtres.

\begin{ejercicio}
Determine el valor de $k$ para que el vector $C=\matrdxd{2k&1}{-3&k}$ pertenezca al espacio generado por $S=\left\{\matrdxd{-1&2}{1&2}, \matrdxd{3&-1}{-2&0}, \matrdxd{-2&0}{3&1}\right\}$.
\end{ejercicio}
%%\sol
Hallemos el espacio generado por dichos vectores. Sea $A=\matrdxd{a&b}{c&d} \in gen(S)$.

\[\forall A\in gen(S):\ 
\matrdxd{a&b}{c&d}=
\beta_1\matrdxd{-1&2}{1&2}+ \beta_2\matrdxd{3&-1}{-2&0}+ \beta_3\matrdxd{-2&0}{3&1}
\]
\[
\left(
\begin{array}{rrr|r}
-1&3&-2&a\\
2&-1&0&b\\
1&-2&3&c\\
2&0&1&d
\end{array}
\right)
\sim
\ldots
\sim
\left(
\begin{array}{rrr|r}
-1&3&-2&a\\
0&5&-4&2a+b\\
0&0&-9&-3a+b-5c\\
0&0&0&-5a-5b-5c+5d
\end{array}
\right)
\]
Para que el sistema sea consistente debe cumplirse que $-5a-5b-5c+5d=0$. Por lo que el espacio generado por S es:
\[gen(S)=\llaves{
\matrdxd{a&b}{c&d}
}{d=a+b+c}
\]
Para que $C$\ pertenezca a $gen(S)$, debe cumplirse que
\[k=2k+1-3\ \rightarrow\ k=2\]


\section{Problemas}

\begin{enumerate}

\begin{prob}[(3er Examen ESPOL 2012)]
Sea $V=C[0,1]$ y $f, g \in V$. Demuestre que si el conjunto \llav{f, g} es linealmente dependiente entonces
\[W=
\begin{vmatrix}
f(x)&g(x)\\
f'(x)&g'(x)\\
\end{vmatrix}=0
\]
\end{prob}
\sol
Si \llav{f, g} es un conjunto linealmente dependiente, entonces existe un escalar $k\in \dobler$ tal que $g(x)=kf(x)$:

\[\Rightarrow\
W=
\begin{vmatrix}
f(x)&g(x)\\
f'(x)&g'(x)\\
\end{vmatrix}
=
\begin{vmatrix}
f(x)&kf(x)\\
f'(x)&kf'(x)\\
\end{vmatrix}
=
kf(x)f'(x)-kf(x)f'(x)=0
\ \blacksquare\]

\begin{prob}[(1er Examen ESPOL 2012)]
Demuestre:
\\Sea $S=\conjvect{v}{n}$ un subconjunto linealmente independiente de vectores del espacio vectorial $V$ y sea $x$ un vector de $V$ que no puede ser expresado como una combinaci\'on lineal de los vectores de S, entonces $\llav{\kvect{v}{n}, x}$
tambi\'en es linealmente independiente.
\end{prob}

\sol
Tenemos que demostrar que $\llav{\kvect{v}{n}, x}$ es l.i.
Consideremos que

\begin{equation}\label{eq1}
\alpha_1 v_1+
\alpha_2 v_2+
\ldots+
\alpha_n v_n+
\beta x =0_v
\end{equation}
Para algunos escalares $ \alpha_i, \beta \in \dobler; i=1...n$
Al operar esta ecuaci\'on tenemos lo siguiente
\begin{equation}\label{eq2}
\beta x=(-\alpha_1) v_1+
(-\alpha_2) v_2+
\ldots+
(-\alpha_n) v_n
\end{equation}
Existen dos casos, que $\beta\neq 0$ o que $\beta=0$
~\\
\underline{Si $\beta\neq 0$}:\ 
Si en \ref{eq2}, multiplicamos por $\frac{1}{\beta}$ (estamos seguro de que existe, ya que $\beta$ es diferente de cero) obtendremos lo siguiente:
\begin{equation*}
x=(-\frac{\alpha_1}{\beta}) v_1+
(-\frac{\alpha_2}{\beta}) v_2+
\ldots+
(-\frac{\alpha_n}{\beta}) v_n
\end{equation*} 
Lo anterior es imposible ya que contradice la hip\'otesis de que x no puede escribirse como combinaci\'on lineal de los vectores de S, as\'i que se concluye que $\beta=0$.
Ya conociendo que $\beta=0$, en \ref{eq1} reemplazamos y tenemos
\begin{equation}
\alpha_1 v_1+
\alpha_2 v_2+
\ldots+
\alpha_n v_n=0_v
\end{equation}
Pero como S es un conjunto l.i. implica que $\alpha_i=0; i=1,...n$
Con esto se concluye que el conjunto $\llav{\kvect{v}{n}, x}$ es l.i.$\blacksquare$

\begin{prob}[(1er Examen Espol 2009)]
Sean $u=\vectrdos{x_1}{y_1}, v=\vectrdos{x_2}{y_2}, w=\vectrdos{x_3}{y_3}$ elementos de $ \rdos$ y sea
\[A=
\left(
\begin{array}{rrr}
x_1&y_1&1\\
x_2&y_2&1\\
x_3&y_3&1
\end{array}
\right)
\]
\begin{enumerate}
\item[(a)] Si u, v, w son colineales ?`cu\'al es el rango de A?
\item[(b)] Determine una base para el espacio fila de A.
\end{enumerate}
\end{prob}

\sol
(a) Si u, v, w son colineales, entonces deben ser m\'ultiplos
 $$v=\vectrdos{a x_1}{a y_1}, w=\vectrdos{b x_1}{b y_1}$$
por lo que la matriz A ser\'ia
\[A=
\left(
\begin{array}{rrr}
x_1&y_1&1\\
a x_1&a y_1&1\\
b x_1&b y_1&1
\end{array}
\right)
\]
\begin{eqnarray*}
C_A &=gen\llav{
\vectrtres{x_1}{ax_1}{b x_1}, 
\vectrtres{y_1}{a y_1}{b y_1}, 
\vectrtres{1}{1}{1}
}\\&=
gen\llav{
\vectrtres{1}{a}{b}, 
\vectrtres{1}{a}{b}, 
\vectrtres{1}{1}{1}
}\\&=
gen\llav{
\vectrtres{1}{a}{b}, 
\vectrtres{1}{1}{1}
}
\end{eqnarray*}
De esta manera, es claro que $\rho(A)=1$ si $a=1 \wedge b=1$
en cualquier otro caso $\rho(A)=2$.
~\\
(b) Si $a=1 \wedge b=1$ entonces una base para el espacio fila ser\'ia $$B_{F_A}=\llav{
\vectrtres{x_1}{ y_1}{ 1}
}$$

Si $a=1 \wedge b\neq 1$ una base de $F_A$ ser\'ia
\[B_{F_A}=\llav{
\vectrtres{x_1}{ y_1}{ 1},
\vectrtres{b x_1}{b y_1}{1}
}\]

Si $a\neq 1 \wedge b=1$ una base de $F_A$ ser\'ia
\[B_{F_A}=\llav{
\vectrtres{x_1}{ y_1}{ 1},
\vectrtres{a x_1}{a y_1}{1}
}\]
\newpage
\begin{prob}
Dada la matriz $A=
\left(
\begin{array}{rrr}
1&10&1\\
2&2&2\\
-1&-8&k
\end{array}
\right)
$, determine los valores de $k$ para que la nulidad de $A$ sea cero.
\end{prob}

\sol

Por el criterio del determinante, la nulidad de A es cero si $det(A)\neq 0$
\[det(A)\neq 0\]
\[1(2k+16)-10(2k+2)+1(-16+2)\neq 0\]
\[-18k-18\neq 0\]
\[k\neq -1\].

\newpage
\begin{prob}[(1er Examen ESPOL 2007)]
Sea $A$ la matriz de coeficientes del sistema lineal
\begin{eqnarray*}
2x+y-z=a\\
x-y+2z=b\\
x+2y-3z=c
\end{eqnarray*}
\begin{enumerate}
\item[a)] Determine el espacio fila, n\'ucleo y recorrido de $A$
\item[b)] Si $c=2a+b$, determine si el vector \vectrtres{a}{b}{c} pertenece a $Im(A)$
\end{enumerate}
\end{prob}

\sol

De acuerdo al sistema se tiene que
\[A=
\left(
\begin{array}{rrr}
2&1&-1\\
1&-1&2\\
1&2&-3\\
\end{array}
\right)
\]
Espacio fila:
\[F_A=gen\llav{
\vectrtres{2}{1}{-1}, \vectrtres{1}{-1}{2}, \vectrtres{1}{2}{-3}
}\]
Lo que conduce al sistema

\[A=
\left(
\begin{array}{rrr|r}
2&1&1&x\\
1&-1&2&y\\
-1&2&-3&z\\
\end{array}
\right)
\sim
\left(
\begin{array}{rrr|r}
2&1&1&x\\
0&3&-3&x-2y\\
0&5&-5&x+2z\\
\end{array}
\right)
\sim
\left(
\begin{array}{rrr|r}
2&1&1&x\\
0&3&-3&x-2y\\
0&0&0&2x-10y-6z\\
\end{array}
\right)
\]
Por lo que
\[F_A=\llaves{\vectrtres{x}{y}{z}}{2x-10y-6z=0}\]

N\'ucleo:

El n\'ucleo es la soluci\'on del sistema homog\'eneo $AX=0$:
\[
\left(
\begin{array}{rrr|r}
2&1&-1&0\\
1&-1&2&0\\
1&2&-3&0\\
\end{array}
\right)
\sim
\left(
\begin{array}{rrr|r}
2&1&-1&0\\
0&3&-5&0\\
0&-3&5&0\\
\end{array}
\right)
\sim
\left(
\begin{array}{rrr|r}
2&1&-1&0\\
0&3&-5&0\\
0&0&0&0\\
\end{array}
\right)
\]

Por lo que:
\[Nu(A)=\llaves{\vectrtres{x_1}{x_2}{x_3}}
{\begin{array}{r}
2x_1+x_2-x_3=0\\
3x_2-5x_3=0
\end{array}}
\]

Recorrido o Imagen de A:

El recorrido es igual al espacio columna, el espacio generado por las columnas:

\[Rec(A)=Im(A)=C_A\]

\[
\left(
\begin{array}{rrr|r}
2&1&-1&x\\
1&-1&2&y\\
1&2&-3&z\\
\end{array}
\right)
\sim
\left(
\begin{array}{rrr|r}
2&1&-1&x\\
0&3&-5&x-2y\\
0&-3&5&x-2z\\
\end{array}
\right)
\sim
\left(
\begin{array}{rrr|r}
2&1&-1&x\\
0&3&-5&x-2y\\
0&0&0&2x-2y-2z\\
\end{array}
\right)
\]


Por lo que
\[C_A=\llaves{\vectrtres{x}{y}{z}}{2x-2y-2z=0}\]

b) Si $c=2a+b$ entonces para que el vector \vectrtres{a}{b}{c} pertenezca a $C_A$ debe cumplirse la condici\'on de exige $C_A$
\[
2x-2y-2z=2(a)-2(b)-2(2a+b)=-2a-4b \neq 0
\]

Por tanto, no pertenece a $Im(A)$.

\newpage
\begin{prob}
Sea $V=\mathcal{M}_{3x2}$. Sean $W_1$ el conjunto de las matrices que tienen la primera y la \'ultima fila iguales, $W_2$ el conjunto de las matrices que tienen la primera columna igual a su segunda columna.
~\\Determine:~\\
\begin{enumerate}
\item[a)] La intersecci\'on entre $W_1, W_2$
\item[b)] La suma entre $W_1, W_2$
\item[c)] Una base para los subespacios intersecci\'on y suma obtenidos en (a) y (b).

\end{enumerate}
\end{prob}

\sol

De acuerdo a lo indicado tenemos que

\[W_1=\llaves{
\left(
\begin{array}{rr}
a_{11}&a_{12}\\
a_{21}&a_{22}\\
a_{31}&a_{32}
\end{array}
\right)
}{
\begin{array}{r}
a_{11}=a_{31}\\
 a_{12}=a_{32}
\end{array}}
\]


\[W_2=\llaves{
\left(
\begin{array}{rr}
a_{11}&a_{12}\\
a_{21}&a_{22}\\
a_{31}&a_{32}
\end{array}
\right)
}{
\begin{array}{r}
a_{11}=a_{12}\\
a_{21}=a_{22}\\
a_{31}=a_{32}
\end{array}
}
\]
Para la intersecci\'on tenemos que

\[W_1\cap W_2=\llaves{
\left(
\begin{array}{rr}
a_{11}&a_{12}\\
a_{21}&a_{22}\\
a_{31}&a_{32}
\end{array}
\right)
}{\begin{array}{rr}
a_{11}=a_{12}&a_{11}=a_{31}\\
a_{21}=a_{22}&a_{12}=a_{32}\\
a_{31}=a_{32}&
\end{array}
}
\]
Si colocamos todas estas ecuaciones en un sistema homog\'eneo e igualamos a cero tendr\'iamos

%\[
%  \begin{blockarray}{rrrrrrrr}
%&a_{11}&a_{12}&a_{21}&a_{22}&a_{31}&a_{32}&\ \\
%    \begin{block}{r(rrrrrr|r)}
%&1    &-1     &0     &0     &0     &0     &0\\
%&0    &0      &1     &-1    &0     &0     &0\\
%&0    &0      &0     &0     &1     &-1     &0\\
%&1    &0      &0     &0     &-1     &0     &0\\
%&0    &1     &0     &0     &0     &-1     &0\\
%    \end{block}
%  \end{blockarray}
%\]

\[
\left(
\begin{array}{rrrrrr|r}
1    &-1     &0     &0     &0     &0     &0\\
0    &0      &1     &-1    &0     &0     &0\\
0    &0      &0     &0     &1     &-1     &0\\
1    &0      &0     &0     &-1     &0     &0\\
0    &1     &0     &0     &0     &-1     &0\\
\end{array}
\right)
\sim
\ldots
\sim
\left(
\begin{array}{rrrrrr|r}
1    &-1     &0     &0     &0     &0     &0\\
0    &1     &0     &0     &0     &-1     &0\\
0    &0      &1     &-1    &0     &0     &0\\
0    &0      &0     &0     &1     &-1     &0\\
0    &0      &0     &0     &0     &0     &0\\
\end{array}
\right)
\]

Donde queda claro que había una ecuaci\'on redundante, por lo tanto


\[W_1\cap W_2=\llaves{
\left(
\begin{array}{rr}
a_{11}&a_{12}\\
a_{21}&a_{22}\\
a_{31}&a_{32}
\end{array}
\right)
}{\begin{array}{r}
a_{11}=a_{32}\\
a_{21}=a_{22}\\
a_{12}=a_{32}\\
a_{31}=a_{32}\\
\end{array}
}
\]
Para hallar una base:
\[\left(
\begin{array}{rr}
a_{32}&a_{32}\\
a_{22}&a_{22}\\
a_{32}&a_{32}
\end{array}
\right)
=a_{22}
\left(
\begin{array}{rr}
0&0\\
1&1\\
0&0
\end{array}
\right)
+a_{32}
\left(
\begin{array}{rr}
1&1\\
0&0\\
1&1
\end{array}
\right)
\]

De esta manera obtenemos una base para la intersecci\'on:


%
%\begin{prob}[(1er Examen ESPOL 2012)]
%Sea $T:\rdos \rightarrow \rdos $ 
%\end{prob}

\newpage
\begin{prob}[(3ra Evaluacion Septiembre 2012)]
(10 puntos)Dado la matriz $A=\left(\begin{matrix}
1&10&1\\
2&2&2\\
-1&-1&k\\
\end{matrix}\right)$
, determine los valores de k para que la nulidad sea cero.
\end{prob}

%
%\begin{prob}[(3ra Evaluacion Septiembre 2012)]
%(10 puntos)Sea $V=C[0,1]$ y $fm g \in V$. Demuestre que si el conjunto ${f, g}$ es linealmente dependiente entonces:
%\[W=\left|\begin{matrix}
%f(x)&g(x)\\
%f'(x)&g'(x)\\
%\end{matrix}\right|=0\]
%\end{prob}
%

\newpage
\begin{prob}[(3ra Evaluacion Febrero 2009)]
(10 puntos) Califique como V o F:Sea $A=\left(\begin{matrix}
2a&2b&-c\\
a&2b&-c\\
-a&-b&c\\
\end{matrix}\right)$ con a, b, c diferentes de cero, entonces el rango de A es igual a 3 y la nulidad de A es igual a 0.
\end{prob}

\newpage

\begin{prob}[(3ra Evaluacion Abril 2011)]
(10 puntos)Califique como Vo F: Sea $A=\mathcal{M}_{2x3}$. Entonces la nulidad de A es mayor o igual a 3,
\end{prob}

\newpage
\begin{prob}[(1ra Evaluacion Julio 2012)]
(10 puntos)Sean $B_1=\{v_1, v_2, v_3\}$ y $B_2=\{v_1-v_2, v_2+v_3, 2v_1\}$ dos bases de un espacio vectorial $V$. Si $[E]_{B_1}=[F]_{B_2}=\vectrtres{1}{1}{1}$ determine $[5E-2F]_{B_2}$
\end{prob}

\newpage
\begin{prob}[(1ra Evaluacion Julio 2012)]
(20 puntos)Sea $V=\ptres$. Considere el conjunto de todos los subespacios de V tal que $$H(a)=gen\llav{1+ax+x^2+x^3, 1+ax+(1-a)x^2+x^3, x+(2a)x^2+2x^3, 1+(1+a)x+(1+a)x^2+3x^3}$$
a) Determine el valor de $a$ para que $dimH=2$ \\
b) Halle una base y la dimensión de los subespacios $H(0)\cap H(1)$ y $H(0)+ H(1)$
\end{prob}


\newpage

\newpage

%Megas 2do parcial~\\
%~\\
%\begin{prob}[(3ra Evaluacion 14 de febrero 2014)]
%(10 puntos) 
%a)Demuestre que $Nu(T)\subseteq Nu(T^{2})$
%~\\
%b) Si ademas $\rho(T)=\rho(T^{2})$ , demuestre que $Nu(T)= Nu(T^{2})$
%\end{prob}

%
%\begin{prob}[(2da Evaluacion X X)]
%(10 puntos) Verdadero o Falso: Sea $L:V->V$ un isomorfismo. Sea $B=\{v_1, v_2, v_3\}$
% una base del espacio vectorial V. Entonces $B2=\{L(v_1), L(v_2), L(v_3)\}$ es una base de V.
%\end{prob} 
%
%\begin{prob}[(2da Evaluacion X X)]
%(20 puntos) Encuentre de ser posible una matriz diagonal D semejante a:
%$A=\left(\begin{array}{rrr}
%-1&2&1\\
%0&-1&0\\
%-1&-3&-3\\
%\end{array}\right)$ 
%\end{prob} 
%
%
%\begin{prob}[(2da Evaluacion X X)]
%(10 puntos) Verdadero o Falso: La matriz $A=\left(\begin{array}{rr}
%k&0\\
%1&3\\
%\end{array}\right)$ es diagonalizable para todo k
%\end{prob}
%
%
%
%\begin{prob}[(2da Evaluacion X X)]
%(10 puntos) Construya de ser posible un operador lineal L en el espacio vectorial \pdos que cumpla las siguientes condiciones: $L(1+x)=-1-x y E(\lambda=3)=gen{2-x}.$ ¿Es diagonalizable este operador? Justifique su respuesta.
%\end{prob}
%
%

%\begin{prob}[(2da Evaluacion X X)]
%(15 puntos) Sea ai=(xi,yi) un vector que indica la posición de una partícula sobre un punto al cabo de t periodos, suponga que:
%$\forall t \in N a_t=\left(\begin{array}{rr}
%1/2&1\\
%0&1/3\\
%\end{array}\right)a_{t-1}$ 
%Si la posición inicial de la partícula es $a_o=(1,-1)$ Determine:
%~\\
%a)La posición de la partícula luego de T=2 periodos
%B)La posicion luego de t periodos
%c) La posicion en estado estable, es decir t=inf.
%
%\end{prob} 
%

%\begin{prob}[(2da Evaluacion X X)]
%(10 puntos) Sean A,B dos matrices semejantes:
%~\\a)Muestre que A y B tienen los mismos valores propios
%~\\b) Tienen los mismos vectores propios? Justifique se respuesta
%
%\end{prob} 
%\begin{prob}[(2da Evaluacion X X)]
%(10 puntos) Sea $A=\left(\begin{array}{rrr}
%1&0&k\\
%3&3&-3\\
%1&0&2\\
%\end{array}\right)$ 
%~\\a) Determine el valor de k para que $\lambda=5$ sea valor propio de A
%~\\b) Para k=20 determine si A es diagonalizable
%\end{prob} 
%
%
%\begin{prob}[(2da Evaluacion X X)]
%(10 puntos) Verdadero o Falso:
%~\\
%La funcion $f:\pdos x \pdos -> R$, definida por $f(p(x),q(x))=p(1)q(1)$ es un producto interno en \pdos
%
%\end{prob} 

%
%\begin{prob}[(2da Evaluacion X X)]
%(15 puntos) Sea (.,.) el producto interno real estandar en el espacio vectorial \rtres. Sea L un operador linel en \rtres
%\[L(x,y,x)=(x,x+y,x+y+z)\]
%Es tambien un producto interno real en \rtres la funcion $(.,.)_{L}$
%\[(v1,v2)_{L}=(L(v1),L(v2))\]
%Sea W un subespacio vectorial tal que
%\[W=gen\{(1,1,1),(-1,0,1)   \}\]
%a) Encuentre una base y determine el complemento ortogonal de W.
%~\\b)Expresar el vector (0,1,1) como la suma de un vector de W y un vector de $W*$
%
%\end{prob} 


%
%\begin{prob}[(2da Evaluacion X X)]
%(10 puntos) Aplicando diagonalizacion ortogonal grafique el lugar geométrico $5x^2+4xy+2y^2-24x-12y+29=0$
%\end{prob} 

\newpage
\subsubsection{Califique como verdaderas o falsas las siguientes proposiciones}
\begin{prop}[(1er Examen ESPOL 2012)]

Sea $V$ un espacio vectorial tal que $H\subseteq V$. Si $H$ es un subespacio vectorial de $V$ entonces $H^C$ es subespacio de $V$.

\end{prop}

\sol
~\\
Debido a que $H$ es un subespacio vectorial(tambi\'en es espacio vectorial), el elemento neutro $0_v$ pertenece a H, pero esto implica que $0_v \notin H^C$.

~\\
Si $0_v \notin H^C$ entonces $H^C$ no puede ser espacio vectorial, luego, tampoco subespacio vectorial de $ V$. 
$\therefore$ La proposici\'on es FALSA.

~\\
~\\
~\\
\begin{prop}[(1er Examen ESPOL 2009)]
Si $(V, \oplus, \odot)$ es un espacio vectorial, y sean $H$ y $W$ dos subespacios de V tales que:
$W=gen\llav{w_1, w_2, w_3}$ y $w_1, w_2  \in H$, entonces es cierto que $dim(H\cap W)=2$
\end{prop}

\sol
Considere el siguiente contraejemplo:
Sea $V={\cal P}_3$ y $W=gen\llav{1, x, x^2}$, y adem\'as $H=gen\llav{1, x, x^2}$ (Por supuesto, esto es intencional). Es claro que $1, x \in H$. Pero $dim(W\cap H)=3$.
$\therefore$ La proposici\'on es FALSA.

\newpage
\begin{prop}[(1er Examen ESPOL 2007)]
Si la matriz B se obtiene a partir de la matriz A por medio de un intercambio de filas entonces $\rho(A)=\rho(B)$
\end{prop}

\sol
Sean $f_1, f_2, \ldots, f_i, \ldots , f_j, \ldots, f_m$ las filas de A, luego $$F_A=gen\llav{f_1, f_2, \ldots, f_i, \ldots , f_j, \ldots, f_m}$$. Si B se obtiene al intercambiar las filas $f_i, f_j$ entonces, 
$$F_B=gen\llav{f_1, f_2, \ldots, f_j, \ldots , f_i, \ldots, f_m}$$
$$F_B=gen\llav{f_1, f_2, \ldots, f_i, \ldots , f_j, \ldots, f_m}$$
$$F_B=F_A$$
Por lo tanto, $dimF_A=dim F_B$, esto es lo mismo que $\rho(A)=\rho(B)$.
$\therefore $ La proposici\'on es VERDADERA.
~\\
~\\
~\\

\begin{prop}[(1er Examen ESPOL 2007)]
Sea $V$ un espacio vectorial. Sean A, B $\subseteq V$, entonces $gen(A\cap B)=gen(A)\cap gen(B)$
\end{prop}
\sol

Consideremos el siguiente contraejemplo:~\\
Sea V=\pdos, y $A=\llav{1,x^2}, B=\llav{1,x^2+1}$, entonces,
\[gen(A)=\llaves{a_0+a_1x+a_2x^2}{a_1=0}\]
\[gen(B)=\llaves{a_0+a_1x+a_2x^2}{a_1=0}\]
Aqu\'i es evidente que el espacio generado por B es el mismo que el espacio generado por A, ya que ambos tienen al vector 1, y el segundo vector de B es una combinaci\'on lineal de los dos vectores de A.
Si los espacios son iguales, la intersecci\'on entre estos espacios es la misma.
\[gen(A)\cap gen(B)=\llaves{a_0+a_1x+a_2x^2}{a_1=0}\]
Ahora, sin embargo, es distinto el espacio generado de la intersecci\'on de A y B:
\[gen(A\cap B)=gen\llav{1}=\llaves{a_0+a_1x+a_2x^2}{a_1=0,\ a_2=0}\]
como vemos, es claro que $gen(A\cap B)\neq gen(A)\cap gen(B)$. $\therefore$ La proposici\'on es FALSA.

\newpage
\begin{prop}[Califique como verdadero o falso]


Sea $V$ un espacio vectorial tal que $H\subseteq V$. Si $H$ es un subespacio vectorial de $V$ entonces $H^C$ es subespacio de $V$.

\end{prop}
~\\~\\~\\
Solución:
~\\
~\\~\\
Si $H$ es un subespacio de $V$ entonces es cierto que $n_v \in H$, el elemento neutro se encuentra en el conjunto H.~\\
~\\
Ya que $n_v \in H$ entonces se cumple que $n_v \notin H^c$, lo cual implica que $H^c$ no puede ser espacio vectorial de $V$ ya que no contiene al neutro.
 
~\\
La proposición es falsa.

\newpage
\newpage


\begin{prob}[]
En el espacio vectorial \mdosxdos, determinar si 
\matrdxd{2&1}{-1&0}
 es combinaci\'on lineal de los vectores 
\matrdxd{1&0}{-1&0}, \matrdxd{-1&2}{0&1}, \matrdxd{0&1}{2&-1}.
\end{prob}

\newpage

\begin{prop}[Califique como verdadero o falso]
Sea $V$ un espacio vectorial. Sean A, B $\subseteq V$, entonces $gen(A\cap B)=gen(A)\cap gen(B)$
\end{prop}
\sol
Por contraejemplo:
~\\
Sea $V=\rtres$, además~\\
\[A=\left\lbrace \vectrtres{1}{0}{0}, \vectrtres{0}{1}{0} \right\rbrace\  ;  B=\left\lbrace \vectrtres{0}{0}{1}, \vectrtres{0}{-1}{0} \right\rbrace\]
Es evidente que:~\\
\[H=gen(A)=\llaves{\vectrtres{a}{b}{c}}{c=0}\]
\[W=gen(B)=\llaves{\vectrtres{a}{b}{c}}{a=0}\]


Si calculamos $H\cap W$ tendríamos:
~\\
\[H\cap W=\llaves{\vectrtres{a}{b}{c}}{a=0, c=0}\]

Ahora, si calculamos $gen(A \cup B)$, (A y B no tienen elementos en común)~\\
\[A\cap B=\phi \]
Con esto se tiene que
\[gen(A\cap B)=\lbrace n_v \rbrace\]~\\
Luego
\[gen(A\cap B)\neq gen(A)\cap gen(B)\]

$\therefore$ La proposición es falsa
\newpage


\begin{prob}[]
Demuestre:
\\Sea $S=\conjvect{v}{n}$ un subconjunto linealmente independiente de vectores del espacio vectorial $V$ y sea $x$ un vector de $V$ que no puede ser expresado como una combinaci\'on lineal de los vectores de S, entonces $\llav{\kvect{v}{n}, x}$
tambi\'en es linealmente independiente.
\end{prob}
~\\
\sol
Demostración:~\\
Consideremos la siguiente ecuación:
~\\
\[\alpha_1 v_1+\alpha_2 v_2+...+\alpha_n v_n+\beta x=0_v\]
Supongamos que $\beta\neq 0$
Si esto es cierto, entonces existe el inverso multiplicativo de $\beta$ y es posible multiplicar a la ecuación anterior por este factor:~\\
\[(\alpha_1/\beta) v_1+(\alpha_2/\beta) v_2+...+(\alpha_n/\beta) v_n+ x=(1/\beta)0_v\]
Y si además, despejamos el vector x tendríamos:
\[x=(-\alpha_1/\beta) v_1+(-\alpha_2/\beta) v_2+...+(-\alpha_n/\beta) v_n\]
Lo que indicaría que x se puede escribir como combinación lineal de los vectores de S, lo cual es imposible por hipótesis. Esto es una contradicción.
~\\
~\\
Por lo tanto, esto implica que $\beta=0$. Reemplazando en la primera ecuación tendríamos que:
\[\alpha_1 v_1+\alpha_2 v_2+...+\alpha_n v_n+0*x=0_v\]
\[\alpha_1 v_1+\alpha_2 v_2+...+\alpha_n v_n=0_v\]
Ya que el conjunto S es linealmente independiente entonces para cualquier $\alpha_i=0$

~\\
Se concluye que $S\cup \lbrace x\rbrace$ es linealmente independiente.



\newpage

%repetido
%\begin{prob}[]
%
%Sea $V=C[0,1]$ y $f, g \in V$. Demuestre que si el conjunto $\{f, g\}$ es linealmente dependiente entonces:
%\[W=\left|\begin{matrix}
%f(x)&g(x)\\
%f'(x)&g'(x)\\
%\end{matrix}\right|=0\]
%\end{prob}

\newpage



\newpage

%
%\begin{prob}
%Sea $A$ la matriz de coeficientes del sistema lineal
%\begin{eqnarray*}
%2x+y-z=a\\
%x-y+2z=b\\
%x+2y-3z=c
%\end{eqnarray*}
%\begin{enumerate}
%\item[a)] Determine el espacio fila, n\'ucleo y recorrido de $A$
%\item[b)] Si $c=2a+b$, determine si el vector \vectrtres{a}{b}{c} pertenece a $Im(A)$
%\end{enumerate}
%\end{prob}
%
%\sol
%
%De acuerdo al sistema se tiene que
%\[A=
%\left(
%\begin{array}{rrr}
%2&1&-1\\
%1&-1&2\\
%1&2&-3\\
%\end{array}
%\right)
%\]
%Espacio fila:
%\[F_A=gen\llav{
%\vectrtres{2}{1}{-1}, \vectrtres{1}{-1}{2}, \vectrtres{1}{2}{-3}
%}\]
%Lo que conduce al sistema
%
%\[A=
%\left(
%\begin{array}{rrr|r}
%2&1&1&x\\
%1&-1&2&y\\
%-1&2&-3&z\\
%\end{array}
%\right)
%\sim
%\left(
%\begin{array}{rrr|r}
%2&1&1&x\\
%0&3&-3&x-2y\\
%0&5&-5&x+2z\\
%\end{array}
%\right)
%\sim
%\left(
%\begin{array}{rrr|r}
%2&1&1&x\\
%0&3&-3&x-2y\\
%0&0&0&2x-10y-6z\\
%\end{array}
%\right)
%\]
%Por lo que
%\[F_A=\llaves{\vectrtres{x}{y}{z}}{2x-10y-6z=0}\]
%
%N\'ucleo:
%
%El n\'ucleo es la soluci\'on del sistema homog\'eneo $AX=0$:
%\[
%\left(
%\begin{array}{rrr|r}
%2&1&-1&0\\
%1&-1&2&0\\
%1&2&-3&0\\
%\end{array}
%\right)
%\sim
%\left(
%\begin{array}{rrr|r}
%2&1&-1&0\\
%0&3&-5&0\\
%0&-3&5&0\\
%\end{array}
%\right)
%\sim
%\left(
%\begin{array}{rrr|r}
%2&1&-1&0\\
%0&3&-5&0\\
%0&0&0&0\\
%\end{array}
%\right)
%\]
%
%Por lo que:
%\[Nu(A)=\llaves{\vectrtres{x_1}{x_2}{x_3}}
%{\begin{array}{r}
%2x_1+x_2-x_3=0\\
%3x_2-5x_3=0
%\end{array}}
%\]
%
%Recorrido o Imagen de A:
%
%El recorrido es igual al espacio columna, el espacio generado por las columnas:
%
%\[Rec(A)=Im(A)=C_A\]
%
%\[
%\left(
%\begin{array}{rrr|r}
%2&1&-1&x\\
%1&-1&2&y\\
%1&2&-3&z\\
%\end{array}
%\right)
%\sim
%\left(
%\begin{array}{rrr|r}
%2&1&-1&x\\
%0&3&-5&x-2y\\
%0&-3&5&x-2z\\
%\end{array}
%\right)
%\sim
%\left(
%\begin{array}{rrr|r}
%2&1&-1&x\\
%0&3&-5&x-2y\\
%0&0&0&2x-2y-2z\\
%\end{array}
%\right)
%\]
%
%
%Por lo que
%\[C_A=\llaves{\vectrtres{x}{y}{z}}{2x-2y-2z=0}\]
%
%b) Si $c=2a+b$ entonces para que el vector \vectrtres{a}{b}{c} pertenezca a $C_A$ debe cumplirse la condici\'on de exige $C_A$
%\[
%2x-2y-2z=2(a)-2(b)-2(2a+b)=-2a-4b \neq 0
%\]
%
%Por tanto, no pertenece a $Im(A)$.
\newpage

%
%\begin{prob}[]
%
%Sean $A\in \mathcal{M}_{mxn}, B \in \mathcal{M}_{nxp}$, demuestre que $C_{AB}\subseteq C_A$
%
%\end{prob}



%
%\begin{prob}[]
%
%Sea $V=\ptres$. Considere el conjunto de todos los subespacios de V tal que $$H(a)=gen\llav{1+ax+x^2+x^3, 1+ax+(1-a)x^2+x^3, x+(2a)x^2+2x^3, 1+(1+a)x+(1+a)x^2+3x^3}$$
%Halle una base y la dimensión de los subespacios $H(0)\cap H(1)$ y $H(0)+ H(1)$
%\end{prob}


%\begin{prob}
%
%Sea $V=\mathcal{M}_{3x2}$. Sean $W_1$ el conjunto de las matrices que tienen la primera y la \'ultima fila iguales, $W_2$ el conjunto de las matrices que tienen la primera columna igual a su segunda columna.
%~\\Determine:~\\
%\begin{enumerate}
%\item[a)] La intersecci\'on entre $W_1, W_2$
%\item[b)] La suma entre $W_1, W_2$
%\item[c)] Una base para los subespacios intersecci\'on y suma obtenidos en (a) y (b).
%
%\end{enumerate}
%
%\end{prob}






\end{enumerate}


%%%%%%%%%%%%%%%%%%%%%%%%%%%%%%%%%%%%%%%%%%%%%%%%
%%%%%%%%%%%%%%%%%%%%%%%%%%%%%%%%%%%%%%%%%%%%%%%
%\section{Combinaciones lineales y subespacios generados}
%\begin{dfn}
%Sea $(E, +, \odot)$ un espacio vectorial real y sean $B=\conjvect{v}{n}$ un subconjunto de vectores en $E$. Un vector $v$ en $E$ se dice que es una combinación lineal de los vectores en $B$ si existen escalares $\alpha_1$, $\alpha_2$, $\ldots$ , $\alpha_n$ en \dobler , tales que $v=\av{\alpha_1}{v_1} + \av{\alpha_2}{v_2} + \ldots + \av{\alpha_n}{v_n}$.
%
%\end{dfn}
%
%\begin{ejemplo}
%Cualquier vector $(a, b, c)$ en \rtres puede ser expresado como una combinación lineal de los vectores  $(1,0,0)$,$(0,1,0)$ y $(0,0,1)$.
%
%\end{ejemplo}
%
%\begin{ejemplo}
%El vector $(4,5,5)$ puede ser expresado como una combinación lineal de los vectores $(1,2,3)$, $(-1,1,4)$ y $(3,3,2)$.
%\end{ejemplo}
%
%\begin{theorem}
%Sea $(E, +, \odot)$ un espacio vectorial y $A$ un subconjunto finito de $E$. El conjunto de todas las combinaciones lineales de elementos de $A$ es un subespacio de $E$ y se llama subespacio generado por $A$ y se denota por $CL(A)$.
%\end{theorem}
%
%\begin{proof}
%
%$\mathbf{0}_v \in CL(A)$ por el comentario anterior.\\
%Sean $u$, $v$ en $CL(A)$, entonces
%\begin{align*}
%u &=\av{\alpha_1}{v_1} + \ldots + \av{\alpha_n}{v_n}\\ 
%v&=\av{\beta_1}{v_1} + \ldots + \av{\beta_n}{v_n}
%\end{align*}
%entonces
%\begin{align*}
%u + v &= (\av{\alpha_1}{v_1} + \ldots + \av{\alpha_n}{v_n}) + (\av{\beta_1}{v_1} + \ldots + \av{\beta_n}{v_n})\\
%u + v &=(\alpha_1 + \beta_1)\,v_1 + \ldots + (\alpha_n + \beta_n)\, v_n
%\end{align*}
%Así
%$$u + v \in CL(A)$$
%Sea $u \in CL(A)$ y $\alpha \in \dobler$
%\begin{align*}
%\alpha u &= \alpha(\av{\alpha_1}{v_1} + \ldots + \av{\alpha_n}{v_n})\\
%\alpha u &= (\alpha \alpha_1)v_1 + \ldots + (\alpha \alpha_n)v_n
%\end{align*}
%luego $\alpha u \in CL(A)$\\
%
%$\therefore$ $CL(A)$ es un subespacio vectorial de $V$. \qedhere \\ %%%%%%%  el \qedhere es el QED
%
%\end{proof}
%El subespacio $CL(A)$ es llamado espacio generado por \conjvect{v}{n}. Si $V = CL(A)$ entonces decimos que $A$ genera al espacio $V$ o que $A$ es conjunto generador de $V$.
%
%\begin{ejemplo}
%Veamos que $A = \llav{(1,2,0),(0,1,-1),(1,1,2)}$ genera a \rtres .\
%En efecto veamos que cualquier vector $(x,y,z)$ es una combinación lineal de los elementos de $A$. Es decir debemos encontrara escalares $\alpha$, $\beta$ y $\lambda \in \dobler$ tales que $$(x_0, y_0, z_0) = \alpha (1,2,0) + \beta (0,1,-1) + \lambda (1,1,2)$$
%lo que nos conduce al estudio del sistema 
%
%$$\left\{
%\begin{array}{rcl}
%\alpha + \lambda &=& x_0\\
%2 \alpha + \beta + \lambda &=& y_0\\
%- \beta + 2\alpha &=& z_0 
%\end{array}
%\right.$$
%
%Donde se obtiene que 
%\begin{align*}
%\alpha &= 3x_0 - y_0 -z_0\\
%\beta &=-4 x_0 +2 y_0 +z_0\\
%\lambda &= -2 x_0 + y_0 +z_0 
%\end{align*}
%
%\end{ejemplo}
