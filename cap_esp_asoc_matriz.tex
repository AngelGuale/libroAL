\chapter{Espacios Asociados a una Matriz}

\section{Espacio Columna}
\begin{dfn}[Espacio Columna]
Sea $A$ una matriz $m\times n$, sean $c_1,c_2,\ldots, c_n \in \mathbb{R}^m$ las columnas de la matriz $A$. Se define el espacio columna de $A$ como el espacio generado por los vectores columnas de $A$:
~\\
\[C_A=gen\{c_1, c_2, ..., c_n\}\]
\end{dfn}
~\\

\begin{ejemplo}
Sea $A=
\left(
\begin{array}{rrr}
3&1&1\\
2&2&3\\
-1&1&2
\end{array}
\right)$ una matriz, determinar el espacio columna de $A$
 
De acuerdo a la definición, debemos hallar el espacio generado por los vectores \llav{\vectrtres{3}{2}{-1}, \vectrtres{1}{2}{1}, \vectrtres{1}{3}{2}} tal como lo vimos secciones anteriores:

$$\reducir{{rrr|l}}{3&1&1&a\\2&2&3&b\\-1&1&2&c}
\underrightarrow{\begin{array}{r}
    2f_1 - 3f_2 \\
    f_1 + 3f_3
\end{array}}
\reducir{rrr|l}{3&1&1&a\\0&-4&-7&2a-3b\\0&4&7&a+3c}
\underrightarrow{f_2 + f_3}
\reducir{rrr|l}{3&1&1&a\\0&-4&-7&2a-3b\\0&0&0&3a-3b+3c}$$

%%subespacio r3
\[C_A=\llaves{\vectrtres{a}{b}{c} \in \rtres}
{\begin{array}{r}
a-b+c=0\\
\end{array}}
\]

\end{ejemplo}

\newpage
\section{Espacio Fila}
\begin{dfn}[Espacio Fila]
Sea $A$ una matriz $m\times n$, sean $f_1,f_2,\ldots, f_m \in \mathbb{R}^n$ las filas de la matriz $A$. Se define el espacio fila de $A$ como el espacio generado por los vectores filas de $A$:

\[F_A=gen\{f_1, f_2, ..., f_m\}\]
\end{dfn}


\begin{ejemplo}
Sea $A=
\left(
\begin{array}{rrrr}
2&1&1&3\\
2&2&0&2\\
1&3&-2&-1\\
\end{array}
\right)$ una matriz, determinar el espacio fila de A

 
De acuerdo a la definición, debemos hallar el espacio generado por los vectores \llav{\vectr4{2}{1}{1}{3}, \vectr4{2}{2}{0}{2}, \vectr4{1}{3}{-2}{-1}} tal como lo vimos secciones anteriores:
$$\reducir{rrr|l}{2&2&1&a\\
1&2&3&b\\
1&0&-2&c\\
3&2&-1&d\\}
\underrightarrow{\begin{array}{r}
    f_1 - 2f_2\\
    f_1 - 2f_3\\
    3f_1 - 2f_4
\end{array}}
\reducir{rrr|r}{2&1&1&a\\
0&-2&-5&a-2b\\
0&2&5&a-2c\\
0&2&5&3a-2d\\}
\underrightarrow{\begin{array}{r}
    f_2 + f_3\\
    f_2 + f_4
\end{array}}
\reducir{rrr|r}{2&1&1&a\\
0&-2&-5&a-2b\\
0&0&0&2a-2b-2c\\
0&0&0&4a-2b-2d\\
}$$

%%subespacio r3
\[F_A=\llaves{\vectr4{a}{b}{c}{d} \in \rcuatro}
{\begin{array}{r}
a-b-c=0\\
2a-b-d=0\\
\end{array}}
\]

\end{ejemplo}


\newpage
\section{Núcleo de una Matriz}
\begin{dfn}[Núcleo de una matriz]
Sea $A$ una matriz $m\times n$, el núcleo de $A$ se define como el conjunto dado por:
~\\
\[Nu(A)=\llaves{X \in \rn }{AX=\cerorm}\]
\end{dfn}
En palabras más específicas, el núcleo es el conjunto solución del sistema homogéneo de una matriz, es decir, el sistema igualado a ceros.
~\\
\begin{theorem}
Sea $A$ una matriz $m \times n$, entonces el núcleo de $A$, es un subespacio vectorial de $\rn$ sobre un campo \dobleK
\end{theorem}
\begin{proof}
    Para esta prueba, utilizaremos la caracterización del subespacio.
    \begin{itemize}
        \item el vector neutro de $\rn$ pertenece a $Nu(A)$ ya que es la solución trivial del sistema homogéneo, por lo que $Nu(A)$ es un conjunto no vacío
        \item $\forall v,w \in Nu(A)$ significa que:
        $$Av = \mathbf{0}_{\rn}$$
        $$Aw = \mathbf{0}_{\rn}$$
        Sumando estas dos ecuaciones, tenemos que:
        $$Av + Aw = \mathbf{0}_{\rn} \rightarrow A(v+w) = \mathbf{0}_{\rn}$$
        por lo que el vector $v+w$ también cumple con la condición de estar en $Nu(A)$, por lo que cumple la cerradura bajo la suma
        \item $\forall v \in Nu(A), \forall \alpha \in \dobleK$, tenemos que:
        $$Av = \mathbf{0}_{\rn}$$
        Al multiplicar esta ecuación por $\alpha$:
        $$\alpha Av = \mathbf{0}_{\rn} \rightarrow A(\alpha v) = \mathbf{0}_{\rn}$$
        por lo que el vector $\alpha v$ también cumple con la condición de estar en $Nu(A)$, por lo que cumple con la cerradura bajo la multiplicación por escalar
    \end{itemize}
\end{proof}

\begin{ejemplo}
Sea $A=
\left(
\begin{array}{rrr}
3&1&1\\
2&2&3\\
-1&1&2
\end{array}
\right)$ una matriz, determinar el núcleo de A
 
Acorde a la definición presentada anteriormente, debemos hallar todas las soluciones del sistema homogéneo, en otras palabras, resolver el sistema igualado a ceros.
$$\reducir{{rrr|l}}{3&1&1&0\\2&2&3&0\\-1&1&2&0}
\underrightarrow{\begin{array}{r}
    2f_1 - 3f_2 \\
    f_1 + 3f_3
\end{array}}
\reducir{rrr|l}{3&1&1&0\\0&-4&-7&0\\0&4&7&0}
\underrightarrow{f_2 + f_3}
\reducir{rrr|l}{3&1&1&0\\0&-4&-7&0\\0&0&0&0}$$

Por lo tanto, este sistema posee infinitas soluciones, las cuales serán los vectores del conjunto núcleo de A, el se expresa como:
%%subespacio r3
\[\svrtres{Nu(A)}{x_1}{x_2}{x_3}{\begin{array}{r}
    3x_1 + x_2 + x_3 = 0  \\
    4x_2 + 7x_3 = 0 
\end{array}}
\]

\end{ejemplo}

\section{Nulidad de una Matriz}
\begin{dfn}
Sea $A$ una matriz de orden $m\times n$ y sea $Nu(A)$ un subespacio vectorial de $\rn$, se denomina a la dimensión de $Nu(A)$ como nulidad de $A$ y se denota como $v(A)$ 
\end{dfn}
\begin{theorem}[Corolario]
Al ser $Nu(A)$ un subespacio de $\rn$, entonces $v(A) \leq~n$
\end{theorem} 

\begin{ejemplo}
Sea $\svrtres{Nu(A)}{x_1}{x_2}{x_3}{\begin{array}{r}
    3x_1 + x_2 + x_3 = 0  \\
    4x_2 + 7x_3 = 0 
\end{array}}$, determine su dimensión.
\end{ejemplo}
\sol 
Para determinar la dimensión de $Nu(A)$, debemos encontrar una base, para eso tomamos un vector genérico y le reemplazamos las condiciones:
\begin{align*}
    4x_2 + 7x_3 = 0 &\rightarrow x_2 = -\frac{7}{4}x_3\\
    3x_1 + x_2 + x_3 = 0 &\rightarrow x_1 = -\frac{1}{3} x_2 - \frac{1}{3}x_3\\ &\rightarrow x_1 = -\frac{1}{3}\left(-\frac{7}{4}x_3 \right) - \frac{1}{3}x_3\\
    &\rightarrow x_1 = \frac{1}{4}x_3
\end{align*}
al reemplazar, obtenemos:
\[ \vectrtres{x_1}{x_2}{x_3} = \vectrtres{\frac{1}{4}x_3}{-\frac{7}{4}x_3}{x_3}  = x_3\vectrtres{\frac{1}{4}}{-\frac{7}{4}}{1}\]
entonces, una base para $Nu(A)$ es :
$$B_{Nu(A)} = \llav{\vectrtres{1}{-7}{4}}$$
Como hay un vector en la base, se puede concluir que $v(A) = 1$



\newpage
\section{Imagen de una Matriz}
\begin{dfn}[Imagen de una matriz]
Sea A una matriz $m\times n$, la imagen de A se define como el conjunto dado por:
~\\
\[Im(A)=\llaves{Y \in \rmm }{\exists X \in \rn,   AX=Y}\]
\end{dfn}

En otras palabras, la imagen es el conjunto de todas los resultados que se obtienen de multiplicar $A$ con algún vector de $\rn$
\begin{ejemplo}
Sea $A=
\left(
\begin{array}{rrr}
3&1&1\\
2&2&3\\
-1&1&2
\end{array}
\right)$ una matriz, determinar la imagen de la matriz A
 
De acuerdo a la definición, debemos resolver el sistema:

$$\reducir{{rrr|l}}{3&1&1&a\\2&2&3&b\\-1&1&2&c}
\underrightarrow{\begin{array}{r}
    2f_1 - 3f_2 \\
    f_1 + 3f_3
\end{array}}
\reducir{rrr|l}{3&1&1&a\\0&-4&-7&2a-3b\\0&4&7&a+3c}
\underrightarrow{f_2 + f_3}
\reducir{rrr|l}{3&1&1&a\\0&-4&-7&2a-3b\\0&0&0&3a-3b+3c}$$
%%subespacio r3
\[Im(A)=\llaves{\vectrtres{a}{b}{c} \in \rtres}
{\begin{array}{r}
a-b+c=0\\
\end{array}}
\]

\end{ejemplo}

\begin{ejercicio}

Sea $A \in M_{nxn} $. Muestre que si $A^2=I$ entonces $Nu(A)=\{n_v\}$
~\\
\sol
~\\
Si $A^2=I$  entonces tenemos que:~\\
\[det(A^2)=det(I)=1\]
\[det(A)det(A)=1\]
\[(det(A))^2=1\]
\[det(A)=1 \vee det(A)=-1\]
~\\
Esto nos indica que el determinante de A siempre es diferente de cero, por lo cual, A es inversible.~\\
~\\
~\\
Sea $X\in Nu(A)$ entonces 
\[AX=0_{\rn}\]
\[A^{-1}AX=A^{-1} 0_{\rn}\]
\[IX=0_{\rn}\]
\[X=0_{\rn}\]
~\\
Por tanto, el único elemento del núcleo de A es $0_{\rn}$
~\\
~\\
$\therefore$ $Nu(A)=\lbrace 0_{\rn} \rbrace$

\end{ejercicio}

\begin{ejercicio}
Sea $A=
\left(
\begin{array}{rrr}
3&1&1\\
2&2&3\\
-1&1&2
\end{array}
\right)$ una matriz, determinar el espacio columna de A y el espacio Fila de A.


\sol
El espacio columna se define como el espacio generado por las columnas de A.
\[C_A=gen\{c_1, c_2, c_3\}\]
~\\
Es decir
$$\left(
\begin{array}{rrr|r}
3&1&1&a\\
2&2&3&b\\
-1&1&2&c
\end{array}
\right)$$
Al reducir este sistema por el método de Gauss tenemos lo siguiente
$$\reducir{{rrr|l}}{3&1&1&a\\2&2&3&b\\-1&1&2&c}
\underrightarrow{\begin{array}{r}
    2f_1 - 3f_2 \\
    f_1 + 3f_3
\end{array}}
\reducir{rrr|l}{3&1&1&a\\0&-4&-7&2a-3b\\0&4&7&a+3c}
\underrightarrow{f_2 + f_3}
\reducir{rrr|l}{3&1&1&a\\0&-4&-7&2a-3b\\0&0&0&3a-3b+3c}$$

 Por tanto el espacio columna de A queda definido por:
 
%%subespacio r3
\[C_A=\llaves{\vectrtres{a}{b}{c} \in \rtres}
{\begin{array}{r}
a-b+c=0\\
\end{array}}
\]

~\\
(b) El espacio fila es el espacio generado por todas las filas de A. 
\[F_A=gen\{f_1, f_2, f_3\}\]
Lo cual implica resolver el siguiente sistema generador
$$\reducir{rrr|r}{
3&2&-1&a\\
1&2&1&b\\
1&3&2&c}$$
Aplicando el método de Gauss tenemos:~
$$\reducir{rrr|r}{
3&2&-1&a\\
1&2&1&b\\
1&3&2&c}
\underrightarrow{\begin{array}{c}
    f_1 - 3f_2\\
    f_1 - 3f_3
\end{array}}
\reducir{rrr|r}{
3&2&-1&a\\
0&-4&-4&a-3b\\
0&-7&-7&a-3c
}
\underrightarrow{7f_2 - 4f_3}
\reducir{rrr|r}{
3&2&-1&a\\
0&-4&-4&a-3b\\
0&0&0&3a-21b+12c}
$$
El espacio fila de A queda definido por:
 
 \[F_A=\left\lbrace  \vectrtres{a}{b}{c} \mid a-7b+4c=0 \right\rbrace  \]



\end{ejercicio}


\newpage
\begin{ejercicio}
Sean $A\in \mathcal{M}_{mxn}, B \in \mathcal{M}_{nxp}$, demuestre que $C_{AB}\subseteq C_A$


\sol
~\\
~\\
Sea Y un elemento de la imagen de $AB$, entonces existe un $X_1$ tal que 
$$Y=(AB)*X_1$$
esto es igual a 
$$Y=A*(BX_1)$$
Donde $BX_1$ es otro vector, es decir $X_2=BX_1$
por lo tanto
$$Y=A*X_2$$
Lo cual indica por definición que "Y" también pertenece a la imagen de A.
En conclusión $$Im(AB) \subseteq Im(A)$$
Por teorema, la imagen es igual al espacio columna, ergo
$$C_{AB} \subseteq C_A$$

\end{ejercicio}

\newpage
\section{Rango de una Matriz}
\begin{dfn}
Sea $A$ una matriz de orden $m\times n$ y sea $Im(A)$ un subespacio vectorial de $\rmm$, se denomina a la dimensión de $Im(A)$ como rango de $A$ y se denota como $\rho(A)$ 
\end{dfn}
\begin{theorem}[Corolario]
Al ser $Im(A)$ un subespacio de $\rmm$, entonces $\rho(A) \leq~m$
\end{theorem} 

\begin{ejemplo}
Sea $\svrtres{Im(A)}{x_1}{x_2}{x_3}{\begin{array}{r}
    x_1 + x_2 - x_3 = 0\\
    2x_2 + x_3 = 0
\end{array}}$, determine su dimensión.
\end{ejemplo}
\sol 
Para determinar la dimensión de $Im(A)$, debemos encontrar una base, para eso tomamos un vector genérico y le reemplazamos las condiciones:
\begin{align*}
    2x_2 + x_3 = 0 &\rightarrow x_3 = -2x_2\\
    x_1 + x_2 - x_3 = 0 &\rightarrow x_1 = x_3 - x_2\\ 
    &\rightarrow x_1 = -2x_2 - x_2\\
    &\rightarrow x_1 = -3x_2
\end{align*}
al reemplazar, obtenemos:
\[ \vectrtres{x_1}{x_2}{x_3} = \vectrtres{-3x_2}{x_2}{-2x_2}  = x_2\vectrtres{-3}{1}{-2}\]
entonces, una base para $Im(A)$ es :
$$B_{Im(A)} = \llav{\vectrtres{-3}{1}{-2}}$$
Como hay un vector en la base, se puede concluir que $\rho(A) = 1$\\

\begin{obsimp}
Para cualquier matriz $A_{m\times n}$, $C_A = Im(A)$. Es decir, el espacio columna es igual a la imagen de $A$. Además se cumple que: $$dim(F_A) = dim(C_A) = \rho(A)$$.
\end{obsimp}

\newpage
\section{Problemas}
\begin{enumerate}
\begin{prop}[]

~\\Sea $A \in M_{nxn} $, si $A^2=A$ muestre que $Nu(A)\cap Im(A)=\{n_v\}$
\end{prop}

\begin{prop}[Califique como verdadero o falso]

Si la matriz B se obtiene a partir de la matriz A por medio de un intercambio de filas entonces $\rho(A)=\rho(B)$
\end{prop}

\begin{prob}[]
Dada la matriz $A=\left(\begin{array}{rrr}
2&12&5\\
1&-5&-3\\
-1&3&2\\
4&-2&-3\\
\end{array}\right)$. ~\\
a) Encuentre una base y determine la dimensi\'on del Espacio Columna de A~\\
b) Encuentre una base y determine la dimensi\'on del N\'ucleo de A.
\end{prob}


\begin{prob}[(1ra Evaluacion Marzo 2013)]
(10 puntos) Sea $A=\left(\begin{matrix}
1&2&1\\
2&a&a\\
2&-1&b\\
\end{matrix}\right)$. Para qu\'e valores de a y b:~\\
a) $\rho(A)=2$~\\
b)$\nu(A)=0$ 
\end{prob}


\begin{prob}[(1ra Evaluacion Septiembre 2013)]
(20 puntos) Dada la matriz $A=\left(\begin{array}{rrr}
1&11&4\\
2&-5&-1\\
-1&10&3\\
5&-2&1\\
\end{array}\right)$. ~\\
a) Encuentre una base y determine la dimensi\'on del Espacio Columna de A~\\
b)Encuentre una base y determine la dimensi\'on del N\'ucleo de A.
\end{prob}



\end{enumerate}