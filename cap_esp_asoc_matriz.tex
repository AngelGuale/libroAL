\chapter{Espacios asociados a una matriz}
\section{Espacio columna}
\begin{dfn}[Espacio Columna]
Sea A una matriz $m\times n$, sean $c_1,c_2,\ldots, c_n \in \mathbb{R}^m$ las columnas de la matriz A. Se define el espacio columna de A como el espacio generado por los vectores columnas de A:
~\\
\[C_A=gen\{c_1, c_2, ..., c_n\}\]
\end{dfn}
~\\

\begin{ejemplo}
Sea $A=
\left(
\begin{array}{rrr}
3&1&1\\
2&2&3\\
-1&1&2
\end{array}
\right)$ una matriz, determinar el espacio columna de A
 
De acuerdo a la definición, debemos hallar el espacio generado por los vectores \vectrtres{3}{2}{-1}, \vectrtres{1}{2}{1}, \vectrtres{1}{3}{2} tal como lo vimos secciones anteriores:

\[
\left(
\begin{array}{rrr|r}
3&1&1&a\\
2&2&3&b\\
-1&1&2&c
\end{array}
\right)
\sim
\left(
\begin{array}{rrr|r}
3&1&1&a\\
0&4&7&3b-2a\\
0&4&7&3c+a
\end{array}
\right)
\sim
\left(
\begin{array}{rrr|r}
3&1&1&a\\
0&4&7&3b-2a\\
0&0&0&3c+3a-3b
\end{array}
\right)
\]

%%subespacio r3
\[C_A=\llaves{\vectrtres{a}{b}{c}}
{\begin{array}{r}
c+a-b=0\\
\end{array}}
\]

\end{ejemplo}

\newpage
\section{Espacio fila}
\begin{dfn}[Espacio Fila]
Sea A una matriz $m\times n$, sean $f_1,f_2,\ldots, f_m \in \mathbb{R}^n$ las filas de la matriz A. Se define el espacio fila de A como el espacio generado por los vectores filas de A:

\[F_A=gen\{f_1, f_2, ..., f_m\}\]
\end{dfn}


\begin{ejemplo}
Sea $A=
\left(
\begin{array}{rrrr}
2&1&1&3\\
2&2&0&2\\
1&3&-2&-1\\
\end{array}
\right)$ una matriz, determinar el espacio fila de A

 
De acuerdo a la definición, debemos hallar el espacio generado por los vectores \vectr4{2}{1}{1}{3}, \vectr4{2}{2}{0}{2}, \vectr4{1}{3}{-2}{-1} tal como lo vimos secciones anteriores:

\[
\left(
\begin{array}{rrr|r}
2&2&1&a\\
1&2&3&b\\
1&0&-2&c\\
3&2&-1&d\\
\end{array}
\right)
\sim
\left(
\begin{array}{rrr|r}
2&1&1&a\\
0&-2&-5&a-2b\\
0&2&5&a-2c\\
0&2&5&3a-2d\\
\end{array}
\right)
\sim
\left(
\begin{array}{rrr|r}
2&1&1&a\\
0&-2&-5&a-2b\\
0&0&0&2a-2b-2c\\
0&0&0&4a-2b-2d\\
\end{array}
\right)
\]

%%subespacio r3
\[F_A=\llaves{\vectr4{a}{b}{c}{d}}
{\begin{array}{r}
a-b-c=0\\
2a-b-d=0\\
\end{array}}
\]

\end{ejemplo}


\newpage
\section{Núcleo de una matriz}
\begin{dfn}[Núcleo de una matriz]
Sea A una matriz $m\times n$, el núcleo de A se define como el conjunto dado por:
~\\
\[Nu(A)=\llaves{X \in \rn }{AX=\cerorm}\]
\end{dfn}
En palabras más específicas, el núcleo es el conjunto de todas las soluciones del sistema homogéneo de una matriz, es decir, el sistema igualado a ceros.
~\\

\begin{ejemplo}
Sea $A=
\left(
\begin{array}{rrr}
3&1&1\\
2&2&3\\
-1&1&2
\end{array}
\right)$ una matriz, determinar el núcleo de A
 
Acorde a la definición presentada anteriormente, debemos hallar todas las soluciones del sistema homogéneo, en otras palabras, resolver el sistema igualado a ceros.
\[
\left(
\begin{array}{rrr}
3&1&1\\
2&2&3\\
-1&1&2
\end{array}
\right)
\vectrtres{x_1}{x_2}{x_3}=
\vectrtres{0}{0}{0}
\]

\[
\left(
\begin{array}{rrr|r}
3&1&1&0\\
2&2&3&0\\
-1&1&2&0
\end{array}
\right)
\sim
\left(
\begin{array}{rrr|r}
3&1&1&0\\
0&4&7&0\\
0&4&7&0
\end{array}
\right)
\sim
\left(
\begin{array}{rrr|r}
3&1&1&0\\
0&4&7&0\\
0&0&0&0
\end{array}
\right)
\]

Por lo tanto, este sistema posee infinitas soluciones, las cuales serán los vectores del conjunto núcleo de A, el se expresa como:
%%subespacio r3
\[Nu(A)=\llaves{\vectrtres{x_1}{x_2}{x_3}}
{\begin{array}{r}
3x_1+x_2+x_3=0\\
4x_2+7x_3=0\\
\end{array}}
\]

\end{ejemplo}

\newpage
\section{Imagen de una matriz}
\begin{dfn}[Imagen de una matriz]
Sea A una matriz $m\times n$, la imagen de A se define como el conjunto dado por:
~\\
\[Im(A)=\llaves{Y \in \rmm }{\exists X \in \rn,   AX=Y}\]
\end{dfn}


\begin{ejemplo}
Sea $A=
\left(
\begin{array}{rrr}
3&1&1\\
2&2&3\\
-1&1&2
\end{array}
\right)$ una matriz, determinar la imagen de la matriz A
 
De acuerdo a la definición, debemos resolver el sistema:

\[
\left(
\begin{array}{rrr|r}
3&1&1&a\\
2&2&3&b\\
-1&1&2&c
\end{array}
\right)
\sim
\left(
\begin{array}{rrr|r}
3&1&1&a\\
0&4&7&3b-2a\\
0&4&7&3c+a
\end{array}
\right)
\sim
\left(
\begin{array}{rrr|r}
3&1&1&a\\
0&4&7&3b-2a\\
0&0&0&3c+3a-3b
\end{array}
\right)
\]

%%subespacio r3
\[Im(A)=\llaves{\vectrtres{a}{b}{c}}
{\begin{array}{r}
c+a-b=0\\
\end{array}}
\]

\end{ejemplo}

\begin{ejercicio}

Sea $A \in M_{nxn} $. Muestre que si $A^2=I$ entonces $Nu(A)=\{n_v\}$
~\\
\sol
~\\
Si $A^2=I$  entonces tenemos que:~\\
\[det(A^2)=det(I)=1\]
\[det(A)det(A)=1\]
\[(det(A))^2=1\]
\[det(A)=1 \vee det(A)=-1\]
~\\
Esto nos indica que el determinante de A siempre es diferente de cero, por lo cual, A es inversible.~\\
~\\
~\\
Sea $X\in Nu(A)$ entonces 
\[AX=0_{\rn}\]
\[A^{-1}AX=A^{-1} 0_{\rn}\]
\[IX=0_{\rn}\]
\[X=0_{\rn}\]
~\\
Por tanto, el único elemento del núcleo de A es $0_{\rn}$
~\\
~\\
$\therefore$ $Nu(A)=\lbrace 0_{\rn} \rbrace$

\end{ejercicio}

\begin{ejercicio}
Sea $A=
\left(
\begin{array}{rrr}
3&1&1\\
2&2&3\\
-1&1&2
\end{array}
\right)$ una matriz, determinar el espacio columna de A y el espacio Fila de A.


\sol
El espacio columna se define como el espacio generado por las columnas de A.
\[C_A=gen\{c_1, c_2, c_3\}\]
~\\
Es decir
$$\left(
\begin{array}{rrr|r}
3&1&1&a\\
2&2&3&b\\
-1&1&2&c
\end{array}
\right)$$
Al reducir este sistema por el método de Gauss tenemos lo siguiente
$$\left(
\begin{array}{rrr|r}
3&1&1&a\\
2&2&3&b\\
-1&1&2&c
\end{array}
\right)
\sim ... \sim
\left(
\begin{array}{rrr|r}
-1&1&2&c\\
0&4&7&2c+b\\
0&0&0&b-c-a
\end{array}
\right)$$
 Por tanto el espacio columna de A queda definido por:
 
 \[C_A=\left\lbrace  \vectrtres{a}{b}{c} \mid b-c-a=0 \right\rbrace  \]
~\\
(b) El espacio fila es el espacio generado por todas las filas de A. 
\[F_A=gen\{f_1, f_2, f_3\}\]
Lo cual implica resolver el siguiente sistema generador
$$\left(
\begin{array}{rrr|r}
3&2&-1&a\\
1&2&1&b\\
1&3&2&c
\end{array}
\right)$$
Aplicando el método de Gauss tenemos:
$$\left(
\begin{array}{rrr|r}
3&2&-1&a\\
1&2&1&b\\
1&3&2&c
\end{array}
\right)
\sim ...\sim
\left(
\begin{array}{rrr|r}
3&2&-1&a\\
0&-4&-4&a-3b\\
0&0&0&-3a+21b-12c
\end{array}
\right)
$$
El espacio fila de A queda definido por:
 
 \[F_A=\left\lbrace  \vectrtres{a}{b}{c} \mid -a+7b-4c=0 \right\rbrace  \]



\end{ejercicio}


\begin{ejercicio}
Sean $A\in \mathcal{M}_{mxn}, B \in \mathcal{M}_{nxp}$, demuestre que $C_{AB}\subseteq C_A$


~\\

\sol
~\\
~\\
Sea Y un elemento de la imagen de $AB$, entonces existe un $X_1$ tal que 
$$Y=(AB)*X_1$$
esto es igual a 
$$Y=A*(BX_1)$$
Donde $BX_1$ es otro vector, es decir $X_2=BX_1$
por lo tanto
$$Y=A*X_2$$
Lo cual indica por definición que "Y" también pertenece a la imagen de A.
En conclusión $$Im(AB) \subseteq Im(A)$$
Por teorema, la imagen es igual al espacio columna, ergo
$$C_{AB} \subseteq C_A$$

\end{ejercicio}

\newpage
\section{Problemas}
\begin{enumerate}
\begin{prop}[]

~\\Sea $A \in M_{nxn} $, si $A^2=A$ muestre que $Nu(A)\cap Im(A)=\{n_v\}$
\end{prop}

\begin{prop}[Califique como verdadero o falso]

Si la matriz B se obtiene a partir de la matriz A por medio de un intercambio de filas entonces $\rho(A)=\rho(B)$
\end{prop}

\begin{prob}[]
Dada la matriz $A=\left(\begin{array}{rrr}
2&12&5\\
1&-5&-3\\
-1&3&2\\
4&-2&-3\\
\end{array}\right)$. ~\\
a) Encuentre una base y determine la dimensi\'on del Espacio Columna de A~\\
b) Encuentre una base y determine la dimensi\'on del N\'ucleo de A.
\end{prob}


\begin{prob}[(1ra Evaluacion Marzo 2013)]
(10 puntos) Sea $A=\left(\begin{matrix}
1&2&1\\
2&a&a\\
2&-1&b\\
\end{matrix}\right)$. Para qu\'e valores de a y b:~\\
a) $\rho(A)=2$~\\
b)$\nu(A)=0$ 
\end{prob}


\begin{prob}[(1ra Evaluacion Septiembre 2013)]
(20 puntos) Dada la matriz $A=\left(\begin{array}{rrr}
1&11&4\\
2&-5&-1\\
-1&10&3\\
5&-2&1\\
\end{array}\right)$. ~\\
a) Encuentre una base y determine la dimensi\'on del Espacio Columna de A~\\
b)Encuentre una base y determine la dimensi\'on del N\'ucleo de A.
\end{prob}



\end{enumerate}