\chapter{Espacios asociados a una matriz}
\section{Espacio columna}
\begin{dfn}[Espacio Columna]
Sea A una matriz $m\times n$, sean $c_1,c_2,\ldots, c_n \in \mathbb{R}^m$ las columnas de la matriz A. Se define el espacio columna de A como el espacio generado por los vectores columnas de A:
~\\
\[C_A=gen\{c_1, c_2, ..., c_n\}\]
\end{dfn}
~\\

\begin{ejemplo}
Sea $A=
\left(
\begin{array}{rrr}
3&1&1\\
2&2&3\\
-1&1&2
\end{array}
\right)$ una matriz, determinar el espacio columna de A
 
De acuerdo a la definición, debemos hallar el espacio generado por los vectores \vectrtres{3}{2}{-1}, \vectrtres{1}{2}{1}, \vectrtres{1}{3}{2} tal como lo vimos secciones anteriores:

\[
\left(
\begin{array}{rrr|r}
3&1&1&a\\
2&2&3&b\\
-1&1&2&c
\end{array}
\right)
\sim
\left(
\begin{array}{rrr|r}
3&1&1&a\\
0&4&7&3b-2a\\
0&4&7&3c+a
\end{array}
\right)
\sim
\left(
\begin{array}{rrr|r}
3&1&1&a\\
0&4&7&3b-2a\\
0&0&0&3c+3a-3b
\end{array}
\right)
\]

%%subespacio r3
\[C_A=\llaves{\vectrtres{a}{b}{c}}
{\begin{array}{r}
c+a-b=0\\
\end{array}}
\]

\end{ejemplo}

\newpage
\section{Espacio fila}
\begin{dfn}[Espacio Fila]
Sea A una matriz $m\times n$, sean $f_1,f_2,\ldots, f_m \in \mathbb{R}^n$ las filas de la matriz A. Se define el espacio fila de A como el espacio generado por los vectores filas de A:

\[F_A=gen\{f_1, f_2, ..., f_m\}\]
\end{dfn}


\begin{ejemplo}
Sea $A=
\left(
\begin{array}{rrrr}
2&1&1&3\\
2&2&0&2\\
1&3&-2&-1\\
\end{array}
\right)$ una matriz, determinar el espacio fila de A

 
De acuerdo a la definición, debemos hallar el espacio generado por los vectores \vectr4{2}{1}{1}{3}, \vectr4{2}{2}{0}{2}, \vectr4{1}{3}{-2}{-1} tal como lo vimos secciones anteriores:

\[
\left(
\begin{array}{rrr|r}
2&2&1&a\\
1&2&3&b\\
1&0&-2&c\\
3&2&-1&d\\
\end{array}
\right)
\sim
\left(
\begin{array}{rrr|r}
2&1&1&a\\
0&-2&-5&a-2b\\
0&2&5&a-2c\\
0&2&5&3a-2d\\
\end{array}
\right)
\sim
\left(
\begin{array}{rrr|r}
2&1&1&a\\
0&-2&-5&a-2b\\
0&0&0&2a-2b-2c\\
0&0&0&4a-2b-2d\\
\end{array}
\right)
\]

%%subespacio r3
\[F_A=\llaves{\vectr4{a}{b}{c}{d}}
{\begin{array}{r}
a-b-c=0\\
2a-b-d=0\\
\end{array}}
\]

\end{ejemplo}


\newpage
\section{Núcleo de una matriz}
\begin{dfn}[Núcleo de una matriz]
Sea A una matriz $m\times n$, el núcleo de A se define como el conjunto dado por:
~\\
\[Nu(A)=\llaves{X \in \rn }{AX=\cerorm}\]
\end{dfn}
En palabras más específicas, el núcleo es el conjunto de todas las soluciones del sistema homogéneo de una matriz, es decir, el sistema igualado a ceros.
~\\

\begin{ejemplo}
Sea $A=
\left(
\begin{array}{rrr}
3&1&1\\
2&2&3\\
-1&1&2
\end{array}
\right)$ una matriz, determinar el núcleo de A
 
Acorde a la definición presentada anteriormente, debemos hallar todas las soluciones del sistema homogéneo, en otras palabras, resolver el sistema igualado a ceros.
\[
\left(
\begin{array}{rrr}
3&1&1\\
2&2&3\\
-1&1&2
\end{array}
\right)
\vectrtres{x_1}{x_2}{x_3}=
\vectrtres{0}{0}{0}
\]

\[
\left(
\begin{array}{rrr|r}
3&1&1&0\\
2&2&3&0\\
-1&1&2&0
\end{array}
\right)
\sim
\left(
\begin{array}{rrr|r}
3&1&1&0\\
0&4&7&0\\
0&4&7&0
\end{array}
\right)
\sim
\left(
\begin{array}{rrr|r}
3&1&1&0\\
0&4&7&0\\
0&0&0&0
\end{array}
\right)
\]

Por lo tanto, este sistema posee infinitas soluciones, las cuales serán los vectores del conjunto núcleo de A, el se expresa como:
%%subespacio r3
\[Nu(A)=\llaves{\vectrtres{x_1}{x_2}{x_3}}
{\begin{array}{r}
3x_1+x_2+x_3=0\\
4x_2+7x_3=0\\
\end{array}}
\]

\end{ejemplo}

\newpage
\section{Imagen de una matriz}
\begin{dfn}[Imagen de una matriz]
Sea A una matriz $m\times n$, la imagen de A se define como el conjunto dado por:
~\\
\[Im(A)=\llaves{Y \in \rmm }{\exists X \in \rn,   AX=Y}\]
\end{dfn}


\begin{ejemplo}
Sea $A=
\left(
\begin{array}{rrr}
3&1&1\\
2&2&3\\
-1&1&2
\end{array}
\right)$ una matriz, determinar la imagen de la matriz A
 
De acuerdo a la definición, debemos resolver el sistema:

\[
\left(
\begin{array}{rrr|r}
3&1&1&a\\
2&2&3&b\\
-1&1&2&c
\end{array}
\right)
\sim
\left(
\begin{array}{rrr|r}
3&1&1&a\\
0&4&7&3b-2a\\
0&4&7&3c+a
\end{array}
\right)
\sim
\left(
\begin{array}{rrr|r}
3&1&1&a\\
0&4&7&3b-2a\\
0&0&0&3c+3a-3b
\end{array}
\right)
\]

%%subespacio r3
\[Im(A)=\llaves{\vectrtres{a}{b}{c}}
{\begin{array}{r}
c+a-b=0\\
\end{array}}
\]

\end{ejemplo}
