%espacios vectoriales

\chapter{Espacios Vectoriales}


\section{Espacios Vectoriales reales}
Previo al estudio de álgebra lineal, conocemos los espacios \rdos\ y \rtres, lo cual es una primera idea al hablar de vectores, los cuales son duplas o ternas de números reales. En estos espacios se pueden identificar dos principales operaciones, la suma entre dos vectores, y el producto de un número real por un vector. En \rtres, tales operaciones son:

\begin{align*}
\vectrtres{a_1}{a_2}{a_3}+\vectrtres{b_1}{b_2}{b_3}&=\vectrtres{a_1+b_1}{a_2+b_2}{a_3+b_3}\\
c\vectrtres{a_1}{a_2}{a_3}&=\vectrtres{ca_1}{ca_2}{ca_3}
\end{align*}

Estas operaciones hacen que \rtres\ sea un espacio vectorial. 

\begin{dfn}[Espacio vectorial real]
Un espacio vectorial real es una cuarteta $(V, \dobleK, \oplus, \odot)$ donde V es un conjunto no vacío en el cual se definen dos operaciones binarias, una binaria interna llamada suma o adición $\oplus$ y otra binaria externa llamada producto por escalar $\odot$, $\dobleK$ es el campo escalar, Los escalares son los elementos de $\dobleK$ y que cumplen con los diferentes axiomas. \index{Espacio Vectorial}
\end{dfn}

\begin{table}[htbp]
    \begin{center}
        \begin{tabular}{|l|l|}
            \hline
            \multicolumn{2}{|l|}{Operación Suma} \\
            \hline \hline
            $\forall v_1,v_2 \in V:\ (v_1 \oplus v_2 )\in V$ &[Cerradura de la Suma] \\ \hline
            $\forall v_1,v_2\in V : v_1\oplus v_2=v_2\oplus v_1  $&[Conmutatividad]  \\ \hline
            $\forall v_1,v_2,v_3\in V:(v_1\oplus v_2 )\oplus v_3=v_1\oplus (v_2\oplus v_3 )$&[Asociatividad] \\ 
            \hline
            $\exists n_V\in V\ ,\forall v\in V  :  v\oplus n_V=v$&  [Elemento neutro]  \\ \hline
            $\forall v\in V\ ,\exists v'\in V: v\oplus v'=n_V$&  [Elemento inverso]  \\ \hline \hline
            \multicolumn{2}{|l|}{Operación Producto Escalar} \\
            \hline \hline
            $\forall v \in V \ , \forall \alpha \in \mathbb{K} :\ (\alpha \odot v )\in V$& [Cerradura del Producto] \\ \hline
            $ \forall \alpha \in \mathbb{K}\ , \forall v_1,v_2\in V: \alpha \odot (v_1\oplus v_2 )=(\alpha \odot v_1)\oplus (\alpha \odot v_2) $&  [Distributividad]  \\ \hline
            $\forall \alpha ,\beta \in \mathbb{K}\ , \forall v\in V : (\alpha +\beta )\odot v=(\alpha \odot v)\oplus(\beta \odot v)  $&  [Distributividad]  \\ \hline
            $\forall \alpha ,\beta \in \mathbb{K}\ , \forall  v\in V : (\alpha  \beta )\odot v=\alpha \odot (\beta \odot v) $&  [Distributividad]  \\ \hline
            $\forall  v\in V    :   n_K \odot v=v $&  [Neutro multiplicativo]  \\ \hline
        \end{tabular}
        \caption{Axiomas de un Espacio Vectorial.}
        \label{Tabla: Sencilla}
    \end{center}
\end{table}

\newpage
\begin{trabajoautonomo}
Sea $(E, \oplus, \odot)$ un espacio vectorial real. Pruebe que
\begin{enumerate}
\item El elemento neutro $\mathbf{0}_E$ es único.
\item Para cada $e$ en $E$, el elemento $e'$ es único.
\item Para cada $e$ en $E$, $0 \odot e = \mathbf{0}_E$.
\item Para cada $e$ en $E$, $(-1) \odot e = e'$.
\item Para cada $\lambda$ en $\mathbb{R}$, $\lambda \odot \mathbf{0}_E = \mathbf{0}_E$.
\item Si $\alpha \odot e = \mathbf{0}_E$, entonces $\alpha = 0$ o $v=\mathbf{0}_E$.

\end{enumerate}
\end{trabajoautonomo}
\begin{ejemplo}

El espacio \rdos \ con las operaciones $(x,y)+(w,z) = (x+w,y+z)$ y la multiplicación por un escalar $\lambda$, $\lambda (x,y) = (\lambda x,\lambda y)$.

\end{ejemplo}

\begin{ejemplo}
El espacio de todos los polinomios de grado menor o igual que $n$, para un número natural $n$ fijo, con las operaciones usuales de suma de polinomios y de multiplicación de un polinomio por un número real.
\end{ejemplo}

\begin{ejemplo}
El espacio de todas las matrices $n \times m$ con las operaciones usuales de suma de matrices y de multiplicación de una matriz por un número real.(Aquí $n$ y $m$ son números naturales fijos y distintos de cero).


\end{ejemplo}

Veamos algunos ejemplos de espacios vectoriales conocidos:
\begin{itemize}
    \item Consideremos $V=\mathbb{R}^n$ y $\mathbb{K}=\mathbb{R}$, con la suma y producto por escalar usuales $V$ es un espacio vectorial sobre los Reales.
    \item $V=\mathcal{P}_n$ y $\mathbb{K}=\mathbb{R}$ Los polinomios con coeficientes reales de grado menor o igual a n, con las operaciones usuales de suma y producto por escalar.
    \item $V=\mathcal{M}_{mxn}$ y $\mathbb{K}=\mathbb{R}$ Las matrices m por n con entradas reales con las operaciones usuales de suma y producto por escalar forman un espacio vectorial sobre los reales
    \item $V=\mathbb{C}^n$ y $\mathbb{K}=\mathbb{R}$ Las n-tuplas con entradas complejas con la suma y el producto por escalar usuales en los complejos forman un espacio vectorial sobre el campo de los Complejos.
    \item  $V=\mathbb{B}^n$ y $\mathbb{K}=\mathbb{B}$ Las n-tuplas con entradas binarias con la suma y el producto por escalar binarios forman un espacio vectorial sobre el campo binario. Éste es un ejemplo de espacio vectorial discreto ya que la cardinalidad del conjunto es
    \item $V=\mathcal{C}[a,b]$ y $\mathbb{K}=\mathbb{R}$ Las funciones continuas en un intervalo [a, b] con las operaciones usuales de suma y producto por escalar de funciones forman un Espacio Vectorial sobre los Reales
\end{itemize}


\subsection{Algunos teoremas elementales sobre espacios vectoriales}
\begin{theorem}
Sea $(V,\oplus,\odot)$ un espacio vectorial sobre el campo $\dobleK$. Entonces $n_+\odot v=n_v$
\end{theorem}
Demostración:
~\\

Se conoce que $n_+ + n_+=n_+$. Entonces
\[(n_++n_+ )\odot v=n_+\odot v\]
\[(n_+\odot v)\oplus (n_+\odot v)=n_+\odot v\]
No olvidemos que$ (n_+\odot v)$ es un vector, y por lo tanto tiene un inverso aditivo $(n_+\odot v)'$
Sumando a ambos lados de la ecuación anterior este inverso aditivo tenemos.
\begin{align*}
    ((n_+\odot v)\oplus (n_+\odot v))\oplus (n_+\odot v)'&=(n_+\odot v)\oplus (n_+\odot v)'\\
    (n_+\odot v)\oplus ((n_+\odot v)\oplus (n_+\odot v)' )&=(n_+\odot v)\oplus (n_+\odot v)'\\
    (n_+\odot v)\oplus n_v&=n_v\\
    (n_+\odot v)&=n_v
\end{align*}


Si la dificultad de la notación en la demostración anterior influyó negativamente para su comprensión, recuerde que $ n_v$ es el vector neutro de V, y que $n_+$ es el escalar neutro. Para facilidad de comprensión relacione $n_+$ con el 0 de los números reales. Es decir la tesis de este teorema en el campo Real es $0\odot v=n_v$.



\begin{theorem}
Sea $(V,\oplus ,\odot )$ un espacio vectorial sobre el campo $\dobleK$. Entonces $\forall \alpha \in K,\alpha \odot n_v=n_v$
\end{theorem}

Por el axioma (iv) de espacios vectoriales tenemos que $n_v\oplus n_v=n_v$. Por lo tanto
\begin{align*}
    \alpha \odot (n_v\oplus n_v )&=\alpha \odot n_v\\
    (\alpha \odot n_v )\oplus (\alpha \odot n_v )&=\alpha \odot n_v
\end{align*}


Sea $ (\alpha \odot n_v )'$ el inverso aditivo de $(\alpha \odot n_v )$. Sumando este vector a ambos lados
\begin{align*}
    ((\alpha \odot n_v )\oplus (\alpha \odot n_v ))\oplus (\alpha \odot n_v )'&=(\alpha \odot n_v)\oplus (\alpha \odot n_v )'\\
    (\alpha \odot n_v )\oplus ((\alpha \odot n_v )\oplus (\alpha \odot n_v )' )&=(\alpha \odot n_v)\oplus (\alpha \odot n_v )'\\
    (\alpha \odot n_v )\oplus n_v&=n_v\\
    (\alpha \odot n_v )&=n_v
\end{align*}
               

\begin{theorem}
Sea $(V,\oplus ,\odot )$ un espacio vectorial sobre el campo R,  sea $v$ un vector de V y v su inverso aditivo. 
Entonces $(-1)\odot v=v'$
\end{theorem}

\begin{align*}
    (1+(-1))\odot v&=0\odot v\\
    (1\odot v)\oplus ((-1)\odot v)&=n_v
\end{align*}
Sumando $(1\odot v)'$ el inverso aditivo de $(1\odot v)$ y agrupando
\begin{align*}
    ((1\odot v)'\oplus (1\odot v))\oplus ((-1)\odot v)  &=(1\odot v)'\oplus n_v\\
    n_v\oplus ((-1)\odot v)&=(1\odot v)'\\
    (-1)\odot v&=v'
\end{align*}


\begin{theorem}

Sea $(V,\oplus ,\odot )$ un espacio vectorial sobre el campo $\dobler$. Sea $n_v$ el vector neutro de $V$.
$$\alpha \odot v=n_v   \rightarrow  (  \alpha =0   \vee    v=n_v   )$$
\end{theorem}
Demostración:~\\

Recordemos que la expresión $ p\rightarrow (q\vee r)$ es lógicamente equivalente a la expresión $(p\wedge \neg q)\rightarrow r$
\begin{align*}
    p\rightarrow(q\vee r)&\equiv  \neg p\vee (q\vee r)\\
    &\equiv  (\neg p\vee q)\vee r\\
    &\equiv  \neg (\neg p\vee q)\rightarrow r\\
    &\equiv  (p\wedge \neg q)\rightarrow r
\end{align*}

Por lo que el teorema es lógicamente equivalente a 
$$\alpha \odot v=n_v  \wedge  \alpha \neq 0  \rightarrow   v=n_v $$  
Luego tenemos que:

$$\alpha \odot v=n_v$$
Como $ \alpha \neq 0$ entonces existe $1/\alpha$ , el inverso multiplicativo de $\alpha$ . Multiplicando por $1/\alpha $ a ambos lados 
\begin{align*}
    \frac{1}{\alpha} \odot (\alpha \odot v)&=\frac{1}{\alpha} \odot n_v\\
    \left(\frac{1}{\alpha} \cdot \alpha \right)\odot v&=n_v\\
    1\odot v&=n_v\\
    v&=n_v\\
\end{align*}


~\\
~\\


\begin{theorem}[El teorema de la unicidad del neutro]
Sea $(V,\oplus ,\odot )$ un espacio vectorial sobre el campo $\dobleK$. Sea $ n_v$ vector neutro de $V$. Entonces $n_v$ es único en $V$.
\end{theorem}
Demostración:
~\\

Por contradicción. Supongamos que existen dos vectores neutros en $V$; $n_v$ y $\widetilde{n_v}$ tales que son diferentes $n_v\neq \widetilde{n_v}$
Por el axioma (iv)
$$n_v=n_v\oplus \widetilde{n_v} $$
Por conmutatividad
$$=\widetilde{n_v} \oplus n_v  $$
Por el axioma (iv)
$$=\widetilde{n_v} $$
Lo cual es una contradicción, de donde se concluye que el neutro es único.



\newpage
\section{Subespacios Vectoriales}
\begin{dfn}
Sea $(E, +, \odot)$ un espacio vectorial real y $S$ un subconjunto no vacío de $E$. Entonces se dice que $S$ es un subespacio vectorial de $E$, si con las operaciones heredadas de $E$, $(S, +, \odot)$ es también un espacio vectorial.
\index{Subespacio Vectorial}
\end{dfn}


\begin{ejemplo}
Sea $E = \rtres$ y $S = \llaves{(x, y, z) \in \rtres}{5x + 2y + z = 0}$ entonces $S$ es un subespacio de \rtres.
\end{ejemplo}

\begin{ejemplo}
Sea $E$ el espacio \mdosxdos \ de todas las matrices cuadradas $2 \times 2$ con las operaciones usuales de suma de matrices y de multiplicación por un escalar y $S$ el conjunto de las matrices diagonales, es decir las matrices de la forma \matrdxd{a & 0}{0 & b}, entonces $S$ es un subespacio de $E$.

\end{ejemplo}

\begin{ejemplo}
Sea $E$ el espacio \ptres \ de todos los polinomios de grado menor o igual a 3, con la suma y multiplicación por un escalar usuales. Sea $S$ el conjunto de los polinomios \pdos \ de grado menor o igual a 2, entonces $S$ es un subespacio de $E$.
\end{ejemplo}

\begin{ejemplo}

Sea $E$ un espacio vectorial real, $V$ y $W$ dos subespacios vectoriales de $E$, entonces $V + W =\llaves{a+b}{a \in V \, , \, b \in W}$ y $V \cap W = \llaves{a}{a \in V \cap W}$ son también subespacios vectoriales reales.
\end{ejemplo}



\subsection{Subespacio trivial}
Si tenemos un espacio Vectorial $V$, elijamos el subconjunto que s\'olo contiene al neutro
$W=\lbrace n_v\rbrace$:
~\\	
\begin{itemize}
 \item W es un subconjunto no vac\'io de V.
\item Como mostramos en la sección anterior, $W$ es siempre un espacio Vectorial (Denominado espacio trivial).
 \end{itemize} 
Por lo tanto $W$, el conjunto que s\'olo contiene al vector neutro de un espacio vectorial V es siempre un Subespacio Vectorial de $V$.
\subsubsection{Todo espacio es subespacio de sí mismo}
Ahora, del espacio vectorial V, consideremos al conjunto V (No, no es error de escritura)
\begin{itemize}
\item $V$ siempre es un subconjunto no vac\'io de $V$. (Recuerde que todo conjunto A es subconjunto de s\'i mismo).
\item $V$ es un espacio vectorial.
\end{itemize}

Por lo tanto cualquier espacio vectorial es subespacio vectorial de s\' mismo.

~\\
Todo espacio vectorial $V$ SIEMPRE tiene, por lo menos, dos subespacios vectoriales $V$ y $\lbrace n_v\rbrace$. Aquellos subespacios diferentes a los anteriores reciben el nombre de Subespacios Propios.
~\\
\subsection{Subespacios elementales}
\subsubsection{Subespacios de $\mathbb{R}^2$}
En $\mathbb{R}^2$, el espacio vectorial $bidimensional$ sobre el campo real; se distinguen, aparte de los subespacios propios, otros subespacios vectoriales:


\begin{itemize}
\item Toda recta que pasa por el origen es un subespacio vectorial de $\mathbb{R}^2$

\end{itemize}
\subsubsection{Subespacios de $\mathbb{R}^3$}
Y en $\mathbb{R}^3$, el espacio vectorial tridimensional sobre el campo real; se distinguen, aparte de los subespacios propios, otros subespacios vectoriales:

\begin{itemize}
\item Toda recta que pasa por el origen es un subespacio vectorial de $\mathbb{R}^3$
\item Todo plano que contiene al origen es un subespacio vectorial de $\mathbb{R}^3$
\end{itemize}

En principio para probar esto se necesitaría probar los 10 axiomas de la definición de espacio vectorial, sin embargo existe un teorema que nos ahorra bastante trabajo.


\newpage
\subsection{El teorema de caracterización del subespacio}

\begin{theorem}[Caracterizaci\'on de subespacio]
Sea $\left(V, \oplus, \odot\right)$ un espacio vectorial sobre el campo. Sea $W$ un subconjunto de $V$. $W$ es un subespacio vectorial de $V$ si y solo si:
	
	\begin{itemize}
	\item $W$ no es vac\'io
	\item $\forall w_1, w_2 \in W: w_1\oplus w_2 \in W$
	\item $\forall \alpha \in \mathbb{K} \forall w \in W: \alpha \odot w \in W$
	
	\end{itemize}

\end{theorem}

La prueba de este teorema se la dejamos al lector, indic\'andoles que deben probar que los restantes 8 axiomas de espacios vectoriales se satisfacen a partir de estos axiomas de cerradura y de la hip\'otesis que $W$ es un subconjunto de $V$, con las mismas operaciones de V
\\
\\
Con este teorema demostraremos las afirmaciones sobre los subespacios de $\mathbb{R}^2$ y 
$\mathbb{R}^3$.

\begin{theorem}
 Toda recta que pasa por el origen es un subespacio vectorial de \rdos

\end{theorem}
\begin{proof}

El enunciado algebraicamente se traduce en que $$\svrdos{x}{y}{y=mx}$$
es un subespacio vectorial de \rdos.
\begin{enumerate}


\item[(i)] \vectrdos{0}{0} $\in W$ ya que $0=m\cdot 0$, por lo tanto $W\neq \emptyset$ 

\item[(ii)] $\forall w_1, w_2 \in W: w_1=$ 
\vectrdos{x_1}{y_1} $w_2=$\vectrdos{x_2}{y_2}

\[
w_1+w_2=\mbox{\vectrdos{x_1}{y_1}}+\mbox{\vectrdos{x_2}{y_2}}=
\mbox{\vectrdos{x_1+x_2}{y_1+y_2}}
\]

para que el $vector\ suma$ pertenezca a W debe cumplir la condici{\'o}n de W:
Por hip{\'o}tesis $w_1 \in W $ por lo que $y_1=mx_1$. Asimismo $w_2 \in W$ implica que
$y_2=mx_2$

\[
\begin{array}{c}

y_1+y_2=mx_1+mx_2
\\
y_1+y_2=m\left(x_1+x_2\right)
\end{array}
\]
Por lo que el vector suma pertenece a $W$

\item[(iii)] $\forall \alpha \in \mbox{\dobler}, \forall w \in W: w=$\vectrdos{x}{y}
\[
\alpha\cdot w=\alpha\mbox{\vectrdos{x}{y}}=
\mbox{\vectrdos{\alpha x}{\alpha y}}
\]
Por hip\'otesis $w\in W$ por lo que $y=mx$

\[\begin{array}{c}
\alpha y=\alpha\left(mx\right)
\\
\alpha y=m\left( \alpha x\right)
\end{array}
\]

Por lo que el $vector\ producto\ \alpha w=$\vectrdos{\alpha x}{\alpha y} pertenece a $W$.

\end{enumerate}
Ya que hemos probado $\left( i\right)\, \left(ii\right), \left(iii\right),$ que son las condiciones en el teorema de caraterizaci\'on de subespacio, por el mismo teorema concluimos que $W$ es un subespacio de \rdos. Lo cual prueba que toda recta que pasa por el origen es un subespacio vectorial de \rdos.

\end{proof}


\begin{theorem}
En \rtres \ toda recta que pasa por el origen es un subespacio vectorial de \rtres.
\end{theorem}
\begin{proof}

Esto es equivalente a decir que el subconjunto $$ \svrtres{W}{x}{y}{z}{x=at, y=bt, z=ct ;t\in \dobler}$$ es un subespacio vectorial de \rtres.

\begin{enumerate}
\item[(i)] \vectrtres{0}{0}{0} $\in W \left( \mbox{cumple las tres condiciones} \right)$
por tanto $W$ no es vac\'io.

\item[(ii)] $\forall w_1, w_2 \in W$ $w_1=$\vectrtres{x_1}{y_1}{z_1} , $w_2=$\vectrtres{x_2}{y_2}{z_2}

\[
w_1+w_2=\mbox{\vectrtres{x_1}{y_1}{z_1}}+\mbox{\vectrtres{x_2}{y_2}{z_3}}
=\mbox{\vectrtres{x_1+x_2}{y_1+y_2}{z_1+z_2}}
\]
Tanto $w_1$ como $w_2$ son elementos de W, por lo que
\[
x_1=at_1,\ y_1=bt_1,\ z_1=ct_1
\]
\[
x_2=at_2,\ y_2=bt_2,\ z_2=ct_2
\]
De esto se obtiene que:
\[
\begin{array}{ccc}
x_1+x_2=at_1+at_2&y_1+y_2=bt_1+bt_2
&z_1+z_2=ct_1+ct_2
\\
x_1+x_2=a\left(t_1+t_2\right)&
y_1+y_2=b\left(t_1+t_2\right)&
z_1+z_2=c\left(t_1+t_2\right)
\end{array}
\]

Por tanto el vector $w_1+w_2$ pertenece a $W$.

\item[(iii)] $\forall \alpha \in \mbox{\doblek}
, \forall w \in W: w=$\vectrtres{x}{y}{z}
\[
\alpha w=\alpha \mbox{\vectrtres{x}{y}{z}}=
\mbox{\vectrtres{\alpha x}{\alpha y}{\alpha z}}\]

Por hip\'otesis, $x=at, y=bt, z=ct$:

\[
\begin{array}{ccc}
\alpha x=\alpha \left(at\right)&
\alpha y=\alpha \left(bt\right)&
\alpha z=\alpha \left(ct\right)
\\
\alpha x=a\left(\alpha t\right)&
\alpha y=b\left(\alpha t\right)&
\alpha z=c\left(\alpha t\right)
\end{array}
\]

\end{enumerate}
~\\
Por el teorema del subespacio, concluimos que $W$ es un subespacio 
vectorial de \rtres.

\end{proof}


\begin{theorem}[Intersección de subespacios vectoriales]
Sean $V_1$ y $V_2$ dos subespacios vectoriales de un espacio vectorial E, entonces $V_1 \cap V_2$ es un subespacio vectorial
\end{theorem}
\begin{proof}
~\\ \begin{enumerate}[i.]
\item El cero vector de E pertenece a ambos subespacios, por lo tanto también peertenece a la intersección.
\item Si $v_1$ y $v_2$ pertenecen a $V_1 \cap V_2$ entonces ambos pertenecen a $V_1$ y $V_2$. Debido a que $V_1$ es un subespacio vectorial, entonces $v_1 +v_2$ pertenece a $V_1$. Por el mismo argumento para $V_2$, se tiene que $v_1+v_2$ pertenece a $V_2$. Por lo tanto $v_1+v_2$ pertenece a $V_1\cap V_2$.
\item Si $v$ pertenece a  $V_1 \cap V_2$, entonces  pertenece a $V_1$ y a $V_2$.  Considere $\alpha \in \dobler$,  debido a que $V_1 y V_2$ son subespacios vectoriales entonces $\alpha v$ pertenece a $V_1$ y $\alpha v$ pertenece a $V_2$, es decir $\alpha v$ pertenece a $V_1 \cap V_2$.  
\end{enumerate}
\end{proof}

\section{Problemas}
\subsection{Espacio vectorial}
\begin{enumerate}[1.]
\item Sea $G=\llav{g}$ un conjunto de un elemento en el cual se definen las siguientes operaciones:
\begin{align*}
\forall g \in G\ \  &g+g=g\\
\forall \alpha\in \dobler\ \ &\alpha g=g \\
\end{align*}
Determine si $G$, con las operaciones definidas anteriormente es un espacio vectorial.

\item Muestre que el conjunto de funciones continuas períodicas, con periodo múltiplo entero de $\pi$, forman un espacio vectorial sobre el campo real son la suma y producto por escalar convencionales de las funciones.


\begin{prob}[(1ra Evaluacion Septiembre 2013)]
(10 puntos) Considere el espacio vectorial real:
\[V=\llaves{a+bx \in P_1}{a+b=5}\]
Con las operaciones:
 \[(a_1+b_1 x)\oplus(a_2+b_2 x)=(a_1+a_2+4)+(b_1+b_2-9)x\]
 \[ \forall \alpha \in \dobler:  \alpha\odot (a+b x)=(\alpha a-4+4\alpha)+(\alpha b+9-9\alpha)x\]
 a)Demuestre que la operaci\'on suma es cerrada en V~\\
 b)Encuentre el vector nulo $0_v$ de $V$ y el vector inverso aditivo del vector $u=2+3x$
\end{prob}

\begin{prob}
Sea V=$\dobler^+$, el conjunto de lo reales positivos. Pruebe que V es un espacio vectorial con la suma y el producto por escalar definidos como:
\begin{align*}
\forall u, v \in V: u\oplus v=uv\\
\forall \alpha \in \dobler \forall v \in V: \alpha\odot v =v^\alpha
\end{align*}
\end{prob}

\begin{prob}
Considere V=$\llaves{A\in M_{2x2}}{A\ es\ invertible}$, determinar si V es un espacio vectorial con las operaciones:
\begin{align*}
\forall A, B \in V: A\oplus B=B^T A^T\\
\forall \alpha \in \dobler \forall A \in V: \alpha\odot A =A
\end{align*}
\end{prob}


\begin{prob}
Considere V el espacio vectorial $\mathcal{M}_{3\times 3}$. Determine si el subconjunto $H=\llaves{A \in \mathcal{M}_{3\times 3}}{tr(A)=0} $ es un subespacio vectorial de V.
\end{prob}
\begin{prob}
Considere V el espacio vectorial $\mathcal{M}_{3\times 3}$. Determine si el subconjunto $H=\llaves{A \in \mathcal{M}_{3\times 3}}{tr(A)\neq0} $ es un subespacio vectorial de V.
\end{prob}
\begin{prob}
Considere V el espacio vectorial \pdos. Determine si el subconjunto $H=\llaves{p(x)\in \pdos}{p(1)=p(-2)} $ es un subespacio vectorial de V.
\end{prob}

~\\

\end{enumerate}
.
