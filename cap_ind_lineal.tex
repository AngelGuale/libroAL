%independencia lineal
\chapter{Conjuntos Linealmente Independientes}

\section{Dependencia e Independencia Lineal}
\begin{dfn}[Independencia Lineal]
Sean \kvect{v}{k} vectores de un espacio vectorial $V$. El conjunto $S=\conjvect{v}{k}$\ es linealmente independiente si y sólo si la ÚNICA manera de obtener al vector neutro de V como combinación lineal de los vectores de S es que todos los escalares de la combinación lineal sean cero.
\[\alpha_1 v_1+
\alpha_2 v_2+
\ldots+
\alpha_k v_k =
n_v
\Rightarrow\ 
\forall i \leq k;
\ \alpha_i=0
\] 
\end{dfn}
\begin{dfn}[Dependencia Lineal]
Sean \kvect{v}{k} vectores de un espacio vectorial $V$. El conjunto $S=\conjvect{v}{k}$\ es linealmente dependiente si y sólo si se puede obtener al vector neutro de V como combinación lineal de los vectores de S, y existe un escalar de la combinación lineal diferente de cero.
\[\alpha_1 v_1+
\alpha_2 v_2+
\ldots+
\alpha_k v_k =
n_v
\wedge\ 
\exists i \leq k;
\ \alpha_i\neq 0
\] 
\end{dfn}
En otras palabras, un conjunto es linealmente dependiente si y solo si NO  es linealmente independiente.

~\\
Esta es, quiz\'as, una de las definiciones más confusas o menos inteligibles(pero a la vez una de las más importantes), as\'i que confiamos aclararla en los ejemplos siguientes.
~\\
\begin{ejemplo}
Determine si el conjunto de vectores de \rdos\  es linealmente independiente $\left\{
\vectrdos{1}{2}, \vectrdos{1}{-1}\right\}$.
\end{ejemplo}

%\sol
Para determinar si son linealmente independientes analicemos los escalares al hacer la combinación lineal igualada al neutro de \rdos
\[c_1\vectrdos{1}{2}+
c_2\vectrdos{1}{-1}=
\vectrdos{0}{0}\]
Esto deriva en un sistema de ecuaciones
$$\reducir{rr|c}{1&1&0\\2&-1&0}
\underrightarrow{2f_1 - f_2}
\reducir{rr|c}{1&1&0\\0&3&0}$$
La solución de este sistema es única $c_1=0,\ c_2=0$. El conjunto $\left\{
\vectrdos{1}{2}, \vectrdos{1}{-1}\right\}$ es linealmente independiente.


\begin{ejemplo}
Determine si el siguiente conjunto de vectores de \rtres\ es linealmente independiente \llav{
\vectrtres{2}{4}{1}, \vectrtres{-1}{2}{3}, \vectrtres{3}{2}{-2}
}
\end{ejemplo}
%\sol
Para determinar si el conjunto es linealmente independiente analicemos los escalares al hacer la combinación lineal igualada al neutro de \rtres.
\[
a_1\vectrtres{2}{4}{1}+
a_2\vectrtres{-1}{2}{3}+
a_3\vectrtres{3}{2}{-2}=
\vectrtres{0}{0}{0}
\]
Lo que deriva en un sistema homogéneo (Un sistema homogéneo es un sistema que está igualado todo a cero).
$$\reducir{rrr|r}{2&-1&3&0\\4&2&2&0\\1&3&-2&0}
\underrightarrow{\begin{array}{r}
    -2f_1 + f_2\\
    f_1 - 2f_3
\end{array}}
\reducir{rrr|r}{2&-1&3&0\\0&4&-4&0\\0&-7&7&0}
\underrightarrow{7f_2 + 4f_3}
\reducir{rrr|r}{2&-1&3&0\\0&4&-4&0\\0&0&0&0}$$

\obs\ Este es el "Talón de Aquiles" de muchos; es claro que el sistema tiene infinitas soluciones ya que la última fila se eliminó y nos quedó un sistema con menos ecuaciones que incógnitas, esto indica claramente que existen infinitas maneras de obtener al vectores neutro como combinación lineal de esos tres vectores, analice lo siguiente:

~\\
\textsl{Razonamiento Incorrecto: 
Como tiene infinitas soluciones con $a_1=0, a_2=0, a_3=0$\ se obtiene al vector neutro, por lo tanto el conjunto de vectores es linealmente independiente}

~\\
Lo anterior es erróneo en el sentido de afirmar que los vectores son linealmente independientes,
vayamos a la definición de independencia lineal para notar que lo esencial de la definición está en que DEBE SER LA ÚNICA FORMA DE OBTENER AL NEUTRO CUANDO TODOS LOS ESCALARES SEAN IGUALES A CERO.

~\\
En el ejemplo anterior no es la única de obtenerlo, hay infinitas, por ejemplo con $a_1=-1, a_2=1, a_3=1$
\[-1\cdot\vectrtres{2}{4}{2}+
1\cdot\vectrtres{-1}{2}{-2}+
1\cdot\vectrtres{3}{2}{-2}=
\vectrtres{0}{0}{0}
\]
No es la única manera cuando todos los escalares son ceros. Luego, el conjunto NO es linealmente independiente, es decir es linealmente dependiente.
\section{?`No hay una manera más sencilla?: Reducción por filas}
Es posible determinar si un conjunto es linealmente independiente colocándolos como filas de una matriz y reduciendo la matriz por el método de Reducción por filas de Gauss.

~\\
Veamos los siguientes ejemplos:
~\\

\begin{ejemplo}
Determine si el subconjunto de \rtres, $S=\llav{\vectrtres{4}{2}{3},\vectrtres{1}{3}{4}, \vectrtres{2}{-4}{-5}}$ es linealmente independiente.
\end{ejemplo}


%\sol

Empleamos el método de Reducción por filas de Gauss:
~\\
Colocamos los vectores como filas de una matriz las cuales vamos a reducir por el método de Gauss

\[\reducir{rrr}{
4&2&3\\
1&3&4\\
2&-4&-5
}\]
Reduciendo:
\[
\reducir{rrr}{
4&2&3\\
1&3&4\\
2&-4&-5
}
\underrightarrow{\begin{array}{c}
    f_1 - 4f_2\\
    f_1 - 2f_3 
\end{array}}
\reducir{rrr}{
4&2&3\\
0&-10&-13\\
0&10&13
}
\underrightarrow{f_2 - f_3}
\reducir{rrr}{
4&2&3\\
0&-10&-13\\
0&0&0
}
\]

Debido a que una de las filas, al emplear la Reducción, resultó llena de ceros, entonces el vector correspondiente a esa fila $\vectrtres{2}{-4}{-5}$ es combinación lineal de los vectores anteriores.
El conjunto es linealmente dependiente.


\begin{ejemplo}
Determine la independencia lineal del siguiente conjunto de \ptres, 
$T=\llav{2+x^3, x^2+4x^3, 1+3x+x^2}$
\end{ejemplo}

%\sol

De nuevo, debemos colocarlos a cada vector como una fila de una matriz, respetando el orden de los coeficientes:

\[
\reducir{rrrr}{
2&0&0&1\\
0&0&1&4\\
1&3&1&0
}
\underrightarrow{f_1 - 2f_3}
\reducir{rrrr}{
2&0&0&1\\
0&0&1&4\\
0&-6&-2&1
}
\underrightarrow{f_2 \longleftrightarrow f_3}
\reducir{rrrr}{
2&0&0&1\\
0&-6&-2&1\\
0&0&1&4
}
\]

Ya con la matriz totalmente reducida, vemos que ninguna fila se anul\'o, por lo tanto el conjunto T es linealmente independiente
\section{El Criterio del Determinante}
A\'un se puede mejorar el método de reducción por filas de Gauss, si en la matriz al colocar los vectores como filas(en realidad, aqu\'i funciona inclusive si los coloca como columnas), resulta una matriz cuadrada, entonces es posible determinar la independecia lineal del conjunto por medio del determinante de la matriz

\subsubsection{El criterio del determinante}
\begin{theorem}
Sea $A$ la matriz que se forma al colocar los vectores de un conjunto S como filas(o columnas), si $A$ es cuadrada entonces se cumple que 
\begin{itemize}
\item S es linealmente independiente si y solo si $det(A)\neq 0$
\item S es linealmente dependiente si y solo si $det(A)=0$
\end{itemize}

\end{theorem}


\begin{ejemplo}
Determine si el conjunto de vectores $S$ de \pdos, $S=\llav{1-x, 1+x, x^2+2x-1}$\ es linealmente independiente.
\end{ejemplo}

%\sol
Al colocar los vectores como filas, se forma la matriz
\[A=
\left(
\begin{array}{rrr}
1&-1&0\\
1&1&0\\
-1&2&1
\end{array}
\right)
\]
Ya que la matriz es CUADRADA entonces es posible aplicar el criterio del determinante, así podemos determinar la independencia lineal:
\[det(A)=1(1-0)-(-1)(1-0)=1+1=2\neq 0\]
Debido a que el determinante de la matriz es diferente de cero entonces el conjunto S es linealmente independiente.
~\\
~\\



\section{Problemas}
\begin{enumerate}


\item
Determine el valor de $k$ para que el conjunto S sea linealmente dependiente, donde $$S=\left\{\matrdxd{-1&2}{1&2}, \matrdxd{3&-1}{-2&0}, \matrdxd{-2&0}{3&1},\matrdxd{2k&1}{-3&k}\right\}$$

%\item
%Demuestre:
%\\Sea $S=\conjvect{v}{n}$ un subconjunto linealmente independiente de vectores del espacio vectorial $V$ y sea $x$ un vector de $V$ que no puede ser expresado como una combinación lineal de los vectores de S, entonces $\llav{\kvect{v}{n}, x}$
%también es linealmente independiente.

\item
Sea $p(x)= x^2+2x-3$, $q(x)=2x^2-3x+4$, y $r(x)=ax^2-1$. El conjunto $\{p, q, r\}$ es
linealmente dependiente si $a=\_?\_$.

\item
Sean $f_1(x) = \sen(x)$, $f_2(x) = \cos\left(x+\frac{\pi}{6}\right)$, y $f_3(x) = \sen\left(x-\frac{\pi}{4}\right)$ para $0 \leq x \leq 2\pi$. Muestre que 
$\llav{f_1(x), f_2(x), f_3(x)}$ es linealmente dependiente.


\item
Sean $a, b, c$ números reales distintos. Pruebe que los vectores \llav{\vectrtres{1}{1}{1}, \vectrtres{a}{b}{c}, \vectrtres{a^2}{b^2}{c^2}} forman un conjunto linealmente independiente en \rtres.



\begin{prob}[]
Sea V el espacio vectorial definido por:
\[V=\llaves{\vectrtres{a}{b}{c}}{a>0; b, c\in \dobler}\]
donde
\[ \vectrtres{a_1}{b_1}{c_1}\oplus\vectrtres{a_2}{b_2}{c_2}=\left(
\begin{array}{c}
a_1a_2\\
b_1+b_2-3\\
c_1+c_2+1
\end{array}
\right)
\]
\[\alpha\odot\vectrtres{a}{b}{c}=\left(
\begin{array}{c}
a^\alpha\\
\alpha b-3\alpha+3\\
\alpha c+\alpha-1
\end{array}
\right)\]
a) Determine el vector neutro de $V$
~\\b) Determine el inverso de $\vectrtres{2}{1}{-3}$
~\\c) Determine si los vectores $\vectrtres{4}{-1}{3}$ y $\vectrtres{2}{1}{1}$ forman un conjunto linealmente independiente.

\end{prob}


\begin{prop}[]

Determine el valor de $c$ para que el conjunto $$S=\left\lbrace  \vectrtres{1}{1}{c}, \vectrtres{c}{-1}{c}, \vectrtres{c}{1}{1}\right\rbrace $$ sea linealmente dependiente.
\end{prop}
~\\
\sol
~\\
Para verificar que un conjunto es linealmente dependiente se puede aplicar la reducción por filas de Gauss, o el criterio del determinante si la matriz es cuadrada.
~\\~\\
Aquí usaremos el criterio del determinante para resolver este problema.
~\\
Al colocar los vectores en fila obtenemos la matriz:
~\\
 $$A=\left(\begin{array}{rrr}
1&1&c\\
c&-1&c\\
c&1&1\\
\end{array}\right)$$
Para que los vectores sean linealmente dependientes el determinante de esta matriz debe ser igual a cero. Es decir:~\\
\[det(A)=0\]
\[(-1-c)-1\times(c-c^2)+c(c+c)=0\]
\[(-1-c)-c+c^2+2c^2=0\]
\[3c^2-2c-1=0\]
\[(c-1)(3c+1)=0\]
\[c=1 \vee c=-\frac{1}{3}\]

Para que el conjunto sea linealmente dependiente $c$ debe ser 1 o $-\frac{1}{3}$


\begin{prob}[]

Si $\{v_1, v_2, v_3\} $ es un conjunto linealmente independiente en un espacio vectorial $V$, determine de ser posible el valor de $a$ para que $$\{av_1-v_2, v_1+av_2-3v_3, 3v_1 -4v_2+5v_3 \}$$ sea linealmente dependiente en V.
\end{prob}


%\begin{prob}[]

%Si $\{v_1, v_2, v_3\} $ es un conjunto linealmente independiente en un espacio vectorial $V$, determine si $$\{v_1-v_2+v_3, v_2-v_3, v_1 +v_3 \}$$ es una base de V.
%\end{prob}


\end{enumerate}

%%%%%%%%%%%%%%%%%%%%%%%%%%%%%%
%\section{Conjuntos linealmente independientes}
%El subconjunto $A= \conjvect{v}{n}$ de un espacio vectorial $V$ se dice que es linealmente independiente si la combinación lineal $\lambda_1 v_1 + \lambda v_2 + \ldots + \lambda v_n = 0$ tiene como única solución la trivial, es decir que $\lambda_1 = \lambda_2 = \ldots = \lambda_n = 0$. En caso contrario se dice que $A$ es linealmente dependiente.
%\index{Conjunto linealmente independiente}
%
%\begin{ejemplo}
%Determinar si el conjunto de vectores $A= \{(1,2,3),(-2,1,1),(8,6,10)\}$ es linealmente independiente o no.\\
%Planteamos la igualdad 
%$$\alpha (1,2,3) + \beta (-2,1,1) + \lambda (8,6,10) = (0,0,0)$$
%de donde se obtiene que 
%$$\left\{
%\begin{array}{rcl}
%\alpha - 2\beta +8 \lambda & = & 0\\
%2 \alpha + \beta + 6\lambda & = & 0\\
%3\alpha + \beta + 10 \lambda & = & 0
%\end{array}
%\right.$$
%Al hacer el análisis del sistema obtenemos que existen infinitas soluciones. Así que el conjunto $A$ es linealmente dependiente.
%\end{ejemplo}
%
%\begin{ejemplo}
%Veamos que el conjunto $A= \{(3,-2,2),(3,-1,4),(1,0,5)\}$ es linealmente independiente.\\
%Consideremos $$\alpha (3,-2,2) + \beta (3,-1,4) + \lambda (1,0,5) = (0,0,0)$$
%y obtenemos
%$$\left\{
%\begin{array}{rcl}
%3 \alpha + 3 \beta + \lambda & = & 0\\
%-2 \alpha - \beta & = &0\\
%2 \alpha +4 \beta +5 \lambda & = & 0
%
%\end{array}
%\right.$$
%Al reducir de forma escalonada la matriz
%$$\left(
%\begin{array}{rrr}
%3 & 3 & 1\\
%-2 & -1 & 0\\
%2 & 4 & 5
%\end{array}
%\right)\\
%\sim \\
%\left(
%\begin{array}{rrr}
%1 & 0 & 0\\
%0 & 1 &0\\
%0 &0 &1
%\end{array}
%\right)\\$$
%Entonces el sistema tiene solución y esta es la trivial $\alpha = \beta = \lambda = 0$.
%\end{ejemplo}
%Recordemos que las matrices reales $2 \times 2$ forman un espacio vectorial.
%
%\begin{ejemplo}
%
%Veamos que $H = \left\{ \matrdxd{-1 & 2}{0 & 1}, \matrdxd{2 & 1}{1 & 1} , \matrdxd{0 & 2}{-1 & 0} \right \}$ es linealmente independiente.\\
%Consideremos la igualdad
%$$\alpha \matrdxd {-1 & 2}{0&1} + \beta \matrdxd {2&1}{1&1} + \lambda \matrdxd {0 & 2}{-1 & 0} = \matrdxd {0&0}{0&0}$$
%de donde se obtiene
%$$\left\{
%\begin{array}{rcl}
%-\alpha + 2\beta &=&0\\
%2\alpha +\beta 2\lambda &=&0\\
%\beta - \lambda &=& 0\\
%\alpha + \beta &=&0
%
%\end{array}
%\right.$$
%y obtenemos que la única solución es $\alpha = \beta = \lambda = 0$.\\
%\end{ejemplo}
