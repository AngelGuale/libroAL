%independencia lineal
\chapter{Conjuntos linealmente independientes}

\section{Conjuntos linealmente independientes}
El subconjunto $A= \conjvect{v}{n}$ de un espacio vectorial $V$ se dice que es linealmente independiente si la combinación lineal $\lambda_1 v_1 + \lambda v_2 + \ldots + \lambda v_n = 0$ tiene como única solución la trivial, es decir que $\lambda_1 = \lambda_2 = \ldots = \lambda_n = 0$. En caso contrario se dice que $A$ es linealmente dependiente.
\index{Conjunto linealmente independiente}

\begin{ejemplo}
Determinar si el conjunto de vectores $A= \{(1,2,3),(-2,1,1),(8,6,10)\}$ es linealmente independiente o no.\\
Planteamos la igualdad 
$$\alpha (1,2,3) + \beta (-2,1,1) + \lambda (8,6,10) = (0,0,0)$$
de donde se obtiene que 
$$\left\{
\begin{array}{rcl}
\alpha - 2\beta +8 \lambda & = & 0\\
2 \alpha + \beta + 6\lambda & = & 0\\
3\alpha + \beta + 10 \lambda & = & 0
\end{array}
\right.$$
Al hacer el análisis del sistema obtenemos que existen infinitas soluciones. Así que el conjunto $A$ es linealmente dependiente.
\end{ejemplo}

\begin{ejemplo}
Veamos que el conjunto $A= \{(3,-2,2),(3,-1,4),(1,0,5)\}$ es linealmente independiente.\\
Consideremos $$\alpha (3,-2,2) + \beta (3,-1,4) + \lambda (1,0,5) = (0,0,0)$$
y obtenemos
$$\left\{
\begin{array}{rcl}
3 \alpha + 3 \beta + \lambda & = & 0\\
-2 \alpha - \beta & = &0\\
2 \alpha +4 \beta +5 \lambda & = & 0

\end{array}
\right.$$
Al reducir de forma escalonada la matriz
$$\left(
\begin{array}{rrr}
3 & 3 & 1\\
-2 & -1 & 0\\
2 & 4 & 5
\end{array}
\right)\\
\sim \\
\left(
\begin{array}{rrr}
1 & 0 & 0\\
0 & 1 &0\\
0 &0 &1
\end{array}
\right)\\$$
Entonces el sistema tiene solución y esta es la trivial $\alpha = \beta = \lambda = 0$.
\end{ejemplo}
Recordemos que las matrices reales $2 \times 2$ forman un espacio vectorial.

\begin{ejemplo}

Veamos que $H = \left\{ \matrdxd{-1 & 2}{0 & 1}, \matrdxd{2 & 1}{1 & 1} , \matrdxd{0 & 2}{-1 & 0} \right \}$ es linealmente independiente.\\
Consideremos la igualdad
$$\alpha \matrdxd {-1 & 2}{0&1} + \beta \matrdxd {2&1}{1&1} + \lambda \matrdxd {0 & 2}{-1 & 0} = \matrdxd {0&0}{0&0}$$
de donde se obtiene
$$\left\{
\begin{array}{rcl}
-\alpha + 2\beta &=&0\\
2\alpha +\beta 2\lambda &=&0\\
\beta - \lambda &=& 0\\
\alpha + \beta &=&0

\end{array}
\right.$$
y obtenemos que la única solución es $\alpha = \beta = \lambda = 0$.\\
\end{ejemplo}
