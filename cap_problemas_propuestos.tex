\chapter{Problemas}

\begin{enumerate}

\begin{prob}[(3er Examen ESPOL 2012)]
Sea $V=C[0,1]$ y $f, g \in V$. Demuestre que si el conjunto \llav{f, g} es linealmente dependiente entonces
\[W=
\begin{vmatrix}
f(x)&g(x)\\
f'(x)&g'(x)\\
\end{vmatrix}=0
\]
\end{prob}
\sol
Si \llav{f, g} es un conjunto linealmente dependiente, entonces existe un escalar $k\in \dobler$ tal que $g(x)=kf(x)$:

\[\Rightarrow\
W=
\begin{vmatrix}
f(x)&g(x)\\
f'(x)&g'(x)\\
\end{vmatrix}
=
\begin{vmatrix}
f(x)&kf(x)\\
f'(x)&kf'(x)\\
\end{vmatrix}
=
kf(x)f'(x)-kf(x)f'(x)=0
\ \blacksquare\]

\begin{prob}[(1er Examen ESPOL 2012)]
Demuestre:
\\Sea $S=\conjvect{v}{n}$ un subconjunto linealmente independiente de vectores del espacio vectorial $V$ y sea $x$ un vector de $V$ que no puede ser expresado como una combinación lineal de los vectores de S, entonces $\llav{\kvect{v}{n}, x}$
también es linealmente independiente.
\end{prob}

\sol
Tenemos que demostrar que $\llav{\kvect{v}{n}, x}$ es l.i.
Consideremos que

\begin{equation}\label{eq1}
\alpha_1 v_1+
\alpha_2 v_2+
\ldots+
\alpha_n v_n+
\beta x =0_v
\end{equation}
Para algunos escalares $ \alpha_i, \beta \in \dobler; i=1...n$
Al operar esta ecuación tenemos lo siguiente
\begin{equation}\label{eq2}
\beta x=(-\alpha_1) v_1+
(-\alpha_2) v_2+
\ldots+
(-\alpha_n) v_n
\end{equation}
Existen dos casos, que $\beta\neq 0$ o que $\beta=0$
~\\
\underline{Si $\beta\neq 0$}:\ 
Si en \ref{eq2}, multiplicamos por $\frac{1}{\beta}$ (estamos seguro de que existe, ya que $\beta$ es diferente de cero) obtendremos lo siguiente:
\begin{equation*}
x=(-\frac{\alpha_1}{\beta}) v_1+
(-\frac{\alpha_2}{\beta}) v_2+
\ldots+
(-\frac{\alpha_n}{\beta}) v_n
\end{equation*} 
Lo anterior es imposible ya que contradice la hipótesis de que x no puede escribirse como combinación lineal de los vectores de S, as\'i que se concluye que $\beta=0$.
Ya conociendo que $\beta=0$, en \ref{eq1} reemplazamos y tenemos
\begin{equation}
\alpha_1 v_1+
\alpha_2 v_2+
\ldots+
\alpha_n v_n=0_v
\end{equation}
Pero como S es un conjunto l.i. implica que $\alpha_i=0; i=1,...n$
Con esto se concluye que el conjunto $\llav{\kvect{v}{n}, x}$ es l.i.$\blacksquare$

\begin{prob}[(1er Examen ESPOL 2009)]
Sean $u=\vectrdos{x_1}{y_1}, v=\vectrdos{x_2}{y_2}, w=\vectrdos{x_3}{y_3}$ elementos de $ \rdos$ y sea
\[A=
\left(
\begin{array}{rrr}
x_1&y_1&1\\
x_2&y_2&1\\
x_3&y_3&1
\end{array}
\right)
\]
\begin{enumerate}
\item[(a)] Si u, v, w son colineales ?`cuál es el rango de A?
\item[(b)] Determine una base para el espacio fila de A.
\end{enumerate}
\end{prob}

\sol
(a) Si u, v, w son colineales, entonces deben ser m\'ultiplos
 $$v=\vectrdos{a x_1}{a y_1}, w=\vectrdos{b x_1}{b y_1}$$
por lo que la matriz A ser\'ia
\[A=
\left(
\begin{array}{rrr}
x_1&y_1&1\\
a x_1&a y_1&1\\
b x_1&b y_1&1
\end{array}
\right)
\]
\begin{eqnarray*}
C_A &=gen\llav{
\vectrtres{x_1}{ax_1}{b x_1}, 
\vectrtres{y_1}{a y_1}{b y_1}, 
\vectrtres{1}{1}{1}
}\\&=
gen\llav{
\vectrtres{1}{a}{b}, 
\vectrtres{1}{a}{b}, 
\vectrtres{1}{1}{1}
}\\&=
gen\llav{
\vectrtres{1}{a}{b}, 
\vectrtres{1}{1}{1}
}
\end{eqnarray*}
De esta manera, es claro que $\rho(A)=1$ si $a=1 \wedge b=1$
en cualquier otro caso $\rho(A)=2$.
~\\
(b) Si $a=1 \wedge b=1$ entonces una base para el espacio fila ser\'ia $$B_{F_A}=\llav{
\vectrtres{x_1}{ y_1}{ 1}
}$$

Si $a=1 \wedge b\neq 1$ una base de $F_A$ ser\'ia
\[B_{F_A}=\llav{
\vectrtres{x_1}{ y_1}{ 1},
\vectrtres{b x_1}{b y_1}{1}
}\]

Si $a\neq 1 \wedge b=1$ una base de $F_A$ ser\'ia
\[B_{F_A}=\llav{
\vectrtres{x_1}{ y_1}{ 1},
\vectrtres{a x_1}{a y_1}{1}
}\]
\newpage
\begin{prob}
Dada la matriz $A=
\left(
\begin{array}{rrr}
1&10&1\\
2&2&2\\
-1&-8&k
\end{array}
\right)
$, determine los valores de $k$ para que la nulidad de $A$ sea cero.
\end{prob}

\sol

Por el criterio del determinante, la nulidad de A es cero si $det(A)\neq 0$
\[det(A)\neq 0\]
\[1(2k+16)-10(2k+2)+1(-16+2)\neq 0\]
\[-18k-18\neq 0\]
\[k\neq -1\].

\newpage
\begin{prob}[(1er Examen ESPOL 2007)]
Sea $A$ la matriz de coeficientes del sistema lineal
\begin{eqnarray*}
2x+y-z=a\\
x-y+2z=b\\
x+2y-3z=c
\end{eqnarray*}
\begin{enumerate}
\item[a)] Determine el espacio fila, n\'ucleo y recorrido de $A$
\item[b)] Si $c=2a+b$, determine si el vector \vectrtres{a}{b}{c} pertenece a $Im(A)$
\end{enumerate}
\end{prob}

\sol

De acuerdo al sistema se tiene que
\[A=
\left(
\begin{array}{rrr}
2&1&-1\\
1&-1&2\\
1&2&-3\\
\end{array}
\right)
\]
Espacio fila:
\[F_A=gen\llav{
\vectrtres{2}{1}{-1}, \vectrtres{1}{-1}{2}, \vectrtres{1}{2}{-3}
}\]
Lo que conduce al sistema

\[A=
\left(
\begin{array}{rrr|r}
2&1&1&x\\
1&-1&2&y\\
-1&2&-3&z\\
\end{array}
\right)
\sim
\left(
\begin{array}{rrr|r}
2&1&1&x\\
0&3&-3&x-2y\\
0&5&-5&x+2z\\
\end{array}
\right)
\sim
\left(
\begin{array}{rrr|r}
2&1&1&x\\
0&3&-3&x-2y\\
0&0&0&2x-10y-6z\\
\end{array}
\right)
\]
Por lo que
\[F_A=\llaves{\vectrtres{x}{y}{z}}{2x-10y-6z=0}\]

N\'ucleo:

El n\'ucleo es la solución del sistema homog\'eneo $AX=0$:
\[
\left(
\begin{array}{rrr|r}
2&1&-1&0\\
1&-1&2&0\\
1&2&-3&0\\
\end{array}
\right)
\sim
\left(
\begin{array}{rrr|r}
2&1&-1&0\\
0&3&-5&0\\
0&-3&5&0\\
\end{array}
\right)
\sim
\left(
\begin{array}{rrr|r}
2&1&-1&0\\
0&3&-5&0\\
0&0&0&0\\
\end{array}
\right)
\]

Por lo que:
\[Nu(A)=\llaves{\vectrtres{x_1}{x_2}{x_3}}
{\begin{array}{r}
2x_1+x_2-x_3=0\\
3x_2-5x_3=0
\end{array}}
\]

Recorrido o Imagen de A:

El recorrido es igual al espacio columna, el espacio generado por las columnas:

\[Rec(A)=Im(A)=C_A\]

\[
\left(
\begin{array}{rrr|r}
2&1&-1&x\\
1&-1&2&y\\
1&2&-3&z\\
\end{array}
\right)
\sim
\left(
\begin{array}{rrr|r}
2&1&-1&x\\
0&3&-5&x-2y\\
0&-3&5&x-2z\\
\end{array}
\right)
\sim
\left(
\begin{array}{rrr|r}
2&1&-1&x\\
0&3&-5&x-2y\\
0&0&0&2x-2y-2z\\
\end{array}
\right)
\]


Por lo que
\[C_A=\llaves{\vectrtres{x}{y}{z}}{2x-2y-2z=0}\]

b) Si $c=2a+b$ entonces para que el vector \vectrtres{a}{b}{c} pertenezca a $C_A$ debe cumplirse la condición de exige $C_A$
\[
2x-2y-2z=2(a)-2(b)-2(2a+b)=-2a-4b \neq 0
\]

Por tanto, no pertenece a $Im(A)$.

\newpage
\begin{prob}
Sea $V=\mathcal{M}_{3x2}$. Sean $W_1$ el conjunto de las matrices que tienen la primera y la última fila iguales, $W_2$ el conjunto de las matrices que tienen la primera columna igual a su segunda columna.
~\\Determine:~\\
\begin{enumerate}
\item[a)] La intersección entre $W_1, W_2$
\item[b)] La suma entre $W_1, W_2$
\item[c)] Una base para los subespacios intersección y suma obtenidos en (a) y (b).

\end{enumerate}
\end{prob}

\sol

De acuerdo a lo indicado tenemos que

\[W_1=\llaves{
\left(
\begin{array}{rr}
a_{11}&a_{12}\\
a_{21}&a_{22}\\
a_{31}&a_{32}
\end{array}
\right)
}{
\begin{array}{r}
a_{11}=a_{31}\\
 a_{12}=a_{32}
\end{array}}
\]


\[W_2=\llaves{
\left(
\begin{array}{rr}
a_{11}&a_{12}\\
a_{21}&a_{22}\\
a_{31}&a_{32}
\end{array}
\right)
}{
\begin{array}{r}
a_{11}=a_{12}\\
a_{21}=a_{22}\\
a_{31}=a_{32}
\end{array}
}
\]
Para la intersección tenemos que

\[W_1\cap W_2=\llaves{
\left(
\begin{array}{rr}
a_{11}&a_{12}\\
a_{21}&a_{22}\\
a_{31}&a_{32}
\end{array}
\right)
}{\begin{array}{rr}
a_{11}=a_{12}&a_{11}=a_{31}\\
a_{21}=a_{22}&a_{12}=a_{32}\\
a_{31}=a_{32}&
\end{array}
}
\]
Si colocamos todas estas ecuaciones en un sistema homog\'eneo e igualamos a cero tendr\'iamos

%\[
%  \begin{blockarray}{rrrrrrrr}
%&a_{11}&a_{12}&a_{21}&a_{22}&a_{31}&a_{32}&\ \\
%    \begin{block}{r(rrrrrr|r)}
%&1    &-1     &0     &0     &0     &0     &0\\
%&0    &0      &1     &-1    &0     &0     &0\\
%&0    &0      &0     &0     &1     &-1     &0\\
%&1    &0      &0     &0     &-1     &0     &0\\
%&0    &1     &0     &0     &0     &-1     &0\\
%    \end{block}
%  \end{blockarray}
%\]

\[
\left(
\begin{array}{rrrrrr|r}
1    &-1     &0     &0     &0     &0     &0\\
0    &0      &1     &-1    &0     &0     &0\\
0    &0      &0     &0     &1     &-1     &0\\
1    &0      &0     &0     &-1     &0     &0\\
0    &1     &0     &0     &0     &-1     &0\\
\end{array}
\right)
\sim
\ldots
\sim
\left(
\begin{array}{rrrrrr|r}
1    &-1     &0     &0     &0     &0     &0\\
0    &1     &0     &0     &0     &-1     &0\\
0    &0      &1     &-1    &0     &0     &0\\
0    &0      &0     &0     &1     &-1     &0\\
0    &0      &0     &0     &0     &0     &0\\
\end{array}
\right)
\]

Donde queda claro que había una ecuación redundante, por lo tanto


\[W_1\cap W_2=\llaves{
\left(
\begin{array}{rr}
a_{11}&a_{12}\\
a_{21}&a_{22}\\
a_{31}&a_{32}
\end{array}
\right)
}{\begin{array}{r}
a_{11}=a_{32}\\
a_{21}=a_{22}\\
a_{12}=a_{32}\\
a_{31}=a_{32}\\
\end{array}
}
\]
Para hallar una base:
\[\left(
\begin{array}{rr}
a_{32}&a_{32}\\
a_{22}&a_{22}\\
a_{32}&a_{32}
\end{array}
\right)
=a_{22}
\left(
\begin{array}{rr}
0&0\\
1&1\\
0&0
\end{array}
\right)
+a_{32}
\left(
\begin{array}{rr}
1&1\\
0&0\\
1&1
\end{array}
\right)
\]

De esta manera obtenemos una base para la intersección:


%
%\begin{prob}[(1er Examen ESPOL 2012)]
%Sea $T:\rdos \rightarrow \rdos $ 
%\end{prob}

\newpage
\begin{prob}[(3ra Evaluacion Septiembre 2012)]
(10 puntos)Dado la matriz $A=\left(\begin{matrix}
1&10&1\\
2&2&2\\
-1&-1&k\\
\end{matrix}\right)$
, determine los valores de k para que la nulidad sea cero.
\end{prob}

%
%\begin{prob}[(3ra Evaluacion Septiembre 2012)]
%(10 puntos)Sea $V=C[0,1]$ y $fm g \in V$. Demuestre que si el conjunto ${f, g}$ es linealmente dependiente entonces:
%\[W=\left|\begin{matrix}
%f(x)&g(x)\\
%f'(x)&g'(x)\\
%\end{matrix}\right|=0\]
%\end{prob}
%

\newpage
\begin{prob}[(3ra Evaluacion Febrero 2009)]
(10 puntos) Califique como V o F:Sea $A=\left(\begin{matrix}
2a&2b&-c\\
a&2b&-c\\
-a&-b&c\\
\end{matrix}\right)$ con a, b, c diferentes de cero, entonces el rango de A es igual a 3 y la nulidad de A es igual a 0.
\end{prob}

\newpage

\begin{prob}[(3ra Evaluacion Abril 2011)]
(10 puntos)Califique como Vo F: Sea $A=\mathcal{M}_{2x3}$. Entonces la nulidad de A es mayor o igual a 3,
\end{prob}



\newpage
\begin{prob}[(1ra Evaluacion Julio 2012)]
(20 puntos)Sea $V=\ptres$. Considere el conjunto de todos los subespacios de V tal que $$H(a)=gen\llav{1+ax+x^2+x^3, 1+ax+(1-a)x^2+x^3, x+(2a)x^2+2x^3, 1+(1+a)x+(1+a)x^2+3x^3}$$
a) Determine el valor de $a$ para que $dimH=2$ \\
b) Halle una base y la dimensión de los subespacios $H(0)\cap H(1)$ y $H(0)+ H(1)$
\end{prob}


\newpage

\newpage

%Megas 2do parcial~\\
%~\\
%\begin{prob}[(3ra Evaluacion 14 de febrero 2014)]
%(10 puntos) 
%a)Demuestre que $Nu(T)\subseteq Nu(T^{2})$
%~\\
%b) Si ademas $\rho(T)=\rho(T^{2})$ , demuestre que $Nu(T)= Nu(T^{2})$
%\end{prob}

%
%\begin{prob}[(2da Evaluacion X X)]
%(10 puntos) Verdadero o Falso: Sea $L:V->V$ un isomorfismo. Sea $B=\{v_1, v_2, v_3\}$
% una base del espacio vectorial V. Entonces $B2=\{L(v_1), L(v_2), L(v_3)\}$ es una base de V.
%\end{prob} 
%
%\begin{prob}[(2da Evaluacion X X)]
%(20 puntos) Encuentre de ser posible una matriz diagonal D semejante a:
%$A=\left(\begin{array}{rrr}
%-1&2&1\\
%0&-1&0\\
%-1&-3&-3\\
%\end{array}\right)$ 
%\end{prob} 
%
%
%\begin{prob}[(2da Evaluacion X X)]
%(10 puntos) Verdadero o Falso: La matriz $A=\left(\begin{array}{rr}
%k&0\\
%1&3\\
%\end{array}\right)$ es diagonalizable para todo k
%\end{prob}
%
%
%
%\begin{prob}[(2da Evaluacion X X)]
%(10 puntos) Construya de ser posible un operador lineal L en el espacio vectorial \pdos que cumpla las siguientes condiciones: $L(1+x)=-1-x y E(\lambda=3)=gen{2-x}.$ ¿Es diagonalizable este operador? Justifique su respuesta.
%\end{prob}
%
%

%\begin{prob}[(2da Evaluacion X X)]
%(15 puntos) Sea ai=(xi,yi) un vector que indica la posición de una partícula sobre un punto al cabo de t periodos, suponga que:
%$\forall t \in N a_t=\left(\begin{array}{rr}
%1/2&1\\
%0&1/3\\
%\end{array}\right)a_{t-1}$ 
%Si la posición inicial de la partícula es $a_o=(1,-1)$ Determine:
%~\\
%a)La posición de la partícula luego de T=2 periodos
%B)La posicion luego de t periodos
%c) La posicion en estado estable, es decir t=inf.
%
%\end{prob} 
%

%\begin{prob}[(2da Evaluacion X X)]
%(10 puntos) Sean A,B dos matrices semejantes:
%~\\a)Muestre que A y B tienen los mismos valores propios
%~\\b) Tienen los mismos vectores propios? Justifique se respuesta
%
%\end{prob} 
%\begin{prob}[(2da Evaluacion X X)]
%(10 puntos) Sea $A=\left(\begin{array}{rrr}
%1&0&k\\
%3&3&-3\\
%1&0&2\\
%\end{array}\right)$ 
%~\\a) Determine el valor de k para que $\lambda=5$ sea valor propio de A
%~\\b) Para k=20 determine si A es diagonalizable
%\end{prob} 
%
%
%\begin{prob}[(2da Evaluacion X X)]
%(10 puntos) Verdadero o Falso:
%~\\
%La funcion $f:\pdos x \pdos -> R$, definida por $f(p(x),q(x))=p(1)q(1)$ es un producto interno en \pdos
%
%\end{prob} 

%
%\begin{prob}[(2da Evaluacion X X)]
%(15 puntos) Sea (.,.) el producto interno real estandar en el espacio vectorial \rtres. Sea L un operador linel en \rtres
%\[L(x,y,x)=(x,x+y,x+y+z)\]
%Es tambien un producto interno real en \rtres la funcion $(.,.)_{L}$
%\[(v1,v2)_{L}=(L(v1),L(v2))\]
%Sea W un subespacio vectorial tal que
%\[W=gen\{(1,1,1),(-1,0,1)   \}\]
%a) Encuentre una base y determine el complemento ortogonal de W.
%~\\b)Expresar el vector (0,1,1) como la suma de un vector de W y un vector de $W*$
%
%\end{prob} 


%
%\begin{prob}[(2da Evaluacion X X)]
%(10 puntos) Aplicando diagonalizacion ortogonal grafique el lugar geométrico $5x^2+4xy+2y^2-24x-12y+29=0$
%\end{prob} 

\newpage
\subsubsection{Califique como verdaderas o falsas las siguientes proposiciones}
\begin{prop}[(1er Examen ESPOL 2012)]

Sea $V$ un espacio vectorial tal que $H\subseteq V$. Si $H$ es un subespacio vectorial de $V$ entonces $H^C$ es subespacio de $V$.

\end{prop}

\sol
~\\
Debido a que $H$ es un subespacio vectorial(también es espacio vectorial), el elemento neutro $0_v$ pertenece a H, pero esto implica que $0_v \notin H^C$.

~\\
Si $0_v \notin H^C$ entonces $H^C$ no puede ser espacio vectorial, luego, tampoco subespacio vectorial de $ V$. 
$\therefore$ La proposición es FALSA.

~\\
~\\
~\\
\begin{prop}[(1er Examen ESPOL 2009)]
Si $(V, \oplus, \odot)$ es un espacio vectorial, y sean $H$ y $W$ dos subespacios de V tales que:
$W=gen\llav{w_1, w_2, w_3}$ y $w_1, w_2  \in H$, entonces es cierto que $dim(H\cap W)=2$
\end{prop}

\sol
Considere el siguiente contraejemplo:
Sea $V={\cal P}_3$ y $W=gen\llav{1, x, x^2}$, y además $H=gen\llav{1, x, x^2}$ (Por supuesto, esto es intencional). Es claro que $1, x \in H$. Pero $dim(W\cap H)=3$.
$\therefore$ La proposición es FALSA.

\newpage
\begin{prop}[(1er Examen ESPOL 2007)]
Si la matriz B se obtiene a partir de la matriz A por medio de un intercambio de filas entonces $\rho(A)=\rho(B)$
\end{prop}

\sol
Sean $f_1, f_2, \ldots, f_i, \ldots , f_j, \ldots, f_m$ las filas de A, luego $$F_A=gen\llav{f_1, f_2, \ldots, f_i, \ldots , f_j, \ldots, f_m}$$. Si B se obtiene al intercambiar las filas $f_i, f_j$ entonces, 
$$F_B=gen\llav{f_1, f_2, \ldots, f_j, \ldots , f_i, \ldots, f_m}$$
$$F_B=gen\llav{f_1, f_2, \ldots, f_i, \ldots , f_j, \ldots, f_m}$$
$$F_B=F_A$$
Por lo tanto, $dimF_A=dim F_B$, esto es lo mismo que $\rho(A)=\rho(B)$.
$\therefore $ La proposición es VERDADERA.
~\\
~\\
~\\



\newpage


\newpage
\newpage


\begin{prob}[]
En el espacio vectorial \mdosxdos, determinar si 
\matrdxd{2&1}{-1&0}
 es combinación lineal de los vectores 
\matrdxd{1&0}{-1&0}, \matrdxd{-1&2}{0&1}, \matrdxd{0&1}{2&-1}.
\end{prob}

\newpage

\begin{prop}[Califique como verdadero o falso]
Sea $V$ un espacio vectorial. Sean A, B $\subseteq V$, entonces $gen(A\cap B)=gen(A)\cap gen(B)$
\end{prop}
\sol
Por contraejemplo:
~\\
Sea $V=\rtres$, además~\\
\[A=\left\lbrace \vectrtres{1}{0}{0}, \vectrtres{0}{1}{0} \right\rbrace\  ;  B=\left\lbrace \vectrtres{0}{0}{1}, \vectrtres{0}{-1}{0} \right\rbrace\]
Es evidente que:~\\
\[H=gen(A)=\llaves{\vectrtres{a}{b}{c}}{c=0}\]
\[W=gen(B)=\llaves{\vectrtres{a}{b}{c}}{a=0}\]


Si calculamos $H\cap W$ tendríamos:
~\\
\[H\cap W=\llaves{\vectrtres{a}{b}{c}}{a=0, c=0}\]

Ahora, si calculamos $gen(A \cup B)$, (A y B no tienen elementos en común)~\\
\[A\cap B=\phi \]
Con esto se tiene que
\[gen(A\cap B)=\lbrace n_v \rbrace\]~\\
Luego
\[gen(A\cap B)\neq gen(A)\cap gen(B)\]

$\therefore$ La proposición es falsa
\newpage






\newpage

%repetido
%\begin{prob}[]
%
%Sea $V=C[0,1]$ y $f, g \in V$. Demuestre que si el conjunto $\{f, g\}$ es linealmente dependiente entonces:
%\[W=\left|\begin{matrix}
%f(x)&g(x)\\
%f'(x)&g'(x)\\
%\end{matrix}\right|=0\]
%\end{prob}

\newpage



\newpage

%
%\begin{prob}
%Sea $A$ la matriz de coeficientes del sistema lineal
%\begin{eqnarray*}
%2x+y-z=a\\
%x-y+2z=b\\
%x+2y-3z=c
%\end{eqnarray*}
%\begin{enumerate}
%\item[a)] Determine el espacio fila, n\'ucleo y recorrido de $A$
%\item[b)] Si $c=2a+b$, determine si el vector \vectrtres{a}{b}{c} pertenece a $Im(A)$
%\end{enumerate}
%\end{prob}
%
%\sol
%
%De acuerdo al sistema se tiene que
%\[A=
%\left(
%\begin{array}{rrr}
%2&1&-1\\
%1&-1&2\\
%1&2&-3\\
%\end{array}
%\right)
%\]
%Espacio fila:
%\[F_A=gen\llav{
%\vectrtres{2}{1}{-1}, \vectrtres{1}{-1}{2}, \vectrtres{1}{2}{-3}
%}\]
%Lo que conduce al sistema
%
%\[A=
%\left(
%\begin{array}{rrr|r}
%2&1&1&x\\
%1&-1&2&y\\
%-1&2&-3&z\\
%\end{array}
%\right)
%\sim
%\left(
%\begin{array}{rrr|r}
%2&1&1&x\\
%0&3&-3&x-2y\\
%0&5&-5&x+2z\\
%\end{array}
%\right)
%\sim
%\left(
%\begin{array}{rrr|r}
%2&1&1&x\\
%0&3&-3&x-2y\\
%0&0&0&2x-10y-6z\\
%\end{array}
%\right)
%\]
%Por lo que
%\[F_A=\llaves{\vectrtres{x}{y}{z}}{2x-10y-6z=0}\]
%
%N\'ucleo:
%
%El n\'ucleo es la solución del sistema homog\'eneo $AX=0$:
%\[
%\left(
%\begin{array}{rrr|r}
%2&1&-1&0\\
%1&-1&2&0\\
%1&2&-3&0\\
%\end{array}
%\right)
%\sim
%\left(
%\begin{array}{rrr|r}
%2&1&-1&0\\
%0&3&-5&0\\
%0&-3&5&0\\
%\end{array}
%\right)
%\sim
%\left(
%\begin{array}{rrr|r}
%2&1&-1&0\\
%0&3&-5&0\\
%0&0&0&0\\
%\end{array}
%\right)
%\]
%
%Por lo que:
%\[Nu(A)=\llaves{\vectrtres{x_1}{x_2}{x_3}}
%{\begin{array}{r}
%2x_1+x_2-x_3=0\\
%3x_2-5x_3=0
%\end{array}}
%\]
%
%Recorrido o Imagen de A:
%
%El recorrido es igual al espacio columna, el espacio generado por las columnas:
%
%\[Rec(A)=Im(A)=C_A\]
%
%\[
%\left(
%\begin{array}{rrr|r}
%2&1&-1&x\\
%1&-1&2&y\\
%1&2&-3&z\\
%\end{array}
%\right)
%\sim
%\left(
%\begin{array}{rrr|r}
%2&1&-1&x\\
%0&3&-5&x-2y\\
%0&-3&5&x-2z\\
%\end{array}
%\right)
%\sim
%\left(
%\begin{array}{rrr|r}
%2&1&-1&x\\
%0&3&-5&x-2y\\
%0&0&0&2x-2y-2z\\
%\end{array}
%\right)
%\]
%
%
%Por lo que
%\[C_A=\llaves{\vectrtres{x}{y}{z}}{2x-2y-2z=0}\]
%
%b) Si $c=2a+b$ entonces para que el vector \vectrtres{a}{b}{c} pertenezca a $C_A$ debe cumplirse la condición de exige $C_A$
%\[
%2x-2y-2z=2(a)-2(b)-2(2a+b)=-2a-4b \neq 0
%\]
%
%Por tanto, no pertenece a $Im(A)$.
\newpage

%
%\begin{prob}[]
%
%Sean $A\in \mathcal{M}_{mxn}, B \in \mathcal{M}_{nxp}$, demuestre que $C_{AB}\subseteq C_A$
%
%\end{prob}



%
%\begin{prob}[]
%
%Sea $V=\ptres$. Considere el conjunto de todos los subespacios de V tal que $$H(a)=gen\llav{1+ax+x^2+x^3, 1+ax+(1-a)x^2+x^3, x+(2a)x^2+2x^3, 1+(1+a)x+(1+a)x^2+3x^3}$$
%Halle una base y la dimensión de los subespacios $H(0)\cap H(1)$ y $H(0)+ H(1)$
%\end{prob}


%\begin{prob}
%
%Sea $V=\mathcal{M}_{3x2}$. Sean $W_1$ el conjunto de las matrices que tienen la primera y la última fila iguales, $W_2$ el conjunto de las matrices que tienen la primera columna igual a su segunda columna.
%~\\Determine:~\\
%\begin{enumerate}
%\item[a)] La intersección entre $W_1, W_2$
%\item[b)] La suma entre $W_1, W_2$
%\item[c)] Una base para los subespacios intersección y suma obtenidos en (a) y (b).
%
%\end{enumerate}
%
%\end{prob}






\end{enumerate}


%%%%%%%%%%%%%%%%%%%%%%%%%%%%%%%%%%%%%%%%%%%%%%%%
%%%%%%%%%%%%%%%%%%%%%%%%%%%%%%%%%%%%%%%%%%%%%%%
%\section{Combinaciones lineales y subespacios generados}
%\begin{dfn}
%Sea $(E, +, \odot)$ un espacio vectorial real y sean $B=\conjvect{v}{n}$ un subconjunto de vectores en $E$. Un vector $v$ en $E$ se dice que es una combinación lineal de los vectores en $B$ si existen escalares $\alpha_1$, $\alpha_2$, $\ldots$ , $\alpha_n$ en \dobler , tales que $v=\av{\alpha_1}{v_1} + \av{\alpha_2}{v_2} + \ldots + \av{\alpha_n}{v_n}$.
%
%\end{dfn}
%
%\begin{ejemplo}
%Cualquier vector $(a, b, c)$ en \rtres puede ser expresado como una combinación lineal de los vectores  $(1,0,0)$,$(0,1,0)$ y $(0,0,1)$.
%
%\end{ejemplo}
%
%\begin{ejemplo}
%El vector $(4,5,5)$ puede ser expresado como una combinación lineal de los vectores $(1,2,3)$, $(-1,1,4)$ y $(3,3,2)$.
%\end{ejemplo}
%
%\begin{theorem}
%Sea $(E, +, \odot)$ un espacio vectorial y $A$ un subconjunto finito de $E$. El conjunto de todas las combinaciones lineales de elementos de $A$ es un subespacio de $E$ y se llama subespacio generado por $A$ y se denota por $CL(A)$.
%\end{theorem}
%
%\begin{proof}
%
%$\mathbf{0}_v \in CL(A)$ por el comentario anterior.\\
%Sean $u$, $v$ en $CL(A)$, entonces
%\begin{align*}
%u &=\av{\alpha_1}{v_1} + \ldots + \av{\alpha_n}{v_n}\\ 
%v&=\av{\beta_1}{v_1} + \ldots + \av{\beta_n}{v_n}
%\end{align*}
%entonces
%\begin{align*}
%u + v &= (\av{\alpha_1}{v_1} + \ldots + \av{\alpha_n}{v_n}) + (\av{\beta_1}{v_1} + \ldots + \av{\beta_n}{v_n})\\
%u + v &=(\alpha_1 + \beta_1)\,v_1 + \ldots + (\alpha_n + \beta_n)\, v_n
%\end{align*}
%Así
%$$u + v \in CL(A)$$
%Sea $u \in CL(A)$ y $\alpha \in \dobler$
%\begin{align*}
%\alpha u &= \alpha(\av{\alpha_1}{v_1} + \ldots + \av{\alpha_n}{v_n})\\
%\alpha u &= (\alpha \alpha_1)v_1 + \ldots + (\alpha \alpha_n)v_n
%\end{align*}
%luego $\alpha u \in CL(A)$\\
%
%$\therefore$ $CL(A)$ es un subespacio vectorial de $V$. \qedhere \\ %%%%%%%  el \qedhere es el QED
%
%\end{proof}
%El subespacio $CL(A)$ es llamado espacio generado por \conjvect{v}{n}. Si $V = CL(A)$ entonces decimos que $A$ genera al espacio $V$ o que $A$ es conjunto generador de $V$.
%
%\begin{ejemplo}
%Veamos que $A = \llav{(1,2,0),(0,1,-1),(1,1,2)}$ genera a \rtres .\
%En efecto veamos que cualquier vector $(x,y,z)$ es una combinación lineal de los elementos de $A$. Es decir debemos encontrara escalares $\alpha$, $\beta$ y $\lambda \in \dobler$ tales que $$(x_0, y_0, z_0) = \alpha (1,2,0) + \beta (0,1,-1) + \lambda (1,1,2)$$
%lo que nos conduce al estudio del sistema 
%
%$$\left\{
%\begin{array}{rcl}
%\alpha + \lambda &=& x_0\\
%2 \alpha + \beta + \lambda &=& y_0\\
%- \beta + 2\alpha &=& z_0 
%\end{array}
%\right.$$
%
%Donde se obtiene que 
%\begin{align*}
%\alpha &= 3x_0 - y_0 -z_0\\
%\beta &=-4 x_0 +2 y_0 +z_0\\
%\lambda &= -2 x_0 + y_0 +z_0 
%\end{align*}
%
%\end{ejemplo}

\begin{enumerate}
\begin{prob}[]
Demuestre:
Sea $B=\conjvect{v}{n}$ una base del espacio vectorial $V$, entonces cualquier conjunto de m\'as de n vectores es linealmente dependiente
\end{prob}

\end{enumerate}