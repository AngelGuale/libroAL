%%transformaciones
\chapter{Transformaciones Lineales}

\begin{dfn}
Sean $V$ y $W$ dos espacios vectoriales sobre un campo $K$. Una función $L: V \to W$ se dice que es lineal si para todo $u$, $v$ en $V$ y todo $\lambda \in K$, se cumple que
\begin{enumerate}
\item $L (u+v) = L(u) + L(v)$
\item $L(\lambda u) = \lambda L(u)$
\end{enumerate}
\end{dfn}

\begin{ejemplo}
Sea $r_0 \in K$ y $V$ un espacio vectorial sobre un campo $\mathbb{K}$. La función 
\begin{align*}
L: V &\to V\\
L(V) &= r_0 v
\end{align*}
es una transformación lineal.\\
En efecto 
\begin{align*}
L(u+v) &= r_0 (u+v)\\
&= r_0 u + r_0 v\\
&= L(u)+L(v)
\end{align*}
Si $\lambda \in \mathbb{K}$
\begin{align*}
L(\lambda u) &= r_0 (\lambda u)\\
&= \lambda (r_0 u)\\
&=\lambda (u)
\end{align*}
\end{ejemplo}

\begin{ejemplo}
Sea $L: \rdos \to \rdos$ definida por
$$L(x,y) = (x \cos(\theta) - y\, \mathrm{sen} (\theta) ,\, x\, \mathrm{sen}(\theta) + y \cos (\theta))$$
$L$ es una transformación lineal que consiste en una rotación, de $\theta$ grados, alrededor del origen.
\end{ejemplo}

\begin{ejemplo}
Sea $L: \mathbb{R} \to \mathbb{R}$ definida por 
$$L(x) = x^2$$
Claramente $L$ no es una transformación lineal pues
$$(x+y)^2 \neq x^2 + y^2  \ \ \ \textup{si} \; x\neq 0 ,\; y \neq 0$$
\end{ejemplo}

\begin{theorem}
Sean $U$ y $V$ dos espacios vectoriales y $T: U \to V$ una transformación lineal, entonces se cumple que
\begin{itemize}
\item[a. ] $T(\mathbf{0}_u) = \mathbf{0}_v$
\item[b. ] $T(-v) = -T(v)$
\item[c. ] $T(u - v) = T(u) - T(v)$
\item[d. ] Si $v = \alpha_1 v_1 + \alpha_2 v_2 + \ldots + \alpha_n v_n$, entonces $T(v) = \alpha_1 T(v_1) + \alpha_2 T(v_2) + \ldots + \alpha_n T(v_n)$
\end{itemize}
\end{theorem}

\begin{proof}
\begin{itemize}
\item[a. ] $T(\mathbf{0}_u) = T(\mathbf{0}_u + \mathbf{0}_u) = T(\mathbf{0}_u) + T(\mathbf{0}_u)$, entonces $\mathbf{0}_v = T(\mathbf{0}_u)- T(\mathbf{0}_u) = T(\mathbf{0}_u)$
\item[b. ] $T(-v) = T((-1)v) = (-1) T(v) = -T(v)$
\item [c. ] $T(u-v) = T(u+(-1)v) = T(u) + T((-1)v) = T(u) - T(v)$
\item [d. ] Aplicando $n$ veces la linealidad de $T$ se obtiene la igualdad deseada.
\end{itemize}
\end{proof}

\begin{theorem}
Sean $V$ y $W$ dos espacios vectoriales y $T: V \to W$ una transformación lineal. Si $B = \{v_1, v_2, \ldots , v_n\}$ es una base para $V$ entonces $L$ está unívocamente determinada por los vectores $T(v_1), T(v_2), \ldots , T(v_n)$.
\end{theorem}

\begin{proof}
En efecto, probemos que si $L: V \to W$ es una transformación lineal tal que $L(v_i) = T(v_i)$ para $i = 1, 2, \ldots , n$ entonces $L = T$. Necesitamos probar que para todo $v \in V$, $L(v) = T(v)$.
\\

Como $B$ es una base para $V$, se tiene que, para cualquier $v \in V$, existen escalares $\alpha_1, \alpha_2, \ldots , \alpha_n$ tales que $v = \alpha_1 v_1 + \alpha_2 v_2 + \ldots + \alpha_n v_n$.\\

Ahora
\begin{align*}
L(v) &= L(\alpha_1 v_1 + \alpha_2 v_2 + \ldots + \alpha_n v_n)\\
&= L(\alpha_1 v_1) + L(\alpha_2 v_2) + \ldots + L(\alpha_n v_n)\\
&=\alpha_1 L(v_1) + \alpha_2 L(v_2) + \ldots + \alpha_n L(v_n)\\ 
&=\alpha_1 T(v_1) + \alpha_2 T(v_2) + \ldots + \alpha_n T(v_n)\\
&= T(\alpha_1 v_1 + \alpha_2 v_2 + \ldots + \alpha_n v_n)\\
&= T(v) 
\end{align*}
\end{proof}

\begin{theorem}
Sean $V$ y $W$ dos espacios vectoriales y $B = \conjvect{v}{n}$ una base de $V$. Sean \conjvect{w}{n} elementos de $W$. Entonces existe una única transformación lineal $L: V \to W$ tal que 
$$w_i = L(v_i)\qquad \forall i = 1, 2, \ldots, n$$
\end{theorem}

\begin{proof}
En efecto sea $L: V \to W$ la función definida por $L(v) = L(\sum \alpha_i v_i) = \sum \alpha_i L(v_i) = \sum \alpha_i w_i$.
\\
%%%%%%%%%%%%%%%%%%%%%%%%%%%%%%%%%%%%%%%%%%%
$L$ es claramente lineal y por el teorema %9.3 
es única.
%%%%%%%%%%%%%%%%%%%%%%%%%%%%%%%%%%%%%%
\end{proof}

\begin{theorem}
Sea $A$ una matriz $m \times n$ con entradas reales. Entonces la función $L : \rn \to \rmm $ definida por $$L(v)= A v$$
es una transformación lineal.
\end{theorem}

\begin{proof}
Sin pérdida de rigurosidad representamos a un elemento $u$ de \rn \ como un vector columna $n \times 1$ y a los elementos de \rmm \ como vectores columna $m \times 1$.
\end{proof}




