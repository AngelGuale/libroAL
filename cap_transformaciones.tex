%%transformaciones
\chapter{Transformaciones Lineales}

\begin{dfn}
Sean $V$ y $W$ dos espacios vectoriales sobre un campo $K$. Una función $L: V \to W$ se dice que es lineal si para todo $u$, $v$ en $V$ y todo $\lambda \in K$, se cumple que
\begin{enumerate}
\item $L (u+v) = L(u) + L(v)$
\item $L(\lambda u) = \lambda L(u)$
\end{enumerate}
\end{dfn}

\begin{ejemplo}
Sea $r_0 \in K$ y $V$ un espacio vectorial sobre un campo $\mathbb{K}$. La función 
\begin{align*}
L: V &\to V\\
L(V) &= r_0 v
\end{align*}
es una transformación lineal.\\
En efecto 
\begin{align*}
L(u+v) &= r_0 (u+v)\\
&= r_0 u + r_0 v\\
&= L(u)+L(v)
\end{align*}
Si $\lambda \in \mathbb{K}$
\begin{align*}
L(\lambda u) &= r_0 (\lambda u)\\
&= \lambda (r_0 u)\\
&=\lambda (u)
\end{align*}
\end{ejemplo}

\begin{ejemplo}
Sea $L: \rdos \to \rdos$ definida por
$$L(x,y) = (x \cos(\theta) - y\, \mathrm{sen} (\theta) ,\, x\, \mathrm{sen}(\theta) + y \cos (\theta))$$
$L$ es una transformación lineal que consiste en una rotación, de $\theta$ grados, alrededor del origen.
\end{ejemplo}

\begin{ejemplo}
Sea $L: \mathbb{R} \to \mathbb{R}$ definida por 
$$L(x) = x^2$$
Claramente $L$ no es una transformación lineal pues
$$(x+y)^2 \neq x^2 + y^2  \ \ \ \textup{si} \; x\neq 0 ,\; y \neq 0$$
\end{ejemplo}

\begin{theorem}
Sean $U$ y $V$ dos espacios vectoriales y $T: U \to V$ una transformación lineal, entonces se cumple que
\begin{itemize}
\item[a. ] $T(\mathbf{0}_u) = \mathbf{0}_v$
\item[b. ] $T(-v) = -T(v)$
\item[c. ] $T(u - v) = T(u) - T(v)$
\item[d. ] Si $v = \alpha_1 v_1 + \alpha_2 v_2 + \ldots + \alpha_n v_n$, entonces $T(v) = \alpha_1 T(v_1) + \alpha_2 T(v_2) + \ldots + \alpha_n T(v_n)$
\end{itemize}
\end{theorem}

\begin{proof}
\begin{itemize}
\item[a. ] $T(\mathbf{0}_u) = T(\mathbf{0}_u + \mathbf{0}_u) = T(\mathbf{0}_u) + T(\mathbf{0}_u)$, entonces $\mathbf{0}_v = T(\mathbf{0}_u)- T(\mathbf{0}_u) = T(\mathbf{0}_u)$
\item[b. ] $T(-v) = T((-1)v) = (-1) T(v) = -T(v)$
\item [c. ] $T(u-v) = T(u+(-1)v) = T(u) + T((-1)v) = T(u) - T(v)$
\item [d. ] Aplicando $n$ veces la linealidad de $T$ se obtiene la igualdad deseada.
\end{itemize}
\end{proof}

\begin{theorem}
Sean $V$ y $W$ dos espacios vectoriales y $T: V \to W$ una transformación lineal. Si $B = \{v_1, v_2, \ldots , v_n\}$ es una base para $V$ entonces $L$ está unívocamente determinada por los vectores $T(v_1), T(v_2), \ldots , T(v_n)$.
\end{theorem}

\begin{proof}
En efecto, probemos que si $L: V \to W$ es una transformación lineal tal que $L(v_i) = T(v_i)$ para $i = 1, 2, \ldots , n$ entonces $L = T$. Necesitamos probar que para todo $v \in V$, $L(v) = T(v)$.
\\

Como $B$ es una base para $V$, se tiene que, para cualquier $v \in V$, existen escalares $\alpha_1, \alpha_2, \ldots , \alpha_n$ tales que $v = \alpha_1 v_1 + \alpha_2 v_2 + \ldots + \alpha_n v_n$.\\

Ahora
\begin{align*}
L(v) &= L(\alpha_1 v_1 + \alpha_2 v_2 + \ldots + \alpha_n v_n)\\
&= L(\alpha_1 v_1) + L(\alpha_2 v_2) + \ldots + L(\alpha_n v_n)\\
&=\alpha_1 L(v_1) + \alpha_2 L(v_2) + \ldots + \alpha_n L(v_n)\\ 
&=\alpha_1 T(v_1) + \alpha_2 T(v_2) + \ldots + \alpha_n T(v_n)\\
&= T(\alpha_1 v_1 + \alpha_2 v_2 + \ldots + \alpha_n v_n)\\
&= T(v) 
\end{align*}
\end{proof}

\begin{theorem}
Sean $V$ y $W$ dos espacios vectoriales y $B = \conjvect{v}{n}$ una base de $V$. Sean \conjvect{w}{n} elementos de $W$. Entonces existe una única transformación lineal $L: V \to W$ tal que 
$$w_i = L(v_i)\qquad \forall i = 1, 2, \ldots, n$$
\end{theorem}

\begin{proof}
En efecto sea $L: V \to W$ la función definida por $L(v) = L(\sum \alpha_i v_i) = \sum \alpha_i L(v_i) = \sum \alpha_i w_i$.
\\
%%%%%%%%%%%%%%%%%%%%%%%%%%%%%%%%%%%%%%%%%%%
$L$ es claramente lineal y por el teorema %9.3 
es única.
%%%%%%%%%%%%%%%%%%%%%%%%%%%%%%%%%%%%%%
\end{proof}

\begin{theorem}
Sea $A$ una matriz $m \times n$ con entradas reales. Entonces la función $L : \rn \to \rmm $ definida por $$L(v)= A v$$
es una transformación lineal.
\end{theorem}

\begin{proof}
Sin pérdida de rigurosidad representamos a un elemento $u$ de \rn \ como un vector columna $n \times 1$ y a los elementos de \rmm \ como vectores columna $m \times 1$.
\end{proof}

\section{Isomorfismos entre espacios vectoriales}
\begin{dfn}
Sean $V$ y $W$ dos espacios vectoriales. Si existe una transformación lineal $L : V \to W$ que es inyectiva y sobreyectiva, decimos que $V$ y $W$ son isomorfos y que $L$ es un isomorfismo de espacios vectoriales.
\end{dfn}

\begin{dfn}
Sea $V$ un espacio vectorial sobre el campo $K$ y sea $B = \conjvect{v}{n}$ una base para $V$, entonces la función $L : V \to \rn$ tal que $L(v) = [v]_B$ es una transformación lineal llamada transformación de coordenadas de $V$ con respecto a la base $B$.
\end{dfn}

\begin{theorem}
Sea $V$ un espacio vectorial de dimensión $n$ sobre $\mathbb{R}$. Entonces la transformación de coordenadas respecto a $B$ $L : V \to \rn$ es un isomorfismo.
\end{theorem}

\begin{proof}
La prueba sigue fácilmente del hecho de observar que $[v + w]_B = [v]_B + [w]_B$ y que $[\alpha v]_B = \alpha [v]_B$.
\end{proof}

\section{Núcleo e imagen de una transformación lineal}
\begin{dfn}
Sean $V$ y $W$ dos espacios vectoriales y $T : V \to W$ una transformación lineal. Entonces el núcleo de $T$, denotado por $Nu(T) = \{v \in V : T(v) = \mathbf{0}_W\}$ y la imagen de $T$, denotado por $Img(T) = \{w : w \neq T(v) \ para \ v \in V\}$.
\end{dfn}

\begin{theorem}
Sean $V$ y $W$ dos espacios vectoriales y $T : V \to W$ una transformación lineal. Entonces $Nu(T)$ y $Img(T)$ son subespacios vectoriales de $V$ y $W$ respectivamente.
\end{theorem}

\begin{proof}
Veamos que $Nu(T)$ es un subespacio vectorial de $V$.
\begin{enumerate}
\item Como $T(\mathbf{0}_V)=\mathbf{0}_W$ , entonces $\mathbf{0}_V \in Nu(T)$.
\item Supongamos que $u , v \in Nu(T)$, entonces $T(u+v) = T(u) + T(v) = \mathbf{0}_W + \mathbf{0}_W = \mathbf{0}_w$. Por lo tanto $u+v \in Nu(T)$.
\item Si $\alpha \in \mathbb{K}$ y $u \in Nu(T)$, se tiene que $T(\alpha u) = \alpha T(u) = \alpha \mathbf{0}_W = \mathbf{0}_W$ lo cual prueba que $\alpha u \in Nu(T)$.
\end{enumerate}
Así tenemos que $Nu(T)$ es un subespacio vectorial de $V$. Similarmente se prueba que $Img(T)$ es un subespacio vectorial de $W$.
\end{proof}

\begin{dfn}
Sean $V$ y $W$ dos espacios vectoriales sobre $\mathbb{K}$ y $T$ es una transformación lineal. La nulidad de $T$ se define como la dimensión de $Nu(T)$ y se denota por $\eta (T)$. El $rank$ de $T$ se define como la dimensión de $Img(T)$.
\end{dfn}

\begin{theorem}
Una transformación lineal $L : V \to W$ es inyectiva si y solo si $Nu(L) = \{\mathbf{0}_V\}$.
\end{theorem}

\begin{proof}
Supongamos que $L$ es inyectiva, entonces si $L(v) = \mathbf{0}_W$ se tiene que $L(v) = L(\mathbf{0}_V)$ y por la inyectividad de $L$ concluimos que $u = \mathbf{0}_V$. Así $Nu(L) = \{\mathbf{0}_V\}$.\\
Recíprocamente, si $Nu(L) = \{\mathbf{0}_V\}$ y $L(u) = L(v)$ se tiene que $L(u - v) = \mathbf{0_W}$. Entonces $u -v = \mathbf{0}_V$ lo cual implica que $u = v + \mathbf{0}_V = v$. Concluimos entonces que $L$ es inyectiva.
\end{proof}

\begin{theorem}
Sea $V$ y $W$ dos espacios vectoriales sobre $\mathbb{K}$ y $L : V \to W$ una transformación lineal inyectiva. Sea $B$ un subconjunto de $V$ linealmente independiente, entonces $L(B)$ es un subconjunto de $W$ linealmente independiente. 
\end{theorem}

\begin{proof}
Sea $B = \conjvect{v}{n}$ un conjunto linealmente independiente y $L(B) = \{L(v_1), L(v_2), \ldots , L(v_k)\}$. Si $\alpha_1 L(v_1) + \ldots + \alpha_k L(v_k) = 0$ entonces $L(\alpha_1 v_1 + \ldots +\alpha_k v_k) = \mathbf{0}_W$. Como $L$ es inyectiva se tiene que $\alpha_1 v_1 + \ldots + \alpha_k v_k = \mathbf{0}_V$ y por el hecho de ser $B$ linealmente independiente se tiene que $\alpha_i = 0$ para todo $i = 1, 2, \ldots , k$. Obtenemos entonces que $\{L(v_1), \ldots , L(v_k)\}$ es linealmente independiente.
\end{proof}

\begin{theorem}
Sean $V$ y $W$ dos espacios vectoriales sobre $\mathbb{K}$.Si $B$ es una base para $V$ y $L : V \to W$ es una transformación lineal inyectiva, entonces $L(B)$ es una base para la $Img(L)$ y si $V$ es de dimensión finita $n$ se tiene que $n = rank(L)$.\\

La demostración se deja como ejercicio. %%% :c 
\end{theorem}

\begin{theorem}
Sean $V$ y $W$ dos espacios vectoriales sobre $\mathbb{K}$ y $L : V \to W$ una transformación lineal. Si $B = \conjvect{v}{k}$ es una base para $Nu(L)$, entonces si $\{v_1 , \ldots , v_{k}, \ldots , v_n\}$ es una base para $V$, se tiene que $\{T(v_{k+1}), \ldots , T(v_n)\}$ es una bse para $Img(L)$.
\end{theorem}

\begin{proof}
.
\end{proof}

\section{Operaciones con transformaciones lineales}
\begin{dfn}
Sean $V$ y $W$ dos espacios vectoriales sobre un campo $\mathbb{K}$ y $T$ y $L$ dos transformaciones lineales de $V$ en $W$ y $\lambda \in \mathbb{K}$. Entonces 
\begin{enumerate}
\item $T + L : V \to W \qquad (T + L)(v) = T(v) + L(v)$
\item $\lambda T : V \to W \qquad (\lambda T)(v) = \lambda T(v)$
\end{enumerate}
Son transformaciones lineales.
\end{dfn}

\begin{proof}
\begin{align*}
(T+L)(u+v) &=T(u+v) + L(u+v)\\
&= T(u) + T(v) + L(u) + L(v)\\
&= (T+L)(u) + (T+L)(v)\\
\\
(T+L)(\alpha u) &= T(\lambda u) + L(\lambda u)\\
&= \lambda T(u)+ \lambda L(u)\\
&= \lambda (T+L)(u) 
\end{align*}
%%%%%%%%%%%%%%%%%%%%%%%%%%%%%%%%%%%%%%
\begin{align*}
\lambda T(u+v) &= \lambda [T(u+v)]\\
&= \lambda [T(u) + T(v)]\\
&= (\lambda T)(u) + (\lambda T)(v)\\
\\
(\lambda T)(\beta u) &= \lambda T(\beta u)\\
&=\lambda \beta T(u)\\
&= \beta (\lambda T)(u)
\end{align*}
\end{proof}

\begin{theorem}
Sean $V$ y $W$ espacios vectoriales y sea $L(V, W) = \{T : V \to W \; ; T \ una \ transformaci\acute{o}n \ lineal\}$. Entonces $L(V, W)$ con las operaciones definidas anteriormente es un espacio vectorial.
\end{theorem}

\begin{proof}
.
\end{proof}

