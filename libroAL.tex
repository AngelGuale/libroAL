\documentclass[10pt,a4paper]{amsbook}
\usepackage[utf8]{inputenc}
\usepackage[spanish]{babel}
\usepackage{amsmath}
\usepackage{amsthm}
\usepackage{amsfonts}
\usepackage{amssymb}
\usepackage{graphicx}

\RequirePackage{algebraLineal}


\newtheorem{theorem}{Teorema}[chapter]
\newtheorem{lemma}[theorem]{Lema}


\theoremstyle{definition}
\newtheorem{dfn}[theorem]{Definición}
\newtheorem{ejemplo}[theorem]{Ejemplo}
\newtheorem{ejercicio}[theorem]{Ejercicio}


\theoremstyle{remark}
\newtheorem{remark}[theorem]{Remark}

\numberwithin{section}{chapter}
\numberwithin{equation}{chapter}

%    For a single index; for multiple indexes, see the manual
%    "Instructions for preparation of papers and monographs:
%    AMS-LaTeX" (instr-l.pdf in the AMS-LaTeX distribution).

\author{Jorge Vielma, Angel Guale}
\title{Álgebra Lineal}
\begin{document}
\maketitle
%    Dedication.  If the dedication is longer than a line or two,
%    remove the centering instructions and the line break.
%\cleardoublepage
%\thispagestyle{empty}
%\vspace*{13.5pc}
%\begin{center}
%  Dedication text (use \\[2pt] for line break if necessary)
%\end{center}
%\cleardoublepage
%    Change page number to 7 if a dedication is present.
\setcounter{page}{4}
\tableofcontents


\chapter{Espacios Vectoriales Reales}
\section{Espacios Vectoriales reales}

\begin{dfn}[Espacio vectorial real]
Un espacio vectorial real es una cuarteta $(V, \dobler, +, \odot)$ donde V es un conjunto no vacío , $\dobler$ es el campo de los números reales, + es una operación en V llamada suma o adición y $\odot$ es una operación en V llamada multiplicación por un escalar. Los escalares son los elementos de $\dobler$ y que cumplen con los diferentes axiomas.
\end{dfn}
\subsection*{Axiomas para la suma}
\begin{enumerate}
\item Si $u$ y $v$ son elementos de $V$, $u+v$ es un elemento de V.
\item Si $u$, $v$, y $w$ son elementos de $V$, entonces $u+v=v+u$. Es decir, la suma es una operación conmutativa.
\item Si $u$, $v$, y $w$ son elementos de $V$, entonces  $(u+v)+w=u+(v+w)$. Es decir, la operación es asociativa.
\end{enumerate}

\subsection*{Axiomas para la multiplicación por escalares}
\begin{enumerate}
\item Si $\alpha$ es un número real y $u$ es un elemento de $V$, entonces $\alpha u$ es un elemento de $V$.
\item Si $\alpha$ es un número real y $u$ y $v$ son elementos de $V$, entonces $\alpha (u+v) = \alpha u + \alpha v$. Es decir la multiplicación por un escalar es distributiva con respecto a la suma de vectores.
\item Si $\alpha$ y $\beta$ son números reales y $u$ es un elemento de $V$, entonces $(\alpha + \beta)u = \alpha u + \beta u$. Es decir la multiplicación por un escalar es distributiva con respecto a la suma de escalares.
\item Si $\alpha$ y $\beta$ son números reales y $u$ es un elemento de $V$, entonces $(\alpha \beta)u = \beta(\alpha u)$.
\item Si $u$ es un elemento de $V$, entonces $1 \odot u = u$.

\end{enumerate}

\textbf{Trabajo autónomo 1}
Sea $(E, +, \odot)$ un espacio vectorial real. Pruebe que
\begin{enumerate}
\item El elemento neutro $\mathbf{0}_E$ es único.
\item Para cada $e$ en $E$, el elemento $-e$ es único.
\item Para cada $e$ en $E$, $\mathbf{0}_E \odot e = 0$.
\item Para cada $e$ en $E$, $(-1)e = -e$.
\item Para cada $\lambda$ en $\mathbb{R}$, $\lambda \odot \mathbf{0}_E = \mathbf{0}_E$.
\item Si $\alpha v = \mathbf{0}_E$, entonces $\alpha = 0$ o $v=\mathbf{0}_E$.

\end{enumerate}

\begin{ejemplo}
\begin{enumerate}
\item El espacio \rdos \ con las operaciones $(x+y)+(w,z) = (x+w,y+z)$ y la multiplicación por un escalar $\lambda$, $\lambda (x,y) = (\lambda x,\lambda y)$.
\item El espacio de todos los polinomios de grado menor o igual que $n$, para un número natural $n$ fijo, con la operación normal de suma de polinomios y la multiplicación usual de un polinomio por un número real.
\item El espacio de todas las matrices $n \times m$ con la operación usual de suma de matrices y multiplicación de una matriz por un número real.(Aquí $n$ y $m$ son números naturales fijos y distintos de cero).

\end{enumerate}

\end{ejemplo}

\section{Subespacios Vectoriales}
\begin{dfn}
Sea $(E, +, \odot)$ un espacio vectorial real y $S$ un subconjunto no vacío de $E$. Entonces se dice que $S$ es un subespacio vectorial de $E$, si con las operaciones heredadas de $E$, $(S, +, \odot)$ es también un espacio vectorial.

\end{dfn}

\begin{ejemplo}
\begin{enumerate}
\item Sea $E = \rtres$ y $S = \llaves{(x, y, z) \in \rtres}{5x + 2y + z = 0}$ entonces $S$ es un subespacio de \rtres.
\item Sea $E$ el espacio $\mathcal{M_{2 \times 2}}$ \ de todas las matrices cuadradas $2 \times 2$ con las operaciones usuales de suma de matrices y de multiplicación por un escalar y $S$ el conjunto de las matrices diagonales, es decir las matrices de la forma \matrdxd{a & 0}{0 & b} , entonces $S$ es un subespacio de $E$.

\end{enumerate}
\end{ejemplo}

%%%%%%%% esta seccion son los modelos de comandos para secciones
%comentalas luego
%%ejercicio
\begin{ejercicio}
Sea V un espacio vectorial etc etc
\end{ejercicio}

\begin{ejemplo}
Sea V un espacio vectorial etc etc
\end{ejemplo}

\begin{theorem}[Teorema de las dimensiones -es opcional]
Sea V un espacio vectorial etc etc
\end{theorem}

\begin{lemma}[Nombre del lema- es opcional]
Sea V un espacio vectorial etc etc
\end{lemma}

\begin{dfn}[Base]
Sea V un espacio vectorial etc etc

\end{dfn}

%%%%%%%%%%%%%%%%


Trabajo autonomo


\end{document}