\documentclass[10pt,a4paper]{amsbook}
\usepackage[utf8]{inputenc}
\usepackage[spanish]{babel}
\usepackage{amsmath}
\usepackage{amsthm}
\usepackage{amsfonts}
\usepackage{amssymb}
\usepackage{graphicx}

\RequirePackage{algebraLineal}


\newtheorem{theorem}{Teorema}[chapter]
\newtheorem{lemma}[theorem]{Lemma}


\theoremstyle{definition}
\newtheorem{dfn}[theorem]{Definición}
\newtheorem{example}[theorem]{Example}
\newtheorem{xca}[theorem]{Exercise}

\theoremstyle{remark}
\newtheorem{remark}[theorem]{Remark}

\numberwithin{section}{chapter}
\numberwithin{equation}{chapter}

%    For a single index; for multiple indexes, see the manual
%    "Instructions for preparation of papers and monographs:
%    AMS-LaTeX" (instr-l.pdf in the AMS-LaTeX distribution).

\author{Jorge Vielma, Angel Guale}
\title{Álgebra Lineal}
\begin{document}
\maketitle
%    Dedication.  If the dedication is longer than a line or two,
%    remove the centering instructions and the line break.
%\cleardoublepage
%\thispagestyle{empty}
%\vspace*{13.5pc}
%\begin{center}
%  Dedication text (use \\[2pt] for line break if necessary)
%\end{center}
%\cleardoublepage
%    Change page number to 7 if a dedication is present.
\setcounter{page}{4}
\tableofcontents


\chapter{Espacios Vectoriales Reales}
\begin{dfn}[Espacio vectorial real]
Un espacio vectorial real es una cuarteta $(V, \dobler, +, \odot)$ donde V es un conjunto no vacío , $\dobler$ es el campo de los números reales, + es una operación en V llamada suma o adición y $\odot$ es una operación en V llamada multiplicación por un escalar. Los escalares son los elementos de $\dobler$ y que cumplen con los diferentes axiomas.
\end{dfn}
\subsection*{Axiomas para la suma}
\begin{enumerate}
\item Si $u$ y $v$ son elementos de $V$, $u+v$ es un elemento de V.
\item Si $u$, $v$, y $w$ son elementos de $V$, entonces $u+v=v+u$. Es decir, la suma es una operación conmutativa.
\item Si $u$, $v$, y $w$ son elementos de $V$, entonces  $(u+v)+w=u+(v+w)$. Es decir, la operación es asociativa.
\end{enumerate}

\subsection*{Axiomas para la multiplicación por escalares}
\begin{enumerate}
\item Si $\alpha$ es un número real y $u$ es un elemento de $V$, entonces $\alpha u$ es un elemento de $V$.
\item Si $\alpha$ es un número real y $u$ y $v$ son elementos de $V$, entonces $\alpha (u+v) = \alpha u + \alpha v$. Es decir la multiplicación por un escalar es distributiva con respecto a la suma de vectores.
\item Si $\alpha$ y $\beta$ son números reales y $u$ es un elemento de $V$, entonces $(\alpha + \beta)u = \alpha u + \beta u$. Es decir la multiplicación por un escalar es distributiva con respecto a la suma de escalares.
\item Si $\alpha$ y $\beta$ son números reales y $u$ es un elemento de $V$, entonces $(\alpha \beta)u = \beta(\alpha u)$.
\item Si $u$ es un elemento de $V$, entonces $1 \odot u = u$.
  
\end{enumerate}


\end{document}