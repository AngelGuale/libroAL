\documentclass[10pt,a4paper]{book}
\usepackage[spanish]{babel}
\usepackage[utf8]{inputenc}
\usepackage{amsmath}
\usepackage{amsthm}
\usepackage{amsfonts}
\usepackage{amssymb}
\usepackage{graphicx}
\usepackage{makeidx}
\usepackage{enumerate}


\RequirePackage{algebraLineal}



\newcommand{\sol}{
~\\
\underline{SOLUCI\'ON}:
~\\
}
\newcommand{\obs}{
~\\
\underline{Observaci\'on}: 
}


\newtheorem{theorem}{Teorema}[chapter]
\newtheorem{lemma}[theorem]{Lema}


\theoremstyle{definition}
\newtheorem{dfn}{Definición}[chapter]
\newtheorem{ejemplo}{Ejemplo}[chapter]
\newtheorem{ejercicio}{Ejercicio}[chapter]


\theoremstyle{remark}
\newtheorem{remark}[theorem]{Remark}

\numberwithin{section}{chapter}
\numberwithin{equation}{chapter}

\newcounter{tacounter}
\newenvironment{trabajoautonomo}[0]
    {\begin{center}
    \refstepcounter{tacounter}
     \textbf{Trabajo autónomo \thetacounter}\\[1ex]
    \begin{tabular}{|p{0.9\textwidth}|}
    \hline\\
    }
    { 
    \\\\\hline
    \end{tabular} 
    \end{center}
    }
\numberwithin{tacounter}{chapter}

\newenvironment{obsimp}[0]
    {\noindent$\star$ \textbf{Observación importante}:
  \itshape
    }
    { 
    \\
    }


\newenvironment{prob}[1][]{
\item {\bfseries Problema  #1}: 
}{
}


\newenvironment{prop}[1][]{
\item {\bfseries Proposición #1}: 
}{
}

%    For a single index; for multiple indexes, see the manual
%    "Instructions for preparation of papers and monographs:
%    AMS-LaTeX" (instr-l.pdf in the AMS-LaTeX distribution).

\author{Jorge Vielma, Angel Guale}
\title{Álgebra Lineal}

\makeindex

\begin{document}
\maketitle
%    Dedication.  If the dedication is longer than a line or two,
%    remove the centering instructions and the line break.
%\cleardoublepage
%\thispagestyle{empty}
%\vspace*{13.5pc}
%\begin{center}
%  Dedication text (use \\[2pt] for line break if necessary)
%\end{center}
%\cleardoublepage
%    Change page number to 7 if a dedication is present.
\setcounter{page}{4}
\tableofcontents

\printindex
\chapter{Nociones Preliminares}

\section{Campo Escalar}
Sea $\mathbf{K}$ un conjunto no vac\'io en el cual se definen dos operaciones binarias. La terna $\left( K,\ast,\bigtriangleup \right)$ es un campo escalar si cumple: 



\begin{table}[htbp]
\begin{center}
\begin{tabular}{|l|l|}
\hline
\multicolumn{2}{|l|}{Operación suma escalar } \\
\hline \hline
$\forall a,b\in \doubleK (a*b)\in \doubleK $ &[Cerradura de la suma] \\ \hline
$ \forall  a,b\in K a*b=b*a $&[Conmutatividad]  \\ \hline
$\forall  a,b,c\in K (a*b)*c=a*(b*c)$&[Asociativa] \\ 
\hline
$\exists n_*\in K \forall  a\in K a*n_*=a$&  [elemento neutro]  \\ \hline
$\forall  a\in K \exists a'\in K a*a'=n_*$&  [elemento inverso]  \\ \hline
Operación producto escalar &\\ \hline \hline
 $\forall  a,b\in K (a\bigtriangleup b)\in K $& [Cerradura de producto] \\ \hline
$\forall  a,b\in K a\bigtriangleup b=b\bigtriangleup a$&  [Conmutativa]  \\ \hline
$\forall  a,b,c\in K (a\bigtriangleup b)\bigtriangleup c=a\bigtriangleup (b\bigtriangleup c) $&  [Asociativa]  \\ \hline
$\exists n'\in K \forall  a\in K  a*n_\bigtriangleup =a $&  [Elemento neutro]  \\ \hline
$\forall  a\in K a\neq n_*  \exists a''\in K a*a''=n_\bigtriangleup  $&  [Elemento inverso]  \\ \hline
$\exists n'\in K \forall  a\in K  a*n_\bigtriangleup =a $&  [Elemento neutro]  \\ \hline
\multicolumn{2}{|c|}{$\forall a,b,c\in K  a\bigtriangleup (b*c)=(a\bigtriangleup b)*(a\bigtriangleup c)
 $ } \\ \hline
\end{tabular}
\caption{Axiomas de un campo escalar.}
\label{tabla:sencilla}
\end{center}
\end{table}


\textbf{Ejemplos:}




$\left( \mathbb{N},+ ,\cdot \right) $
El conjunto de los Naturales con la suma y multiplicación usuales NO es un campo escalar.
Basta el hecho de no cumplir uno de los axiomas para dejar de ser un campo. En este caso, por ejemplo; no existe un elemento neutro para la adicion (recuerde que el 0 no forma parte del conjunto de los naturales).

~\\
$\left( \mathbb{Z},+ ,\cdot \right) $ 
El conjunto de los Enteros con la suma y multiplicación usuales NO es un campo escalar.
Está claro que en este conjunto con las operaciones usuales, el neutro multiplicativo es el 1. Por tanto ningún elemento de este conjunto tiene inverso multiplicativo.

~\\
$\left( \mathbb{Q},+ ,\cdot \right) $ 
El conjunto de los Racionales con la suma y multiplicación usuales ES un CAMPO ESCALAR.

~\\
$\left( \mathbb{R},+ ,\cdot \right) $ 
El conjunto de los Reales con la suma y multiplicación usuales ES un CAMPO ESCALAR.

~\\
$\left( \mathbb{C},+ ,\cdot \right) $ 
El conjunto de los Complejos con la suma y multiplicación usuales ES un CAMPO ESCALAR.


\section{Sistema de ecuaciones lineales}

Una ecuación lineal es una ecuación de la forma
\begin{align*}
\alpha_1 x_1+\alpha_2 x_2+...+\alpha_n x_n=\beta_1
\end{align*} 
donde las variables $x_1, x_2, .., x_n$ son las incógnitas del sistema y los $\alpha_i, \beta_i$ son elementos del campo real o complejo.

Una solución de una ecuación lineal es una colección de n elementos del campo $(c_1, c_2, ..., c_n)$ de tal manera que al ser sustituídos en la ecuación se obtiene una igualdad. 
\begin{ejemplo}
La ecuación 2x+3y=1 es una ecuación lineal en las variables x, y. Una solución del sistema sería la colección de números reales (8, -5) ya que al sustituírlos en la ecuación se obtiene una igualdad
\begin{align*}
2x+3y&=1\\
2(8)+3(-5)&=1\\
16-15&=1\\
1&=1
\end{align*}
\end{ejemplo}
Un sistema de ecuaciones lineales es un conjunto de ecuaciones lineales que deben satifacerse simultáneamente.
\begin{align*}
\alpha_{11} x_1+\alpha_{12} x_2+...+\alpha_{1n} x_n&=\beta_1\\
\alpha_{21} x_1+\alpha_{22} x_2+...+\alpha_{2n} x_n&=\beta_2\\
\alpha_{31} x_1+\alpha_{32} x_2+...+\alpha_{3n} x_n&=\beta_3\\
\vdots\\
\alpha_{m1} x_1+\alpha_{m2} x_2+...+\alpha_{mn} x_n&=\beta_m\\
\end{align*} 
\begin{ejemplo}
Considere las incógnitas x, y, z. Entonces el conjunto de ecuaciones
\begin{align*}
3x-5y-z=2\\
4x+y+3z=6\\
7x-4y+2z=8\\
\end{align*}
es un sistema de ecuaciones lineales. La terna (2, 1, -1) forma una solución de este sistema de ecuaciones lineales, ya que satisface todas las ecuaciones lineales del mismo.
\begin{align*}
3(2)-5(1)-(-1)=2\\
4(2)+(1)+3(-1)=6\\
7(2)-4(1)+2(-1)=8\\
\end{align*}
\end{ejemplo}

\section{Representación Matricial de un S.E.L}
Un sistema de ecuaciones lineales puede representarse por medio de la ecuación matricial AX=B, donde A es la matriz de coeficientes, X es el conjunto de incógnitas y B el conjunto de términos independientes. De esta forma el sistema de ecuaciones lineales

\begin{align*}
\alpha_{11} x_1+\alpha_{12} x_2+...+\alpha_{1n} x_n&=\beta_1\\
\alpha_{21} x_1+\alpha_{22} x_2+...+\alpha_{2n} x_n&=\beta_2\\
\alpha_{31} x_1+\alpha_{32} x_2+...+\alpha_{3n} x_n&=\beta_3\\
\vdots\\
\alpha_{m1} x_1+\alpha_{m2} x_2+...+\alpha_{mn} x_n&=\beta_m\\
\end{align*} 
se puede representar matricialmente como
\begin{align*}
\left(
\begin{array}{rrrr}
\alpha_{11} &\alpha_{12} &...&\alpha_{1n} \\
\alpha_{21} &\alpha_{22} &...&\alpha_{2n} \\
\alpha_{31} &\alpha_{32} &...&\alpha_{3n} \\
\vdots&&&\\
\alpha_{m1} &\alpha_{m2} &...&\alpha_{mn}\\
\end{array}
\right)
\left(
\begin{array}{r}
x_1\\x_2\\\vdots\\x_n
\end{array}
\right)
=
\left(
\begin{array}{r}
\beta_1\\\beta_2\\\beta_3\\\vdots\\\beta_m
\end{array}
\right)
\end{align*} 
\begin{align*}
AX=B
\end{align*}
\subsection{Matriz aumentada}
La representación matricial mostrada en líneas anteriores tiene una forma abreviada llamada representación mediante matriz aumentada. El mismo sistema puede representarse como:
\begin{align*}
\reducir{rrrr|r}{
\alpha_{11} &\alpha_{12} &...&\alpha_{1n}&\beta_1 \\
\alpha_{21} &\alpha_{22} &...&\alpha_{2n}&\beta_2 \\
\alpha_{31} &\alpha_{32} &...&\alpha_{3n}&\beta_3 \\
\vdots&&&\\
\alpha_{m1} &\alpha_{m2} &...&\alpha_{mn}&\beta_m\\
}
\end{align*}

\section{Tipo de solución de un S.E.L}
Un sistema de ecuaciones lineales pertenece a sólo uno de los siguientes casos:
\begin{itemize}
\item Es inconsistente. (No tiene solución)
\item Es consistente con solución única. (Solo una n-tupla satisface las ecuaciones)
\item Es consistente con infinitas soluciones. (Existen infinitas n-tuplas que satisfacen todas las ecuaciones)
\end{itemize}
\subsection{S.E.L inconsistente}
Un sistema de ecuaciones lineales se dice inconsistente si no existe n-tupla que pueda satisfacer todas las ecuaciones a la vez. 
\begin{ejemplo}
El sistema 
\begin{align*}
2x&-y=2\\
4x&-2y=1\\
\end{align*}
Es un sistema de ecuaciones lineales inconsistente 
\end{ejemplo}
Podemos comprobar que el sistema es inconsistente utilizando el método de Gauss, al representar al sistema por medio de una matriz aumentada
\begin{align*}
\left(
\begin{array}{rr|r}
2& -1 & 2\\
4& -2 & 1\\
\end{array}
\right)
\sim
\left(
\begin{array}{rr|r}
2& -1 & 2\\
0& 0 & -3\\
\end{array}
\right)
\end{align*}
La última ecuación del sistema reducido nos indica una ecuación $0x+0y=-3$, es decir, sin importar el valor de x e y, la ecuación resulta ser $0=-3$ lo cual es una inconsistencia. De esto se concluye que el sistema no posee solución.~\\

De manera general se puede afirmar que si al reducir completamente el sistema por Gauss se obtiene una fila de la matriz de coeficientes llena de ceros y su correspondiente valor independiente distinto de cero, entonces el sistema es inconsistente.
\begin{ejemplo}
Considere el sistema de ecuaciones lineales
\begin{align*}
2x-y+3z=2\\
x+2y+z=3\\
x-3y+2z=1\\
\end{align*}
La representación en matriz aumentada del sistema anterior es
\begin{align*}
\reducir{rrr|r}{
2&-1&3&2\\
1&2&1&3\\
1&-3&2&1\\
}
\end{align*}
Y al reducir tendremos
\begin{align*}
\reducir{rrr|r}{
2&-1&3&2\\
1&2&1&3\\
1&-3&2&1\\
}
\sim
\reducir{rrr|r}{
2&-1&3&2\\
0&-5&1&-4\\
0&5&-1&0\\
}
\sim
\reducir{rrr|r}{
2&-1&3&2\\
0&-5&1&-4\\
0&0&0&-4\\
}
\end{align*}
En donde la última fila del sistema reducido muestra ceros en todos los coeficientes de la matriz del sistema y un número distinto de cero (en este caso -4) como término independiente, esto nos indica que el sistema de ecuaciones lineales es inconsistente.

\end{ejemplo}

\subsection{S.E.L Consistentes con solución única}
Un sistema de ecuaciones lineales se dice que es consistente si tiene al menos una solución. Cuando un sistema consistente tiene exactamente una n-tupla que satisface las ecuaciones, se dice que tiene solución única.

\begin{ejemplo}
Considere el sistema de ecuaciones lineales
\begin{align*}
2x-3y=9\\
x+y=2\\
\end{align*}
Este sistema de ecuaciones posee exactamente una solución, la dupla (3, -1). Podemos obtener este resultado empleando el método de Gauss
\begin{align*}
\reducir{rr|r}{
2&-3&9\\
1&1&2\\
}\sim
\reducir{rr|r}{
2&-3&9\\
0&-5&5\\
}
\end{align*}
De la última ecuación se puede obtener $y=-1$ y reemplazando en la primera $2x-3(-1)=9$ se deduce que $x=3$, y el conjunto solución estaría conformado sólo por la dupla (3, -1).
\end{ejemplo}


En términos generales podemos afirmar que un sistema consistente tiene solución única si en el sistema reducido por filas, el número de \textit{filas válidas} - filas que no están completamente llenas de ceros - es igual al número de incógnitas del sistema. 


\begin{ejemplo}
El sisguiente sistema de ecuaciones lineales también posee solución única
\begin{align*}
4x-y&=7\\
x+y&=3\\
2x-3y&=1
\end{align*}
Por el método de Gauss tenemos:
\begin{align*}
\reducir{rr|r}{
4&-1&7\\
1&1&3\\
2&-3&1\\
}\sim
\reducir{rr|r}{
4&-1&7\\
0&-5&-5\\
0&5&5\\
}\sim
\reducir{rr|r}{
4&-1&7\\
0&-5&-5\\
0&0&0\\
}
\end{align*}
De la última ecuación se puede obtener que $y=1$, y reemplazando en la primera ecuación: $4x-(1)=7$, se tiene que $x=2$. Y la única solución del sistema sería $(2, 1)$.

\end{ejemplo}

\subsection{S.E.L Consistentes con infinitas soluciones}
Cuando un sistema consistente tiene más de una n-tupla que satisface las ecuaciones, se dice que tiene infinitas soluciones.

\begin{ejemplo}
El sistema de ecuaciones conformado por las siguientes ecuaciones 
\begin{align*}
3x-4y+z=1\\
x-y+2z=4\\
2x-3y-z=-3\\
\end{align*}
Si resolvemos este sistema por Gauss tendremos
\begin{align*}
\reducir{rrr|r}{
3&-4&1&1\\
1&-1&2&4\\
2&-3&-1&-3\\
}\sim
\reducir{rrr|r}{
3&-4&1&1\\
0&-1&-5&-11\\
0&1&5&11\\
}\sim
\reducir{rrr|r}{
3&-4&1&1\\
0&-1&-5&-11\\
0&0&0&0\\
}\sim
\end{align*}
De la última ecuación del sistema reducido podemos despejar 
\begin{align*}
y=&-5z+11\\
\end{align*}

Y de la primera ecuación se obtiene que
\begin{align*}
3x=&4y-z+1\\
3x=&4(-5z+11)-z+1\\
3x=&-21z+45\\
x=&-7z+15\\
\end{align*}
Aquí no es posible definir un valor único para x, y, z. En estos casos se conviene dejar expresado la solución en función un parámetro, al cual le llamaremos variable libre, en esta ocasión, la variable z. Las demás incógnitas del sistema: x e y quedan condicionadas de acuerdo al valor que tome z, por lo que se denominarán variables condicionadas. El conjunto solución del sistema sería:
\begin{align*}
Sol(x,y, z)=\llaves{\vectrtres{x}{y}{z}}{\begin{array}{r}x=-7z+15\\y=-5z+11\\z\in \dobler
\end{array}}
\end{align*}

\end{ejemplo}

De forma similar que en los casos anteriores se puede establecer una relación entre el sistema reducido y el tipo de solución. Un sistema de ecuaciones lineales tiene infinitas soluciones si el número de incógnitas es mayor que el número de fílas válidas en el sistema reducido.
\newpage
\section{Problemas propuestos}


\begin{enumerate}[1.]
\item Demostrar que el conjunto $\mathbb{Z}_2$ representa un campo con la suma módulo 2.
\item Demostrar que el conjunto $\mathbb{Z}_3$ representa un campo con la suma módulo 3.
\item Determinar el tipo de solución de los siguientes sistemas

\begin{enumerate}
\item $\left\lbrace \begin{aligned}
2x-y+z=1\\
x-y+3z=0\\
-x+2y-8z=1\\
\end{aligned}\right.$
\item $\left\lbrace \begin{aligned}
2x-y+z=1\\
x-y+3z=0\\
-x+2y-8z=1\\
\end{aligned}\right.$
\item \sisteq{
3x-y+6z=0\\
5x+y-6z=1\\
7x-4y+z=2\\
}

\end{enumerate}
\item Demostrar que el conjunto $\mathbb{Z}_5$ representa un campo escalar con la suma módulo 5.
\item Construya de ser posible un sistema de ecuaciones son 4 incógnitas, 4 ecuaciones, con solución única.
\item Construya de ser posible un sistema de ecuaciones son 3 incógnitas, 4 ecuaciones, con solución única.
\item Construya de ser posible un sistema de ecuaciones son 2 incógnitas, 4 ecuaciones, con solución única.

\end{enumerate}


%espacios vectoriales

\chapter{Espacios Vectoriales}
\section{Espacio Vectorial}
\begin{dfn}
Sea $V$ un conjunto no vac\'io, en el cual se definen dos operaciones binarias, una binaria interna llamada suma $\oplus$ y otra binaria externa, sobre un campo $\mathbb{K}$, llamada producto por escalar $\odot$. Se dice que $\left( V,\oplus ,\odot \right) $ es un espacio vectorial sobre el campo $\mathbb{K}$ si cumple:
\end{dfn}

\begin{table}[htbp]
\begin{center}
\begin{tabular}{|l|l|}
\hline
\multicolumn{2}{|l|}{Operación suma } \\
\hline \hline
$\forall v_1,v_2 \in V:\ (v_1 \oplus v_2 )\in V$ &[Cerradura de la suma] \\ \hline
$\forall v_1,v_2\in V : v_1\oplus v_2=v_2\oplus v_1  $&[Conmutatividad]  \\ \hline
$\forall v_1,v_2,v_3\in V:(v_1\oplus v_2 )\oplus v_3=v_1\oplus (v_2\oplus v_3 )$&[Asociativa] \\ 
\hline

$\exists n_V\in V\ \forall v\in V  :  v\oplus n_V=v$&  [elemento neutro]  \\ \hline
$\forall v\in V\ \exists v'\in V: v\oplus v'=n_V$&  [elemento inverso]  \\ \hline
\multicolumn{2}{|l|}{Operación producto escalar } \\
\hline \hline
 $\forall v \in V \forall \alpha \in \mathbb{K} :\ (\alpha \odot v )\in V$& [Cerradura de producto] \\ \hline
$ \forall \alpha \in \mathbb{K}\ \forall v_1,v_2\in V: \alpha \odot (v_1\oplus v_2 )=(\alpha \odot v_1)\oplus (\alpha \odot v_2) $&  [Distributiva]  \\ \hline
$\forall \alpha ,\beta \in \mathbb{K}\ \forall v\in V : (\alpha +\beta )\odot v=(\alpha \odot v)\oplus(\beta \odot v)  $&  [Distributiva]  \\ \hline
$\forall \alpha ,\beta \in \mathbb{K}\ \forall  v\in V : (\alpha  \beta )\odot v=\alpha \odot (\beta \odot v) $&  [Distributiva]  \\ \hline
$\forall  v\in V    :   n_K \odot v=v $&  [Neutro multiplicativo]  \\ \hline

\end{tabular}
\caption{Axiomas de un espacio vectorial.}
\label{tabla:sencilla}
\end{center}
\end{table}

Veamos algunos ejemplos de espacios vectoriales conocidos:
~\\
Consideremos $V=\mathbb{R}^n$ y $\mathbb{K}=\mathbb{R}$, con la suma y producto por escalar usuales $V$ es un espacio vectorial sobre los Reales.
~\\
$V=\mathcal{P}_n$
y $\mathbb{K}=\mathbb{R}$
Los polinomios con coeficientes reales de grado menor o igual a n, con las
operaciones usuales de suma y producto por escalar.

~\\
$V=\mathcal{M}_{mxn}$
y $\mathbb{K}=\mathbb{R}$
Las matrices m por n con entradas reales con las operaciones usuales
de suma y producto por escalar forman un espacio vectorial sobre los reales
~\\\\
$V=\mathbb{C}^n$
y $\mathbb{K}=\mathbb{R}$
Las n-tuplas con entradas complejas con la suma y el producto por escalar
usuales en los complejos forman un espacio vectorial sobre el campo de los Complejos.
~\\\\
$V=\mathbb{B}^n$
y $\mathbb{K}=\mathbb{B}$
Las n-tuplas con entradas binarias con la suma y el producto por escalar
binarios forman un espacio vectorial sobre el campo binario. Éste es un ejemplo de espacio
vectorial discreto ya que la cardinalidad del conjunto es
~\\\\
$V=\mathcal{C}[a,b]$
y $\mathbb{K}=\mathbb{R}$
Las funciones continuas en un intervalo [a, b] con las operaciones
usuales de suma y producto por escalar de funciones forman un Espacio Vectorial sobre los
Reales


\section{Espacios Vectoriales reales}

\begin{dfn}[Espacio vectorial real]
Un espacio vectorial real es una cuarteta $(V, \dobler, +, \odot)$ donde V es un conjunto no vacío , $\dobler$ es el campo de los números reales, + es una operación en V llamada suma o adición y $\odot$ es una operación en V llamada multiplicación por un escalar. Los escalares son los elementos de $\dobler$ y que cumplen con los diferentes axiomas. \index{Espacio Vectorial}


\end{dfn}
\subsubsection*{Axiomas para la suma}
\begin{enumerate}
\item Si $u$ y $v$ son elementos de $V$, $u+v$ es un elemento de V.
\item Si $u$, $v$, y $w$ son elementos de $V$, entonces $u+v=v+u$. Es decir, la suma es una operación conmutativa.
\item Si $u$, $v$, y $w$ son elementos de $V$, entonces  $(u+v)+w=u+(v+w)$. Es decir, la operación es asociativa.
\item Existe un elemento de $V$, llamado elemento nulo y denotado por $\mathbf{0}_v$, tal que para todo $u$ en $V$, $u + \mathbf{0}_v = \mathbf{0}_v +u = u$.
\item Para cada $u$ en $V$ existe un único elemento en $V$, llamado $-u$, tal que $u+(-u)=\mathbf{0}_v$.
\end{enumerate}

\subsubsection*{Axiomas para la multiplicación por escalares}
\begin{enumerate}
\item Si $\alpha$ es un número real y $u$ es un elemento de $V$, entonces $\alpha u$ es un elemento de $V$.
\item Si $\alpha$ es un número real y $u$ y $v$ son elementos de $V$, entonces $\alpha (u+v) = \alpha u + \alpha v$. Es decir la multiplicación por un escalar es distributiva con respecto a la suma de vectores.
\item Si $\alpha$ y $\beta$ son números reales y $u$ es un elemento de $V$, entonces $(\alpha + \beta)u = \alpha u + \beta u$. Es decir la multiplicación por un escalar es distributiva con respecto a la suma de escalares.
\item Si $\alpha$ y $\beta$ son números reales y $u$ es un elemento de $V$, entonces $(\alpha \beta)u = \beta(\alpha u)$.
\item Si $u$ es un elemento de $V$, entonces $1 \odot u = u$.

\end{enumerate}

\begin{trabajoautonomo}
Sea $(E, +, \odot)$ un espacio vectorial real. Pruebe que
\begin{enumerate}
\item El elemento neutro $\mathbf{0}_E$ es único.
\item Para cada $e$ en $E$, el elemento $-e$ es único.
\item Para cada $e$ en $E$, $\mathbf{0}_E \odot e = 0$.
\item Para cada $e$ en $E$, $(-1)e = -e$.
\item Para cada $\lambda$ en $\mathbb{R}$, $\lambda \odot \mathbf{0}_E = \mathbf{0}_E$.
\item Si $\alpha v = \mathbf{0}_E$, entonces $\alpha = 0$ o $v=\mathbf{0}_E$.

\end{enumerate}
\end{trabajoautonomo}
\begin{ejemplo}

El espacio \rdos \ con las operaciones $(x,y)+(w,z) = (x+w,y+z)$ y la multiplicación por un escalar $\lambda$, $\lambda (x,y) = (\lambda x,\lambda y)$.

\end{ejemplo}

\begin{ejemplo}
El espacio de todos los polinomios de grado menor o igual que $n$, para un número natural $n$ fijo, con la operación normal de suma de polinomios y la multiplicación usual de un polinomio por un número real.
\end{ejemplo}

\begin{ejemplo}
El espacio de todas las matrices $n \times m$ con la operación usual de suma de matrices y multiplicación de una matriz por un número real.(Aquí $n$ y $m$ son números naturales fijos y distintos de cero).


\end{ejemplo}



\section{Subespacios Vectoriales}
\begin{dfn}
Sea $(E, +, \odot)$ un espacio vectorial real y $S$ un subconjunto no vacío de $E$. Entonces se dice que $S$ es un subespacio vectorial de $E$, si con las operaciones heredadas de $E$, $(S, +, \odot)$ es también un espacio vectorial.
\index{Subespacio Vectorial}
\end{dfn}


\begin{ejemplo}
Sea $E = \rtres$ y $S = \llaves{(x, y, z) \in \rtres}{5x + 2y + z = 0}$ entonces $S$ es un subespacio de \rtres.
\end{ejemplo}

\begin{ejemplo}
Sea $E$ el espacio \mdosxdos \ de todas las matrices cuadradas $2 \times 2$ con las operaciones usuales de suma de matrices y de multiplicación por un escalar y $S$ el conjunto de las matrices diagonales, es decir las matrices de la forma \matrdxd{a & 0}{0 & b}, entonces $S$ es un subespacio de $E$.

\end{ejemplo}

\begin{ejemplo}
Sea $E$ el espacio \ptres \ de todos los polinomios de grado menor o igual a 3, con la suma y multiplicación por un escalar usuales. Sea $S$ el conjunto de los polinomios \pdos \ de grado menor o igual a 2, entonces $S$ es un subespacio de $E$.
\end{ejemplo}

\begin{ejemplo}

Sea $E$ un espacio vectorial real, $V$ y $W$ dos subespacios vectoriales de $E$, entonces $V + W =\llaves{a+b}{a \in V \, , \, b \in W}$ y $V \cap W = \llaves{a}{a \in V \cap W}$ son también subespacios vectoriales reales.
\end{ejemplo}



\subsection{Subespacio trivial}
Si tenemos un espacio Vectorial $V$, elijamos el subconjunto que s\'olo contiene al neutro
$W=\lbrace n_v\rbrace$:
~\\	
\begin{itemize}
 \item W es un subconjunto no vac\'io de V.
\item Como mostramos en la sección anterior, $W$ es siempre un espacio Vectorial (Denominado espacio trivial).
 \end{itemize} 
Por lo tanto $W$, el conjunto que s\'olo contiene al vector neutro de un espacio vectorial V es siempre un Subespacio Vectorial de $V$.
\subsubsection{Todo espacio es subespacio de si mismo}
Ahora, del espacio vectorial V, consideremos al conjunto V (No, no es error de escritura)
\begin{itemize}
\item $V$ siempre es un subconjunto no vac\'io de $V$. (Recuerde que todo conjunto A es subconjunto de s\'i mismo).
\item $V$ es un espacio vectorial.
\end{itemize}

Por lo tanto cualquier espacio vectorial es subespacio vectorial de s\' mismo.

~\\
Todo espacio vectorial $V$ SIEMPRE tiene, por lo menos, dos subespacios vectoriales $V$ y $\lbrace n_v\rbrace$. Aquellos subespacios diferentes a los anteriores reciben el nombre de Subespacios Propios.
~\\
\subsection{Subespacios elementales}
\subsubsection{Subespacios de $\mathbb{R}^2$}
En $\mathbb{R}^2$, el espacio vectorial $bidimensional$ sobre el campo real; se distinguen, aparte de los subespacios propios, otros subespacios vectoriales:


\begin{itemize}
\item Toda recta que pasa por el origen es un subespacio vectorial de $\mathbb{R}^2$

\end{itemize}
\subsubsection{Subespacios de $\mathbb{R}^3$}
Y en $\mathbb{R}^3$, el espacio vectorial tridimensional sobre el campo real; se distinguen, aparte de los subespacios propios, otros subespacios vectoriales:

\begin{itemize}
\item Toda recta que pasa por el origen es un subespacio vectorial de $\mathbb{R}^3$
\item Todo plano que contiene al origen es un subespacio vectorial de $\mathbb{R}^3$
\end{itemize}

En principio para probar esto se necesitaría probar los 10 axiomas de la definición de espacio vectorial, sin embargo existe un teorema que nos ahorra bastante trabajo.


\newpage
\subsection{El teorema de caracterización del subespacio}

\begin{theorem}[Caracterizaci\'on de subespacio]
Sea $\left(V, \oplus, \odot\right)$ un espacio vectorial sobre el campo. Sea $W$ un subconjunto de $V$. $W$ es un subespacio vectorial de $V$ si y solo si:
	
	\begin{itemize}
	\item $W$ no es vac\'io
	\item $\forall w_1, w_2 \in W: w_1\oplus w_2 \in W$
	\item $\forall \alpha \in \mathbb{K} \forall w \in W: \alpha \odot w \in W$
	
	\end{itemize}

\end{theorem}

La prueba de este teorema se la dejamos al lector, indic\'andoles que deben probar que los restantes 8 axiomas de espacios vectoriales se satisfacen a partir de estos axiomas de cerradura y de la hip\'otesis que $W$ es un subconjunto de $V$, con las mismas operaciones de V
\\
\\
Con este teorema demostraremos las afirmaciones sobre los subespacios de $\mathbb{R}^2$ y 
$\mathbb{R}^3$.

\begin{theorem}
 Toda recta que pasa por el origen es un subespacio vectorial de \rdos

\end{theorem}
\begin{proof}

El enunciado algebraicamente se traduce en que $$\svrdos{x}{y}{y=mx}$$
es un subespacio vectorial de \rdos.
\begin{enumerate}


\item[(i)] \vectrdos{0}{0} $\in W$ ya que $0=m\cdot 0$, por lo tanto $W\neq \phi$ 

\item[(ii)] $\forall w_1, w_2 \in W: w_1=$ 
\vectrdos{x_1}{y_1} $w_2=$\vectrdos{x_2}{y_2}

\[
w_1+w_2=\mbox{\vectrdos{x_1}{y_1}}+\mbox{\vectrdos{x_2}{y_2}}=
\mbox{\vectrdos{x_1+x_2}{y_1+y_2}}
\]

para que el $vector\ suma$ pertenezca a W debe cumplir la condici{\'o}n de W:
Por hip{\'o}tesis $w_1 \in W $ por lo que $y_1=mx_1$. Asimismo $w_2 \in W$ implica que
$y_2=mx_2$

\[
\begin{array}{c}

y_1+y_2=mx_1+mx_2
\\
y_1+y_2=m\left(x_1+x_2\right)
\end{array}
\]
Por lo que el vector suma pertenece a $W$

\item[(iii)] $\forall \alpha \in \mbox{\dobler}, \forall w \in W: w=$\vectrdos{x}{y}
\[
\alpha\cdot w=\alpha\mbox{\vectrdos{x}{y}}=
\mbox{\vectrdos{\alpha x}{\alpha y}}
\]
Por hip\'otesis $w\in W$ por lo que $y=mx$

\[\begin{array}{c}
\alpha y=\alpha\left(mx\right)
\\
\alpha y=m\left( \alpha x\right)
\end{array}
\]

Por lo que el $vector\ producto\ \alpha w=$\vectrdos{\alpha x}{\alpha y} pertenece a $W$.

\end{enumerate}
Ya que hemos probado $\left( i\right)\, \left(ii\right), \left(iii\right),$ que son las condiciones en el teorema de caraterizaci\'on de subespacio, por el mismo teorema concluimos que $W$ es un subespacio de \rdos. Lo cual prueba que toda recta que pasa por el origen es un subespacio vectorial de \rdos.

\end{proof}


\begin{theorem}
En \rtres \ toda recta que pasa por el origen es un subespacio vectorial de \rtres.
\end{theorem}
\begin{proof}

Esto es equivalente a decir que el subconjunto $$ \svrtres{W}{x}{y}{z}{x=at, y=bt, z=ct ;t\in \dobler}$$ es un subespacio vectorial de \rtres.

\begin{enumerate}
\item[(i)] \vectrtres{0}{0}{0} $\in W \left( \mbox{cumple las tres condiciones} \right)$
por tanto $W$ no es vac\'io.

\item[(ii)] $\forall w_1, w_2 \in W$ $w_1=$\vectrtres{x_1}{y_1}{z_1} , $w_2=$\vectrtres{x_2}{y_2}{z_2}

\[
w_1+w_2=\mbox{\vectrtres{x_1}{y_1}{z_1}}+\mbox{\vectrtres{x_2}{y_2}{z_3}}
=\mbox{\vectrtres{x_1+x_2}{y_1+y_2}{z_1+z_2}}
\]
Tanto $w_1, w_2, w_3$ son elementos de W, por lo que
\[
x_1=at_1,\ y_1=bt_1,\ z_1=ct_1
\]
\[
x_2=at_2,\ y_2=bt_2,\ z_2=ct_2
\]
De esto se obtiene que:
\[
\begin{array}{ccc}
x_1+x_2=at_1+at_2&y_1+y_2=bt_1+bt_2
&z_1+z_2=ct_1+ct_2
\\
x_1+x_2=a\left(t_1+t_2\right)&
y_1+y_2=b\left(t_1+t_2\right)&
z_1+z_2=c\left(t_1+t_2\right)
\end{array}
\]

Por tanto el vector $w_1+w_2$ pertenece a $W$.

\item[(iii)] $\forall \alpha \in \mbox{\doblek}
, \forall w \in W: w=$\vectrtres{x}{y}{z}
\[
\alpha w=\alpha \mbox{\vectrtres{x}{y}{z}}=
\mbox{\vectrtres{\alpha x}{\alpha y}{\alpha z}}\]

Por hip\'otesis, $x=at, y=bt, z=ct$:

\[
\begin{array}{ccc}
\alpha x=\alpha \left(at\right)&
\alpha y=\alpha \left(bt\right)&
\alpha z=\alpha \left(ct\right)
\\
\alpha x=a\left(\alpha t\right)&
\alpha y=b\left(\alpha t\right)&
\alpha z=c\left(\alpha t\right)
\end{array}
\]

\end{enumerate}
~\\
Por el teorema del subespacio, concluimos que $W$ es un subespacio 
vectorial de \rtres.

\end{proof}


\begin{theorem}[Intersección de subespacios vectoriales]
Sean $V_1$ y $V_2$ dos subespacios vectoriales de un espacio vectorial E, entonces $V_1 \cap V_2$ es un subespacio vectorial
\end{theorem}
\begin{proof}
~\\ \begin{enumerate}[i.]
\item El cero vector de E pertenece a ambos subespacios, por lo tanto también peertenece a la intersección.
\item Si $v_1$ y $v_2$ pertenecen a $V_1 \cap V_2$ entonces ambos pertenecen a $V_1$ y $V_2$. Debido a que $V_1$ es un subespacio vectorial, entonces $v_1 +v_2$ pertenece a $V_1$. Por el mismo argumento para $V_2$, se tiene que $v_1+v_2$ pertenece a $V_2$. Por lo tanto $v_1+v_2$ pertenece a $V_1\cap V_2$.
\item Si $v$ pertenece a  $V_1 \cap V_2$, entonces  pertenece a $V_1$ y a $V_2$.  Considere $\alpha \in \dobler$,  debido a que $V_1 y V_2$ son subespacios vectoriales entonces $\alpha v$ pertenece a $V_1$ y $\alpha v$ pertenece a $V_2$, es decir $\alpha v$ pertenece a $V_1 \cap V_2$.  
\end{enumerate}
\end{proof}

\section{Problemas}
\subsection{Espacio vectorial}
\begin{enumerate}[1.]
\item Sea $G=\llav{g}$ un conjunto de un elemento en el cual se definen las siguientes operaciones:
\begin{align*}
\forall g \in G\ \  &g+g=g\\
\forall \alpha\in \dobler\ \ &\alpha g=g \\
\end{align*}
Determine si $G$, con las operaciones definidas anteriormente es un espacio vectorial.

\item Muestre que el conjunto de funciones continuas períodicas, con periodo múltiplo entero de $\pi$, forman un espacio vectorial sobre el campo real son la suma y producto por escalar convencionales de las funciones.

\end{enumerate}
.


%espacio generado combinación lineal
\chapter{Combinaciones Lineales}

\section{Combinación Lineal}
\begin{dfn}[Combinación Lineal]
Sea $v\in V$, sean \conjvect{v}{n} vectores de un espacio vectorial $V$ sobre el campo \doblek , se dice que $v$ es una combinación lineal de los vectores \conjvect{v}{n} si y solo si existen los escalares $\alpha_1, \alpha_2, \ldots, \alpha_n \in \dobleK$ tal que 
\[
v=\left(\alpha_1\odot v_1\right)\oplus
\left(\alpha_2\odot v_2\right)\oplus
\ldots \oplus
\left(\alpha_n\odot v_n\right)
\]
\end{dfn}
Nótese que la palabra clave de la definición anterior es EXISTEN, para ilustrar esto considere los
dos siguientes ejemplos:
\begin{ejemplo}
Considere el espacio vectorial \rdos; ?` es el vector \vectrdos{6}{-1} combinación lineal de los vectores \vectrdos{3}{2}, \vectrdos{1}{-1}?
\end{ejemplo}

Para resolver este problema debemos verificar si EXISTEN escalares $\alpha$, $\beta$ tales que hagan posible la combinación lineal:
\[\mbox{\vectrdos{6}{-1}} =\alpha\mbox{\vectrdos{3}{2}}+\beta\mbox{\vectrdos{1}{-1}}\]

Esto nos llevar\'a a un sistema de ecuaciones
\[
\left\{
\begin{array}{l}
3\alpha+\beta=\ 6\\
2\alpha-\beta=-1
\end{array}
\right.
\]

Si el sistema es consistente, entonces los escalares si existen y por lo tanto, dicho vector es combinación lineal de los otros dos, pero; si el sistema es inconsistente, los escalares no existen
por lo que el vector no sería combinación lineal de los otros dos.

~\\
Al resolver este sistema llegamos a la solución:
$\alpha=1$ y $\beta=3$.

~\\
Los escalares $\alpha$ y $\beta$ existen, el sistema es consistente con solución \'unica, luego el vector \vectrdos{6}{-1} es combinación lineal de los vectores \vectrdos{3}{2}, \vectrdos{1}{-1}. Ver figura ~\ref{fig:figura1}

\imagen{dosvect1}{combinación lineal de vectores en \rdos. Método del paralelogramo}{figura1}{0.5}

\begin{ejemplo}
?`El vector \vectrdos{2}{2} es combinación lineal de los vectores 
\vectrdos{-1}{2}, \vectrdos{2}{-4}?
\end{ejemplo}


Para que sea combinación lineal deben existir los escalares
$\alpha$ y $\beta$ tales que:

\[\mbox{\vectrdos{2}{2}}=\alpha\mbox{\vectrdos{-1}{2}}+
\beta\mbox{\vectrdos{2}{-4}}\]
~\\
Resolviendo el sistema por el metodo de Gauss-Jordan:

\[
\left(
\begin{array}{rr|r}
-1 & 2 & 2\\
2&-4&2
\end{array}
\right)
\underrightarrow{2f_1 + f_2}
\left(
\begin{array}{rr|r}
-1 & 2 & 2\\
0&0&6
\end{array}
\right)
\]
~\\
La última fila indica un absurdo matem\'atico, lo cual implica que el sistema es inconsistente, no
tiene solución. Como los escalares no existen, el vector \vectrdos{2}{2} NO es combinación lineal de los vectores 
\vectrdos{-1}{2}, \vectrdos{2}{-4}

\newpage
%\subsection{?`Espacio de colores?}
%Uno de los m\'etodos de explicar la combinación lineal con un ejemplo inteligible son los colores:
%
%~\\
%Suponga que usted tiene un conjunto de bolitas de plastilina con los colores Azul, Rojo, Amarillo.
%Tomamos una “bolita” amarilla y otra “bolita” roja y las combinamos (esperamos que el lector conozca qué color se formar\'a), como era de suponerse, tendremos ahora una bolita de color Naranja; es decir, el color Naranja es una Combinación Lineal de los colores Amarillo y Rojo.
%
%\imagen{amarillorojo}{combinación lineal}{figura2}{0.5}
%
%Asimismo si tomamos una bolita de color Azul y una bolita de color Amarillo, obtendremos una
%de color Verde, luego el color Verde es Combinación Lineal de los colores Azul y Amarillo.
%
%\imagen{azulamarillo}{combinación lineal}{figura3}{0.5}
%
%Ahora suponga que no toma una bolita de Azul, sino TRES bolitas de Azul y una de Amarillo; y las
%combinamos, ¡por supuesto! Tendremos otra clase de color Verde, pero un poco más oscuro, es
%decir la cantidad de bolitas son los ESCALARES de la combinación lineal y si los escalares
%cambian tendremos otro resultado.
%
%
%\imagen{verdeoscuro}{combinación lineal}{figura4}{0.4}
%
%?`Ahora est\'a más claro el concepto de combinación lineal?

\subsection{M\'as ejemplos de combinación lineal}

De ahora en adelante trabajaremos sobre espacios vectoriales sobre campo Real y con operaciones usuales, por lo que obviaremos $\oplus$ y $\odot$ por los usuales $+$ y $\cdot$

\begin{ejemplo}
Considere el espacio vectorial \pdos\ sobre el campo real, determinar si el vector $x^2+x-1$ es combinación lineal de los vectores $x+2, x^2-3, x^2+2x+1$.
\end{ejemplo}
~\\
solución

~\\
Veamos si existen los escalares $\alpha_1, \alpha_2, \alpha_3$ tales que
\[x^2+x-1=\alpha_1\left(x+2\right)+
\alpha_2\left(x^2-3\right)+
\alpha_3\left(x^2+2x+1\right)
\]

~\\
Dos polinomios son iguales si y solos si cada uno de sus coeficientes lo son, por tanto:

~\\
\[x^2+x-1=\left(\alpha_2+\alpha_3\right)x^2+
\left(\alpha_1+2\alpha_3\right)x+
\left(2\alpha_2-3\alpha_2+\alpha_3\right)
\]

Al igualar coeficientes, se origina el siguiente sistema de ecuaciones:

\begin{eqnarray*}
\alpha_2+\alpha_3=1\\
\alpha_1+2\alpha_3=1\\
2\alpha_1-3\alpha_2+\alpha_3=-1\\
\end{eqnarray*}

Resolviéndolo por Gauss:

\[
\reducir{rrr|r}{
0 &1 &1 &1\\
1&0&2&1\\
2 &-3&1&-1
}
\underrightarrow{\begin{array}{c}
    -2f_2 + f_3\\
    f_2 \longleftrightarrow f_1
\end{array}}
\reducir{rrr|r}{
1 &0 &2 &1\\
0&1&1&1\\
0 &-3&-3&-3
}
\underrightarrow{3f_2 + f_3}
\reducir{rrr|r}{
1 &0 &2 &1\\
0&1&1&1\\
0 &0&0&0
}\]

De la última fila del sistema inferimos que el sistema tiene infinitas soluciones, es decir; es consistente, que es lo que nos interesa. Como es consistente entonces EXISTEN escalares (en este caso infinitos) que hacen posible la combinación lineal.

\begin{ejemplo}
En el espacio vectorial \mdosxdos, determinar si 
\matrdxd{2&1}{-1&0}
 es combinación lineal de los vectores 
\matrdxd{1&0}{-1&0}, \matrdxd{-1&2}{0&1}, \matrdxd{0&1}{2&-1}.
\end{ejemplo}


\[\matrdxd{2&1}{-1&0}=
\alpha_1\matrdxd{1&0}{-1&0}
+{\alpha_2}{\matrdxd{-1&2}{0&1}}
+{\alpha_3}{\matrdxd{0&1}{2&-1}}
\]

Dos matrices son iguales si y solo si cada uno de sus coeficientes
lo son, por lo que resolver la ecuación anterior es equivalente a resolver el sistema de ecuaciones siguiente:
\begin{align*}
\alpha_1+\alpha_2=&2\\
2\alpha_2+\alpha_3=&1\\
-\alpha_1+2\alpha_3=&-1\\
\alpha_2-\alpha_3=&0\\
\end{align*}
Por el método de Gauss obtenemos un sistema equivalente por filas
\[
\reducir{rrr|r}{
1 &-1 &0 &2\\
0&2&1&1\\
-1&0&2&-1\\
0&1&-1&0
}
\underrightarrow{f_1 + f_3}
\reducir{rrr|r}{
1 &-1 &0 &2\\
0&2&1&1\\
0&-1&2&1\\
0&1&-1&0
}
\underrightarrow{\begin{array}{c}
    f_2 + 2f_3\\
    f_2 - 2f_4 
\end{array}}
\reducir{rrr|r}{
1 &-1 &0 &2\\
0&2&1&1\\
0&0&5&3\\
0&0&3&1
}\]

~\\
En las dos últimas filas se obtiene una inconsistencia matem\'atica, ya que en la tercera tenemos que 
$\alpha_3=\frac{3}{5}$ 
pero en la cuarta fila 
$\alpha_3=\frac{1}{3}$
lo cual es una contradicción. Por lo tanto el sistema es inconsistente, no existen escalares que satisfagan dicho sistema. Por tanto \matrdxd{2&1}{-1&0} NO
 es combinación lineal de los vectores \matrdxd{1&0}{-1&0}, \matrdxd{-1&2}{0&1}, \matrdxd{0&1}{2&-1}.

~\\
\begin{ejercicio}
Determinar de ser posible los valores de $k$ para los cuales el vector 
\vectrtres{1}{1}{k} es una combinación lineal de los vectores
 \vectrtres{1}{2}{k}, \vectrtres{k}{-k}{1}.
\end{ejercicio}
~\\
\underline{solución}:
\\
analicémoslo de la misma manera y planteemos la combinación lineal
\[
\vectrtres{1}{1}{k}=
 \alpha_1\vectrtres{1}{2}{k}+\alpha_2\vectrtres{k}{-k}{1}
\]
Para que la combinación lineal exista el sistema debe ser consistente. Realizando operaciones elementales tenemos:

\[
\reducir{rr|r}{
1 &k&1\\
2&-k&1\\
k&1&k
}
\underrightarrow{\begin{array}{c}
    -2f_1 + f_2\\
    -kf_1 + f_3 
\end{array}}
\reducir{rr|r}{
1 &k&1\\
0&-3k&-1\\
0&1-k^2&0
}
\underrightarrow{(1-k^2)f_2 - (-3k)f_3 }
\reducir{rr|l}{
1 &k&1\\
0&-3k&-1\\
0&0&k^2-1
}
\]

Atención a las dos últimas filas, para que el sistema sea consistente el valor de $k^2-1$ debe ser igual a cero (si es diferente de cero: $k^2-1=m, m\neq 0$ el sistema es inconsistente ya que quedar\'ia una ecuación $0\alpha_1+0\alpha_2=m$\, lo cual es un absurdo), es decir es consistente cuando:
\[k^2-1=0  \longrightarrow  k=-1 \vee k=1\]

Debemos tener en cuenta también que $k\neq 0$
ya que con eso el sistema es inconsistente en la
segunda fila, pero esto no afecta los valores obtenidos ya que ninguno de ellos es cero.


Por lo tanto para que \vectrtres {1}{1}{k} sea combinación lineal de \vectrtres{1}{2}{k}, \vectrtres{k}{-k}{1}; $k=-1 \vee k=1$.

\newpage
\section{Cápsula Lineal}
\begin{dfn}[Cápsula Lineal]
Sean \conjvect{v}{k} k vectores de un espacio vectorial $(V,\dobleK,\oplus,\odot)$, se denomina Cápsula Lineal de \conjvect{v}{k} al conjunto de todas las combinaciones lineales posibles de los vectores \conjvect{v}{k}
\[
\mathcal{L} \left(\kvect{v}{k}\right)=
\lbrace v \in V|v= \av{}{\alpha_1 \odot v_1} \oplus
\av{}{\alpha_2 \odot v_2} \oplus 
\hdots \oplus
\av{}{\alpha_k \odot v_k}
\rbrace
\]

\end{dfn}

\begin{theorem}
Sean \conjvect{v}{k} vectores de un espacio vectorial $(V,\dobleK,\oplus,\odot)$, entonces la cápsula lineal de \conjvect{v}{k} es un subespacio vectorial de $V$
\end{theorem}
\begin{proof}~\\
\begin{enumerate}
\item Puesto que $0_V= \av{}{0 \odot v_1} \oplus
\av{}{0 \odot v_2} \oplus 
\hdots \oplus
\av{}{0 \odot v_k}$ se tiene que $0_V$ es un elemento de la cápsula lineal de \conjvect{v}{k}
\item Sean $w_1, w_2$ dos elementos de la cápsula lineal de \conjvect{v}{k}, entonces existen escalares $\alpha_i, \beta_i \in \dobleK$ tales que \begin{align*}
w_1=\av{}{\alpha_1 \odot v_1} \oplus
\av{}{\alpha_2 \odot v_2} \oplus 
\hdots \oplus
\av{}{\alpha_k \odot v_k}\\
w_2=\av{}{\beta_1 \odot v_1} \oplus
\av{}{\beta_2 \odot v_2} \oplus 
\hdots \oplus
\av{}{\beta_k \odot v_k}\\
\end{align*}
Ahora como
\begin{align*}
w_1\oplus w_2=\av{}{(\alpha_1 + \beta_1) \odot v_1} \oplus
\av{}{(\alpha_2 + \beta_2) \odot v_2} \oplus 
\hdots \oplus
\av{}{(\alpha_k +\beta_k) \odot v_k}
\end{align*}
Concluimos que $w_1\oplus w_2$ es un elemento de la cápsula lineal de \conjvect{v}{k}.

\item Sean $c \in \dobleK$ y $w$ un elemento de la cápsula lineal de \conjvect{v}{k}. Entonces 
\begin{align*}
c\odot w&=c\odot (\av{}{\alpha_1 \odot v_1} \oplus
\av{}{\alpha_2 \odot v_2} \oplus 
\hdots \oplus
\av{}{\alpha_k \odot v_k})\\
c\odot w&=\av{}{(c\alpha_1) \odot v_1} \oplus
\av{}{(c\alpha_2) \odot v_2} \oplus 
\hdots \oplus
\av{}{(c\alpha_k) \odot v_k}\\
\end{align*}
Y así, $c\odot w$ es un elemento de la cápsula lineal de \conjvect{v}{k}.
\end{enumerate}
Por lo tanto la cápsula lineal es un subespacio de V.
\end{proof}


~\\
Por esto a la cápsula lineal se le denomina espacio generado.

\section{Conjunto Generador y Espacio Generado}(?`Qué?\ ?`no es lo mismo?)


\begin{dfn}
Sea $S=$\conjvect{v}{k} un subconjunto de un espacio vectorial $V$, se denomina Espacio Generado por S al conjunto de todas las combinaciones posibles de los vectores de S. Dado esto, el conjunto S se denomina Conjunto Generador. Se denota:
\[gen(S)=
\lbrace v \in V|v=\av{}{\alpha_1 \odot v_1} \oplus
\av{}{\alpha_2 \odot v_2} \oplus 
\hdots \oplus
\av{}{\alpha_k \odot v_k}
\rbrace
\]
al espacio generado por S.
\end{dfn}

\begin{ejemplo}
Halle el espacio generado por el subconjuto S de \pdos; $S=\lbrace
x+1, x^2-1, 2x^2+x-1\rbrace$. Luego, determine si 
\begin{enumerate}
\item[(a)]$x^2-2x+5 \in gen(S)$
\item[(b)]$3x^2+4x+1 \in gen(S)$
\end{enumerate}
\end{ejemplo}

~\\
\underline{solución}
~\\
Por definición $gen(S)=
\lbrace v \in V|v=
\beta_1 v_2+
\beta_2 v_2+
\ldots+
\beta_k v_k
\rbrace
$

~\\
Elegimos ahora un vector típico de \pdos, que es el espacio referencial
\[p(x)=a+b x+c x^2\]
$$\forall p(x)\in gen(S):\  
a+b x+c x^2=
\av{\beta_1}{x+1}+
\av{\beta_2}{x^2-1}+
\av{\beta_3}{2x^2+x-1}
$$
Igualando coeficientes llegamos al sistema de ecuaciones
\begin{eqnarray*}
\beta_1-\beta_2-\beta_3=a\\
\beta_1+\beta_3=b\\
\beta_2+2\beta_3=c
\end{eqnarray*}

\[
\reducir{rrr|c}{
1&-1&-1&a\\
1&0&1&b\\
0&1&2&c
}
\underrightarrow{-f_1 + f_2}
\reducir{rrr|c}{
1&-1&-1&a\\
0&1&2&b-a\\
0&1&2&c
}
\underrightarrow{f_2 - f_3}
\reducir{rrr|c}{
1&-1&-1&a\\
0&1&2&b-a\\
0&0&0&b-a-c
}
\]

\underline{Observación}: Atención aquí, las incógnitas de este sistema por naturaleza serían $\beta_1, \beta_2, \beta_3$
; sin embargo cuando vamos a determinar el espacio generado no nos interesa quienes sean los escalares, sólo nos interesa que existan, lo que si nos interesa son las condiciones que recaen sobre $a,b,c$
para que el sistema sea consistente (en otras palabras, para que los escalares existan).


En la última fila tenemos una fila de llena de ceros pero al final tenemos el factor $b-a-c$
analicémosla, si $b-a-c\neq 0$
,entonces el sistema no tendría solución (?`Por qué?), lo cual va
en contra de la definición de combinación lineal, que es esencial en la definición de espacio
generado, por lo que esto es lo que debe ser evitado. Para que el sistema sea consistente debe cumplirse que $b-a-c=0$
, y ésta será la condición de todos los vectores que estén en el
espacio generado por S.
\[gen(S)=
\lbrace
a+b x+c x^2 \in \pdos|
b-a-c=0
\rbrace
\]
\begin{enumerate}

\item[(a)] El vector $x^2-2x+5$ no pertenece a $gen(S)$\ ya que $(-2)-1-5=-8\neq 0$
\item[(b)] El vector $3x^2+4x+1 \in gen(S)$, ya que $4-3-1=0$
\end{enumerate}

\begin{ejemplo}
Determine el espacio generado por los vectores $\left\{ \vectrtres{1}{2}{1}, \vectrtres{3}{0}{2}, \vectrtres{3}{2}{0}\right\}$.
\end{ejemplo}

~\\
\underline{solución}:
~\\

Sea $v=\vectrtres{a}{b}{c}; a, b, c, \in \dobler$

\[
\forall v\in gen(S):\ 
\vectrtres{a}{b}{c}=
\beta_1\vectrtres{1}{2}{1}+
\beta_2\vectrtres{3}{0}{2}+
\beta_3\vectrtres{3}{2}{0}
\]

\[
\reducir{rrr|r}{
1&3&3&a\\
2&0&2&b\\
1&2&0&c
}
\underrightarrow{\begin{array}{cc}
    -2f_1 + f_2\\
    f_1 - f_3
\end{array}}
\reducir{rrr|l}{
1&3&3&a\\
0&-6&-4&-2a+b\\
0&1&3&a-c
}
\underrightarrow{f_2 + 6f_3}
\reducir{rrr|l}{
1&3&3&a\\
0&-6&-4&-2a+b\\
0&0&14&4a+b-6c
}
\]
Este sistema siempre ser\'a consistente para cualesquiera que sean los valores de $a, b, c$ por lo
tanto se dice que no hay condiciones sobre ellos, en otras palabras $a, b, c \in \dobler$; puede ser
cualquier real.
\[
gen(S)=
\left\{
\left.
\vectrtres{a}{b}{c} \in \rtres \right|
a, b, c \in \dobler
\right\}=\rtres
\]
Se dice que S genera a \rtres.

\begin{ejercicio}
Determine el valor de $k$ para que el vector $C=\matrdxd{2k&1}{-3&k}$ pertenezca al espacio generado por $S=\left\{\matrdxd{-1&2}{1&2}, \matrdxd{3&-1}{-2&0}, \matrdxd{-2&0}{3&1}\right\}$.
\end{ejercicio}
%%\sol
Hallemos el espacio generado por dichos vectores. Sea $A=\matrdxd{a&b}{c&d} \in gen(S)$.

\[\forall A\in gen(S):\ 
\matrdxd{a&b}{c&d}=
\beta_1\matrdxd{-1&2}{1&2}+ \beta_2\matrdxd{3&-1}{-2&0}+ \beta_3\matrdxd{-2&0}{3&1}
\]
\[
\left(
\begin{array}{rrr|r}
-1&3&-2&a\\
2&-1&0&b\\
1&-2&3&c\\
2&0&1&d
\end{array}
\right)
\sim
\ldots
\sim
\left(
\begin{array}{rrr|l}
-1&3&-2&a\\
0&5&-4&2a+b\\
0&0&-9&-3a+b-5c\\
0&0&0&-5a-5b-5c+5d
\end{array}
\right)
\]
Para que el sistema sea consistente debe cumplirse que $-5a-5b-5c+5d=0$. Por lo que el espacio generado por S es:
\[gen(S)=\llaves{
\matrdxd{a&b}{c&d}
}{d=a+b+c}
\]
Para que $C$\ pertenezca a $gen(S)$, debe cumplirse que
\[k=2k+1-3\ \rightarrow\ k=2\]





%independencia lineal
\chapter{Conjuntos Linealmente Independientes}

\section{Dependencia e Independencia Lineal}
\begin{dfn}[Independencia Lineal]
Sean \kvect{v}{k} vectores de un espacio vectorial $V$. El conjunto $S=\conjvect{v}{k}$\ es linealmente independiente si y sólo si la ÚNICA manera de obtener al vector neutro de V como combinación lineal de los vectores de S es que todos los escalares de la combinación lineal sean cero.
\[\alpha_1 v_1+
\alpha_2 v_2+
\ldots+
\alpha_k v_k =
n_v
\Rightarrow\ 
\forall i \leq k;
\ \alpha_i=0
\] 
\end{dfn}
\begin{dfn}[Dependencia Lineal]
Sean \kvect{v}{k} vectores de un espacio vectorial $V$. El conjunto $S=\conjvect{v}{k}$\ es linealmente dependiente si y sólo si se puede obtener al vector neutro de V como combinación lineal de los vectores de S, y existe un escalar de la combinación lineal diferente de cero.
\[\alpha_1 v_1+
\alpha_2 v_2+
\ldots+
\alpha_k v_k =
n_v
\wedge\ 
\exists i \leq k;
\ \alpha_i\neq 0
\] 
\end{dfn}
En otras palabras, un conjunto es linealmente dependiente si y solo si NO  es linealmente independiente.

~\\
Esta es, quiz\'as, una de las definiciones más confusas o menos inteligibles(pero a la vez una de las más importantes), as\'i que confiamos aclararla en los ejemplos siguientes.
~\\
\begin{ejemplo}
Determine si el conjunto de vectores de \rdos\  es linealmente independiente $\left\{
\vectrdos{1}{2}, \vectrdos{1}{-1}\right\}$.
\end{ejemplo}

%\sol
Para determinar si son linealmente independientes analicemos los escalares al hacer la combinación lineal igualada al neutro de \rdos
\[c_1\vectrdos{1}{2}+
c_2\vectrdos{1}{-1}=
\vectrdos{0}{0}\]
Esto deriva en un sistema de ecuaciones
$$\reducir{rr|c}{1&1&0\\2&-1&0}
\underrightarrow{2f_1 - f_2}
\reducir{rr|c}{1&1&0\\0&3&0}$$
La solución de este sistema es única $c_1=0,\ c_2=0$. El conjunto $\left\{
\vectrdos{1}{2}, \vectrdos{1}{-1}\right\}$ es linealmente independiente.


\begin{ejemplo}
Determine si el siguiente conjunto de vectores de \rtres\ es linealmente independiente \llav{
\vectrtres{2}{4}{1}, \vectrtres{-1}{2}{3}, \vectrtres{3}{2}{-2}
}
\end{ejemplo}
%\sol
Para determinar si el conjunto es linealmente independiente analicemos los escalares al hacer la combinación lineal igualada al neutro de \rtres.
\[
a_1\vectrtres{2}{4}{1}+
a_2\vectrtres{-1}{2}{3}+
a_3\vectrtres{3}{2}{-2}=
\vectrtres{0}{0}{0}
\]
Lo que deriva en un sistema homogéneo (Un sistema homogéneo es un sistema que está igualado todo a cero).
$$\reducir{rrr|r}{2&-1&3&0\\4&2&2&0\\1&3&-2&0}
\underrightarrow{\begin{array}{r}
    -2f_1 + f_2\\
    f_1 - 2f_3
\end{array}}
\reducir{rrr|r}{2&-1&3&0\\0&4&-4&0\\0&-7&7&0}
\underrightarrow{7f_2 + 4f_3}
\reducir{rrr|r}{2&-1&3&0\\0&4&-4&0\\0&0&0&0}$$

\obs\ Este es el "Talón de Aquiles" de muchos; es claro que el sistema tiene infinitas soluciones ya que la última fila se eliminó y nos quedó un sistema con menos ecuaciones que incógnitas, esto indica claramente que existen infinitas maneras de obtener al vectores neutro como combinación lineal de esos tres vectores, analice lo siguiente:

~\\
\textsl{Razonamiento Incorrecto: 
Como tiene infinitas soluciones con $a_1=0, a_2=0, a_3=0$\ se obtiene al vector neutro, por lo tanto el conjunto de vectores es linealmente independiente}

~\\
Lo anterior es erróneo en el sentido de afirmar que los vectores son linealmente independientes,
vayamos a la definición de independencia lineal para notar que lo esencial de la definición está en que DEBE SER LA ÚNICA FORMA DE OBTENER AL NEUTRO CUANDO TODOS LOS ESCALARES SEAN IGUALES A CERO.

~\\
En el ejemplo anterior no es la única de obtenerlo, hay infinitas, por ejemplo con $a_1=-1, a_2=1, a_3=1$
\[-1\cdot\vectrtres{2}{4}{2}+
1\cdot\vectrtres{-1}{2}{-2}+
1\cdot\vectrtres{3}{2}{-2}=
\vectrtres{0}{0}{0}
\]
No es la única manera cuando todos los escalares son ceros. Luego, el conjunto NO es linealmente independiente, es decir es linealmente dependiente.
\section{?`No hay una manera más sencilla?: Reducción por filas}
Es posible determinar si un conjunto es linealmente independiente colocándolos como filas de una matriz y reduciendo la matriz por el método de Reducción por filas de Gauss.

~\\
Veamos los siguientes ejemplos:
~\\

\begin{ejemplo}
Determine si el subconjunto de \rtres, $S=\llav{\vectrtres{4}{2}{3},\vectrtres{1}{3}{4}, \vectrtres{2}{-4}{-5}}$ es linealmente independiente.
\end{ejemplo}


%\sol

Empleamos el método de Reducción por filas de Gauss:
~\\
Colocamos los vectores como filas de una matriz las cuales vamos a reducir por el método de Gauss

\[\reducir{rrr}{
4&2&3\\
1&3&4\\
2&-4&-5
}\]
Reduciendo:
\[
\reducir{rrr}{
4&2&3\\
1&3&4\\
2&-4&-5
}
\underrightarrow{\begin{array}{c}
    f_1 - 4f_2\\
    f_1 - 2f_3 
\end{array}}
\reducir{rrr}{
4&2&3\\
0&-10&-13\\
0&10&13
}
\underrightarrow{f_2 - f_3}
\reducir{rrr}{
4&2&3\\
0&-10&-13\\
0&0&0
}
\]

Debido a que una de las filas, al emplear la Reducción, resultó llena de ceros, entonces el vector correspondiente a esa fila $\vectrtres{2}{-4}{-5}$ es combinación lineal de los vectores anteriores.
El conjunto es linealmente dependiente.


\begin{ejemplo}
Determine la independencia lineal del siguiente conjunto de \ptres, 
$T=\llav{2+x^3, x^2+4x^3, 1+3x+x^2}$
\end{ejemplo}

%\sol

De nuevo, debemos colocarlos a cada vector como una fila de una matriz, respetando el orden de los coeficientes:

\[
\reducir{rrrr}{
2&0&0&1\\
0&0&1&4\\
1&3&1&0
}
\underrightarrow{f_1 - 2f_3}
\reducir{rrrr}{
2&0&0&1\\
0&0&1&4\\
0&-6&-2&1
}
\underrightarrow{f_2 \longleftrightarrow f_3}
\reducir{rrrr}{
2&0&0&1\\
0&-6&-2&1\\
0&0&1&4
}
\]

Ya con la matriz totalmente reducida, vemos que ninguna fila se anul\'o, por lo tanto el conjunto T es linealmente independiente
\section{El Criterio del Determinante}
A\'un se puede mejorar el método de reducción por filas de Gauss, si en la matriz al colocar los vectores como filas(en realidad, aqu\'i funciona inclusive si los coloca como columnas), resulta una matriz cuadrada, entonces es posible determinar la independecia lineal del conjunto por medio del determinante de la matriz

\subsubsection{El criterio del determinante}
\begin{theorem}
Sea $A$ la matriz que se forma al colocar los vectores de un conjunto S como filas(o columnas), si $A$ es cuadrada entonces se cumple que 
\begin{itemize}
\item S es linealmente independiente si y solo si $det(A)\neq 0$
\item S es linealmente dependiente si y solo si $det(A)=0$
\end{itemize}

\end{theorem}


\begin{ejemplo}
Determine si el conjunto de vectores $S$ de \pdos, $S=\llav{1-x, 1+x, x^2+2x-1}$\ es linealmente independiente.
\end{ejemplo}

%\sol
Al colocar los vectores como filas, se forma la matriz
\[A=
\left(
\begin{array}{rrr}
1&-1&0\\
1&1&0\\
-1&2&1
\end{array}
\right)
\]
Ya que la matriz es CUADRADA entonces es posible aplicar el criterio del determinante, así podemos determinar la independencia lineal:
\[det(A)=1(1-0)-(-1)(1-0)=1+1=2\neq 0\]
Debido a que el determinante de la matriz es diferente de cero entonces el conjunto S es linealmente independiente.
~\\
~\\



\section{Problemas}
\begin{enumerate}


\item
Determine el valor de $k$ para que el conjunto S sea linealmente dependiente, donde $$S=\left\{\matrdxd{-1&2}{1&2}, \matrdxd{3&-1}{-2&0}, \matrdxd{-2&0}{3&1},\matrdxd{2k&1}{-3&k}\right\}$$

%\item
%Demuestre:
%\\Sea $S=\conjvect{v}{n}$ un subconjunto linealmente independiente de vectores del espacio vectorial $V$ y sea $x$ un vector de $V$ que no puede ser expresado como una combinación lineal de los vectores de S, entonces $\llav{\kvect{v}{n}, x}$
%también es linealmente independiente.

\item
Sea $p(x)= x^2+2x-3$, $q(x)=2x^2-3x+4$, y $r(x)=ax^2-1$. El conjunto $\{p, q, r\}$ es
linealmente dependiente si $a=\_?\_$.

\item
Sean $f_1(x) = \sen(x)$, $f_2(x) = \cos\left(x+\frac{\pi}{6}\right)$, y $f_3(x) = \sen\left(x-\frac{\pi}{4}\right)$ para $0 \leq x \leq 2\pi$. Muestre que 
$\llav{f_1(x), f_2(x), f_3(x)}$ es linealmente dependiente.


\item
Sean $a, b, c$ números reales distintos. Pruebe que los vectores \llav{\vectrtres{1}{1}{1}, \vectrtres{a}{b}{c}, \vectrtres{a^2}{b^2}{c^2}} forman un conjunto linealmente independiente en \rtres.



\begin{prob}[]
Sea V el espacio vectorial definido por:
\[V=\llaves{\vectrtres{a}{b}{c}}{a>0; b, c\in \dobler}\]
donde
\[ \vectrtres{a_1}{b_1}{c_1}\oplus\vectrtres{a_2}{b_2}{c_2}=\left(
\begin{array}{c}
a_1a_2\\
b_1+b_2-3\\
c_1+c_2+1
\end{array}
\right)
\]
\[\alpha\odot\vectrtres{a}{b}{c}=\left(
\begin{array}{c}
a^\alpha\\
\alpha b-3\alpha+3\\
\alpha c+\alpha-1
\end{array}
\right)\]
a) Determine el vector neutro de $V$
~\\b) Determine el inverso de $\vectrtres{2}{1}{-3}$
~\\c) Determine si los vectores $\vectrtres{4}{-1}{3}$ y $\vectrtres{2}{1}{1}$ forman un conjunto linealmente independiente.

\end{prob}


\begin{prop}[]

Determine el valor de $c$ para que el conjunto $$S=\left\lbrace  \vectrtres{1}{1}{c}, \vectrtres{c}{-1}{c}, \vectrtres{c}{1}{1}\right\rbrace $$ sea linealmente dependiente.
\end{prop}
~\\
\sol
~\\
Para verificar que un conjunto es linealmente dependiente se puede aplicar la reducción por filas de Gauss, o el criterio del determinante si la matriz es cuadrada.
~\\~\\
Aquí usaremos el criterio del determinante para resolver este problema.
~\\
Al colocar los vectores en fila obtenemos la matriz:
~\\
 $$A=\left(\begin{array}{rrr}
1&1&c\\
c&-1&c\\
c&1&1\\
\end{array}\right)$$
Para que los vectores sean linealmente dependientes el determinante de esta matriz debe ser igual a cero. Es decir:~\\
\[det(A)=0\]
\[(-1-c)-1\times(c-c^2)+c(c+c)=0\]
\[(-1-c)-c+c^2+2c^2=0\]
\[3c^2-2c-1=0\]
\[(c-1)(3c+1)=0\]
\[c=1 \vee c=-\frac{1}{3}\]

Para que el conjunto sea linealmente dependiente $c$ debe ser 1 o $-\frac{1}{3}$


\begin{prob}[]

Si $\{v_1, v_2, v_3\} $ es un conjunto linealmente independiente en un espacio vectorial $V$, determine de ser posible el valor de $a$ para que $$\{av_1-v_2, v_1+av_2-3v_3, 3v_1 -4v_2+5v_3 \}$$ sea linealmente dependiente en V.
\end{prob}


%\begin{prob}[]

%Si $\{v_1, v_2, v_3\} $ es un conjunto linealmente independiente en un espacio vectorial $V$, determine si $$\{v_1-v_2+v_3, v_2-v_3, v_1 +v_3 \}$$ es una base de V.
%\end{prob}


\end{enumerate}

%%%%%%%%%%%%%%%%%%%%%%%%%%%%%%
%\section{Conjuntos linealmente independientes}
%El subconjunto $A= \conjvect{v}{n}$ de un espacio vectorial $V$ se dice que es linealmente independiente si la combinación lineal $\lambda_1 v_1 + \lambda v_2 + \ldots + \lambda v_n = 0$ tiene como única solución la trivial, es decir que $\lambda_1 = \lambda_2 = \ldots = \lambda_n = 0$. En caso contrario se dice que $A$ es linealmente dependiente.
%\index{Conjunto linealmente independiente}
%
%\begin{ejemplo}
%Determinar si el conjunto de vectores $A= \{(1,2,3),(-2,1,1),(8,6,10)\}$ es linealmente independiente o no.\\
%Planteamos la igualdad 
%$$\alpha (1,2,3) + \beta (-2,1,1) + \lambda (8,6,10) = (0,0,0)$$
%de donde se obtiene que 
%$$\left\{
%\begin{array}{rcl}
%\alpha - 2\beta +8 \lambda & = & 0\\
%2 \alpha + \beta + 6\lambda & = & 0\\
%3\alpha + \beta + 10 \lambda & = & 0
%\end{array}
%\right.$$
%Al hacer el análisis del sistema obtenemos que existen infinitas soluciones. Así que el conjunto $A$ es linealmente dependiente.
%\end{ejemplo}
%
%\begin{ejemplo}
%Veamos que el conjunto $A= \{(3,-2,2),(3,-1,4),(1,0,5)\}$ es linealmente independiente.\\
%Consideremos $$\alpha (3,-2,2) + \beta (3,-1,4) + \lambda (1,0,5) = (0,0,0)$$
%y obtenemos
%$$\left\{
%\begin{array}{rcl}
%3 \alpha + 3 \beta + \lambda & = & 0\\
%-2 \alpha - \beta & = &0\\
%2 \alpha +4 \beta +5 \lambda & = & 0
%
%\end{array}
%\right.$$
%Al reducir de forma escalonada la matriz
%$$\left(
%\begin{array}{rrr}
%3 & 3 & 1\\
%-2 & -1 & 0\\
%2 & 4 & 5
%\end{array}
%\right)\\
%\sim \\
%\left(
%\begin{array}{rrr}
%1 & 0 & 0\\
%0 & 1 &0\\
%0 &0 &1
%\end{array}
%\right)\\$$
%Entonces el sistema tiene solución y esta es la trivial $\alpha = \beta = \lambda = 0$.
%\end{ejemplo}
%Recordemos que las matrices reales $2 \times 2$ forman un espacio vectorial.
%
%\begin{ejemplo}
%
%Veamos que $H = \left\{ \matrdxd{-1 & 2}{0 & 1}, \matrdxd{2 & 1}{1 & 1} , \matrdxd{0 & 2}{-1 & 0} \right \}$ es linealmente independiente.\\
%Consideremos la igualdad
%$$\alpha \matrdxd {-1 & 2}{0&1} + \beta \matrdxd {2&1}{1&1} + \lambda \matrdxd {0 & 2}{-1 & 0} = \matrdxd {0&0}{0&0}$$
%de donde se obtiene
%$$\left\{
%\begin{array}{rcl}
%-\alpha + 2\beta &=&0\\
%2\alpha +\beta 2\lambda &=&0\\
%\beta - \lambda &=& 0\\
%\alpha + \beta &=&0
%
%\end{array}
%\right.$$
%y obtenemos que la única solución es $\alpha = \beta = \lambda = 0$.\\
%\end{ejemplo}


%bases
\chapter{Base y Dimensión}

\section{Base}
\begin{dfn}[Base]
Un conjunto $B=\conjvect{v}{n}$\ es una $base$ de un espacio vectorial $V$ si y s\'olo si satisface lo siguiente:
\begin{itemize}
\item B genera a V
\item B es linealmente independiente
\end{itemize}

%Se conviene decir que si $V=\llav{0_v}$, entonces su base es el conjunto vac\'io $B_V=\llav{\ }$
\end{dfn}

\begin{ejemplo}
Determine si el conjunto $B=\llav{
1+x, 3x+x^2, x^2+1
}$\ es una base de \pdos
\end{ejemplo}
\sol
Para determinar si B genera a V debemos comprobar que existen escalares tales:~\\

\[a_0+a_1 x+a_2 x^2=
\beta_1(1+x)+
\beta_2(3x+x^2)+
\beta_3(x^2+1)
\]
Lo cual provoca el siguiente sistema de ecuaciones, el cual tendría que ser consistente:
$$\sisteq{\beta_1 + \beta_3 = a_0\\\beta_1 + 3\beta_2 = a_1\\\beta_2 + \beta_3 = a_2}$$
que desemboca en el siguiente sistema reducido:
$$\reducir{{rrr|l}}{1&0&1&a_0\\1&3&0&a_1\\0&1&1&a_2}
\underrightarrow{-f_1+f_2}
\reducir{rrr|l}{1&0&1&a_0\\0&3&-1&a_1-a_0\\0&1&1&a_2}
\underrightarrow{f_2 - 3f_3}
\reducir{rrr|l}{1&0&1&a_0\\0&3&-1&a_1-a_0\\0&0&-4&a_1-a_0-3a_2}$$

Este sistema tiene solución siempre, por lo que no hay condiciones sobre $a_0, a_1, a_2$
\[gen(B)=
\llaves{a_0+a_1 x+a_2 x^2 \in \pdos}{
a_0, a_1, a_2 \in \dobler
}=\pdos
\]

Ahora veamos la independencia lineal, debe haber solución única para

\[\beta_1(1+x)+
\beta_2(3x+x^2)+
\beta_3(x^2+1)=
0+0x+0x^2\]
Esto nos conduce a un sistema similar al anterior
$$\reducir{{rrr|l}}{1&0&1&0\\1&3&0&0\\0&1&1&0}
\underrightarrow{-f_1+f_2}
\reducir{rrr|l}{1&0&1&0\\0&3&-1&0\\0&1&1&0}
\underrightarrow{f_2 - 3f_3}
\reducir{rrr|l}{1&0&1&0\\0&3&-1&0\\0&0&-4&0}$$
Como vemos, tiene solución única para los escalares, la solución trivial $\beta_1=0, \beta_2=0, \beta_3=0$. El conjunto es linealmente independiente.


Por tanto como B genera a V y es linealmente independiente, B es una Base de V.



\begin{theorem}\label{th_masde_n}
Sea $B=\conjvect{v}{n}$ una base del espacio vectorial $V$, entonces cualquier conjunto de m\'as de n vectores es linealmente dependiente

\end{theorem}

\begin{proof}
Por contradicción ~\\
Sea $S=\conjvect{w}{m}$ un conjunto de vectores de $V$ con $m>n$, supongamos que $S$ es linealmente independiente. Como $B$ es una base entonces cualquier elemento se puede escribir como combinación lineal de los elementos de $B$

\begin{align*}
    \begin{array}{ccccccccc}
        w_1 &=& a_{11} v_1&+&a_{12} v_2 &+&\ldots &+&a_{1n} v_n\\
        w_2&=&a_{21} v_1&+&a_{22} v_2 &+&\ldots &+&a_{2n} v_n\\
        \vdots&\vdots&\vdots&\vdots&\cdots&\vdots&\vdots&\vdots&\vdots\\
        w_m&=&a_{m1} v_1&+&a_{n2} v_2 &+&\ldots &+&a_{mn} v_n\\
    \end{array}
\end{align*}

Si $S$ es linealmente independiente, entonces al tener una combinación lineal igualada al vector cero implicaría que todos esos escalares son iguales a cero

\begin{eqnarray*}
\beta_1 w_1+\beta_2 w_2 +\ldots +\beta_m w_m=0_v
\end{eqnarray*}

Al combinar estas ecuaciones tenemos
\begin{eqnarray*}
\beta_1 (a_{11} v_1+a_{12} v_2 +\ldots +a_{1n} v_n)+
\beta_2 (a_{21} v_1+a_{22} v_2 +\ldots +a_{2n} v_n) +\ldots \\+
\beta_m (a_{m1} v_1+a_{n2} v_2 +\ldots +a_{mn} v_n)
=0_v
\end{eqnarray*}

Reordenando esta ecuación tendríamos


\begin{eqnarray*}
(\beta_1 a_{11}+\beta_2 a_{21}+\ldots+\beta_m a_{m1})v1+
(\beta_1 a_{12}+\beta_2 a_{22}+\ldots+\beta_m a_{m2})v_2+
\ldots\\+
(\beta_1 a_{1n}+\beta_2 a_{2n}+\ldots+\beta_m a_{mn})v_n=0_v
\end{eqnarray*}

Pero $B$ es una base, por tanto es linealmente independiente, de esta manera, los escalares de esta combinación lineal deben ser igual a cero

\begin{align*}
\begin{array}{ccccccccc}
\beta_1 a_{11}&+&\beta_2 a_{21}&+&\ldots&+&\beta_m a_{m1}&=&0\\
\beta_1 a_{12}&+&\beta_2 a_{22}&+&\ldots&+&\beta_m a_{m2}&=&0\\
\vdots&\vdots&\vdots&\vdots&\cdots&\vdots&\vdots&\vdots&\vdots\\
\beta_1 a_{1n}&+&\beta_2 a_{2n}&+&\ldots&+&\beta_m a_{mn}&=&0\\
\end{array}
\end{align*}


Este es un sistema homogéneo (un sistema igualado todo a cero), si $m>n$ habrían más incógnitas que ecuaciones y por tanto el sistema tendría infinitas soluciones para $\beta_i$, lo cual contradice la suposición de que $S$ es un conjunto linealmente independiente. Contradicción. Luego, $S$ es linealmente dependiente.

\end{proof}

\begin{theorem}
Sea $V$ un espacio vectorial, entonces cualquier base $B$ de $V$ tiene el mismo número de elementos
\end{theorem}

\begin{proof}
Sean $B_1=\conjvect{v}{n}$ y $B_2=\conjvect{u}{m}$ dos bases de $V$, demostraremos que estos conjuntos tienen igual cantidad de elementos, es decir $m=n$.
Si $m>n$ entonces por el teorema ~\ref{th_masde_n}, ya que $B_1$ es base de $V$, $B_2$ ser\'ia linealmente dependiente, lo cual es una contradicción a la hipótesis de que $B_2$ es una base de $V$.

Si $n>m$ entonces por el mismo teorema ~\ref{th_masde_n}, debido a que $B_2$ es base de $V$, entonces $B_1$ ser\'ia linealmente dependiente, lo cual es una contradicción.

Luego, la única posibilidad es que $m=n$. $\blacksquare$
\end{proof}

Este resultado anterior nos indica que todas las bases de un mismo espacio vectorial tienen el mismo número de vectores, esto permite dar una caracter\'istica propia del espacio vectorial, que a continuación ser\'a definido como dimensión.

~\\
~\\
%\begin{quote}
%\small Un matemático, como un pintor o un poeta, es un fabricante de modelos. Si sus modelos son más duraderos que los de estos últimos, es debido a que están hechos de ideas. Los modelos del matemático, como los del pintor o los del poeta deben ser hermosos. La belleza es la primera prueba; no hay lugar permanente en el mundo para unas matemáticas feas. ~\\-G.H.HARDY
%\end{quote}
\newpage
\section{Dimensión de un espacio vectorial}
\begin{dfn}[Dimensión]
Sea $V$ un espacio vectorial sobre un campo \doblek, y $B$ una base de $V$, con un número finito de vectores. Se define como dimensión de $V$ al número de elementos de $B$ y se denota por $dimV$.
~\\
%Se conviene decir que si $V=\llav{0_v}$ entonces $dimV=0$
\end{dfn}

Si la base de un espacio vectorial $V$ posee infinitos vectores, entonces se dice que $V$ es de  \textit{dimensión infinita} 

~\\
Por ejemplo, los espacios clásicos que hemos trabajado tienen dimensiones bien conocidas. El espacio $\rdos $ posee dimensión 2 ya que $\llav{\vectrdos{1}{0}, \vectrdos{0}{1}}$ es una base de $\rdos$ y posee dos vectores. Además esto nos asegura que cualquier otra base de $ \rdos$ tendrá exactamente 2 vectores.
~\\

Las dimensiones del resto de espacios conocidos la mostramos a continuación:

\begin{itemize}
\item $\rtres $ posee dimensión 3
\item $\rn$ posee dimensión $n$
\item $\puno$ posee dimensión 2
\item $\pdos$ posee dimensión 3
\item $\pn$ posee dimensión $n+1$
\item $\mdosxdos$ posee  dimensión 4
\item $\mmxn$ posee  dimensión $m\times n$ 
\item $S_{nxn}$ posee  dimensión $\frac{n(n+1)}{2}$ 
\item El espacio trivial $\{\mathbf{n}_v\}$ posee dimensión 0

\end{itemize}
\begin{ejemplo}
Calcule la dimensión del subespacio $\svrtres{H}{a}{b}{c}{\begin{array}{c}
a+b-c=0\\
b=2c
\end{array}}$
~\\
\sol
Si obtenemos una base del subespacio H tendríamos:
\[b=2c\]
\[ a=c-b=c-2c=-c\]
\[B_H=\llav{\vectrtres{-1}{2}{1}}\]
Como solo hay un vector en la base, la dimensión de $H$ es igual a 1.

\end{ejemplo}
%%%%%%%%%%%%%%%%%%%%%%%%%%%%%%%%%%%%%%%%

\begin{obsimp}
En cualquier espacio vectorial, el vector nulo $\mathbf{0}_v$ es combinación lineal de cualquier conjunto finito de elementos de $V$ \conjvect {v}{n} por $\mathbf{0}_v = 0 v_1 + 0 v_2 + \ldots + 0 v_n$.\\
\end{obsimp}

\begin{dfn}
Un subconjunto $S$ de un espacio vectorial $(V, +, \cdot)$ se dice que es un subconjunto linealmente independiente maximal de $V$ si\\
\begin{enumerate}
\item $S$ es linealmente independiente.
\item Para cualquier $w \in V$ y $w \notin S$ el subconjunto $S'$ de $V$ formado por $S \cup \{w\}$ es linealmente dependiente.
\end{enumerate}
\end{dfn}

\begin{theorem}
Sea A un subconjunto de un espacio vectorial V. Si A es linealmente independiente maximal, entonces A es una base para V. Recíprocamente si A es una base entonces es un conjunto linealmente independiente maximal.
\end{theorem}
\begin{proof}
Veamos que $A$ genera a $V$. Sea $w \in V$ y $w \notin A$, entonces $\{w\}\cup A$ es un conjunto linealmente dependiente. Así la combinación lineal $\alpha_0 w+\alpha_1v_1+\alpha_2v_2+\hdots+\alpha_nv_n=0_V$ donde $A=\conjvect{v}{n}$ tiene solución no trivial. Si $\alpha_0=0$, entonces algún $\alpha_i;\ i=1, 2, \hdots, n$, debe ser diferente de cero. Esto último implica que A no es un conjunto linealmente independiente, por tanto $\alpha_0\neq 0$ y así obtendríamos que $w=\left(\frac{-\alpha_1}{\alpha_0}\right)v_1+\left(\frac{-\alpha_2}{\alpha_0}\right)v_2+\hdots+\left(\frac{-\alpha_n}{\alpha_0}\right)v_n$, por lo que $A$ es un conjunto generador de $V$.
\end{proof}
\begin{dfn}
Un subconjunto $M$ de un espacio vectorial se dice que es un subconjunto generador minimal de $V$ si
\begin{enumerate}
\item $M$ genera a $V$.
\item Si a $M$ se le quita un elemento $v_i$ entonces $M \setminus \{v_i\}$ no es un conjunto generador.
\end{enumerate}
\end{dfn}

\begin{theorem}
Si el espacio vectorial $(V , + , \cdot)$ tiene un subconjunto generador minimal $B$, entonces $B$ es una base para $V$.
\end{theorem}
\begin{proof}
Veamos que $B$ es linealmente independiente, para eso igualamos una combinación lineal de esos vectores al cero vector:
$$\alpha_1 v_1 + \alpha_2  v_2 + \hdots + \alpha_n v_n = \mathbf{0}_V $$
Por contradicción, asumiremos que $B$ es linealmente dependiente, eso implica que: $\exists \alpha_i \neq 0: i \leq n$\\
Por lo tanto, podemos escribir el vector $v_i$ correspondiente al escalar $\alpha_i$ como combinación lineal de los $n-1$ vectores restantes:
$$v_i = \left( \frac{-\alpha_1}{\alpha_i} \right)v_1 + \left( \frac{-\alpha_2}{\alpha_i} \right)v_2 + \hdots + \left( \frac{-\alpha_n}{\alpha_i} \right)v_n$$
Y esto implica que: 
$$gen\conjvect{v}{n} \subseteq gen\conjvect{v}{{n-1}}$$
Luego:
$$gen\{B\} \subseteq gen\{B\} \setminus \{v_i\}$$
Pero esto contradice la hipótesis de que $B$ es generador minimal, por lo tanto, este conjunto debe ser linealmente independiente.

\end{proof}


\section{Propiedades de las bases}
%
%Si $V$ es un subespacio vectorial y $B = \conjvect{v}{n}$ es una base para $V$, entonces los coeficientes de la combinación lineal $\alpha_1 v_1 + \alpha_2 v_2 + \ldots + \alpha_n v_n = v$ se denominan las coordenadas de $v$ respecto a la base $B$ y será representado por 
%$$(v)_B = \left( \begin{array}{c}
%\alpha_1\\
%\alpha_2\\
%\vdots\\
%\alpha_n  \end{array} \right)$$

%\begin{obsimp}
%Sea $V$ un espacio vectorial. Si una base está conformada por $n$ vectores, entonces todas las demás bases tendrán $n$ vectores.
%\end{obsimp}

%\begin{obsimp}
%Sea $V$ un espacio vectorial. Si existe una base con $n$ vectores, diremos que $n$ es la dimensión de $V$ y será denotado como $dim (V) = n$.
%\end{obsimp}
%
%\begin{theorem}
%Si el espacio vectorial $(V , + , \odot)$ tiene un subconjunto linealmente independiente maximal \conjvect{v}{n} entonces es una base para $V$.
%
%\end{theorem}

\begin{theorem}
Sea A=$\conjvect{v}{n}$ un subconjunto linealmente independiente de un espacio vectorial V. Sea $w\in V$ y $A'=A\cup\llav{w}$. Entonces A' es linealmente independiente si y solo si $ w \notin gen\{A\}$
\end{theorem}
 
\begin{proof}
Si $w$ es un elemento de $gen\{A\}$, entonces existen escalares $\alpha_1, \alpha_2,\\ \hdots, \alpha_n$ tales que $w=\alpha_1v_1+\alpha_2v_2+\hdots+\alpha_nv_n$. Así $(-1)w+\alpha_1v_1+\alpha_2v_2+\hdots+\alpha_nv_n=0_V$ y esto implica que $A'$ es linealmente dependiente.\\
Supongamos ahora que $A'$ es linealmente dependiente entonces existen escalares  $\alpha_1, \alpha_2, \hdots, \alpha_n$ no todos iguales a cero tales que $\alpha_0w+\alpha_1v_1+\alpha_2v_2+\hdots+\alpha_nv_n=0_V$. Si $\alpha_0=0$ entonces $\alpha_1v_1+\alpha_2v_2+\hdots+\alpha_nv_n=0_V$ esto implicaría que $\alpha_1= \alpha_2= \hdots=\alpha_n=0$ ya que $A$ es linealmente independiente, pero esto es imposible ya que $A'$ es linealmente dependiente; por lo tanto $\alpha_0\neq0$ y así $w$ es una combinación lineal de $A$, por lo tanto $w\in gen\{A\}$.
\end{proof}
\begin{theorem}
Si  A=$\conjvect{v}{n}$ es un conjunto generador de un espacio vectorial $V$ y la dimensión de $V$ es mayor o igual a 2, entonces existe un conjunto linealmente independiente contenido en $A$ que también genera a $V$.
\end{theorem}

\begin{proof}
Si $A$ es linealmente independiente no hay nada que demostrar. Supongamos que $A$ es linealmente dependiente. Sea $w_r$ el primer elemento no cero de $A$. Así $gen\{w_r\}$ es un subespacio de $V$ y $\{w_r\}$ es un conjunto linealmente independiente. Si $gen\{w_r\}=V$ se termina la prueba. Supongamos que $gen\{w_r\}\neq V$, entonces existe un elemento $w_q \in A$ que no pertenece a $gen\{w_r\}$ y por el teorema anterior, $\{w_r,w_q\}$ es un conjunto linealmente independiente. Recursivamente se obtienen vectores que no pertenezcan a el espacio generado por el conjunto linealmente independiente hasta que en un momento existirá un vector $w_p \in gen\{w_q,w_r, \hdots \}$, que al agregarlo, hará al conjunto linealmente dependiente, demostrando así que ese conjunto será linealmente independiente maximal y al mismo tiempo generador minimal de $V$.

\end{proof} 






\chapter{Coordenadas y Cambio de Base}
\section{Coordenadas de un vector}
\begin{dfn} [Coordenadas de un vector]

Sea $B=\conjvect{v}{n}$ una base ordenada del espacio vectorial $V$ sobre un campo \dobleK. Se denominan coordenadas de $v$ con respecto a $B; \forall v \in V$ a los escalares $\alpha _1,\alpha_2, \hdots, \alpha _n$ tales que
$v=\alpha_1 v_1+\alpha_2 v_2+ \hdots + \alpha_n v_n$
Y se define el vector $$[v]_B = \vectrcuatcent{\alpha_1}{\alpha_2}{\vdots}{\alpha_n} \in  \rn$$ el cual se denomina vector de coordenadas de $v$ con respecto a $B$. 
\end{dfn}

Por ejemplo, en $P_1$, si tenemos la base $B=\{1+x,1-2x\}$, entonces el vector $1+7x$ se escribe como
$$1+7x=3(1+x)-2(1-2x)$$
Entonces, las coordenadas de $1+7x$ con respecto a B son 3 y -2, el vector de coordenadas de $v$ con respecto a $B$ es $$[v]_B=\vectrdos{3}{-2}$$

Como podemos observar, el vector de coordenadas de $v$, depende directamente de la base, inclusive, si cambiamos el orden de la base, cambia el vector de coordenadas, de ahí la importancia de la “base ordenada” en la definición.

\section{Matriz de Cambio de Base}
\begin{dfn} Sea $V$ un espacio vectorial de dimensión $n$, $ B_1$  y$ B_2$  dos bases ordenadas $V$, y $C$ una matriz cuadrada de orden $n$, se dice que $C$ es la matriz de cambio de base de $B_1$ a $B_2$ si se cumple que:
$$\forall v \in V \ ; \ [v]_{B_2}=C[v]_{B_1}$$
\end{dfn}

\begin{theorem}


Sea V un espacio vectorial de dimensión n, $B_1={v_1,v_2,…,v_n }$  y $B_2={u_1,u_2,…,u_n}$  dos bases ordenadas V, y C es la matriz de cambio de base de $B_1$ a $B_2$ entonces:
\[C_{[B_1 \rightarrow B_2]}= \begin{pmatrix}
\vdots & \vdots & \vdots & \vdots\\
\left[v_1\right]_{B_2}&\left[v_2\right]_{B_2}& \hdots & \left[v_n\right]_{B_2}\\
\vdots & \vdots & \vdots & \vdots\\
\end{pmatrix}\]
\end{theorem}

\begin{theorem}[Corolario]
Sea V un espacio vectorial de dimensión $n$, $B_1$ y $B_2$  dos bases ordenadas $V$, y $C$ es la matriz de cambio de base de $B_1$ a $B_2$ entonces $C$ es invertible, y $C^{-1}$ es la matriz de cambio de base de $B_2$ a $B_1$.
\[
C_{[B_2\rightarrow B_1] }={C^{-1}}_{[B_1 \rightarrow B_2]}\]
\end{theorem}

\begin{ejemplo}
Sea $V=\pdos$ , y sean $B_1, B_2$ dos bases de $V$. Determine la matriz de cambio de base de $B_2$ a $B_1$. $ B_1=\llav{x^2+4x-2,x-3,x^2+1}$  y $B_2=\llav{2x^2+5x-4,x^2+2x+4,-x^2+x-4}$
~\\

La matriz de cambio de base de $B_2$ a $B_1$, estará formada por las coordenadas de los vectores de la base $B_2$ con respecto a la base $B_1$, es decir:
\[C_{[B_2 \rightarrow B_1]}= \left(\begin{array}{ccc}
\vdots & \vdots & \vdots \\
\left[2x^2+5x-4\right]_{B_1}&\left[x^2+2x+4\right]_{B_1}&  \left[-x^2+x-4\right]_{B_1}\\
\vdots & \vdots &  \vdots\\
\end{array}
\right)\]

Calculando estas coordenadas tenemos:
\[\left[2x^2+5x-4\right]_{B_1}=\vectrtres{1}{1}{1}\]
\[\left[x^2+2x+4\right]_{B_1}=\vectrtres{1}{-2}{0}\]
\[\left[-x^2+x-4\right]_{B_1}=\vectrtres{0}{1}{-1}\]

Por tanto, la matriz que nos pide el ejercicio es:

\[C_{B_2 \rightarrow B_1}= \left(\begin{array}{rrr}
1 & 1 & 0 \\
1&-2& 1\\
1 & 0 &  -1\\
\end{array}
\right)\]




\end{ejemplo}

\begin{ejercicio}
Sean $B_1=\llav{\vectrtres{1}{0}{2}, \vectrtres{1}{-2}{0}, \vectrtres{0}{1}{2}}$ y $B=\{u_1, u_2, u_3\}$ bases del espacio vectorial \rtres, y 

\[M_{B_1 \rightarrow B}=\left(\begin{array}{rrr}
1&-1&1\\
0&-2&1\\
1&1&1\\
\end{array}\right)\]
a)Determine los vectores de la base B
~\\b) Determine $[v]_{B_1}$ si $v=\vectrtres{1}{2}{3}$

\sol
Recordemos que las columnas de la matriz son las coordenadas de los vectores de la base de partida con respecto a la base de llegada.~\\
Por lo tanto, para resolver el literal (a) de nuestro problema necesitaríamos la matriz inversa de $M_{B_1 \rightarrow B}$~\\
Al calcularla obtenemos:~\\

\[M_{B \rightarrow B_1}=\left(\begin{array}{rrr}
3/2&-1&-1/2\\
-1/2&0&1/2\\
-1&1&1\\
\end{array}\right)\]
Las columnas de esta matriz son las coordenadas de los vectores de la base B, con respecto a la base B1:
~\\
Para el vector $u_1$~\\
$$[u_1]_{B_1}=\vectrtres{3/2}{-1/2}{-1}$$
$$u_1=(3/2)\vectrtres{1}{0}{2}+(-1/2)\vectrtres{1}{-2}{0}+(-1)\vectrtres{0}{1}{2}$$
$$u_1=\vectrtres{1}{0}{1}$$
Para el vector $u_2$~\\
$$[u_2]_{B_1}=\vectrtres{-1}{0}{1}$$
$$u_1=(-1)\vectrtres{1}{0}{2}+(0)\vectrtres{1}{-2}{0}+(1)\vectrtres{0}{1}{2}$$
$$u_1=\vectrtres{-1}{1}{0}$$
Para el vector $u_3$~\\
$$[u_1]_{B_1}=\vectrtres{-1/2}{1/2}{1}$$
$$u_1=(-1/2)\vectrtres{1}{0}{2}+(1/2)\vectrtres{1}{-2}{0}+(1)\vectrtres{0}{1}{2}$$
$$u_1=\vectrtres{0}{0}{1}$$
~\\
~\\
El literal b se reduce a simplemente calcular las coordenadas del vector $v=\vectrtres{1}{2}{3}$ con respecto a los vectores de la base $B_1$ los cuales se los coloca como columnas del sistema:
\[\left(\begin{array}{rrr|r}
1&1&0&1\\
0&-2&1&2\\
2&0&2&3\\
\end{array}\right)
\sim...\sim
\left(\begin{array}{rrr|r}
1&1&0&1\\
0&-2&1&2\\
0&0&-1&1\\
\end{array}\right)
\]

\[\alpha_3=-1, \alpha_2=-3/2, \alpha_1=5/2\]
Por lo tanto, el vector de coordenadas del vector $v$ con respecto a la base B1 es:
\[[v]_{B_1}=\vectrtres{5/2}{-3/2}{-1}\]


\end{ejercicio}

\section{Problemas}
\begin{enumerate}

\begin{prob}[]

Sean $B_1=\llav{x^2+x+1, x-1, 1-x^2}$ y $B_2=\{v_1, v_2, v_3\}$ bases del espacio vectorial \pdos, y 

\[M_{B_1 \rightarrow B_2}=\left(\begin{array}{rrr}
2&0&2\\
1&1&0\\
-1&-1&1\\
\end{array}\right)\]
a)Determine los vectores de la base $B_2$
~\\b) Determine $[v]_{B_1}=\vectrtres{1}{2}{3}$ y $[u]_{B_2}=\vectrtres{2}{1}{-1}$, determine $[2v-3u]_{B_2}$
\end{prob}

\begin{prob}[(1ra Evaluación Diciembre 2013)]
(10 puntos) Sean $B_1=\llav{v_1, v_2, v_3}$ y $B_2=\llav{u_1, u_2, u_3}$ dos bases ordenadas del espacio vectorial funcional V. Se conoce que $V=Gen\llav{\sen(x), \cos(x), x}$ y que $u_1=v_1+v_3, u_2=v_2+v_3 $ y $u_3= v_1+v_2$.
a) Calcule la matriz de transición de $B_1$ a $B2$ ~\\
b) Si $[2x]_{B_2}=\vectrtres{1}{1}{1}, [x-\sen(x)]_{B_2}=\vectrtres{1}{0}{0}, [2\sen(x)+\cos(x)]_{B_2}=\vectrtres{1}{0}{1}$, determine los vectores de $B_1$
\end{prob}



\begin{prob}[(1ra Evaluación Julio 2013)]
(10 puntos) Sean $v_1, v_2, v_3 $ vectores de un espacio vectorial $V$. Si $u_1, u_2, u_3 $ son linealmente independientes y son, respectivamente los vectores coordenados de $ v_1, v_2, v_3$ respecto a de una base $ B $ de $V$ entonces $dim V \geq 3 $
\end{prob}
\newpage
\begin{prob}[(1ra Evaluación Julio 2012)]
(10 puntos)Sean $B_1=\{v_1, v_2, v_3\}$ y $B_2=\{v_1-v_2, v_2+v_3, 2v_1\}$ dos bases de un espacio vectorial $V$. Si $[E]_{B_1}=[F]_{B_2}=\vectrtres{1}{1}{1}$ determine $[5E-2F]_{B_2}$
\end{prob}

\end{enumerate}

\chapter{Espacios Asociados a una Matriz}

\section{Espacio Columna}
\begin{dfn}[Espacio Columna]
Sea $A$ una matriz $m\times n$, sean $c_1,c_2,\ldots, c_n \in \mathbb{R}^m$ las columnas de la matriz $A$. Se define el espacio columna de $A$ como el espacio generado por los vectores columnas de $A$:
~\\
\[C_A=gen\{c_1, c_2, ..., c_n\}\]
\end{dfn}
~\\

\begin{ejemplo}
Sea $A=
\left(
\begin{array}{rrr}
3&1&1\\
2&2&3\\
-1&1&2
\end{array}
\right)$ una matriz, determinar el espacio columna de $A$
 
De acuerdo a la definición, debemos hallar el espacio generado por los vectores \llav{\vectrtres{3}{2}{-1}, \vectrtres{1}{2}{1}, \vectrtres{1}{3}{2}} tal como lo vimos secciones anteriores:

$$\reducir{{rrr|l}}{3&1&1&a\\2&2&3&b\\-1&1&2&c}
\underrightarrow{\begin{array}{r}
    2f_1 - 3f_2 \\
    f_1 + 3f_3
\end{array}}
\reducir{rrr|l}{3&1&1&a\\0&-4&-7&2a-3b\\0&4&7&a+3c}
\underrightarrow{f_2 + f_3}
\reducir{rrr|l}{3&1&1&a\\0&-4&-7&2a-3b\\0&0&0&3a-3b+3c}$$

%%subespacio r3
\[C_A=\llaves{\vectrtres{a}{b}{c} \in \rtres}
{\begin{array}{r}
a-b+c=0\\
\end{array}}
\]

\end{ejemplo}

\newpage
\section{Espacio Fila}
\begin{dfn}[Espacio Fila]
Sea $A$ una matriz $m\times n$, sean $f_1,f_2,\ldots, f_m \in \mathbb{R}^n$ las filas de la matriz $A$. Se define el espacio fila de $A$ como el espacio generado por los vectores filas de $A$:

\[F_A=gen\{f_1, f_2, ..., f_m\}\]
\end{dfn}


\begin{ejemplo}
Sea $A=
\left(
\begin{array}{rrrr}
2&1&1&3\\
2&2&0&2\\
1&3&-2&-1\\
\end{array}
\right)$ una matriz, determinar el espacio fila de A

 
De acuerdo a la definición, debemos hallar el espacio generado por los vectores \llav{\vectr4{2}{1}{1}{3}, \vectr4{2}{2}{0}{2}, \vectr4{1}{3}{-2}{-1}} tal como lo vimos secciones anteriores:
$$\reducir{rrr|l}{2&2&1&a\\
1&2&3&b\\
1&0&-2&c\\
3&2&-1&d\\}
\underrightarrow{\begin{array}{r}
    f_1 - 2f_2\\
    f_1 - 2f_3\\
    3f_1 - 2f_4
\end{array}}
\reducir{rrr|r}{2&1&1&a\\
0&-2&-5&a-2b\\
0&2&5&a-2c\\
0&2&5&3a-2d\\}
\underrightarrow{\begin{array}{r}
    f_2 + f_3\\
    f_2 + f_4
\end{array}}
\reducir{rrr|r}{2&1&1&a\\
0&-2&-5&a-2b\\
0&0&0&2a-2b-2c\\
0&0&0&4a-2b-2d\\
}$$

%%subespacio r3
\[F_A=\llaves{\vectr4{a}{b}{c}{d} \in \rcuatro}
{\begin{array}{r}
a-b-c=0\\
2a-b-d=0\\
\end{array}}
\]

\end{ejemplo}


\newpage
\section{Núcleo de una Matriz}
\begin{dfn}[Núcleo de una matriz]
Sea $A$ una matriz $m\times n$, el núcleo de $A$ se define como el conjunto dado por:
~\\
\[Nu(A)=\llaves{X \in \rn }{AX=\cerorm}\]
\end{dfn}
En palabras más específicas, el núcleo es el conjunto solución del sistema homogéneo de una matriz, es decir, el sistema igualado a ceros.
~\\
\begin{theorem}
Sea $A$ una matriz $m \times n$, entonces el núcleo de $A$, es un subespacio vectorial de $\rn$ sobre un campo \dobleK
\end{theorem}
\begin{proof}
    Para esta prueba, utilizaremos la caracterización del subespacio.
    \begin{itemize}
        \item el vector neutro de $\rn$ pertenece a $Nu(A)$ ya que es la solución trivial del sistema homogéneo, por lo que $Nu(A)$ es un conjunto no vacío
        \item $\forall v,w \in Nu(A)$ significa que:
        $$Av = \mathbf{0}_{\rn}$$
        $$Aw = \mathbf{0}_{\rn}$$
        Sumando estas dos ecuaciones, tenemos que:
        $$Av + Aw = \mathbf{0}_{\rn} \rightarrow A(v+w) = \mathbf{0}_{\rn}$$
        por lo que el vector $v+w$ también cumple con la condición de estar en $Nu(A)$, por lo que cumple la cerradura bajo la suma
        \item $\forall v \in Nu(A), \forall \alpha \in \dobleK$, tenemos que:
        $$Av = \mathbf{0}_{\rn}$$
        Al multiplicar esta ecuación por $\alpha$:
        $$\alpha Av = \mathbf{0}_{\rn} \rightarrow A(\alpha v) = \mathbf{0}_{\rn}$$
        por lo que el vector $\alpha v$ también cumple con la condición de estar en $Nu(A)$, por lo que cumple con la cerradura bajo la multiplicación por escalar
    \end{itemize}
\end{proof}

\begin{ejemplo}
Sea $A=
\left(
\begin{array}{rrr}
3&1&1\\
2&2&3\\
-1&1&2
\end{array}
\right)$ una matriz, determinar el núcleo de A
 
Acorde a la definición presentada anteriormente, debemos hallar todas las soluciones del sistema homogéneo, en otras palabras, resolver el sistema igualado a ceros.
$$\reducir{{rrr|l}}{3&1&1&0\\2&2&3&0\\-1&1&2&0}
\underrightarrow{\begin{array}{r}
    2f_1 - 3f_2 \\
    f_1 + 3f_3
\end{array}}
\reducir{rrr|l}{3&1&1&0\\0&-4&-7&0\\0&4&7&0}
\underrightarrow{f_2 + f_3}
\reducir{rrr|l}{3&1&1&0\\0&-4&-7&0\\0&0&0&0}$$

Por lo tanto, este sistema posee infinitas soluciones, las cuales serán los vectores del conjunto núcleo de A, el se expresa como:
%%subespacio r3
\[\svrtres{Nu(A)}{x_1}{x_2}{x_3}{\begin{array}{r}
    3x_1 + x_2 + x_3 = 0  \\
    4x_2 + 7x_3 = 0 
\end{array}}
\]

\end{ejemplo}

\section{Nulidad de una Matriz}
\begin{dfn}
Sea $A$ una matriz de orden $m\times n$ y sea $Nu(A)$ un subespacio vectorial de $\rn$, se denomina a la dimensión de $Nu(A)$ como nulidad de $A$ y se denota como $v(A)$ 
\end{dfn}
\begin{theorem}[Corolario]
Al ser $Nu(A)$ un subespacio de $\rn$, entonces $v(A) \leq~n$
\end{theorem} 

\begin{ejemplo}
Sea $\svrtres{Nu(A)}{x_1}{x_2}{x_3}{\begin{array}{r}
    3x_1 + x_2 + x_3 = 0  \\
    4x_2 + 7x_3 = 0 
\end{array}}$, determine su dimensión.
\end{ejemplo}
\sol 
Para determinar la dimensión de $Nu(A)$, debemos encontrar una base, para eso tomamos un vector genérico y le reemplazamos las condiciones:
\begin{align*}
    4x_2 + 7x_3 = 0 &\rightarrow x_2 = -\frac{7}{4}x_3\\
    3x_1 + x_2 + x_3 = 0 &\rightarrow x_1 = -\frac{1}{3} x_2 - \frac{1}{3}x_3\\ &\rightarrow x_1 = -\frac{1}{3}\left(-\frac{7}{4}x_3 \right) - \frac{1}{3}x_3\\
    &\rightarrow x_1 = \frac{1}{4}x_3
\end{align*}
al reemplazar, obtenemos:
\[ \vectrtres{x_1}{x_2}{x_3} = \vectrtres{\frac{1}{4}x_3}{-\frac{7}{4}x_3}{x_3}  = x_3\vectrtres{\frac{1}{4}}{-\frac{7}{4}}{1}\]
entonces, una base para $Nu(A)$ es :
$$B_{Nu(A)} = \llav{\vectrtres{1}{-7}{4}}$$
Como hay un vector en la base, se puede concluir que $v(A) = 1$



\newpage
\section{Imagen de una Matriz}
\begin{dfn}[Imagen de una matriz]
Sea A una matriz $m\times n$, la imagen de A se define como el conjunto dado por:
~\\
\[Im(A)=\llaves{Y \in \rmm }{\exists X \in \rn,   AX=Y}\]
\end{dfn}

En otras palabras, la imagen es el conjunto de todas los resultados que se obtienen de multiplicar $A$ con algún vector de $\rn$
\begin{ejemplo}
Sea $A=
\left(
\begin{array}{rrr}
3&1&1\\
2&2&3\\
-1&1&2
\end{array}
\right)$ una matriz, determinar la imagen de la matriz A
 
De acuerdo a la definición, debemos resolver el sistema:

$$\reducir{{rrr|l}}{3&1&1&a\\2&2&3&b\\-1&1&2&c}
\underrightarrow{\begin{array}{r}
    2f_1 - 3f_2 \\
    f_1 + 3f_3
\end{array}}
\reducir{rrr|l}{3&1&1&a\\0&-4&-7&2a-3b\\0&4&7&a+3c}
\underrightarrow{f_2 + f_3}
\reducir{rrr|l}{3&1&1&a\\0&-4&-7&2a-3b\\0&0&0&3a-3b+3c}$$
%%subespacio r3
\[Im(A)=\llaves{\vectrtres{a}{b}{c} \in \rtres}
{\begin{array}{r}
a-b+c=0\\
\end{array}}
\]

\end{ejemplo}

\begin{ejercicio}

Sea $A \in M_{nxn} $. Muestre que si $A^2=I$ entonces $Nu(A)=\{n_v\}$
~\\
\sol
~\\
Si $A^2=I$  entonces tenemos que:~\\
\[det(A^2)=det(I)=1\]
\[det(A)det(A)=1\]
\[(det(A))^2=1\]
\[det(A)=1 \vee det(A)=-1\]
~\\
Esto nos indica que el determinante de A siempre es diferente de cero, por lo cual, A es inversible.~\\
~\\
~\\
Sea $X\in Nu(A)$ entonces 
\[AX=0_{\rn}\]
\[A^{-1}AX=A^{-1} 0_{\rn}\]
\[IX=0_{\rn}\]
\[X=0_{\rn}\]
~\\
Por tanto, el único elemento del núcleo de A es $0_{\rn}$
~\\
~\\
$\therefore$ $Nu(A)=\lbrace 0_{\rn} \rbrace$

\end{ejercicio}

\begin{ejercicio}
Sea $A=
\left(
\begin{array}{rrr}
3&1&1\\
2&2&3\\
-1&1&2
\end{array}
\right)$ una matriz, determinar el espacio columna de A y el espacio Fila de A.


\sol
El espacio columna se define como el espacio generado por las columnas de A.
\[C_A=gen\{c_1, c_2, c_3\}\]
~\\
Es decir
$$\left(
\begin{array}{rrr|r}
3&1&1&a\\
2&2&3&b\\
-1&1&2&c
\end{array}
\right)$$
Al reducir este sistema por el método de Gauss tenemos lo siguiente
$$\left(
\begin{array}{rrr|r}
3&1&1&a\\
2&2&3&b\\
-1&1&2&c
\end{array}
\right)
\sim ... \sim
\left(
\begin{array}{rrr|r}
-1&1&2&c\\
0&4&7&2c+b\\
0&0&0&b-c-a
\end{array}
\right)$$
 Por tanto el espacio columna de A queda definido por:
 
 \[C_A=\left\lbrace  \vectrtres{a}{b}{c} \mid b-c-a=0 \right\rbrace  \]
~\\
(b) El espacio fila es el espacio generado por todas las filas de A. 
\[F_A=gen\{f_1, f_2, f_3\}\]
Lo cual implica resolver el siguiente sistema generador
$$\left(
\begin{array}{rrr|r}
3&2&-1&a\\
1&2&1&b\\
1&3&2&c
\end{array}
\right)$$
Aplicando el método de Gauss tenemos:
$$\left(
\begin{array}{rrr|r}
3&2&-1&a\\
1&2&1&b\\
1&3&2&c
\end{array}
\right)
\sim ...\sim
\left(
\begin{array}{rrr|r}
3&2&-1&a\\
0&-4&-4&a-3b\\
0&0&0&-3a+21b-12c
\end{array}
\right)
$$
El espacio fila de A queda definido por:
 
 \[F_A=\left\lbrace  \vectrtres{a}{b}{c} \mid -a+7b-4c=0 \right\rbrace  \]



\end{ejercicio}


\begin{ejercicio}
Sean $A\in \mathcal{M}_{mxn}, B \in \mathcal{M}_{nxp}$, demuestre que $C_{AB}\subseteq C_A$


~\\

\sol
~\\
~\\
Sea Y un elemento de la imagen de $AB$, entonces existe un $X_1$ tal que 
$$Y=(AB)*X_1$$
esto es igual a 
$$Y=A*(BX_1)$$
Donde $BX_1$ es otro vector, es decir $X_2=BX_1$
por lo tanto
$$Y=A*X_2$$
Lo cual indica por definición que "Y" también pertenece a la imagen de A.
En conclusión $$Im(AB) \subseteq Im(A)$$
Por teorema, la imagen es igual al espacio columna, ergo
$$C_{AB} \subseteq C_A$$

\end{ejercicio}

\section{Rango de una Matriz}

\newpage
\section{Problemas}
\begin{enumerate}
\begin{prop}[]

~\\Sea $A \in M_{nxn} $, si $A^2=A$ muestre que $Nu(A)\cap Im(A)=\{n_v\}$
\end{prop}

\begin{prop}[Califique como verdadero o falso]

Si la matriz B se obtiene a partir de la matriz A por medio de un intercambio de filas entonces $\rho(A)=\rho(B)$
\end{prop}

\begin{prob}[]
Dada la matriz $A=\left(\begin{array}{rrr}
2&12&5\\
1&-5&-3\\
-1&3&2\\
4&-2&-3\\
\end{array}\right)$. ~\\
a) Encuentre una base y determine la dimensi\'on del Espacio Columna de A~\\
b) Encuentre una base y determine la dimensi\'on del N\'ucleo de A.
\end{prob}


\begin{prob}[(1ra Evaluacion Marzo 2013)]
(10 puntos) Sea $A=\left(\begin{matrix}
1&2&1\\
2&a&a\\
2&-1&b\\
\end{matrix}\right)$. Para qu\'e valores de a y b:~\\
a) $\rho(A)=2$~\\
b)$\nu(A)=0$ 
\end{prob}


\begin{prob}[(1ra Evaluacion Septiembre 2013)]
(20 puntos) Dada la matriz $A=\left(\begin{array}{rrr}
1&11&4\\
2&-5&-1\\
-1&10&3\\
5&-2&1\\
\end{array}\right)$. ~\\
a) Encuentre una base y determine la dimensi\'on del Espacio Columna de A~\\
b)Encuentre una base y determine la dimensi\'on del N\'ucleo de A.
\end{prob}



\end{enumerate}

\chapter{Operaciones entre Subespacios Vectoriales}
Al igual que los conjuntos, los espacios adoptan operaciones que se pueden
realizar entre ellos, la intersección y la unión, y además, se define una nueva
operación para subespacios, la suma de subespacios.
\section{Intersección de Subespacios Vectoriales}
\begin{dfn}
Sean $H$ y $W$ dos subespacios vectoriales de un espacio vectorial $V$. Se define la intersección de $H$ y $W$ como:
$$H \cap W = \left\{ v \in V | v \in H \land v \in W \right\}$$
\end{dfn}

\begin{theorem}
Sean $H$ y $W$ dos subespacios vectoriales de un espacio vectorial $V$, entonces $H \cap W$ es un subespacio vectorial de $V$
\end{theorem}
\begin{proof}
Para demostrar que $H \cap W$ es un subespacio vectorial de $V$, utilizaremos el teorema de caracterización de subespacios
~\\ \begin{enumerate}[i.]
\item Como $H$ y $W$ son subespacios de $V$, el cero vector de $V$ pertenece a ambos subespacios, por lo tanto también pertenece a la intersección. Probando así que $H \cap W$ es no vacío.
\item Si $v_1$ y $v_2$ pertenecen a $H \cap W$ entonces ambos pertenecen a $H$ y $W$. Debido a que $H$ es un subespacio vectorial, entonces $v_1 \oplus v_2$ pertenece a $H$. Por el mismo argumento para $W$, se tiene que $v_1\oplus v_2$ pertenece a $W$. Por lo tanto $v_1\oplus v_2$ pertenece a $H\cap W$.
\item Si $v$ pertenece a  $H \cap W$, entonces  pertenece a $H$ y a $W$.  Considere $\alpha \in \dobleK$,  debido a que $H$ y $W$ son subespacios vectoriales entonces $\alpha \odot v$ pertenece a $H$ y $\alpha \odot v$ pertenece a $W$, es decir $\alpha \odot v$ pertenece a $H \cap W$.  
\end{enumerate}
\end{proof}

\begin{ejemplo}
sean $H = gen\llav{\vectrtres{1}{1}{0},\vectrtres{-2}{0}{1}}$ y \svrtres{W}{a}{b}{c}{\begin{array}{c}
    b = 3c\\
    a -c =0
\end{array}}
dos subespacios vectoriales de $\rtres$. Determine el espacio $H \cap W$
\end{ejemplo}

\begin{sol}
Para determinar la intersección de subespacios necesitamos las condiciones en forma de ecuaciones de $H$ y $W$. Las condiciones de $W$ ya las tenemos, por lo que necesitamos las de $H$, por lo que debemos encontrar el espacio generado por esos 2 vectores, entonces:

$$\vectrtres{a}{b}{c} = \alpha_1 \vectrtres{1}{1}{0}+ \alpha_2 \vectrtres{-2}{0}{1}$$

Por lo que obtenemos lo siguiente:

$$\reducir{rr|l}{1&-2&a\\1&0&b\\0&1&c} 
\underrightarrow{f_1 - f_2}
\reducir{rr|l}{1&-2&a\\0&-2&a-b\\0&1&c}
\underrightarrow{f_2 + 2f_3} 
\reducir{rr|l}{1&-2&a\\0&-2&a-b\\0&0&a-b+2c}$$

Por lo tanto:
$$\svrtres{H}{a}{b}{c}{a-b+2c=0}$$

Luego, colocamos las condiciones de ambos subespacios en un mismo sistema, tenemos que:
$$\svrtres{H \cap W}{a}{b}{c}{\begin{array}{c}
    a-b+2c=0  \\
    b=3c\\
    a-c=0
\end{array}}$$

Al final, reducimos las condiciones y las parametrizamos

$$\reducir{rrr|c}{1&-1&2&0\\0&1&-3&0\\1&0&-1&0}
\underrightarrow{f_1 - f_3}
\reducir{rrr|c}{1&-1&2&0\\0&1&-3&0\\0&-1&3&0}
\underrightarrow{f_2 + f_3}
\reducir{rrr|c}{1&-1&2&0\\0&1&-3&0\\0&0&0&0}$$

De la última fila válida obtenemos que: $b - 3c = 0$, entonces $b = 3c$\\
De la primera fila obtenemos que: $a-b+2c = 0$, con lo obtenido anteriormente: $a-3c +2c = 0 \longrightarrow a-c = 0 \longrightarrow a= c$\\

De donde concluimos que las condiciones de $H \cap W$ son:

$$\svrtres{H \cap W}{a}{b}{c}{\begin{array}{c}
    b=3c\\
    a=c
\end{array}}$$
\end{sol}

\newpage
\section{Suma de Subespacios Vectoriales}
\begin{dfn}
Sean $H$ y $W$ dos subespacios vectoriales de un espacio vectorial $V$, se define la suma de subespacios, denotada por $H+W$ como el conjunto tal que:
$$H+W = \llav{v \in V | v = h\oplus w ; h \in H \land w \in W}$$
\end{dfn}

es decir, la suma de dos subespacios es aquel conjunto que está formado por todos los vectores que se pueden escribir como la suma entre un vector de $H$ y un vector de $W$

\begin{theorem}
Sean $H$ y $W$ dos subespacios vectoriales de un espacio vectorial $V$ sobre un campo $\dobleK$, entonces $H+W$ es un subespacio vectorial de $V$
\end{theorem}
\begin{proof}
Para esta demostración utilizaremos, de nuevo, el teorema de caracterización de un subespacio
~\\\begin{enumerate}[i.]
    \item Probaremos que $H+W$ no es vacío. Dado que $n_v = n_v \oplus n_v$, se puede expresar como la suma de un vector de $H$ y un vector de $W$ donde $n_v \in H$ y $n_v \in W$ ya que son subespacios de $V$, entonces $n_v \in H+W$, por lo tanto es no vacío.
    
    \item Sean $v_1, v_2 \in H+W$, entonces existen $h_1 , h_2 \in H$ y $w_1 , w_2 \in W$ tales que:
    $$u_1 = h_1 \oplus w_1$$ 
    $$u_2 = h_2 \oplus w_2$$
    Por tanto, el vector $u_1 \oplus u_2 = (h_1 \oplus h_2) \oplus (h_2 \oplus w_2) = (h_1 \oplus h_2) \oplus (w_1 \oplus w_2)$ se puede expresar como la suma de un vector de $H$ y un vector de $W$, donde por la cerradura del subespacio de $H$, el vector $h_1 \oplus h_2 \in H$ y por la cerradura del subespacio de $W$, el vector $w_1 \oplus w_2 \in W$. Se concluye que $u_1 \oplus u_2 \in H+W$.
    
    \item Sean $\alpha \in \dobleK, u \in H+W$, entonces existen vectores $h \in H , w \in W$ tales que $u = h\oplus w$. Luego, el vector $\alpha \odot u = \alpha\odot(h\oplus w) = (\alpha \odot h) \oplus (\alpha \odot w)$ se puede escribir como la suma de un vector de $H$ y un vector de $W$, donde $\alpha \odot h \in H$ y $\alpha \odot w \in W$.
\end{enumerate}
\end{proof}

\begin{theorem}
Sean $H$ y $W$ dos subespacios vectoriales de un espacio vectorial $V$, tales que $H = gen(S_1)$ y $W = gen(S_2)$, entonces $H+W = gen(S_1 \cup S_2)$.
\end{theorem}
\begin{proof}
Para todo elemento $u \in H+W$, se escribe como la suma de un vector de $H$ y un vector de $W, u = h \oplus w$. Sea $S_1 = \conjvect{h}{k}$ el conjunto generador de $H$ y $S_2 = \conjvect{w}{m}$ el conjunto generador de $W$, entonces es cierto que:
$$h = (\alpha_1 \odot h_1) \oplus (\alpha_2 \odot h_2) \oplus \hdots \oplus (\alpha_k \odot h_k)$$
$$w = (\beta_1 \odot w_1) \oplus (\beta_2 \odot w_2) \oplus \hdots \oplus (\beta_m \odot w_m)$$
Por tanto $u$ se puede escribir como $u = (\alpha_1 \odot h_1) \oplus (\alpha_2 \odot h_2) \oplus \hdots \oplus (\alpha_k \odot h_k) \oplus (\beta_1 \odot w_1) \oplus (\beta_2 \odot w_2) \oplus \hdots \oplus (\beta_m \odot w_m)$, es decir $u$ es combinación lineal del conjunto $\llav{h_1, h_2, \hdots, h_k, w_1, w_2, \hdots, w_m}$ que es, precisamente, la unión de $S_1$ y $S_2$. Luego $H+W = gen(S_1 \cup S_2)$
\end{proof}

\begin{ejemplo}
Sean \svrtres{H}{a}{b}{c}{a = b -2c}, \svrtres{W}{a}{b}{c}{\begin{array}{c}
    b = 2c\\
    a-c = 0
\end{array}} dos subespacios vectoriales. Determine el subespacio $H+W$
\end{ejemplo}
\begin{sol}
Para determinar la suma de subespacios necesitamos los conjuntos generadores de $H$ y $W$, ya que toda base es un conjunto generador, entonces necesitamos una base de $H$ y $W$.

~\\Para $H$:
$$\vectrtres{a}{b}{c} = \vectrtrescent{b-2c}{b}{c} = b\vectrtres{1}{1}{0} + c\vectrtres{-2}{0}{1}$$
$$B_H = \llav{\vectrtres{1}{1}{0},\vectrtres{-2}{0}{1}}$$
Para $W$:
$$\vectrtres{a}{b}{c} = \vectrtrescent{c}{2c}{c} = c\vectrtres{1}{2}{1}$$
$$B_W = \llav{\vectrtres{1}{2}{1}}$$

Luego, por le teorema del generador de la suma, tenemos que:
$$H+W = gen\llav{\vectrtres{1}{1}{0},\vectrtres{-2}{0}{1},\vectrtres{1}{2}{1}}$$

Para hallar las condiciones en ecuaciones de $H+W$, procedemos a hallar el espacio generado por ese conjunto.

$$\reducir{rrr|l}{1&-2&1&a\\1&0&2&b\\0&1&1&c} 
\underrightarrow{f_1 - f_2}
\reducir{rrr|l}{1&-2&1&a\\0&-2&-1&a-b\\0&1&1&c} 
\underrightarrow{f_2 + 2f_3}
\reducir{rrr|l}{1&-2&1&a\\0&-2&-1&a-b\\0&0&1&a-b+2c}$$

~\\De donde concluimos que no hay condiciones sobre $a,b,c$; es decir $H+W = \rtres$
\end{sol}

\newpage

\section{Unión de Subespacios Vectoriales}
La intersección y la suma de subespacios vectoriales siempre da como resultado otro subespacio, sin embargo esto no es así con la unión. No siempre la unión de subespacios es un subespacio, solo en ciertos casos, descrito por el siguiente teorema.
\begin{theorem}
Sean $H$ y $W$ dos subespacios vectoriales de un espacio vectorial $V$, se cumple que: $H \cup W$ es un subespacio vectorial si y solo si $H \subseteq W \ \lor \ W \subseteq H$
\end{theorem}

\begin{ejemplo}
Sean $H = gen\llav{\vectrtres{1}{1}{0},\vectrtres{-2}{0}{1}}, \svrtres{W}{a}{b}{c}{\begin{array}{c}
    b = 3c\\
    a-c=0
\end{array}}$ dos subespacios vectoriales. Determine si el conjunto $H \cup W$ es un subespacio vectorial de $V$
\end{ejemplo}
\begin{sol}
Para resolver esto, hacemos uso del teorema de unión de subespacios. Necesitaremos tanto las condiciones de $H$ y $W$, como las bases de $H$ y $W$.
$$\text{Condiciones de }H: a-b+2c = 0$$
$$\text{Condiciones de }W: \begin{array}{c}
    b = 3c\\
    a-c=0
\end{array}$$

$$\text{Base de }H: B_H = \llav{\vectrtres{1}{1}{0},\vectrtres{-2}{0}{1}}$$
$$\text{Base de }W: B_W = \llav{\vectrtres{1}{2}{1}}$$

Entonces para probar que $H$ está contenido en $W$, debemos comprobar que todos los vectores de la base de $H$ cumplan todas las condiciones de $W$

$$\text{Condiciones de }W: \begin{array}{c}
    b = 3c\\
    a-c=0
\end{array}$$
$$\text{Base de }H: B_H = \llav{\vectrtres{1}{1}{0},\vectrtres{-2}{0}{1}}$$

Para el primer vector:
$$b = 3c \longrightarrow 1 \neq 3(0)$$
Basta que falle una ecuación para afirmar que $H$ no está contenido en $W$. Es decir, $H \nsubseteq W$, sin embargo aún no podemos afirmar que la unión no es subespacio, antes debemos comprobar también si sucede lo contrario. Para probar que $W$ está contenido en $H$, debemos comprobar que todos los vectores de la base de $W$ cumplan todas las condiciones de $H$.

$$\text{Condiciones de }H: a-b+2c = 0$$
$$\text{Base de }W: B_W = \llav{\vectrtres{1}{2}{1}}$$

Para el único vector que hay, tenemos que:
$$a -b +2c=0 \longrightarrow 1-3 +2(1)=0$$

Lo cual es cierto, es decir se cumple que: 
$$W \subseteq H$$
Por lo que se concluye que $H \cup W$ es un subespacio vectorial.
\end{sol}






%%transformaciones
\chapter{Transformaciones Lineales}

TL

\chapter{Producto Interno}

\begin{dfn}
Sea $V$ un \dobleK-espacio vectorial. Una función $\prodInt{\cdot}{\cdot}: V\times V \to \dobleK$ es un producto interno sobre $V$ si y solo si, se cumple que:
\begin{enumerate}
    \item Para todo $v \in V$,  \prodInt{v}{v} es un número real no negativo, el cual es cero si y solo si $v = 0_V$.  
    \item $\prodInt{v_1}{v_2} = \overline{\prodInt{v_2}{v_1}}$ para todo $v_1,v_2 \in V$.
    \item $\prodInt{v_1 + v_2}{v_3} = \prodInt{v_1}{v_3} + \prodInt{v_2}{v_3}$ para todo $v_1, v_2, v_3 \in V$.
    \item $\prodInt{\alpha v_1}{v_2} = \alpha \prodInt{v_1}{v_2}$ para todo $\alpha \in \dobleK$ y para todo $v_1, v_2 \in V$.
\end{enumerate}
\begin{obs}
    \begin{enumerate}[i.]
        \item $\overline{\prodInt{v_2}{v_1}}$ es el conjugado de \prodInt{v_2}{v_1}.
        \item Las propiedades 3 y 4 indican que \prodInt{\cdot}{\cdot} es una función lineal.
    \end{enumerate}
    
    
\end{obs}
\end{dfn}

\begin{ejemplo}
    Sean $v_1 = \vectrdos{a_1}{b_1}$ y $v_2 = \vectrdos{a_2}{b_2}$. Sea $\prodInt{\cdot}{\cdot}: \rdos \times \rdos \to \dobler$ una función definida por: 
    $$\prodInt{v_1}{v_2} = 3a_1a_2 + 3b_1b_2$$
    es un producto interno, así:
    \begin{enumerate}[i.]
        \item sea $v \in V$, entonces
        $\prodInt{v}{v} = 3a^2 + 3b^2$, al tener una suma de números elevados al cuadrado y multiplicados por 3, entonces \prodInt{v}{v} es mayor que cero. Por lo que la única manera en la que $\prodInt{v}{v} = 0$ es que $v = 0_V$.
        \item Sean $v_1, v_2 \in V$, entonces $\prodInt{v_1}{v_2} = 3a_1a_2 + 3b_1b_2 = 3a_2a_1 + 3b_2b_1 = \prodInt{v_2}{v_1}$ esto es posible gracias a la conmutatividad de la multiplicación de números reales y además, el conjugado de un número real, sigue siendo el mismo número real.
        \item Sean $v_1, v_2, v_3 \in V$, entonces $\prodInt{v_1 + v_2}{v_3} = 3(a_1 + a_2)a_3 + 3(b_1 + b_2)b_3$ aplicando distributividad y conmutatividad, nos queda que\\
        $\prodInt{v_1 + v_2}{v_3} = 3a_1a_3+ 3b_1b_3+ 3a_2a_3+ 3b_2b_3 = \prodInt{v_1}{v_3} + \prodInt{v_2}{v_3}$.
        \item Sea $\alpha \in \dobleK$ y sean $v_1, v_2 \in V$, entonces $\prodInt{\alpha v_1}{v_2} = 3(\alpha a_1)a_2 + 3(\alpha b_1)b_2 = \alpha(3a_1a_2 + 3b_1b_2) = \alpha \prodInt{v_1}{v_2}$ 
    \end{enumerate}
\end{ejemplo}
\newpage

\begin{ejemplo}
    Sean $p(x) = a_1x+b_1$ y $q(x)= a_2x+b_2$ $\prodInt{p(x)}{q(x)}: \puno \times \puno \to \dobler$ una función definida por:
    $$\prodInt{p(x)}{q(x)} = 2a_1a_2 - 3b_1b_2$$
    NO es un producto interno sobre \puno, porque:\\
    Sea $p(x) = x+1$
    $$\prodInt{p(x)}{p(x)} = 2(1)(1) - 3(1)(1) = -1$$
    Lo cual no cumple con la primera propiedad de que el producto interno de un vector consigo mismo, debe de ser mayor que cero
\end{ejemplo}

\begin{theorem}
Sea $V$ un \dobleK-espacio vectorial y $\prodInt{\cdot}{\cdot}: V \times V \to \dobleK$ un producto interno sobre $V$, entonces se cumple que:
\begin{enumerate}[a.]
    \item Sean $v_1, v_2, v_3 \in V$, entonces $\prodInt{v_1}{v_2 + v_3} = \prodInt{v_1}{v_2} + \prodInt{v_1}{v_3}$.
    \item Sean $v_1, v_2 \in V$ y sea $\alpha \in \dobleK$, entonces $\prodInt{v_1}{\alpha v_2} = \overline{\alpha}\prodInt{v_1}{v_2}$.
    \item $\forall v \in V$, entonces $\prodInt{0_V}{v} = \prodInt{v}{0_V} = 0$.
\end{enumerate}
\end{theorem}

Las siguientes funciones son conocidas como productos internos canónicos de los distintos espacios vectoriales, a continuación:
\begin{itemize}
    \item Producto interno canónico en \rn:
    $$\prodInt{\vectrtrescent{x_1}{x_2\\\vdots}{x_n}}{\vectrtrescent{y_1}{y_2\\\vdots}{y_n}} = x_1y_1 + x_2y_2 + \cdots + x_ny_n$$
    \item Producto interno canónico en \cn:
    $$\prodInt{\vectrtrescent{x_1}{x_2\\\vdots}{x_n}}{\vectrtrescent{y_1}{y_2\\\vdots}{y_n}} = x_1\overline{y_1} + x_2\overline{y_2} + \cdots + x_n\overline{y_n}$$
    \item Producto interno canónico en \mmxn:
    $$\prodInt{A}{B} = tr(A\cdot B^T)$$
    \hspace{5,5cm} o
    $$\prodInt{A}{B} = \prodInt{[A]_B}{[B]_B}$$
    donde $B$ es la base canónica de \mmxn
    
    \item Producto interno canónico en \pn:
    $$\prodInt{a_0x^n + a_1x^{n-1}+ \cdots + a_{n}}{b_0x^n + b_1x^{n-1} +\cdots + b_{n}} = a_0 b_0 + a_1 b_1 + \cdots + a_n b_n$$
    
    \item Producto interno canónico en $C[a,b]$:
    $$\prodInt{f}{g} = \int_a^b f(x)g(x) dx$$
\end{itemize}

\begin{dfn}
A un espacio vectorial sobre un campo \dobleK, que posee un producto interno, se denomina \textit{Espacio Euclídeo}
\end{dfn}

\section{Norma de un Vector}
\begin{dfn}
Sea $V$ un \dobleK-espacio vectorial con producto interno $\prodInt{\cdot}{\cdot}: V \times V \to \dobleK$ y sea $v \in V$. Se define la norma de $v$ como el escalar:
$$\norm{v} = \sqrt{\prodInt{v}{v}} $$
\end{dfn}

\begin{theorem}
 Sea $V$ un \dobleK-espacio vectorial con producto interno $\prodInt{\cdot}{\cdot}: V \times V \to \dobleK$ y sea $v \in V$. entonces se cumple que:
 \begin{enumerate}[a.]
     \item Para todo $v \in V$, $\norm{v} \geq 0$,  y $\norm{v} = 0$ si y solo si $v = 0$.
     \item Sean $\alpha \in \dobleK$ y $v \in V$, entonces $\norm{\alpha v} = |\alpha| \norm{v}$.
     \item Sean $v_1, v_2 \in V$, entonces $\left|\prodInt{v_1}{v_2}\right| \leq \norm{v_1} \cdot \norm{v_2}$. A esta desigualdad se la conoce como \textbf{Desigualdad de Cauchy-Schwartz}
     \item Sean $v_1, v_2 \in V$, entonces $\norm{v_1 + v_2} \leq \norm{v_1} + \norm{v_2}$. A esta desigualdad se la conoce como \textbf{Desigualdad Triangular}
 \end{enumerate}
\end{theorem}

\begin{ejemplo}
    Sea $V = \pdos(\dobler)$ con el producto interno canónico, entonces:
    $$\norm{2x^2 + 3x -1} = \sqrt{(2)^2 + (3)^2 + (-1)^2} = \sqrt{15}$$
\end{ejemplo}

\begin{ejemplo}
    Sea $V = \dobleC^2$ con el producto interno canónico y sea $v = \vectrdos{-i}{i}$, entonces:
    $$\norm{v} = \sqrt{\prodInt{v}{v}} = \sqrt{(-i)(i) + (i)(-i)} = \sqrt{-(-1) -(-1)} = \sqrt{2}$$
    
\end{ejemplo}
\section{Distancia entre vectores}
\begin{dfn}
Sea $V$ un \dobleK-espacio vectorial con producto interno $\prodInt{\cdot}{\cdot}: V \times V \to \dobleK$, la distancia entre dos vectores se denota por $d(v_1, v_2)$ y se define como:
$$d(v_1,v_2) = \norm{v_1 - v_2}$$
\end{dfn}

\begin{theorem}
 Sea $V$ un \dobleK-espacio vectorial con producto interno $\prodInt{\cdot}{\cdot}: V \times V \to \dobleK$ y sea $v \in V$. entonces se cumple que:
 \begin{enumerate}[a.]
     \item Para todo $v_1, v_2 \in V$, entonces $d(v_1,v_2) \geq 0$.
     \item $d(v_1, v_2) = 0$ si y solo si $v_1 = v_2$.
     \item Para todo $v_1, v_2 \in V$, entonces $d(v_1, v_2) = d(v_2, v_1)$.
     \item Para todo $v_1, v_2, v_3 \in V$, entonces $(v_1, v_3) \leq d(v_1, v_2) + d(v_2, v_3)$
 \end{enumerate}
\end{theorem}

\begin{ejemplo}
    Sea $V = \pdos$, sean $p(x) = x^2 + 2x + 3$ y $q(x) = x + 2$, la distancia entre $p(x)$ y $q(x)$ se calcula:
    $$d(p(x),q(x)) = \norm{x^2 + 2x +3 - x -2} = \norm{x^2 +x +1} = \sqrt{1^2 + 1^2 + 1^2}  = \sqrt{3}$$
\end{ejemplo}

\begin{ejemplo}
    Sea $V = \dobleC^2$ y sean $u = \vectrdos{1+i}{i}$ y $v = \vectrdos{1}{2i}$, la distancia entre $u$ y $v$ se calcula:
    $$d(u,v) = \norm{\vectrdos{1+i}{i} - \vectrdos{1}{2i}} = \norm{\vectrdos{i}{-i}} = \sqrt{(i)(-i) + (-i)(i)} = \sqrt{-(-1) -(-1)} = \sqrt{2}$$
\end{ejemplo}

\section{Ángulo entre vectores}

\begin{dfn}
Sea $V$ un \dobleK-espacio vectorial con producto interno, y sean $v_1,v_2$ dos vectores no nulos de $V$. Se define el ángulo entre $v_1$ y $v_2$ como el único número real $\theta \in [0,\pi]$ tal que: 
$$\cos(\theta) = \frac{\prodInt{v_1}{v_2}}{\norm{v_1} \cdot \norm{v_2}}$$
\end{dfn}

\begin{obs}
    Nótese que si $\prodInt{v_1}{v_2} = 0$, entonces de la expresión anterior, obtenemos que el ángulo entre $v_1$ y $v_2$ es $\theta = \frac{\pi}{2}$, dado el caso, los vectores se dice que son ortogonales con respecto al producto interno definido
\end{obs}

\begin{ejemplo}
    Sea $V = \pdos(\dobler)$ con el producto interno canónico. La medida del ángulo entre los vectores $p(x) = x^2 + 2x$ y $q(x) = 2x^2 +1$
    \begin{align*}
        \cos(\theta) &= \frac{\prodInt{p(x)}{q(x)}}{\norm{p(x)} \cdot \norm{q(x)}}\\
        \cos(\theta) &= \frac{(1)(2) + (2)(0) + (0)(1)}{\sqrt{(1)^2 + (2^2)} \cdot \sqrt{(2)^2 + (1)^2}} = \frac{2}{\sqrt{5}\cdot \sqrt{5}} = \frac{2}{5}\\
        \theta &= \cos^{-1}\left(\frac{2}{5}\right) \approx 66.42^\circ
    \end{align*}

\end{ejemplo}

\begin{theorem}[Ley del Coseno]
Si $v_1$ y $v_2$ son dos vectores en un espacio euclídeo, y $\theta$ es la medida del ángulo entre $v_1$ y $v_2$, entonces:
$$\norm{v_1 \pm v_2}^2 = \norm{v_1}^2 \pm 2\cos(\theta)\norm{v_1}\cdot\norm{v_2}+ \norm{v_2}^2$$
\end{theorem}

\section{Matriz de un Producto Interno}
\begin{dfn}
Sea $V$ un \dobleK-espacio vectorial con producto interno \prodInt{\cdot}{\cdot} y $B = \conjvect{v}{n}$ una base de $V$. Se define la matriz del producto interno o \textbf{Matriz de Gram} en la base $B$ como la matriz $A \in \mathcal{M}_{n\times n}(\dobleK)$ tal que:
$$A_{ij} = \prodInt{v_i}{v_j}, \forall i,j \  1 \leq i,j \leq n$$
\end{dfn}

\begin{ejemplo}
    Sea $V = \rtres$ con el producto interno canónico y sea $B = \llav{\vectrtres{1}{2}{3},\vectrtres{0}{1}{2}, \vectrtres{0}{0}{1}}$ una base de \rtres. La matriz de Gram se constituye:
    $$A = \reducir{rrr}{
        \prodInt{v_1}{v_1} & \prodInt{v_1}{v_2} & \prodInt{v_1}{v_3}\\\\
        \prodInt{v_2}{v_1} & \prodInt{v_2}{v_2} & \prodInt{v_2}{v_3}\\\\
        \prodInt{v_3}{v_1} & \prodInt{v_3}{v_2} & \prodInt{v_3}{v_3}
    }
    \text{Entonces: \ }
    A = \reducir{ccc}{
        14 & 8 & 3\\
        8 & 5 & 2\\
        3 & 2 & 1
    }$$
\end{ejemplo}

\begin{obs}
    La diagonal principal de la matriz de Gram no puede contener un cero ya que eso significa que $\prodInt{v_i}{v_i} = 0, \forall i, 1 \leq i \leq n$ y por lo tanto que $v = 0_V$, pero eso significaría que el vector neutro está en la base de $V$, lo cual no es posible
\end{obs}


\section{Conjuntos Ortogonales y Ortonormales}
\begin{dfn}
Sea $V$ un \dobleK-espacio vectorial con producto interno, dos vectores $v_1, v_2 \in V$ se dicen \textit{ortogonales} si $\prodInt{v_1}{v_2} = 0$
\end{dfn}
\begin{dfn}
Sea $V$ un \dobleK-espacio vectorial con producto interno. Se dice que $\conjvect{v}{k} \subseteq V$ es un conjunto ortogonal si $\prodInt{v_i}{v_j} = 0, \forall i \neq j$ donde $1 \leq i,j \leq k$. 
\end{dfn}

\begin{ejemplo}
    En \rdos con el producto interno canónico, el conjunto \llav{\vectrdos{1}{1},\vectrdos{1}{-1}} es un conjunto ortogonal:
    \begin{align*}
        \prodInt{\vectrdos{1}{1}}{\vectrdos{1}{-1}} = (1)(1) + (1)(-1) = 0
    \end{align*}
\end{ejemplo}

\begin{dfn}
Sea $V$ un \dobleK-espacio vectorial con producto interno. Se dice que $\conjvect{v}{k} \subseteq V$ es un conjunto ortonormal si es ortogonal y además $\norm{v_i} = 1, \forall i$ donde $1 \leq i \leq k$ 
\end{dfn}

\begin{obs}
    El vector nulo de $V$ es ortogonal a todo vector $v \in V$, ya que $\prodInt{0_V}{v} = 0$
\end{obs}

\begin{ejemplo}
    En \rtres con el producto interno canónico, la base canónica es un conjunto ortonormal:
    $$B = \llav{\vectrtres{1}{0}{0}, \vectrtres{0}{1}{0}, \vectrtres{0}{0}{1}}$$
    Así entonces:
    \begin{align*}
        \prodInt{\vectrtres{1}{0}{0}}{\vectrtres{1}{0}{0}} = 1 &&
        \prodInt{\vectrtres{1}{0}{0}}{\vectrtres{0}{1}{0}} = 0 && \prodInt{\vectrtres{1}{0}{0}}{\vectrtres{0}{0}{1}} = 0\\
        \prodInt{\vectrtres{0}{1}{0}}{\vectrtres{0}{1}{0}} = 1 &&
        \prodInt{\vectrtres{0}{1}{0}}{\vectrtres{1}{0}{0}} = 0 &&
        \prodInt{\vectrtres{0}{1}{0}}{\vectrtres{0}{0}{1}} = 0\\
        \prodInt{\vectrtres{0}{0}{1}}{\vectrtres{0}{0}{1}} = 1 &&
        \prodInt{\vectrtres{0}{0}{1}}{\vectrtres{0}{1}{0}} = 0 &&
        \prodInt{\vectrtres{0}{0}{1}}{\vectrtres{1}{0}{0}} = 0
    \end{align*}
\end{ejemplo}

\begin{theorem}
Sea $V$ un \dobleK-espacio vectorial con producto interno y $S$ un conjunto ortogonal de vectores no nulos en $V$. Entonces $S$ es un conjunto linealmente independiente
\end{theorem}
\begin{proof}
Sean \conjvect{v}{k} vectores de $S$. Suponga que:
$$\alpha_1 v_1 + \alpha_2 v_2 + \cdots + \alpha_k v_k = 0_V$$
Dado que $\prodInt{0_V}{v_i} = 0$ para cada $1 \leq i \leq k$ y haciendo uso de las propiedades de linealidad del producto interno, se obtiene que:
\begin{align*}
    \prodInt{0_v}{v_i} &= 0\\
    \prodInt{\alpha_1 v_1 + \alpha_2 v_2 + \cdots + \alpha_k v_k}{v_i} &= 0 \\
    \alpha_1 \prodInt{v_1}{v_i} + \alpha_2 \prodInt{v_2}{v_i} + \hdots + \alpha_k \prodInt{v_k}{v_i} &= 0
\end{align*}
Sabiendo que el conjunto $S$ es ortogonal, lo cual significa que:
$$\prodInt{v_i}{v_j} = \left\{\begin{array}{cll}
    \norm{v_i}^2& \forall i,j & i = j\\
    0 & \forall i,j & i \neq j 
\end{array}\right.$$

Reemplazando estos valores en la combinación lineal, se obtiene que:
$$\alpha_j \norm{v_j}^2$$
Por hipótesis, los vectores de $S$ son no nulos, por lo tanto se concluye que:
$$\forall j , 1 \leq j \leq k \ ; \ \alpha_j = 0$$
Así entonces todo conjunto ortogonal es linealmente independiente
\end{proof}
\begin{theorem}[Corolario]
Todo conjunto ortonormal de vectores en un espacio con producto interno es linealmente independiente
\end{theorem}

\begin{theorem}
Sea $V$ un espacio con producto interno, y sea $H$ el espacio generado por el conjunto ortonormal de vectores $S = \conjvect{v}{k}$. El vector $u \in V$ pertenece a $H$ si y solo si $u$ puede escribirse como:
$$u = \prodInt{u}{v_1}v_1 + \prodInt{u}{v_2}v_2 + \cdots + \prodInt{u}{v_k}$$
\end{theorem}
\begin{proof}
Si el vector $u \in V$ se expresa como: 
$$u = \prodInt{u}{v_1}v_1 + \prodInt{u}{v_2}v_2 + \cdots + \prodInt{u}{v_k}$$
es obvio que $u \in H$, en tal caso $u$ es una combinación lineal de los vectores de $S$ que generan a $H$.\\
Suponga entonces que si $u \in H$, entonces exiten escalares tales que:
$$u = \alpha_1v_1 + \alpha_2v_2 + \cdots + \alpha_k v_k$$
Al tomar el producto interno del vector $u$ con el vector $v_i \in S$ \prodInt{u}{v_i},  se tiene que :
$$\prodInt{\alpha_1v_1 + \alpha_2v_2 + \cdots + \alpha_k v_k}{v_i} = \alpha_1\prodInt{v_1}{v_i} + \alpha_2\prodInt{v_2}{v_i} + \cdots + \alpha_k \prodInt{v_k}{v_i}$$
Ahora, como $S$ es un conjunto ortonormal, entonces 
$$\prodInt{v_i}{v_j} = \left\{\begin{array}{cll}
    1 & \forall i,j & i = j\\
    0 & \forall i,j & i \neq j 
\end{array}\right.$$
y por lo tanto $\prodInt{u}{v_i} = \alpha_i$, lo que significa que:
$$u = \prodInt{u}{v_1}v_1 + \prodInt{u}{v_2}v_2 + \cdots + \prodInt{u}{v_k}$$
\end{proof}

\section{Bases Ortonormales}
\begin{dfn}
Sea $V$ un \dobleK-espacio vectorial con producto interno de dimensión finita.
\begin{itemize}
    \item Se dice que la base $B = \conjvect{v}{n}$ de $V$, es una base ortogonal si el conjunto $B$ es un conjunto ortogonal de vectores en $V$
    \item Se dice que la base $B = \conjvect{v}{n}$ de $V$, es una base ortonormal si el conjunto $B$ es un conjunto ortonormal de vectores en $V$
\end{itemize}
\end{dfn}
\begin{obs}
    \begin{itemize}
        \item Un conjunto ortonormal de vectores en $V$ es una base ortonormal si y solo si este conjunto genera a $V$.
        \item La matriz de Gram asociada a una base ortogonal, es una matriz escalar.
        \item La matriz de Gram asociada a una base ortonormal, es la matriz identidad.
    \end{itemize}
    
\end{obs}
\begin{ejemplo}
    En \rdos con el producto interno canónico, la base canónica es una base ortonormal:
    $$B = \llav{\vectrdos{1}{0}, \vectrdos{0}{1}}$$
    Así entonces:
    \begin{align*}
        \prodInt{\vectrdos{1}{0}}{\vectrdos{0}{1}} = 0&&
        \prodInt{\vectrdos{1}{0}}{\vectrdos{1}{0}} = 1&&
        \prodInt{\vectrdos{0}{1}}{\vectrdos{0}{1}} = 1
    \end{align*}
\end{ejemplo}

\begin{theorem}[Proceso de Gram-Schmidt]
    Sea $B=\conjvect{v}{n}$ una base del espacio $V$ con producto interno \prodInt{\cdot}{\cdot}. Entonces el conjunto $B_{ON} = \conjvect{u}{n}$ en donde los vectores \conjvect{u}{n} son definidos inductivamente como:
    \begin{align*}
        u_1 &= \frac{v_1}{\norm{v_1}}\\
        u_2 &= \frac{v_2 - \prodInt{v_2}{u_1}u_1}{\norm{v_2 - \prodInt{v_2}{u_1}u_1}}\\
        u_3 &= \frac{v_3 - \prodInt{v_3}{u_2}u_2 - \prodInt{v_3}{u_1}u_1}{\norm{v_3 - \prodInt{v_3}{u_2}u_2 - \prodInt{v_3}{u_1}u_1}}\\
        & \vdots\\
        u_n &= \frac{v_n - \displaystyle \sum_{i=1}^{n-1}\prodInt{v_n}{u_i}u_i}{\norm{v_n - \displaystyle \sum_{i=1}^{n-1}\prodInt{v_n}{u_i}u_i}}
    \end{align*}
    en donde el conjunto $\conjvect{u}{n}$ es una base ortonormal
\end{theorem}

\begin{theorem}[corolario]
    Todo espacio vectorial de dimensión finita con producto interno, posee una base ortonormal
\end{theorem}

\begin{ejemplo}
    Sea $V = \rtres$ con el producto interno canónico, sea $B = \llav{\vectrtres{1}{1}{0}, \vectrtres{1}{1}{1},\vectrtres{0}{1}{1}}$ una base de \rtres, determine una base ortonormal $B_{NO}$ de \rtres
\end{ejemplo}
\begin{sol}
Para encontrar una base ortonormal, haremos uso del proceso de ortonormalización de Gram-Schmidt, entonces:
\begin{align*}
    v_1 = \vectrtres{1}{1}{0} && v_2 = \vectrtres{1}{1}{1} && v_3 = \vectrtres{0}{1}{1}
\end{align*}
Luego: 
$$u_1 = \frac{v_1}{\norm{v_1}} = \frac{1}{\sqrt{(1)^2 + (1)^2 + (0)^2}}\vectrtres{1}{1}{0} = \frac{1}{\sqrt{2}} \vectrtres{1}{1}{0}$$
Para determinar $u_2$ se calcula:
$$u_2 = \frac{w_2}{\norm{w_2}} \text{, donde $w_2 = v_2 - \prodInt{v_2}{u_1}u_1$}$$
Ahora:
\begin{align*}
    w_2 &= \vectrtres{1}{1}{1} - \prodInt{\vectrtres{1}{1}{1}}{\frac{1}{\sqrt{2}}\vectrtres{1}{1}{0}}\frac{1}{\sqrt{2}}\vectrtres{1}{1}{0}\\
    &= \vectrtres{1}{1}{1} - \prodInt{\vectrtres{1}{1}{1}}{\vectrtres{1}{1}{0}}\left|\frac{1}{\sqrt{2}}\right|\left(\frac{1}{\sqrt{2}}\right) \vectrtres{1}{1}{0}\\
    &= \vectrtres{1}{1}{1} - (2)\left(\frac{1}{2}\right)\vectrtres{1}{1}{0}\\
    w_2 &= \vectrtres{0}{0}{1}
\end{align*}
y $\norm{w_2} = 1$, entonces el vector $u_2$ nos queda $u_2 = \vectrtres{0}{0}{1}$\\
Ahora para determinar $u_3$, se calcula:
$$u_3 = \frac{w_3}{\norm{w_3}} \text{, donde $w_3 = v_3 - \prodInt{v_3}{u_2}u_2 - \prodInt{v_3}{u_1}u_1$}$$
Ahora:

\begin{align*}
    w_3 &= \vectrtres{0}{1}{1} - \prodInt{\vectrtres{0}{1}{1}}{\vectrtres{0}{0}{1}}\vectrtres{0}{0}{1} - \prodInt{\vectrtres{0}{1}{1}}{\frac{1}{\sqrt{2}}\vectrtres{1}{1}{0}}\frac{1}{\sqrt{2}}\vectrtres{1}{1}{0}\\
    &= \vectrtres{0}{1}{1} - (1)\vectrtres{0}{0}{1} - \prodInt{\vectrtres{0}{1}{1}}{\vectrtres{1}{1}{0}} \left|\frac{1}{\sqrt{2}}\right|\left(\frac{1}{\sqrt{2}}\right)\vectrtres{1}{1}{0}\\
    &= \vectrtres{0}{1}{1} - \vectrtres{0}{0}{1} - (1)\left(\frac{1}{2}\right)\vectrtres{1}{1}{0}\\
    w_3 &= \vectrtres{-\frac{1}{2}}{\frac{1}{2}}{0}
\end{align*}

y $\norm{w_3} = \frac{1}{\sqrt{2}}$, entonces el vector $u_3$ nos queda $u_3 =(\sqrt{2}) \vectrtres{-\frac{1}{2}}{\frac{1}{2}}{0}$\\

Así termina el algoritmo de Gram-Schmidt, por lo que una base ortonormal para \rtres es la siguiente:
$$B_{ON} = \llav{\vectrtres{\frac{1}{\sqrt{2}}}{\frac{1}{\sqrt{2}}}{0} , \vectrtres{0}{0}{1}, \vectrtres{-\frac{\sqrt{2}}{2}}{\frac{\sqrt{2}}{2}}{0}}$$
\end{sol}

\section{Complemento Ortogonal}
\begin{dfn}
Sea $V$ un \dobleK-espacio vectorial con producto interno y sea $W$ un subespacio de $V$. Se define el conjunto:
$$\subesp{W^\perp}{v \in V}{\prodInt{v}{w} = 0 ; w \in W}$$
el cual se denomina como el \textit{complemento ortogonal} de $W$
\end{dfn}
Es decir, el complemento ortogonal de $W$ está conformado por aquellos vectores de $V$ que son ortogonales a todos los vectores de $W$

\begin{theorem}
    Sea $V$ un \dobleK-espacio vectorial con producto interno y sea $W$ un subespacio de $V$, entonces $W^\perp$ es un subespacio de $V$
\end{theorem}
\begin{proof}
Veamos que $W^\perp$ es un subespacio vectorial de $V$.
\begin{enumerate}[i.]
\item Como el vector $0_V$ es ortogonal a todos los demás vectores, entonces $W^\perp$ es no vacío
\item Supongamos que $u , u' \in W^\perp$, entonces para cada $v \in V, \prodInt{u}{v} = 0$ y $\prodInt{u'}{v} = 0$, ahora veamos si $\prodInt{u + u'}{v} = 0$, por propiedades del producto interno: $\prodInt{u}{v} + \prodInt{u'}{v} = 0+ 0 =0$, así que $u+u' \in W^\perp$
\item Si $\alpha \in \mathbb{K}$ y $u \in W^\perp$, se tiene que para cada $v \in V$, se cumple que $\prodInt{\alpha u}{v} = \alpha \prodInt{u}{v} = \alpha \cdot 0 = 0$, así entonces $\alpha u \in W^\perp$ 
\end{enumerate}
\end{proof}

\begin{theorem}
    Sea $V$ un \dobleK-espacio vectorial con producto interno. Si $W$ es un subespacio de $V$ y $B$ es una base de $W$ entonces $v \in W^\perp$ si y solo si $v$ es ortogonal a cada elemento de $B$
\end{theorem}

\begin{ejemplo}
    Sea $V = \rtres$ con el producto interno canónico, determine el complemento ortogonal de $W = gen\llav{\vectrtres{1}{0}{1}, \vectrtres{1}{1}{1}}$
\end{ejemplo}
\begin{sol}
Por definición, se tiene que:
$$\svrtres{W^\perp}{a}{b}{c}{\prodInt{\vectrtres{a}{b}{c}}{\vectrtres{1}{0}{1}} = 0 \land \prodInt{\vectrtres{a}{b}{c}}{\vectrtres{1}{1}{1}} = 0}$$
Podemos reescribir las condiciones como:
$$\svrtres{W^\perp}{a}{b}{c}{\begin{array}{rl}
    a + c &= 0\\ 
    a+b+c &= 0
\end{array}}$$
Así entonces:
$$W^\perp = gen\llav{\vectrtres{-1}{0}{1}}$$
\end{sol}

\begin{theorem}
    Sea $V$ un \dobleK-espacio vectorial con producto interno y sea $W$ un subespacio de $V$, entonces:
    $$V = W \oplus W^\perp $$
\end{theorem}
\begin{obs}
    La notación $\oplus$ representa que el espacio vectorial $V$ es \textit{suma directa} de $W$ y $W^\perp$.
\end{obs}
\begin{theorem}[corolario]
    Sea $V$ un \dobleK-espacio vectorial con producto interno y sea $W$ un subespacio de $V$, entonces:
    $$dimV = dimW + dimW^\perp $$
\end{theorem}
\begin{theorem}[corolario]
    Sea $V$ un \dobleK-espacio vectorial con producto interno y sea $W$ un subespacio de $V$, entonces:
    $$W \cap W^\perp = \llav{0_V}$$
\end{theorem}

\begin{theorem}
    Sea $V$ un \dobleK-espacio vectorial con producto interno y sea $W$ un subespacio de $V$, entonces:
    $$(W^\perp)^\perp = W$$
\end{theorem}

\section{Proyecciones}

\begin{dfn}
Sea $V = W \oplus W^\perp$, entonces cada $v \in V$ puede ser expresado de manera única como $v = w + w'$ donde $w \in W$ y $w' \in W^\perp$
\begin{itemize}
    \item Se define a $w$ como la proyección de $v$ a lo largo de $W$ y es denotado como $proy_{_W} v$.
    \item Se define a $w'$ como la proyección de $v$ a lo largo de $W^\perp$ y es denotado por $proy_{_{W^\perp}}v$.
\end{itemize}
En particular, si $W = gen\conjvect{w}{n}$, donde \conjvect{w}{n} es un conjunto de vectores ortonormales, entonces:
$$proy_{_W} v = \prodInt{v}{w_1}w_1 + \prodInt{v}{w_2}w_2 + \hdots + \prodInt{v}{w_n}w_n$$

~\\Además, la proyección de un vector $x$ sobre otro vector $v$, se define como:
$$proy_v x = \frac{\prodInt{x}{v}v}{\prodInt{v}{v}}$$
\end{dfn}

\begin{ejemplo}
    Considere el espacio $V = \rdos$ con el producto interno canónico. Sea 
    $$W = gen\llav{\vectrdos{1}{2}, \vectrdos{0}{1}}$$
    Expresar el vector $v = \vectrdos{2}{3} \in \rdos$ como $v = v_1 + v_2$ donde $v_1 \in W$ y $v_2 \in W^\perp$
\end{ejemplo}
\begin{sol}
Primero necesitamos determinar una base ortonormal a partir de $B_W$, entonces:
\begin{align*}
    u_1 &= \frac{v_1}{\norm{v_1}} = \frac{1}{\sqrt{(1)^2 + (2)^2}}\vectrdos{1}{2} = \frac{1}{\sqrt{5}}\vectrdos{1}{2}
\end{align*}
Luego, para determinar $u_2$ se calcula:
$$u_2 = \frac{w_2}{\norm{w_2}} \text{, donde $w_2 = v_2 - \prodInt{v_2}{u_1}u_1$}$$
Ahora:
$$w_2 = \vectrdos{0}{1} - \prodInt{\vectrdos{0}{1}}{\frac{1}{\sqrt{5}}\vectrdos{1}{2}}\frac{1}{\sqrt{5}}\vectrdos{1}{2}$$
Entonces:
\begin{align*}
    w_2 &= \vectrdos{0}{1} - (2)\left(\frac{1}{5}\right)\vectrdos{1}{2}\\
    w_2 &= \vectrdos{- 2/5}{1/5}
\end{align*}
Y $\norm{w_2} = \frac{\sqrt{5}}{5}$
Por lo tanto el vector $u_2$ nos queda:
$$u_2 = \frac{5}{\sqrt{5}}\vectrdos{-2/5}{1/5}$$

Entonces una base ortonormal para \rdos nos queda:
$$B_W = \llav{ \frac{1}{\sqrt{5}}\vectrdos{1}{2},\frac{5}{\sqrt{5}}\vectrdos{-2/5}{1/5}}$$

Entonces para determinar las proyecciones, tenemos:
\begin{align*}
    proy_{_W}v = v_1 &= \prodInt{v}{u_1}u_1 + \prodInt{v}{u_2}u_2\\
    v_1 &= \prodInt{\vectrdos{2}{3}}{\frac{1}{\sqrt{5}}\vectrdos{1}{2}}\frac{1}{\sqrt{5}}\vectrdos{1}{2} + \prodInt{\vectrdos{2}{3}}{\frac{5}{\sqrt{5}}\vectrdos{-2/5}{1/5}}\frac{5}{\sqrt{5}}\vectrdos{-2/5}{1/5}\\
    v_1 &= (8) \left( \frac{1}{5} \right) \vectrdos{1}{2} + \left(-\frac{1}{5}\right) (5) \vectrdos{-2/5}{1/5}\\
    v_1 &= \vectrdos{8/5}{16/5} + \vectrdos{2/5}{-1/5}\\
    v_1 &= \vectrdos{2}{3}
\end{align*}
Ahora para determinar $v_2$, lo podemos despejar de la igualdad $v = v_1 + v_2$, entonces:
\begin{align*}
    v_2 &= v - v_1\\
    v_2 &= \vectrdos{2}{3} - \vectrdos{2}{3}\\
    v_2 &= \vectrdos{0}{0}
\end{align*}
Así entonces el vector $v = \vectrdos{2}{3} \in V$ puede escribirse como:
$$\vectrdos{2}{3} = \vectrdos{2}{3} + \vectrdos{0}{0}$$
donde $v_1 = \vectrdos{2}{3} \in W$ y $v_2 = \vectrdos{0}{0} \in W^\perp$
\end{sol}

\begin{ejemplo}
    Sea $V = \rtres$, determine la proyección del vector $v = \vectrtres{1}{1}{0}$ sobre el vector $u = \vectrtres{0}{1}{1}$ 
\end{ejemplo}
\begin{sol}
Para esto, sabemos que:
$$proy_u v = \frac{\prodInt{\vectrtres{1}{1}{0}}{\vectrtres{0}{1}{1}}\vectrtres{0}{1}{1}}{\prodInt{\vectrtres{0}{1}{1}}{\vectrtres{0}{1}{1}}} = \frac{1}{2}\vectrtres{0}{1}{1} = \vectrtrescent{0}{1/2}{1/2}$$
\end{sol}

%vectores propios
\chapter{Valores y Vectores Característicos}
\section{Valores y Vectores Característicos}

\begin{dfn}
Sea $A$ una matriz $n\times n$. Un escalar $\lambda$ se dice que es un valor característico de $A$ si existe un vector no nulo $x$ en \rn \ tal que
$$A X = \lambda X$$
El vector $x$ se llama vector característico de $A$ correspondiente a $\lambda$.
\end{dfn}

\begin{ejemplo}
Sea $A = \left( \begin{array}{rrr}
1 & -2 & 1\\
0 & 0 & 0\\
0 & 1 & 1
\end{array} \right)$.\\
Compruebe que $x= (-3, -1, 1)$ es un vector característico de $A$.\\
\noindent En efecto 
\begin{align*}
\left( \begin{array}{rrr}
1 & -2 &1\\
0 & 0 & 0\\
0&1&1
\end{array} \right)
\vectrtres {-3}{-1}{1} &=
\left( \begin{array}{c}
-3+2+1\\
0\\
-1+1
\end{array} \right)\\
&= \vectrtres {0}{0}{0}\\
&= 0 \vectrtres {-3}{-1}{1}
\end{align*}
\end{ejemplo}


\begin{ejemplo}
Sea $A =\matrdxd {10 & -18}{6 & -11}$ y $v= \vectrdos{2}{1}$\\
\begin{align*}
\matrdxd{10 & -18}{6 & -11} \vectrdos{2}{1} &= \vectrdos {20-18}{12-11}\\
&= \vectrdos {2}{1}
\end{align*}
Así tenemos que 
$$\matrdxd{10 & -18}{6 & -11} \vectrdos {2}{1} = 1 \vectrdos {2}{1}$$
Lo cual indica que $\lambda = 1$ es un valor característico correspondiente al vector característico \vectrdos{2}{1}.
\begin{align*}
\matrdxd {10 & -18}{6 & -11} \vectrdos {3}{2} &= \vectrdos {30-36}{18-22}\\
&= \vectrdos {-6}{-4}\\
&= -2 \vectrdos {3}{2}
\end{align*}
De igual forma $\lambda = -2$ es el valor característico correspondiente al vector característico \vectrdos{3}{2}.
\end{ejemplo}


\begin{ejemplo}
Sea $A = \left( \begin{array}{rrr}
1 & 0 & 0\\
0 & 1 & 0\\
0 & 0 & 1
\end{array} \right)$ y $v = \vectrtres {a}{b}{c}$ un vector cualquiera
\begin{align*}
Av &= \left( \begin{array}{rrr}
1 & 0 & 0\\
0 & 1 & 0\\
0 & 0 & 1
\end{array} \right) \vectrtres{a}{b}{c}\\
&= \vectrtres{a}{b}{c}
\end{align*}
Entonces $\lambda = 1$ es un valor característico de cualquier vector $v$ y todo vector $v \neq 0$ es un vector característico de $A$.
\end{ejemplo}


\noindent Sea $\lambda$ un valor característico de $A$ correspondiente al vector característico $v \neq 0$. Entonces 
$$Av = \lambda v$$
Ahora como $\lambda v = \lambda I v$ donde $I$ es la matriz identidad, obtenemos que 
$$Av = \lambda I v$$
De donde se obtiene la igualdad 
$$(A - \lambda I) v = 0$$
\begin{align*}
\left[ \left( \begin{array}{cccc}
a_{11} & a_{12} & \ldots & a_{1n}\\
a_{21} & a_{22} & \ldots & a_{2n}\\
\vdots & \vdots & \ddots & \vdots\\
a_{n1} & a_{n2} & \ldots & a_{nn}
\end{array} \right) - \lambda \left( \begin{array}{cccc}
1 & 0 & \ldots & 0\\
0 & 1 & \ldots & 0\\
\vdots & \vdots & \ddots & \vdots\\
0 & 0 & \ldots & 1
\end{array} \right) \right] \left( \begin{array}{c}
x_1\\
x_2\\
\vdots\\
x_n
\end{array} \right) &= \left( \begin{array}{c}
0\\
0\\
\vdots\\
0
\end{array} \right)\\
\left( \begin{array}{cccc}
a_{11} - \lambda & a_{12} & \ldots & a_{1n}\\
a_{21} & a_{22} - \lambda & \ldots & a_{2n}\\
\vdots & \vdots & \ddots & \vdots\\
a_{n1} & a_{n2} & \ldots & a_{nn} - \lambda
\end{array} \right) \left( \begin{array}{c}
x_1\\
x_2\\
\vdots\\
x_n
\end{array} \right) &= \left( \begin{array}{c}
0\\
0\\
\vdots\\
0
\end{array} \right)
\end{align*}
$$\left\{ \begin{array}{ccl}
(a_{11}- \lambda)x_1 + a_{12} x_2 + \ldots + a_{1n} x_n & = & 0\\
a_{21}x_1 + (a_{22} - \lambda)x_2 + \ldots + a_{2n} x_n & = & 0\\
\vdots && \vdots\\
a_{n1} x_1 + a_{n2} x_2 + \ldots + (a_{nn} - \lambda) x_n & = & 0
\end{array}\right.$$

\noindent $An$, si $x = (x_1 , x_2 , \ldots , x_n)$ es un vector característico $\lambda$, se concluye que el sistema homogéneo anterior de $n$ ecuaciones tiene solución no trivial. Entonces $det (A - \lambda I) = 0$.\\
Recíprocamente, si $det(A- \lambda I)=0$, entonces el sistema homogéneo tiene soluciones no triviales y $\lambda$ es en consecuencia un valor característico.\\
Los argumentos anteriores son una prueba del siguiente teorema.
\begin{theorem}
Sea $A$ una matriz $n \times n$. Entonces $\lambda$ es un valor característico de $A$ si y solo si 
$$det(A- \lambda I) =0$$
\end{theorem}


\begin{theorem}
Sea $A$ una matriz $n \times n$ y $\lambda$ un valor característico de $A$. Entonces el conjunto $V_{\lambda} = \{v \in \rn : Av = \lambda v \} \cup  \{\mathbf{0}_{\rn}\}$ es un subespacio vectorial de \rn .
\end{theorem}

\begin{proof}
Por definición $\mathbf{0}_{\rn} \in \mathbb{V}_{\lambda}$.\\
Sean $v_1$ y $v_2$ $\in V_{\lambda}$, entonces 
\begin{align*}
A(v_1 + v_2) &= A v_1 + A v_2\\
&= \lambda v_1 + \lambda v_2\\
&= \lambda (v_1 + v_2)
\end{align*}
Por lo tanto $v_1 + v_2 \in V_{\lambda}$.
Sea $\lambda \in \mathbb{R}$ y $v \in \mathbb{V}_{\lambda}$
\begin{align*}
A(\alpha V) &= \alpha A v\\
&= \alpha \lambda V\\
&=\lambda (\alpha V)
\end{align*}
Lo cual implica que $\alpha V \in \mathbb{V}_{\lambda}$.\\
Por lo tanto $\mathbb{V}_{\lambda}$ es un subespacio vectorial de \rn .
\end{proof}


\begin{theorem}
Sea $A$ una matriz $n \times n$, \kvect{\lambda}{k} valores característicos de $A$ distintos. Entonces los vectores característicos \kvect{v}{k} relativos a los valores característícos \kvect{\lambda}{k}, son linealmente independientes.
\end{theorem}

\begin{proof}
Probemos el teorema por inducción.\\
Sea $n=2$\\
Entonces si $\lambda_1 \neq \lambda_2$ y $v_1 , v_2$ son los correspondientes vectores característicos, probaremos que son linealmente independientes.\\
Así, si $\alpha_1 v_1 + \alpha_2 v_2 = 0 \: (*)$ y multiplicamos por $A$ obtenemos 
\begin{align*}
A(\alpha_1 v_1) + A(\alpha_2 v_2) &= A \cdot 0 = 0\\
\alpha_1 A v_1 + \alpha_2 A v_2 &=0\\
\alpha_1 \lambda_1 v_1 + \alpha_2 \lambda_2 v_2 &= 0
\end{align*}
Si multiplicamos las ecuación $(*)$ por $\lambda_1$, obtenemos
$$\alpha_1 \lambda_1 v_1 + \alpha_2 \lambda_1 v_2 = 0$$
y si restamos estas dos últimas ecuaciones obtenemos $$\alpha_2 \lambda_2 v_2 - \alpha_2 \lambda_1 v_2 = 0$$
Es decir tenemos que $$\alpha_2 (\lambda_2 - \lambda_1)v_2 = 0$$
Por hipótesis $\lambda_2 -\lambda_1 \neq 0$ y $v_2 \neq \mathbf{0}$.\\
Entonces $\alpha_2 = 0$. Esto último implica que $\alpha_1 v_1 =0$ y como $v_1 \neq \mathbf{0}$ se concluye que $\alpha_1 =0$. Por lo tanto se prueba que $v_1$ y $v_2$ son linealmente independientes.\\
Supongamos ahora que para $1\leq n \leq k$ el teorema es cierto y veamos que también lo es para $n =k+1$.\\
Supongamos entonces que 
$$\beta_1 v_1 +\beta_2 v_2 + \ldots + \beta_k v_k + \beta_{k+1} v_{k+1}= 0\;(**)$$
y si multiplicamos por $A$ la ecuación $(**)$ obtenemos
$$\beta_1 A\, v_1 + \beta_2 A\, v_2 + \ldots +\beta_k A\, v_k + \beta_{k+1} A\, v_{k+1}$$
$$\beta_1 \lambda_1 v_1 + \beta_2 \lambda_2 v_2 + \ldots + \beta_k \lambda_k v_k + \beta_{k+1} \lambda_{k+1} v_{k+1}\; (***)$$
multipliquemos la ecuación $(**)$ por $\lambda_{k+1}$
$$\beta_1 \lambda_{k+1} v_1 + \beta_{2} \lambda_{k+1} v_2 + \ldots + \beta_{k} \lambda_{k+1} v_k + \beta_{k+1} \lambda_{k+1} v_{k+1} = 0$$
Si la última ecuación la restamos de $(***)$ obtenemos
$$\beta_1 (\lambda_{k+1} - \lambda_1)v_1 + \beta_2(\lambda_{k+1}- \lambda_2)v_2 + \ldots + \beta_k (\lambda{k+1} - \lambda_k)v_k + \beta{k+1}(\lambda_{k+1} - \lambda_{k+1})v_{k+1} =0$$
el cual queda como
$$\beta_1(\lambda_{k+1}-\lambda_1)v_1 + \beta_2 (\lambda_{k+1} - \lambda_2)v_2 + \ldots + \beta_k(\lambda_{k+1}-\lambda_k)v_k =0$$
Por hipótesis inductiva \conjvect{v}{R} son linealmente independientes y como $\lambda_{k+1}-\lambda_j \neq 0$ para $j = 1 \ldots k$. Obtenemos que $\beta_1 = \beta_2 = \ldots = \beta_k =0$.
La ecuación $(**)$ queda como
$$\beta_{k+1} v_{k+1} =0$$
lo cual implica que $\beta_{k+1}=0$ por cuanto $v_{k+1}\neq \mathbf{0}$
\end{proof}




\section{Multiplicidad Algebraica y Multiplicidad Geométrica de Valores Característicos}
\begin{dfn}
Sea $A$ una matriz $n \times n$ y sea $p(\lambda) = det (A - \lambda I)$ si $p(\lambda) = (\lambda - \lambda_1)^{r_1} (\lambda - \lambda_2)^{r_2} \ldots (\lambda- \lambda_m)^{r_m}$. Entonces los valores $r_i : i : 1, \ldots, m$ se llaman las multiplicidades algebraicas de los valores característicos $\lambda_i : i:1, \ldots, m$.
\end{dfn}

\begin{dfn}
Sea $\lambda$ un valor característico de la matriz $A$. Entonces la multiplicidad geométrica de $\lambda$ es la dimensión del espacio $\mathbb{V}_{\lambda}$ el cual es igual a la nulidad de la transformación lineal $T(v) = (A- \lambda I)v$
\end{dfn}

\begin{theorem}
Sea $A$ una matriz $n \times n$. A tiene $n$ vectores característicos linealmente independientes si y solo si cada uno de sus valores característicos tiene multiplicidad algebraica igual a 1.
\end{theorem}

\section{Diagonalización de Matrices Cuadradas}
\begin{dfn}
Sean $A$ y $B$ dos matrices $n \times n$. $A$ y $B$ se dicen que son semejantes si existe una matriz invertible $P$ tal que 
$$B= P^{-1} AP$$
\end{dfn}

\begin{dfn}
Una matriz $A$ $n \times n$ es diagonalizable si es semejante a una matriz diagonal. En otra palabras si existe una matriz diagonal $D$ y una matriz invertible $P$ tal que 
$$A=P^{-1}DP$$
\end{dfn}



\chapter{Problemas}

\begin{enumerate}

\begin{prob}[(3er Examen ESPOL 2012)]
Sea $V=C[0,1]$ y $f, g \in V$. Demuestre que si el conjunto \llav{f, g} es linealmente dependiente entonces
\[W=
\begin{vmatrix}
f(x)&g(x)\\
f'(x)&g'(x)\\
\end{vmatrix}=0
\]
\end{prob}
\sol
Si \llav{f, g} es un conjunto linealmente dependiente, entonces existe un escalar $k\in \dobler$ tal que $g(x)=kf(x)$:

\[\Rightarrow\
W=
\begin{vmatrix}
f(x)&g(x)\\
f'(x)&g'(x)\\
\end{vmatrix}
=
\begin{vmatrix}
f(x)&kf(x)\\
f'(x)&kf'(x)\\
\end{vmatrix}
=
kf(x)f'(x)-kf(x)f'(x)=0
\ \blacksquare\]

\begin{prob}[(1er Examen ESPOL 2012)]
Demuestre:
\\Sea $S=\conjvect{v}{n}$ un subconjunto linealmente independiente de vectores del espacio vectorial $V$ y sea $x$ un vector de $V$ que no puede ser expresado como una combinación lineal de los vectores de S, entonces $\llav{\kvect{v}{n}, x}$
también es linealmente independiente.
\end{prob}

\sol
Tenemos que demostrar que $\llav{\kvect{v}{n}, x}$ es l.i.
Consideremos que

\begin{equation}\label{eq1}
\alpha_1 v_1+
\alpha_2 v_2+
\ldots+
\alpha_n v_n+
\beta x =0_v
\end{equation}
Para algunos escalares $ \alpha_i, \beta \in \dobler; i=1...n$
Al operar esta ecuación tenemos lo siguiente
\begin{equation}\label{eq2}
\beta x=(-\alpha_1) v_1+
(-\alpha_2) v_2+
\ldots+
(-\alpha_n) v_n
\end{equation}
Existen dos casos, que $\beta\neq 0$ o que $\beta=0$
~\\
\underline{Si $\beta\neq 0$}:\ 
Si en \ref{eq2}, multiplicamos por $\frac{1}{\beta}$ (estamos seguro de que existe, ya que $\beta$ es diferente de cero) obtendremos lo siguiente:
\begin{equation*}
x=(-\frac{\alpha_1}{\beta}) v_1+
(-\frac{\alpha_2}{\beta}) v_2+
\ldots+
(-\frac{\alpha_n}{\beta}) v_n
\end{equation*} 
Lo anterior es imposible ya que contradice la hipótesis de que x no puede escribirse como combinación lineal de los vectores de S, as\'i que se concluye que $\beta=0$.
Ya conociendo que $\beta=0$, en \ref{eq1} reemplazamos y tenemos
\begin{equation}
\alpha_1 v_1+
\alpha_2 v_2+
\ldots+
\alpha_n v_n=0_v
\end{equation}
Pero como S es un conjunto l.i. implica que $\alpha_i=0; i=1,...n$
Con esto se concluye que el conjunto $\llav{\kvect{v}{n}, x}$ es l.i.$\blacksquare$

\begin{prob}[(1er Examen ESPOL 2009)]
Sean $u=\vectrdos{x_1}{y_1}, v=\vectrdos{x_2}{y_2}, w=\vectrdos{x_3}{y_3}$ elementos de $ \rdos$ y sea
\[A=
\left(
\begin{array}{rrr}
x_1&y_1&1\\
x_2&y_2&1\\
x_3&y_3&1
\end{array}
\right)
\]
\begin{enumerate}
\item[(a)] Si u, v, w son colineales ?`cuál es el rango de A?
\item[(b)] Determine una base para el espacio fila de A.
\end{enumerate}
\end{prob}

\sol
(a) Si u, v, w son colineales, entonces deben ser m\'ultiplos
 $$v=\vectrdos{a x_1}{a y_1}, w=\vectrdos{b x_1}{b y_1}$$
por lo que la matriz A ser\'ia
\[A=
\left(
\begin{array}{rrr}
x_1&y_1&1\\
a x_1&a y_1&1\\
b x_1&b y_1&1
\end{array}
\right)
\]
\begin{eqnarray*}
C_A &=gen\llav{
\vectrtres{x_1}{ax_1}{b x_1}, 
\vectrtres{y_1}{a y_1}{b y_1}, 
\vectrtres{1}{1}{1}
}\\&=
gen\llav{
\vectrtres{1}{a}{b}, 
\vectrtres{1}{a}{b}, 
\vectrtres{1}{1}{1}
}\\&=
gen\llav{
\vectrtres{1}{a}{b}, 
\vectrtres{1}{1}{1}
}
\end{eqnarray*}
De esta manera, es claro que $\rho(A)=1$ si $a=1 \wedge b=1$
en cualquier otro caso $\rho(A)=2$.
~\\
(b) Si $a=1 \wedge b=1$ entonces una base para el espacio fila ser\'ia $$B_{F_A}=\llav{
\vectrtres{x_1}{ y_1}{ 1}
}$$

Si $a=1 \wedge b\neq 1$ una base de $F_A$ ser\'ia
\[B_{F_A}=\llav{
\vectrtres{x_1}{ y_1}{ 1},
\vectrtres{b x_1}{b y_1}{1}
}\]

Si $a\neq 1 \wedge b=1$ una base de $F_A$ ser\'ia
\[B_{F_A}=\llav{
\vectrtres{x_1}{ y_1}{ 1},
\vectrtres{a x_1}{a y_1}{1}
}\]
\newpage
\begin{prob}
Dada la matriz $A=
\left(
\begin{array}{rrr}
1&10&1\\
2&2&2\\
-1&-8&k
\end{array}
\right)
$, determine los valores de $k$ para que la nulidad de $A$ sea cero.
\end{prob}

\sol

Por el criterio del determinante, la nulidad de A es cero si $det(A)\neq 0$
\[det(A)\neq 0\]
\[1(2k+16)-10(2k+2)+1(-16+2)\neq 0\]
\[-18k-18\neq 0\]
\[k\neq -1\].

\newpage
\begin{prob}[(1er Examen ESPOL 2007)]
Sea $A$ la matriz de coeficientes del sistema lineal
\begin{eqnarray*}
2x+y-z=a\\
x-y+2z=b\\
x+2y-3z=c
\end{eqnarray*}
\begin{enumerate}
\item[a)] Determine el espacio fila, n\'ucleo y recorrido de $A$
\item[b)] Si $c=2a+b$, determine si el vector \vectrtres{a}{b}{c} pertenece a $Im(A)$
\end{enumerate}
\end{prob}

\sol

De acuerdo al sistema se tiene que
\[A=
\left(
\begin{array}{rrr}
2&1&-1\\
1&-1&2\\
1&2&-3\\
\end{array}
\right)
\]
Espacio fila:
\[F_A=gen\llav{
\vectrtres{2}{1}{-1}, \vectrtres{1}{-1}{2}, \vectrtres{1}{2}{-3}
}\]
Lo que conduce al sistema

\[A=
\left(
\begin{array}{rrr|r}
2&1&1&x\\
1&-1&2&y\\
-1&2&-3&z\\
\end{array}
\right)
\sim
\left(
\begin{array}{rrr|r}
2&1&1&x\\
0&3&-3&x-2y\\
0&5&-5&x+2z\\
\end{array}
\right)
\sim
\left(
\begin{array}{rrr|r}
2&1&1&x\\
0&3&-3&x-2y\\
0&0&0&2x-10y-6z\\
\end{array}
\right)
\]
Por lo que
\[F_A=\llaves{\vectrtres{x}{y}{z}}{2x-10y-6z=0}\]

N\'ucleo:

El n\'ucleo es la solución del sistema homog\'eneo $AX=0$:
\[
\left(
\begin{array}{rrr|r}
2&1&-1&0\\
1&-1&2&0\\
1&2&-3&0\\
\end{array}
\right)
\sim
\left(
\begin{array}{rrr|r}
2&1&-1&0\\
0&3&-5&0\\
0&-3&5&0\\
\end{array}
\right)
\sim
\left(
\begin{array}{rrr|r}
2&1&-1&0\\
0&3&-5&0\\
0&0&0&0\\
\end{array}
\right)
\]

Por lo que:
\[Nu(A)=\llaves{\vectrtres{x_1}{x_2}{x_3}}
{\begin{array}{r}
2x_1+x_2-x_3=0\\
3x_2-5x_3=0
\end{array}}
\]

Recorrido o Imagen de A:

El recorrido es igual al espacio columna, el espacio generado por las columnas:

\[Rec(A)=Im(A)=C_A\]

\[
\left(
\begin{array}{rrr|r}
2&1&-1&x\\
1&-1&2&y\\
1&2&-3&z\\
\end{array}
\right)
\sim
\left(
\begin{array}{rrr|r}
2&1&-1&x\\
0&3&-5&x-2y\\
0&-3&5&x-2z\\
\end{array}
\right)
\sim
\left(
\begin{array}{rrr|r}
2&1&-1&x\\
0&3&-5&x-2y\\
0&0&0&2x-2y-2z\\
\end{array}
\right)
\]


Por lo que
\[C_A=\llaves{\vectrtres{x}{y}{z}}{2x-2y-2z=0}\]

b) Si $c=2a+b$ entonces para que el vector \vectrtres{a}{b}{c} pertenezca a $C_A$ debe cumplirse la condición de exige $C_A$
\[
2x-2y-2z=2(a)-2(b)-2(2a+b)=-2a-4b \neq 0
\]

Por tanto, no pertenece a $Im(A)$.

\newpage
\begin{prob}
Sea $V=\mathcal{M}_{3x2}$. Sean $W_1$ el conjunto de las matrices que tienen la primera y la última fila iguales, $W_2$ el conjunto de las matrices que tienen la primera columna igual a su segunda columna.
~\\Determine:~\\
\begin{enumerate}
\item[a)] La intersección entre $W_1, W_2$
\item[b)] La suma entre $W_1, W_2$
\item[c)] Una base para los subespacios intersección y suma obtenidos en (a) y (b).

\end{enumerate}
\end{prob}

\sol

De acuerdo a lo indicado tenemos que

\[W_1=\llaves{
\left(
\begin{array}{rr}
a_{11}&a_{12}\\
a_{21}&a_{22}\\
a_{31}&a_{32}
\end{array}
\right)
}{
\begin{array}{r}
a_{11}=a_{31}\\
 a_{12}=a_{32}
\end{array}}
\]


\[W_2=\llaves{
\left(
\begin{array}{rr}
a_{11}&a_{12}\\
a_{21}&a_{22}\\
a_{31}&a_{32}
\end{array}
\right)
}{
\begin{array}{r}
a_{11}=a_{12}\\
a_{21}=a_{22}\\
a_{31}=a_{32}
\end{array}
}
\]
Para la intersección tenemos que

\[W_1\cap W_2=\llaves{
\left(
\begin{array}{rr}
a_{11}&a_{12}\\
a_{21}&a_{22}\\
a_{31}&a_{32}
\end{array}
\right)
}{\begin{array}{rr}
a_{11}=a_{12}&a_{11}=a_{31}\\
a_{21}=a_{22}&a_{12}=a_{32}\\
a_{31}=a_{32}&
\end{array}
}
\]
Si colocamos todas estas ecuaciones en un sistema homog\'eneo e igualamos a cero tendr\'iamos

%\[
%  \begin{blockarray}{rrrrrrrr}
%&a_{11}&a_{12}&a_{21}&a_{22}&a_{31}&a_{32}&\ \\
%    \begin{block}{r(rrrrrr|r)}
%&1    &-1     &0     &0     &0     &0     &0\\
%&0    &0      &1     &-1    &0     &0     &0\\
%&0    &0      &0     &0     &1     &-1     &0\\
%&1    &0      &0     &0     &-1     &0     &0\\
%&0    &1     &0     &0     &0     &-1     &0\\
%    \end{block}
%  \end{blockarray}
%\]

\[
\left(
\begin{array}{rrrrrr|r}
1    &-1     &0     &0     &0     &0     &0\\
0    &0      &1     &-1    &0     &0     &0\\
0    &0      &0     &0     &1     &-1     &0\\
1    &0      &0     &0     &-1     &0     &0\\
0    &1     &0     &0     &0     &-1     &0\\
\end{array}
\right)
\sim
\ldots
\sim
\left(
\begin{array}{rrrrrr|r}
1    &-1     &0     &0     &0     &0     &0\\
0    &1     &0     &0     &0     &-1     &0\\
0    &0      &1     &-1    &0     &0     &0\\
0    &0      &0     &0     &1     &-1     &0\\
0    &0      &0     &0     &0     &0     &0\\
\end{array}
\right)
\]

Donde queda claro que había una ecuación redundante, por lo tanto


\[W_1\cap W_2=\llaves{
\left(
\begin{array}{rr}
a_{11}&a_{12}\\
a_{21}&a_{22}\\
a_{31}&a_{32}
\end{array}
\right)
}{\begin{array}{r}
a_{11}=a_{32}\\
a_{21}=a_{22}\\
a_{12}=a_{32}\\
a_{31}=a_{32}\\
\end{array}
}
\]
Para hallar una base:
\[\left(
\begin{array}{rr}
a_{32}&a_{32}\\
a_{22}&a_{22}\\
a_{32}&a_{32}
\end{array}
\right)
=a_{22}
\left(
\begin{array}{rr}
0&0\\
1&1\\
0&0
\end{array}
\right)
+a_{32}
\left(
\begin{array}{rr}
1&1\\
0&0\\
1&1
\end{array}
\right)
\]

De esta manera obtenemos una base para la intersección:


%
%\begin{prob}[(1er Examen ESPOL 2012)]
%Sea $T:\rdos \rightarrow \rdos $ 
%\end{prob}

\newpage
\begin{prob}[(3ra Evaluacion Septiembre 2012)]
(10 puntos)Dado la matriz $A=\left(\begin{matrix}
1&10&1\\
2&2&2\\
-1&-1&k\\
\end{matrix}\right)$
, determine los valores de k para que la nulidad sea cero.
\end{prob}

%
%\begin{prob}[(3ra Evaluacion Septiembre 2012)]
%(10 puntos)Sea $V=C[0,1]$ y $fm g \in V$. Demuestre que si el conjunto ${f, g}$ es linealmente dependiente entonces:
%\[W=\left|\begin{matrix}
%f(x)&g(x)\\
%f'(x)&g'(x)\\
%\end{matrix}\right|=0\]
%\end{prob}
%

\newpage
\begin{prob}[(3ra Evaluacion Febrero 2009)]
(10 puntos) Califique como V o F:Sea $A=\left(\begin{matrix}
2a&2b&-c\\
a&2b&-c\\
-a&-b&c\\
\end{matrix}\right)$ con a, b, c diferentes de cero, entonces el rango de A es igual a 3 y la nulidad de A es igual a 0.
\end{prob}

\newpage

\begin{prob}[(3ra Evaluacion Abril 2011)]
(10 puntos)Califique como Vo F: Sea $A=\mathcal{M}_{2x3}$. Entonces la nulidad de A es mayor o igual a 3,
\end{prob}



\newpage
\begin{prob}[(1ra Evaluacion Julio 2012)]
(20 puntos)Sea $V=\ptres$. Considere el conjunto de todos los subespacios de V tal que $$H(a)=gen\llav{1+ax+x^2+x^3, 1+ax+(1-a)x^2+x^3, x+(2a)x^2+2x^3, 1+(1+a)x+(1+a)x^2+3x^3}$$
a) Determine el valor de $a$ para que $dimH=2$ \\
b) Halle una base y la dimensión de los subespacios $H(0)\cap H(1)$ y $H(0)+ H(1)$
\end{prob}


\newpage

\newpage

%Megas 2do parcial~\\
%~\\
%\begin{prob}[(3ra Evaluacion 14 de febrero 2014)]
%(10 puntos) 
%a)Demuestre que $Nu(T)\subseteq Nu(T^{2})$
%~\\
%b) Si ademas $\rho(T)=\rho(T^{2})$ , demuestre que $Nu(T)= Nu(T^{2})$
%\end{prob}

%
%\begin{prob}[(2da Evaluacion X X)]
%(10 puntos) Verdadero o Falso: Sea $L:V->V$ un isomorfismo. Sea $B=\{v_1, v_2, v_3\}$
% una base del espacio vectorial V. Entonces $B2=\{L(v_1), L(v_2), L(v_3)\}$ es una base de V.
%\end{prob} 
%
%\begin{prob}[(2da Evaluacion X X)]
%(20 puntos) Encuentre de ser posible una matriz diagonal D semejante a:
%$A=\left(\begin{array}{rrr}
%-1&2&1\\
%0&-1&0\\
%-1&-3&-3\\
%\end{array}\right)$ 
%\end{prob} 
%
%
%\begin{prob}[(2da Evaluacion X X)]
%(10 puntos) Verdadero o Falso: La matriz $A=\left(\begin{array}{rr}
%k&0\\
%1&3\\
%\end{array}\right)$ es diagonalizable para todo k
%\end{prob}
%
%
%
%\begin{prob}[(2da Evaluacion X X)]
%(10 puntos) Construya de ser posible un operador lineal L en el espacio vectorial \pdos que cumpla las siguientes condiciones: $L(1+x)=-1-x y E(\lambda=3)=gen{2-x}.$ ¿Es diagonalizable este operador? Justifique su respuesta.
%\end{prob}
%
%

%\begin{prob}[(2da Evaluacion X X)]
%(15 puntos) Sea ai=(xi,yi) un vector que indica la posición de una partícula sobre un punto al cabo de t periodos, suponga que:
%$\forall t \in N a_t=\left(\begin{array}{rr}
%1/2&1\\
%0&1/3\\
%\end{array}\right)a_{t-1}$ 
%Si la posición inicial de la partícula es $a_o=(1,-1)$ Determine:
%~\\
%a)La posición de la partícula luego de T=2 periodos
%B)La posicion luego de t periodos
%c) La posicion en estado estable, es decir t=inf.
%
%\end{prob} 
%

%\begin{prob}[(2da Evaluacion X X)]
%(10 puntos) Sean A,B dos matrices semejantes:
%~\\a)Muestre que A y B tienen los mismos valores propios
%~\\b) Tienen los mismos vectores propios? Justifique se respuesta
%
%\end{prob} 
%\begin{prob}[(2da Evaluacion X X)]
%(10 puntos) Sea $A=\left(\begin{array}{rrr}
%1&0&k\\
%3&3&-3\\
%1&0&2\\
%\end{array}\right)$ 
%~\\a) Determine el valor de k para que $\lambda=5$ sea valor propio de A
%~\\b) Para k=20 determine si A es diagonalizable
%\end{prob} 
%
%
%\begin{prob}[(2da Evaluacion X X)]
%(10 puntos) Verdadero o Falso:
%~\\
%La funcion $f:\pdos x \pdos -> R$, definida por $f(p(x),q(x))=p(1)q(1)$ es un producto interno en \pdos
%
%\end{prob} 

%
%\begin{prob}[(2da Evaluacion X X)]
%(15 puntos) Sea (.,.) el producto interno real estandar en el espacio vectorial \rtres. Sea L un operador linel en \rtres
%\[L(x,y,x)=(x,x+y,x+y+z)\]
%Es tambien un producto interno real en \rtres la funcion $(.,.)_{L}$
%\[(v1,v2)_{L}=(L(v1),L(v2))\]
%Sea W un subespacio vectorial tal que
%\[W=gen\{(1,1,1),(-1,0,1)   \}\]
%a) Encuentre una base y determine el complemento ortogonal de W.
%~\\b)Expresar el vector (0,1,1) como la suma de un vector de W y un vector de $W*$
%
%\end{prob} 


%
%\begin{prob}[(2da Evaluacion X X)]
%(10 puntos) Aplicando diagonalizacion ortogonal grafique el lugar geométrico $5x^2+4xy+2y^2-24x-12y+29=0$
%\end{prob} 

\newpage
\subsubsection{Califique como verdaderas o falsas las siguientes proposiciones}
\begin{prop}[(1er Examen ESPOL 2012)]

Sea $V$ un espacio vectorial tal que $H\subseteq V$. Si $H$ es un subespacio vectorial de $V$ entonces $H^C$ es subespacio de $V$.

\end{prop}

\sol
~\\
Debido a que $H$ es un subespacio vectorial(también es espacio vectorial), el elemento neutro $0_v$ pertenece a H, pero esto implica que $0_v \notin H^C$.

~\\
Si $0_v \notin H^C$ entonces $H^C$ no puede ser espacio vectorial, luego, tampoco subespacio vectorial de $ V$. 
$\therefore$ La proposición es FALSA.

~\\
~\\
~\\
\begin{prop}[(1er Examen ESPOL 2009)]
Si $(V, \oplus, \odot)$ es un espacio vectorial, y sean $H$ y $W$ dos subespacios de V tales que:
$W=gen\llav{w_1, w_2, w_3}$ y $w_1, w_2  \in H$, entonces es cierto que $dim(H\cap W)=2$
\end{prop}

\sol
Considere el siguiente contraejemplo:
Sea $V={\cal P}_3$ y $W=gen\llav{1, x, x^2}$, y además $H=gen\llav{1, x, x^2}$ (Por supuesto, esto es intencional). Es claro que $1, x \in H$. Pero $dim(W\cap H)=3$.
$\therefore$ La proposición es FALSA.

\newpage
\begin{prop}[(1er Examen ESPOL 2007)]
Si la matriz B se obtiene a partir de la matriz A por medio de un intercambio de filas entonces $\rho(A)=\rho(B)$
\end{prop}

\sol
Sean $f_1, f_2, \ldots, f_i, \ldots , f_j, \ldots, f_m$ las filas de A, luego $$F_A=gen\llav{f_1, f_2, \ldots, f_i, \ldots , f_j, \ldots, f_m}$$. Si B se obtiene al intercambiar las filas $f_i, f_j$ entonces, 
$$F_B=gen\llav{f_1, f_2, \ldots, f_j, \ldots , f_i, \ldots, f_m}$$
$$F_B=gen\llav{f_1, f_2, \ldots, f_i, \ldots , f_j, \ldots, f_m}$$
$$F_B=F_A$$
Por lo tanto, $dimF_A=dim F_B$, esto es lo mismo que $\rho(A)=\rho(B)$.
$\therefore $ La proposición es VERDADERA.
~\\
~\\
~\\



\newpage


\newpage
\newpage


\begin{prob}[]
En el espacio vectorial \mdosxdos, determinar si 
\matrdxd{2&1}{-1&0}
 es combinación lineal de los vectores 
\matrdxd{1&0}{-1&0}, \matrdxd{-1&2}{0&1}, \matrdxd{0&1}{2&-1}.
\end{prob}

\newpage

\begin{prop}[Califique como verdadero o falso]
Sea $V$ un espacio vectorial. Sean A, B $\subseteq V$, entonces $gen(A\cap B)=gen(A)\cap gen(B)$
\end{prop}
\sol
Por contraejemplo:
~\\
Sea $V=\rtres$, además~\\
\[A=\left\lbrace \vectrtres{1}{0}{0}, \vectrtres{0}{1}{0} \right\rbrace\  ;  B=\left\lbrace \vectrtres{0}{0}{1}, \vectrtres{0}{-1}{0} \right\rbrace\]
Es evidente que:~\\
\[H=gen(A)=\llaves{\vectrtres{a}{b}{c}}{c=0}\]
\[W=gen(B)=\llaves{\vectrtres{a}{b}{c}}{a=0}\]


Si calculamos $H\cap W$ tendríamos:
~\\
\[H\cap W=\llaves{\vectrtres{a}{b}{c}}{a=0, c=0}\]

Ahora, si calculamos $gen(A \cup B)$, (A y B no tienen elementos en común)~\\
\[A\cap B=\phi \]
Con esto se tiene que
\[gen(A\cap B)=\lbrace n_v \rbrace\]~\\
Luego
\[gen(A\cap B)\neq gen(A)\cap gen(B)\]

$\therefore$ La proposición es falsa
\newpage






\newpage

%repetido
%\begin{prob}[]
%
%Sea $V=C[0,1]$ y $f, g \in V$. Demuestre que si el conjunto $\{f, g\}$ es linealmente dependiente entonces:
%\[W=\left|\begin{matrix}
%f(x)&g(x)\\
%f'(x)&g'(x)\\
%\end{matrix}\right|=0\]
%\end{prob}

\newpage



\newpage

%
%\begin{prob}
%Sea $A$ la matriz de coeficientes del sistema lineal
%\begin{eqnarray*}
%2x+y-z=a\\
%x-y+2z=b\\
%x+2y-3z=c
%\end{eqnarray*}
%\begin{enumerate}
%\item[a)] Determine el espacio fila, n\'ucleo y recorrido de $A$
%\item[b)] Si $c=2a+b$, determine si el vector \vectrtres{a}{b}{c} pertenece a $Im(A)$
%\end{enumerate}
%\end{prob}
%
%\sol
%
%De acuerdo al sistema se tiene que
%\[A=
%\left(
%\begin{array}{rrr}
%2&1&-1\\
%1&-1&2\\
%1&2&-3\\
%\end{array}
%\right)
%\]
%Espacio fila:
%\[F_A=gen\llav{
%\vectrtres{2}{1}{-1}, \vectrtres{1}{-1}{2}, \vectrtres{1}{2}{-3}
%}\]
%Lo que conduce al sistema
%
%\[A=
%\left(
%\begin{array}{rrr|r}
%2&1&1&x\\
%1&-1&2&y\\
%-1&2&-3&z\\
%\end{array}
%\right)
%\sim
%\left(
%\begin{array}{rrr|r}
%2&1&1&x\\
%0&3&-3&x-2y\\
%0&5&-5&x+2z\\
%\end{array}
%\right)
%\sim
%\left(
%\begin{array}{rrr|r}
%2&1&1&x\\
%0&3&-3&x-2y\\
%0&0&0&2x-10y-6z\\
%\end{array}
%\right)
%\]
%Por lo que
%\[F_A=\llaves{\vectrtres{x}{y}{z}}{2x-10y-6z=0}\]
%
%N\'ucleo:
%
%El n\'ucleo es la solución del sistema homog\'eneo $AX=0$:
%\[
%\left(
%\begin{array}{rrr|r}
%2&1&-1&0\\
%1&-1&2&0\\
%1&2&-3&0\\
%\end{array}
%\right)
%\sim
%\left(
%\begin{array}{rrr|r}
%2&1&-1&0\\
%0&3&-5&0\\
%0&-3&5&0\\
%\end{array}
%\right)
%\sim
%\left(
%\begin{array}{rrr|r}
%2&1&-1&0\\
%0&3&-5&0\\
%0&0&0&0\\
%\end{array}
%\right)
%\]
%
%Por lo que:
%\[Nu(A)=\llaves{\vectrtres{x_1}{x_2}{x_3}}
%{\begin{array}{r}
%2x_1+x_2-x_3=0\\
%3x_2-5x_3=0
%\end{array}}
%\]
%
%Recorrido o Imagen de A:
%
%El recorrido es igual al espacio columna, el espacio generado por las columnas:
%
%\[Rec(A)=Im(A)=C_A\]
%
%\[
%\left(
%\begin{array}{rrr|r}
%2&1&-1&x\\
%1&-1&2&y\\
%1&2&-3&z\\
%\end{array}
%\right)
%\sim
%\left(
%\begin{array}{rrr|r}
%2&1&-1&x\\
%0&3&-5&x-2y\\
%0&-3&5&x-2z\\
%\end{array}
%\right)
%\sim
%\left(
%\begin{array}{rrr|r}
%2&1&-1&x\\
%0&3&-5&x-2y\\
%0&0&0&2x-2y-2z\\
%\end{array}
%\right)
%\]
%
%
%Por lo que
%\[C_A=\llaves{\vectrtres{x}{y}{z}}{2x-2y-2z=0}\]
%
%b) Si $c=2a+b$ entonces para que el vector \vectrtres{a}{b}{c} pertenezca a $C_A$ debe cumplirse la condición de exige $C_A$
%\[
%2x-2y-2z=2(a)-2(b)-2(2a+b)=-2a-4b \neq 0
%\]
%
%Por tanto, no pertenece a $Im(A)$.
\newpage

%
%\begin{prob}[]
%
%Sean $A\in \mathcal{M}_{mxn}, B \in \mathcal{M}_{nxp}$, demuestre que $C_{AB}\subseteq C_A$
%
%\end{prob}



%
%\begin{prob}[]
%
%Sea $V=\ptres$. Considere el conjunto de todos los subespacios de V tal que $$H(a)=gen\llav{1+ax+x^2+x^3, 1+ax+(1-a)x^2+x^3, x+(2a)x^2+2x^3, 1+(1+a)x+(1+a)x^2+3x^3}$$
%Halle una base y la dimensión de los subespacios $H(0)\cap H(1)$ y $H(0)+ H(1)$
%\end{prob}


%\begin{prob}
%
%Sea $V=\mathcal{M}_{3x2}$. Sean $W_1$ el conjunto de las matrices que tienen la primera y la última fila iguales, $W_2$ el conjunto de las matrices que tienen la primera columna igual a su segunda columna.
%~\\Determine:~\\
%\begin{enumerate}
%\item[a)] La intersección entre $W_1, W_2$
%\item[b)] La suma entre $W_1, W_2$
%\item[c)] Una base para los subespacios intersección y suma obtenidos en (a) y (b).
%
%\end{enumerate}
%
%\end{prob}






\end{enumerate}


%%%%%%%%%%%%%%%%%%%%%%%%%%%%%%%%%%%%%%%%%%%%%%%%
%%%%%%%%%%%%%%%%%%%%%%%%%%%%%%%%%%%%%%%%%%%%%%%
%\section{Combinaciones lineales y subespacios generados}
%\begin{dfn}
%Sea $(E, +, \odot)$ un espacio vectorial real y sean $B=\conjvect{v}{n}$ un subconjunto de vectores en $E$. Un vector $v$ en $E$ se dice que es una combinación lineal de los vectores en $B$ si existen escalares $\alpha_1$, $\alpha_2$, $\ldots$ , $\alpha_n$ en \dobler , tales que $v=\av{\alpha_1}{v_1} + \av{\alpha_2}{v_2} + \ldots + \av{\alpha_n}{v_n}$.
%
%\end{dfn}
%
%\begin{ejemplo}
%Cualquier vector $(a, b, c)$ en \rtres puede ser expresado como una combinación lineal de los vectores  $(1,0,0)$,$(0,1,0)$ y $(0,0,1)$.
%
%\end{ejemplo}
%
%\begin{ejemplo}
%El vector $(4,5,5)$ puede ser expresado como una combinación lineal de los vectores $(1,2,3)$, $(-1,1,4)$ y $(3,3,2)$.
%\end{ejemplo}
%
%\begin{theorem}
%Sea $(E, +, \odot)$ un espacio vectorial y $A$ un subconjunto finito de $E$. El conjunto de todas las combinaciones lineales de elementos de $A$ es un subespacio de $E$ y se llama subespacio generado por $A$ y se denota por $CL(A)$.
%\end{theorem}
%
%\begin{proof}
%
%$\mathbf{0}_v \in CL(A)$ por el comentario anterior.\\
%Sean $u$, $v$ en $CL(A)$, entonces
%\begin{align*}
%u &=\av{\alpha_1}{v_1} + \ldots + \av{\alpha_n}{v_n}\\ 
%v&=\av{\beta_1}{v_1} + \ldots + \av{\beta_n}{v_n}
%\end{align*}
%entonces
%\begin{align*}
%u + v &= (\av{\alpha_1}{v_1} + \ldots + \av{\alpha_n}{v_n}) + (\av{\beta_1}{v_1} + \ldots + \av{\beta_n}{v_n})\\
%u + v &=(\alpha_1 + \beta_1)\,v_1 + \ldots + (\alpha_n + \beta_n)\, v_n
%\end{align*}
%Así
%$$u + v \in CL(A)$$
%Sea $u \in CL(A)$ y $\alpha \in \dobler$
%\begin{align*}
%\alpha u &= \alpha(\av{\alpha_1}{v_1} + \ldots + \av{\alpha_n}{v_n})\\
%\alpha u &= (\alpha \alpha_1)v_1 + \ldots + (\alpha \alpha_n)v_n
%\end{align*}
%luego $\alpha u \in CL(A)$\\
%
%$\therefore$ $CL(A)$ es un subespacio vectorial de $V$. \qedhere \\ %%%%%%%  el \qedhere es el QED
%
%\end{proof}
%El subespacio $CL(A)$ es llamado espacio generado por \conjvect{v}{n}. Si $V = CL(A)$ entonces decimos que $A$ genera al espacio $V$ o que $A$ es conjunto generador de $V$.
%
%\begin{ejemplo}
%Veamos que $A = \llav{(1,2,0),(0,1,-1),(1,1,2)}$ genera a \rtres .\
%En efecto veamos que cualquier vector $(x,y,z)$ es una combinación lineal de los elementos de $A$. Es decir debemos encontrara escalares $\alpha$, $\beta$ y $\lambda \in \dobler$ tales que $$(x_0, y_0, z_0) = \alpha (1,2,0) + \beta (0,1,-1) + \lambda (1,1,2)$$
%lo que nos conduce al estudio del sistema 
%
%$$\left\{
%\begin{array}{rcl}
%\alpha + \lambda &=& x_0\\
%2 \alpha + \beta + \lambda &=& y_0\\
%- \beta + 2\alpha &=& z_0 
%\end{array}
%\right.$$
%
%Donde se obtiene que 
%\begin{align*}
%\alpha &= 3x_0 - y_0 -z_0\\
%\beta &=-4 x_0 +2 y_0 +z_0\\
%\lambda &= -2 x_0 + y_0 +z_0 
%\end{align*}
%
%\end{ejemplo}

\begin{enumerate}
\begin{prob}[]
Demuestre:
Sea $B=\conjvect{v}{n}$ una base del espacio vectorial $V$, entonces cualquier conjunto de m\'as de n vectores es linealmente dependiente
\end{prob}

\end{enumerate}
%%%%%%%% esta sección son los modelos de comandos para secciones
%%ejercicio
%\begin{ejercicio}
%Sea V un espacio vectorial etc etc
%\end{ejercicio}
%
%\begin{ejemplo}
%Sea V un espacio vectorial etc etc
%\end{ejemplo}
%
%\begin{theorem}[Teorema de las dimensiones -es opcional]
%Sea V un espacio vectorial etc etc
%\end{theorem}
%
%\begin{lemma}[Nombre del lema- es opcional]
%Sea V un espacio vectorial etc etc
%\end{lemma}
%
%\begin{dfn}[Base]
%Sea V un espacio vectorial etc etc
%
%\end{dfn}

%%%%%%%%%%%%%%%%


\end{document}